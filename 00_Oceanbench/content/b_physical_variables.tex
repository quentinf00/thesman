\section{Physical Variables} \label{sec:physical_variables}

As alluded to in the main body of the paper, we have access to many physical quantities which can be derived from  sea surface height. 
This gives us a way to analyze how effective and trustworthy are our reconstructions. 
Many machine learning methods are unconstrained so they may provide solutions that are physically inconsistent and visualizing the field is a very easy eye test to assess the validity. 
In addition to post analysis, one could include some of these derived quantities maybe useful as additional inputs to the system and/or constraints to the loss function. 




% \begin{table}[h!]
%   \caption{Table of nomanclature}
%   \label{sample-table}
%   \centering
%   \begin{tabular}{ccl}
%     \toprule
%     Symbol & Size & Description  \\
%     \midrule
%     $\mathbf{x}_s$ & $\mathbb{R}^{D_s}$ & Spatial Coordinates  \\
%     $t$ & $\mathbb{R}^{D_t}$ & Temporal Coordinates  \\
%    $\boldsymbol{f}(\mathbf{x}_s, t)$ & $\mathbb{R}^{D}$ & Spatiotemporal Field  \\
%    $\boldsymbol{y}_{obs}(\mathbf{x}_s, t)$ & $\mathbb{R}^{D_{obs}}$ & Spatiotemporal Observations  \\
%    $\eta$ & $\mathbb{R}$ & Sea Surface Height $[m]$ \\
%    $\bar{\eta}$ & $\mathbb{R}$ & Sea Surface Anomaly $[m]$ \\
%    $u$ & $\mathbb{R}$ & Zonal Velocity $[ms^{-2}]$ \\
%    $v$ & $\mathbb{R}$ & Meridional Velocity $[ms^{-2}]$ \\
%     \bottomrule
%   \end{tabular}
%   \label{tb:notation}
%  \end{table}


% \subsection{Coordinates}
% \textbf{SpatioTemporal Coordiantes}. We define some generic spatiotemporal coordinates.
% 
% We are dealing with satellite observations, so we are interested in the domain across the Earth's surface. 
% Let us define the Earth's domain by some spatial coordinates, $\mathbf{x} = [\text{Longitude},\text{Latitude}]^\top \in\mathbb{R}^{D_s}$, and temporal coordinates, $t=[\text{Time}]\in\mathbb{R}^+$, where $D_s$ is the dimensionality of the coordinate vector.  
% We can define some spatial (sub-)domain, $\Omega\subseteq\mathbb{R}^{D_s}$, and a temporal (sub-)domain, $\mathcal{T}\subseteq\mathbb{R}^+$. 
% This domain could be the entire globe for 10 years or a small region within the North Atlantic for 1 year.
%
%
% \begin{align} \label{eq:spatiotemporal_coords}
%     \text{Spatial Coordinates}: && \mathbf{x} &\in \Omega \subseteq \mathbb{R}^{D_s}\\ 
%     \text{Temporal Coordinates}: && t &\in \mathcal{T} \subseteq \mathbb{R}^{D_t}.
% \end{align}
% %
%
% In this case $D_s=2$ because we only have a two coordinates, however we can do some coordinate transformations like spherical to Cartesian. Likewise, we can do some coordinate transformation for the temporal coordinates like cyclic transformations or sinusoidal embeddings~\cite{ATTENTION}. We have two fields of interest from these spatiotemporal coordinates: the state and the observations.
% %
% %
% \begin{align} \label{eq:state_obs}
%     \text{State}: && \boldsymbol{u}(\mathbf{x},t) &: \Omega\times\mathcal{T}\rightarrow\mathbb{R}^{D_u} \\
%     \text{Observations}: && \boldsymbol{y}_{obs}(\mathbf{x},t) &: \Omega\times\mathcal{T}\rightarrow\mathbb{R}^{D_{obs}}
% \end{align}
% %
% %
% The state domain, $u\in\mathcal{U}$, is a scalar or vector-valued field of size $D_u$ which is typically the quantity of interest and the observation domain, $y_{obs}\in\mathcal{Y}_{obs}$, is the observable quantity which is also a scalar or vector-valued field of size $D_{obs}$. Now, we make the assumption that we have an operator $\mathcal{H}$ that transforms the field from the state space, $\boldsymbol{u}$, to the observation space, $\boldsymbol{y}_{obs}$.
% %
% %
% \begin{align} \label{eq:prob_definition}
%     \boldsymbol{y}_{obs}(\mathbf{x},t) = \mathcal{H}\left(\boldsymbol{u}(\mathbf{x},t), t, \boldsymbol{\varepsilon}, \boldsymbol{\mu}\right) 
% \end{align}
% %
% %


% % \subsection{Field}

% \textbf{Field}. We have the generic definition of a scalar or vector-valued field.

% \begin{align} \label{eq:field}
% \text{Field}:&& f&=\boldsymbol{f}(\mathbf{x},t), && \quad \Omega\times \mathcal{T}\rightarrow\mathbb{R}^{D}
% \end{align}



Recall the spatiotemporal coordinates from equation~\ref{eq:spatiotemporal_coords}, 
we use the same coordinates for the subsequent physical quantities. \textbf{Sea Surface Height} is the deviation of the height of the ocean surface from the geoid of the Earth. We can define it as:
%
\begin{align}
	\text{Sea Surface Height }[m]:&& \quad
 \eta &= \boldsymbol{\eta}(\mathbf{x},t)&& \quad \Omega\times \mathcal{T}\rightarrow\mathbb{R} \label{eq:ssh}
\end{align}
%
This quantity is the actual value that is given from the satellite altimeters and is presented in the products for SSH maps~\cite{DUACS}. An example can be seen in the first row of figure~\ref{fig:oceanbench_maps_4dvarnet}.
%
\textbf{Sea Surface Anomaly} is the anomaly wrt to the spatial mean which is defined by
%
\begin{align}
	\text{Sea Level Anomaly }[m]:&& \quad
 \bar{\eta} &= \boldsymbol{\eta}(\mathbf{x},t) - \bar{\eta}(t) &&
 \quad \Omega\times \mathcal{T}\rightarrow\mathbb{R} \label{eq:sla}
\end{align}
%
where $\bar{\eta}(t)$ is the spatial average of the field at each time step.  
An example can be seen in the first row of figure~\ref{fig:oceanbench_maps}.
%
Another important quantity is the \textbf{geostrophic velocities} in the zonal and meridional directions. This is given by
%
\begin{align}
	\text{Zonal Velocity}[ms^{-2}]:&& \quad
 u &= -\frac{g}{f_0}\frac{\partial \eta}{\partial y} &&
 \quad \Omega\times \mathcal{T}\rightarrow\mathbb{R} \label{eq:u_vel} \\
	\text{Meridional Velocity}[ms^{-2}]:&& \quad
 v &= \frac{g}{f_0}\frac{\partial \eta}{\partial x} &&
 \quad \Omega\times \mathcal{T}\rightarrow\mathbb{R} \label{eq:v_vel}
\end{align}
% 
where $g$ is the gravitational constant and $f_0$ is the mean Coriolis parameter. These quantities are important as they can be an related to the sea surface current. The geostrophic assumption is a very strong assumption however it can still be an important indicator variable. The \textbf{kinetic energy} is a way to summarize the (geostrophic) velocities as the total energy of the system. This is given by
%
\begin{equation} \label{eq:kineticenergy}
    KE = \frac{1}{2}\left(u^2 + v^2\right)
\end{equation}
%
An example can be seen in the second row of figure~\ref{fig:oceanbench_maps_4dvarnet}.
%
%
Another very important quantity is the \textit{vorticity} which measures the spin and rotation of a fluid. In geophysical fluid dynamics, we use the \textbf{relative vorticity} which is the vorticity observed within at rotating frame.
This is given by
%
\begin{equation} \label{eq:relvorticity}
    \zeta = \frac{\partial v}{\partial x} - \frac{\partial u}{\partial y}
\end{equation}
%
An example can be seen in the third row of figure~\ref{fig:oceanbench_maps_4dvarnet}.
%
% \subsection{Absolute Vorticity}

% \begin{equation} \label{eq:absvorticity}
%     |\zeta| = \frac{\partial v}{\partial x} + \frac{\partial u}{\partial y}
% \end{equation}
%
We can also use the \textbf{Enstrophy} to summarize the relative voriticty to measure the total contribution which is given by
%
\begin{equation} \label{eq:enstrophy}
    E = \frac{1}{2}\zeta^2
\end{equation}
%
%
The \textbf{Strain} is a measure of deformation of a fluid flow.
%
\begin{equation} \label{eq:strain}
    \sigma = \sqrt{\sigma_n^2 + \sigma_s^2}
\end{equation}
%
where $\sigma_n$ is the shear strain (aka the shearing deformation) and $\sigma_s$ is the normal strain (aka stretching deformation). An example can be seen in the fourth row of figure~\ref{fig:oceanbench_maps_4dvarnet}.
%
The \textbf{Okubo-Weiss Parameter} is high-order quantity which is a linear combination of the strain and the relative vorticity.
%
\begin{equation} \label{eq:okuboweiss}
    \sigma_{ow} = \sigma_n^2 + \sigma_s^2 - \zeta^2
\end{equation}
%
This quantity is often used as a threshold for determining the location of Eddies in sea surface height and sea surface current fields~\cite{OKUBO, WEISS, OKUBOWEISS}.