\section{Additional Tasks}\label{sec:other_tasks}

In the main paper, we thoroughly outlined the interpolation task to showcase how \texttt{OceanBench} can be used to create automated pipelines for processing and evaluation procedures.
However, there are many other additional tasks that can make use of the \texttt{OceanBench} features. 

\textbf{Denoising}. A simpler problem for interpolation tasks is the denoising problem~\cite{DENOISESURVEY,DENOISESURVEY2}.
The SSH and SST measurements we obtain have inherent noise from the sensors.
A key problem is to calibrate the observations by separating the known noise patterns and the true signal.
There has already been a lot of work from the ML side ranging from amortized predictions~\cite{DENOISESWOT} to end-to-end learning schemes~\cite{DENOISESWOT2}.
Much of this work has been facilitated by the \textit{Ocean-Data-Challenge} group which have a few data challenges related to the denoising problem.
Just like \texttt{OceanBench} was able to create reproducible pipelines from the SSH interpolation challenge listed in section~\ref{sec:data_challenges}, we also believe that one could extend the denoising challenge in the same manner.

\textbf{Forecasting}. This is a special form of extrapolation whereby the temporal domain of the state variable is sufficiently outside of the domain of the observation domain. 
Many previous benchmarking suites already look at forecasting for weather~\cite{weatherbench} and climate~\cite{ClimateBench}.
However, in oceanography, it is also advantageous to do forecasting for problems involving currents~\cite{MLSSC,4DVarNetSSC} and eddies~\cite{OCEANEddyTracking,OKUBO,OKUBOWEISS}.
The \texttt{xrpatcher} will work out of the box for forecasting problems and contributions can be made to \texttt{OceanBench} to include some specific metrics for forecasting as were outlined in~\cite{weatherbench,ClimateBench,ENS10Bench}.

\textbf{Proxy Variables}. There are many other control variables that one could use to improve the interpolation or extrapolation task.
We mentioned SST in section~\ref{sec:data_challenges} because it is the most abundant observations available.
However, there are other important observed variables which could be useful, e.g. Ocean colour, Bigeochemical parameters, and atmospheric variables.
In many other downstream applications, the oceanography community often uses SSH and SST as proxy variables to predict important quantities related to the carbon uptake, e.g. SOCAT~\cite{SOCAT}.
It would be straightforward to include a specific variable (and the associated preprocessing operations) into \texttt{OceanBench}.

\textbf{Dimension Reduction}. We often have very resolution spatiotemporal fields.
which poses a very big challenge for learning due to the high correlations exhibited by spatiotemporal data and high dimensionality.
A workaround for this is to learn a latent representation which retains as much relevant information as possible for the given task.
In the ocean sciences, this is known as \textit{Reduced Order Modeling} (ROM) or more generally dimensionality reduction which has been frequently used for adaptive meshes for physical models~\cite{NEMOEOF}.
This could be used for pretraining fields to latent embeddings which could be useful for downstream tasks like anomaly detection~\cite{SSTFLOWANOMALY}.
% In modern diffusion models, most of the operations are within the latent domain without any loss of quality in the generative results.


\textbf{Surrogate Modeling}. 
Physical model simulations are very expensive and ML has played a role in learning surrogate models to descrease the computational intensity~\cite{ML4OCN,MLCLOSURE}.
We have a decently long spatiotemporal field over a region of interest which could be used to learn a surrogate model to mimic the dynamics of that region.
This is also very useful for hybrid schemes whereby we have parameterizations to account for processes that are missing from low resolution simulations.~\cite{MLOCNPARAMETERIZATION,MLOCNPARAMETERIZATION2, MLOCNPARAMETERIZATION3, MLOCNPARAMETERIZATION4}.



% \subsection{Main Tasks}

% \subsubsection{Interpolation/Extrapolation}

% The two main tasks we can define from this problem setup are: 1) interpolation/extrapolation and 2) forecasting (which can be seen as extrapolation in Time). 
% Both interpolation and extrapolation are when the domain for the $\mathcal{T}$ always falls between the past and the present, i.e., $\mathcal{T}\in[0, T]$ where $T$ is the current Time. 
% We define interpolation as the case where the observations along the boundary of the domain are equal to the boundary of the desired reconstruction domain, i.e., $\partial\boldsymbol{\Omega}_{obs} = \partial\boldsymbol{\Omega}_g$, and extrapolation as the case where the boundary of the observation domain lies entirely inside of the boundary of the reconstructed domain, i.e., $\partial\boldsymbol{\Omega}_{obs} \subset \partial\boldsymbol{\Omega}_g$. 
% Forecasting refers to the problem when the temporal domain $\mathcal{T}$ lies outside of the present, i.e., $T+\tau$ where $\tau$ is some step within the future that is outside of the temporal observation domain, $\mathcal{T}\in[0, T]$. 
% Note that this is irrespective of the spatial domain or its boundaries. 
% In the rest of this paper, we will look exclusively at the interpolation problem, but we refer to the reader to section~\ref{sec:other_tasks} in the appendix for a more detailed look at the other tasks.

% \begin{equation}
%     \mathbf{y} = \mathcal{H}(\mathbf{x})
% \end{equation}


% where $\mathbf{y}\in\boldsymbol{\Omega_p}$ is the incomplete observation within some subdomain and $\mathbf{x}\in\boldsymbol{\Omega}$  is the true observation over the full domain.



% \subsubsection{Forecasting}

% \begin{equation}
%     \boldsymbol{u}_{t+\delta t} = \mathcal{M}(\boldsymbol{u}(\mathbf{x}, t+\delta t), \delta t; \boldsymbol{\theta})
% \end{equation}

% \subsection{SubTasks}

% \subsubsection{Surrogate/Hybrid/Parameterizations}

% \begin{equation}
%     \frac{\partial u}{\partial t} = \mathcal{M}(\boldsymbol{u}(\mathbf{x},t), \mathbf{x}, t; \theta)
% \end{equation}




