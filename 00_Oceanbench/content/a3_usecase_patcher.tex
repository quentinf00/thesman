\section{Use Case III: XRPatcher} \label{sec:xrpatcher}

There are many usecases for the \texttt{XRPatcher}. For example, we can do 1D Time chunking, 2D Spatial-Temporal Patches, or 3D Spatial-Temporal Cubes.


\begin{listing}[h!]
\begin{minted}[frame=lines]{python}
import xarray as xr
import torch
import itertools
from oceanbench import XRPatcher
# Easy Integration with PyTorch Datasets (and DataLoaders)
class XRTorchDataset(torch.utils.data.Dataset):
    def __init__(self, batcher: XRPatcher, item_postpro=None):
        self.batcher = batcher
        self.postpro = item_postpro
    def __getitem__(self, idx: int) -> torch.Tensor:
        item = self.batcher[idx].load().values
        if self.postpro:
            item = self.postpro(item)
        return item
    def reconstruct_from_batches(
            self, batches: list(torch.Tensor), **rec_kws
        ) -> xr.Dataset:
        return self.batcher.reconstruct(
            [*itertools.chain(*batches)], **rec_kws
        )
    def __len__(self) -> int:
        return len(self.batcher)
# load demo dataset
data = xr.tutorial.load_dataset("eraint_uvz")
# Instantiate the patching logic for training
patches = dict(longitude=30, latitude=30)
train_patcher = XRPatcher(
    da=data,
    patches=patches,
    strides=patches,        # No Overlap
    check_full_scan=True    # check no extra dimensions
)
# Instantiate the patching logic for testing
patches = dict(longitude=30, latitude=30)
strides = dict(longitude=5, latitude=5)
test_patcher = XRPatcher(
    da=data,
    patches=patches,
    strides=strides,        # Overlap
    check_full_scan=True    # check no extra dimensions
)
# instantiate PyTorch DataSet
train_ds = XRTorchDataset(train_patcher, item_postpro=TrainingItem._make)
test_ds = XRTorchDataset(test_patcher, item_postpro=TrainingItem._make)
# instantiate PyTorch DataLoader
train_dl = torch.utils.data.DataLoader(train_ds, batch_size=4, shuffle=False)
test_dl = torch.utils.data.DataLoader(test_ds, batch_size=4, shuffle=False)
\end{minted}
\label{listing:xrpatcher}
\caption{This is a snippet showcasing how we can easily integrate PyTorch Datasets within the \texttt{XRPatcher} framework without much overhead. Here we define a custom PyTorch Dataset to handle the \texttt{XRPatcher}. We load an arbitrary dataset with \texttt{xarray}, then we instantiate the \texttt{XRPatcher} with some patching logic, then we instantiate the PyTorch dataset and dataloader as per usual.}
\end{listing}




