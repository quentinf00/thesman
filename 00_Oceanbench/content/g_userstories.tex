\section{Target Audience}
% \subsection{Project Vision}
While this tool is general in scope, we specifically target three audiences: 1) the domain expert who may want to use and understand and investigate SSH in relation to other important EO quantities, 2) machine learning researchers who may want to investigate how to make a better model for SSH interpolation, and 3) downstream users who are interested in adopting some techniques for their own domain-specific applications that may rely on SSH like tracking ocean currents~\tocite{}, investigating biogeochemical transport~\tocite{}, or global climate change~\tocite{}. In the following subsections, we give a more detailed description about the users and how might they benefit from \texttt{OceanBench}.

\textbf{Domain Experts}. We consider \textit{domain experts} who are experts in different domains of oceanography. ...


\textbf{ML Researchers}. We consider those who are expert ML researchers but may lack...



\textbf{Down Stream Users}. We also envision a broader adoption of the framework across research labs interested in having standardized data challenges for their own research purposes. There are independent groups choosing datasets for their specific use cases however, this framework can serve as an easy way to integrate their existing methodologies into the set of common tools to be used and improved by multiple communities. By establishing a relatively consistent problem set, we hope that any innovation can be easily understood and transferred across domains.