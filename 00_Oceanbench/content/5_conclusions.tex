\section{Conclusions} \label{sec:conclusions}

The ocean community faces technological and algorithmic challenges to make the most of available observation and simulation datasets. 
In this context, recent studies evidence the critical role of ML schemes in reaching breakthroughs in our ability to monitor ocean dynamics for various space-time scales and processes. 
Nevertheless, domain-specific preprocessing steps and evaluation procedures slow down the uptake of ML toward real-world applications. 
The application considered here is SSH mapping which facilities the production of many crucial derived products that are used in many downstream tasks like subsequent modeling~\citep{ML4OCN}, ocean health monitoring~\citep{ML4NATURECONSERVATION,OCNHEALTH,OCEANHEALTH2} and maritime risk assessment~\citep{SSHOPERATIONAL}.

We proposed four challenges towards a ML-ready benchmarking suite for ocean observation challenges. 
We outlined the inner workings \texttt{OceanBench} and demonstrated its usefulness by recreating some preprocessing and analysis pipelines from a few data challenges involving SSH interpolation.
We firmly believe that the \texttt{OceanBench} platform is instrumental in fostering greater ML method adoption by the ocean community, while also rallying a larger portion of the ML community to tackle the ocean's scientific complexities.


