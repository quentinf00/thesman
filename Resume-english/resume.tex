\begin{bibunit}[IEEEtran.bst]


This thesis explores how advancements in deep learning can aid in the analysis of satellite measurements of sea surface height (SSH).
Current altimeters provide sparsely and irregularly sampled data, limiting the observation of finer processes.
Pushing this limit would improve our knowledge of ocean surface dynamics and our climate monitoring capabilities.
New opportunities to enhance our observational capabilities have arisen with the deployment of the KaRIn sensor during the SWOT mission.
In recent years, deep learning approaches have demonstrated remarkable capabilities in fields like computer vision and natural language processing.
Contrary to such tasks, ocean observations problems can involve strong physical priors as well as little annotated data.
This work addresses the specific considerations of applying deep learning to altimetry data in three parts.
Firstly, through the calibration of the KaRIn sensor, we demonstrate how specific domain knowledge can be integrated into deep learning frameworks. We show specifically how the spectral error budget of the SWOT mission can inform the design of neural architectures.
Secondly, we address the scarcity of ground truth data when training neural-based methods for interpolating sparse SSH measurements. We illustrate how simulations of ocean models and observing systems can overcome this challenge by providing supervised training environments that generalize to real data.
Finally, our third contribution addresses the challenges encountered in bridging the gap between the ocean observation and deep learning communities. Effective collaboration requires experts from the ocean community to define significant problems and specify relevant ways of to assess the solutions.
We describe how we addressed these aspects throughout the development of the OceanBench project.
%   \section{Summary}
% Understanding and anticipating the ramifications of climate change represents a pressing challenge of our era. Enhancing our knowledge of Earth's systems is a key factor in confronting this challenge.
%   Given that the primary source of factual information about the Earth system is observational data, improving our ability to exploit this data could lead to better monitoring and understanding of our planet.
%   Concurrently, recent advancements in deep learning provide robust tools that continually push performance boundaries across a myriad of tasks. 
%   This situation raises an intriguing question: Can deep learning assist in extracting meaningful insights from Earth observations?

% Observing ocean surface dynamics through satellite altimetry offers a compelling case study.
%   Currently, operational products do not resolve processes below 150 km, which are essential for climate monitoring.
%   This situation underscores a significant gap in our observational capabilities.
%   The recent deployment of a novel sensor during the SWOT satellite mission provides numerous opportunities to address this gap.
%   This new sensor introduces unprecedented calibration challenges due to previously unseen errors, but it also promises to enhance the reconstruction of Sea Surface Height (SSH) maps.

% Despite the significant potential of deep learning as a generic tool, two critical factors seem to determine progress of its application to a specific domain. The first factor is quality and availability of data, indeed the creation of large, curated datasets, like ImageNet in computer vision or ThePile in natural language processing have shown to dramatically expedite the development of novel approaches. The second factor is the design of informed architectural patterns that are particularly suited to the given problem, leading to performance breakthroughs. Examples of these include convolution techniques in computer vision, attention mechanisms in natural language processing, and U-Net architectures for hierarchical data. These two facets — comprehensive, well-curated datasets and efficient architectural patterns — can greatly advance the field, propelling research and application development in exciting directions.

% The transdisciplinary nature of this work also introduces unique challenges. The intricacies of ocean observation data and the criteria for precisely evaluating the estimation of geophysical quantities can represent a considerable barrier for ML scientists. Similarly, the logistical aspects and accumulated best practices required to successfully train and utilize a neural network can deter domain experts from leveraging the latest advancements.

% The first chapter of this thesis will present a generic problem formulation that aligns domain expert methods and deep learning approaches within a unified ontology. This will lay a solid foundation for understanding how these two domains provide complementary perspectives on a problem and can potentially be integrated. Additionally, this chapter will serve as a review of the state-of-the-art in this research area.

% The next two chapters of this dissertation consider two use cases of altimetry data analysis while shedding light on various deep learning challenges.

% The second chapter delves into the calibration of the SWOT KaRIn data, examining how to separate the SSH from error signals. From a learning standpoint, this chapter showcases a method for integrating the a priori knowledge we have about error signals into the architectural design of a neural-based method.

% The third chapter emphasizes the challenges of training neural mapping schemes on observational data due to the lack of knowledge about the true state of the ocean. It focuses on the task of interpolating SSH fields with a high rate of missing data. The chapter further demonstrates that current numerical simulations of the ocean permit the training of a neural data assimilation scheme that generalizes effectively to real data.

% The subsequent two chapters document our efforts to facilitate the collaboration between the ML and ocean observation communities. Chapter four introduces a comprehensive toolbox for designing and evaluating ML problems related to altimetry mapping. Chapter 5 presents a didactic and modular implementation of the 4DVarNet deep learning algorithm, which has seen use in a variety of publications related to ocean observation data.


% \chapter*{Résumé en français}

% \section*{Motivations}

% \section*{Objectifs}

% \section*{Contributions}

% \section*{Contenu du manuscrit}

% % In this manuscript I'd like to cite \cite{remo1,remo2}.

% \putbib[./Resume-Francais/Res-Biblio.bib]
\end{bibunit}
