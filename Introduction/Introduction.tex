\begin{bibunit}[IEEEtran.bst]

\chapter*{Introduction}
\addcontentsline{toc}{chapter}{Introduction}
\chaptermark{Introduction}


\section*{Motivation}
\addcontentsline{toc}{section}{Motivation}

Understanding and anticipating the ramifications of climate change represents a pressing challenge of our era. Enhancing our knowledge of Earth's systems is a key factor in confronting this challenge. Given that the primary source of factual information about the Earth system is observational data, improving our ability to exploit this data could lead to better monitoring and understanding of our planet. Concurrently, recent advancements in deep learning provide robust tools that continually push performance boundaries across a myriad of tasks. This situation raises an intriguing question: Can deep learning assist in extracting meaningful insights from Earth observations?

Observing ocean surface dynamics through satellite altimetry offers a compelling case study. Currently, operational products do not resolve processes below 150 km, which are essential for climate monitoring. This situation underscores a significant gap in our observational capabilities. The recent deployment of a novel sensor during the SWOT satellite mission provides numerous opportunities to address this gap. This new sensor introduces unprecedented calibration challenges due to previously unseen errors, but it also promises to enhance the reconstruction of Sea Surface Height (SSH) maps.

Despite the significant potential of deep learning, two critical factors seem to determine progress in a specific domain. The first factor is quality and availability of data, indeed the creation of large, curated datasets, like ImageNet in computer vision or ThePile in natural language processing have shown to dramatically expedite the development of novel approaches. The second factor is the design of architectural patterns that are particularly suited to given problem, leading to performance breakthroughs. Examples of these include convolution techniques in computer vision, attention mechanisms in natural language processing, and U-Net architectures for hierarchical data. These two facets — comprehensive, well-curated datasets and efficient architectural patterns — can greatly advance the field, propelling research and application development in exciting directions.

The transdisciplinary nature of this work also introduces unique challenges. The intricacies of ocean observation data and the criteria for precisely evaluating the estimation of geophysical quantities can represent a considerable barrier for ML scientists. Similarly, the logistical aspects and accumulated best practices required to successfully train and utilize a neural network can deter domain experts from leveraging the latest advancements.

The first chapter of this thesis will present a generic problem formulation that aligns domain expert methods and deep learning approaches within a unified ontology. This will lay a solid foundation for understanding how these two domains provide complementary perspectives on a problem and can potentially be integrated. Additionally, this chapter will serve as a review of the state-of-the-art in this research area.

The next two chapters of this dissertation consider two use cases that correspond to different stages of altimetry data analysis while shedding light on various deep learning challenges.

The second chapter delves into the calibration of the SWOT KaRIn data, examining how to separate the SSH from error signals. From a learning standpoint, this chapter showcases a method for integrating the a priori knowledge we have about error signals into the architectural design of a neural-based method.

The third chapter emphasizes the challenges of training neural mapping schemes on observational data due to the uncertainty about the true state of the ocean. It focuses on the task of interpolating SSH fields with a high rate of missing data. The chapter further demonstrates that current numerical simulations of the ocean permit the training of a neural data assimilation scheme that generalizes effectively to real data.

The subsequent two chapters document our efforts to facilitate the collaboration between the ML and ocean observation communities. Chapter four introduces a comprehensive toolbox for designing and evaluating ML problems related to altimetry mapping. Chapter 5 presents a didactic and modular implementation of the 4DVarNet deep learning algorithm, which has seen use in a variety of publications related to ocean observation data.


% Understanding and anticipating the implications of climate change is a pressing challenge of our era, and improving our knowledge of Earth's systems is a critical factor in meeting this challenge. Given that factual information about the state of the Earth system is derived from observational data, the recent advancements in deep learning provide potent tools for enhancing our knowledge, pushing performance boundaries across an expanding range of tasks. This raises an intriguing question: Can deep learning assist in extracting meaningful knowledge from Earth observations?
%
%
% The first source of challenge we can encounter comes from the 
% Observing ocean surface dynamics through satellite altimetry provides an engaging case study. Currently, operational products do not resolve processes below 150 km, which play a pivotal role in climate monitoring. This situation reveals a significant gap in our observational capabilities. The recent deployment of a novel sensor during the SWOT satellite mission presents multiple opportunities to bridge this gap. This new sensor brings unprecedented challenges for calibration due to previously unseen errors and promises the prospect of improving the reconstruction of Sea Surface Height (SSH) maps.
%
%
% Deep learning, while offering significant advantages, also presents certain challenges. Two crucial factors appear to be determinants for progress in a specific domain through deep learning. First is the creation of large, curated datasets, such as ImageNet or The Pile, which can dramatically speed up the development of novel approaches. Second, the discovery of architectural patterns that are well-suited for a specific domain can trigger performance leaps. Examples include convolution techniques in computer vision, attention mechanisms in natural language processing, and U-Net architectures for hierarchical data. These two aspects — well-curated datasets and effective architectural patterns — can fundamentally advance the field, propelling research and application development in novel directions.
%
%   The transdisciplinary aspect of this work also poses unique challenges. The specificities of ocean observation data and the criteria for accurately evaluating the estimation of geophysical quantities can pose a substantial barrier of entry for ML scientists. Similarly, the logistics and accumulated tools and tricks necessary to successfully train and use a neural network can be a hindrance for domain experts wishing to leverage the latest advancements.
%
% The first two chapters of this dissertation look at two use cases that point to different stages of altimetry data analysis while highlighting on different deep learning challenges.
%
% The first chapter focuses on the calibration of the SWOT KaRIn data, and how we can separate the SSH from error signals. From a learning standpoint, this chapter demonstrates a method for incorporating the a priori knowledge we have about the error signals into the architectural design of a neural-based method.
%
% The second chapter highlights the challenges of training neural mapping schemes on observational data due to the lack of knowledge about the true state of the ocean. It focuses on the interpolation taks of SSH fields with a high rate of missing data. It further demonstrates that current numerical simulations of the ocean allow for training a neural data assimilation scheme that generalizes well on real data.
%
% The subsequent two chapters present our efforts to strengthen the bridge between the ML and ocean observation community. Chapter 3 introduces a comprehensive toolbox for designing and evaluating ML problems related to altimetry mapping. In chapter 4, we present a pedagogical and modular implementation of the 4DVarNet algorithm, which has been utilized in a variety of publications related to ocean observation data.
%
%
\section*{Tasks}
\addcontentsline{toc}{section}{Tasks}
  Most tasks can formulated as finding a mapping $f$ from available data $y$ to a quantity of interest $u$

\section*{Methodology}
\addcontentsline{toc}{section}{Methodology}



% Understanding and anticipating the implications of climate change is a critical challenge for our time, and improvements in our knowledge of the Earth's systems are one of the most impactful ways we can meet this challenge. Moreover, given that factual informations we have on the earth system state comes from observation data and that recent advances in deep learning offer us powerful tools to enhance our knowledge, pushing the performance boundaries across a growing range of tasks. This raises the question: can deep learning assist in extracting knowledge from earth observations ?
%
% Observation of ocean surface dynamics through satellite altimetry serves as an interesting case study. Presently, operational products do not resolve processes below 150 km which play a key role in climate monitoring, indicating a significant gap in our observational capabilities. And multiple opportunities arise to reduce this gap with recent deployment of a novel sensor during the SWOT satellite mission. This new sensor brings novel need for calibration of previously unseen errors and opens perspectives of improving the reconstruction of SSH maps.
%
%  Two wings are needed to make a machine learning model fly: the optimization and the generalization. Indeed training a machine learning model consist in searching a parameter space (i.e optimization) to find a model configuration that will perform well on unseen data (generalization). Dataset, training objective and optimization procedure will intervene in the parameter search whereas informed decision on the architecture will determine if the trained model behaves well outside the training context.
%
%correlate with  in deep learning One may see deep learning as generic tools, however we can note how advances i significant advances were made when architectural choice 
% To make a machine learning model perform effectively, both optimization and generalization are vital. Training a machine learning model involves searching a parameter space (i.e., optimization) to find a model configuration that will perform well on unseen data (generalization). The dataset, training objective, and optimization procedure participate in the parameter search, while informed decisions on the architecture determine if the trained model behaves appropriately outside the training context. When applying learning-based methods to a new applicative domain, the choice of data and decision about the architecture should be made in accord with the domain requirements. 


% The transdisciplinary aspect of this work also brings its share of challenges, indeed the specificities  ocean observation data and the criteria for correctly evaluating the estimation of geophysical quantities can be a significant barrier of entry for ML scientist.
%  Similarly the logistics and accumulated tools and tricks necessary to successfully train and use a neural network can be a hindrance for domain experts to take advantages of the latest advances made.
%
% This first two parts look at two use cases that point to different stages of an observation data analysis and that tackle different aspects of the learning challenges.
%
% The first chapter will focus on the beginning of the observing system chain which is the calibration of the SWOT KaRIn data and how we can separate the SSH from error signals. This example showcases a way of incorporating the a priori knowledge we have about the error signals into the architectural design of a neural based method.
%
% The second part will focus on the processing of calibrated data and how to interpolate SSH fields with high rate of missing data. This section highlight the challenges of training neural mapping schemes on observation data by lack of knowledge of the true ocean state and demonstrate that current numerical simulations of the ocean allow for training a neural data assimilation scheme that generalizes well on real data.
%
% The next two chapters present our effort in reinforcing the brigde between the ML and ocean observation community by presenting in chapter 3 a comprehensive tooolbox for designing and evaluating ML problems related to altimetry mapping. In chapter 4 we present a pedagogical and modular implementation of the 4DVarNet algorithm which has been used in a variety of publication related with ocean observation data. 
%
% Welcoming applied ml researchers to altimetry mapping
% Sharing generic neural data assimilation framework
% Sensor calibration inductive biases
% Blind optimization of mapping algorithm
%
% The application of deep learning to satellite altimetry poses its own challenges, the first one of which is to formulate a learning problem that is relevant for domain expert through data and evaluation metrics.
%
%  The science of ocean observation implicates various fields of expertise. We can distinguish at the ends of the spectrum the "ocean theoritical scientists" working to improve the physical and numericals model the ocean system and the "ocean observation scientist" trying to improve the information we gather from the ocean observing system through sensor design and data procesing. Both theoritical and observation scientists rely on each other as data is used to calibrate models and model simulation are used to design observing systems. 
%
% In a first part will
% T
%   When formulating a learning problem, two sides have to be considered the optimization and the generalization. Indeed training a machine learning model consist in searching a parameter space (i.e optimization) to find a model configuration that will perform well on unseen data (generalization). Dataset, training objective and optimization procedure will intervene in the parameter search whereas informed decision on the architectures will determine 
% The application of deep learning to satellite altimetry poses its own challenges, one of which is bridging the conceptual differences between deep learning approaches and current domain expert methods.
% This thesis poses the question 
% Whereas some work try to see how deep learning can help in the theoritical and modeling challenges. This thesis focuses on leveraging on the observation processing. has been focus in this thesis 
% Deep learning advances consist in searching high dimensional spaces using large amount of data
%
%
% In a first 
%
% Understanding how deep learning approaches fit in the landscape of domain expert methods
% How 
%
%
%
%
%
% The reliance on 
%   My PhD research focused on this area, specifically, exploring how deep learning advances could benefit the observation of ocean surface dynamics using satellite altimetry data.
%
%
% Improving our knowledge of the earth system may be one the most impactful levers for better understanding and anticipating climate change ramifications.
% Deep learning provides a set of powerful tools that constantly push the boundaries the performances that can be reached on a growing range of tasks.
% So what are the issues met when observing the earth, what tools are brought by deep learning, and can the latters help with the formers ?
%
% The observation of ocean surface dynamics through satellite altimetry are an interesting usecase since the processes below 150km are not resolved in current operational products the deployement 
%
% Earth observation involve different expertise and synergies between different disciplines.
% The 
%   Innovations starts at the sensors, which are 
%
%   , some work on designing sensors that can capture relevant quantities. 
% Some build the theoritical foundations and characterize the physical processes of the earth system while others 
%
% We can 
% Observing the earth
% In order to 
% Observation science spans a wide range of skills
% % Observing and understanding the earth system is one of the biggest challenge in the context climate change.
% However in order to apply the deep learning tool to the earth observation problem, one need 
% In order to explore deep learning can help better observe the earth system, 
% What are the conceptual differences between deep learning approaches and current domain experts
%
%
% During my PhD, my main focus was to explore how deep learning advances could benefit the observation of ocean surface dynamics. 
% To this end I worked with satellite altimetry data which measures the ocean surface topography which itself is related the geostrophic currents.
%
% I worked on two problems along the observation pipeline which are:
% - Estimating the SSH of uncalibrated sensor data
% - Estimating the SSH from partial measurements
%
%
% Like a large range of tasks these problems can be framed as finding the best mapping $f: \Omega_y \to \Omega_u$ between available inputs  $y \Omega_y $ and desired outputs $u \in \Omega_u} $. 
%   $$ u  = f(y)$$
%
% In order to separate the SSH signals from errors or generalize it to unobserved areas, these tasks involve incorporating knowledge about the sensors and the ocean dynamics in $f$.
%
% This is done in two stages:
%   - Defining the set of possible mappings $\cal{F} | f \in \cal{F}$ using a priori theoritical knowledge
%   - Searching the space $\cal{F}$ for the best $f$ using data.
%
% Using this perspective, deep learning approaches consists in searching high dimensional $F$ using a lot of data and efficient optimization procedures.
%
% Such appoaches have managed to broaden the what was achievable in areas such as computer vision and natural language processing.
%
% Geoscience 
%
% This in
% deep learning brings interesting perspectives. Indeed whereas domain experts 
% We 
%
% We can see learning as the combination of optimization and generalization.
% Different approaches with different expertise will 
% Those two tasks highlight the challenges of observation science 
% This task requires to exploit the complex dynamics of the earth and ocean systems. 
%
%
%
%
%
%
% A simple parallel. 
% We have a room and want to know the temperature.
%  - We put some mercury in a tube and put it in the room: How do we link the mercury level to the sensor temperature ?
%  - How do we link the sensor temperature to the room temperature ?
%
%
%
%
% This thesis is organized as follows:
%
%
% \section*{Satellite altimetry calibration and mapping}
% \addcontentsline{toc}{section}{Satellite altimetry}
%
% \section*{Deep learning specificities}
% \addcontentsline{toc}{section}{Satellite altimetry}
% \begin{itemize}
%     \item \autoref{chap:1}
%     \item \autoref{chap:2}
%     \item \autoref{chap:conclusions}
% \end{itemize}
%
% In this manuscript I'd like to cite \cite{remo1,remo2}.

% \addcontentsline{toc}{section}{Bibliography}
% \putbib[./Introduction/Intro-Biblio.bib]
\end{bibunit}

% \begin{bibunit}[IEEEtran.bst]
%
% \chapter*{Introduction}
% \addcontentsline{toc}{chapter}{Introduction}
% \chaptermark{Introduction}
%
% Lorem ipsum dolor sit amet, «~consectetuer~» adipiscing elit. Maecenas fermentum, elit non lobortis cursus, orci velit suscipit est, id mollis turpis mi eget orci.
%
% Lorem ipsum dolor sit amet, consectetuer adipiscing elit. Maecenas fermentum, elit non lobortis cursus, orci velit suscipit est, id mollis turpis mi eget orci.
%
% \section*{Première section de l'introduction}
% \addcontentsline{toc}{section}{Première section de l'intro}
%
% Lorem ipsum dolor sit amet, «~consectetuer~» adipiscing elit. Maecenas fermentum, elit non lobortis cursus, orci velit suscipit est, id mollis turpis mi eget orci.
%
% \textbf{Une boite magique : }
%
% \boitemagique{Titre de la boite}{
% Praesent placerat, ante at venenatis pretium, diam turpis faucibus arcu, nec vehicula quam lorem ut leo. Sed facilisis, augue in pharetra dapibus, ligula justo accumsan massa, eu suscipit felis ipsum eget enim.
% }
%
% Laoreet iaculis, nonummy eget, massa. Phasellus ullamcorper commodo velit. Class aptent taciti sociosqu ad litora torquent per «~conubia nostra~», per inceptos hymenaeos. Phasellus est. Maecenas felis augue, gravida quis, porta adipiscing, iaculis vitae, felis.
%
% \textbf{Une boite simple : }
% \boitesimple{Mauris lorem quam, tristique sollicitudin egestas sed, sodales vel leo. In hac habitasse platea dictumst. Lorem ipsum dolor sit amet, consectetur adipiscing elit. Sed sed lorem lacus, at venenatis elit. Pellentesque nisl arcu, blandit ac eleifend non, sodales a quam.}
%
% Laoreet iaculis, nonummy eget, massa. Phasellus ullamcorper commodo velit.
%
% This thesis is organized as follows:
%
% \begin{itemize}
%     \item \autoref{chap:1}
%     \item \autoref{chap:2}
%     \item \autoref{chap:conclusions}
% \end{itemize}
%
% % In this manuscript I'd like to cite \cite{remo1,remo2}.
%
% % \addcontentsline{toc}{section}{Bibliography}
% % \putbib[./Introduction/Intro-Biblio.bib]
% \end{bibunit}
% \begin{bibunit}[IEEEtran.bst]
%
% \chapter*{Introduction}
% \addcontentsline{toc}{chapter}{Introduction}
% \chaptermark{Introduction}
%
% Lorem ipsum dolor sit amet, «~consectetuer~» adipiscing elit. Maecenas fermentum, elit non lobortis cursus, orci velit suscipit est, id mollis turpis mi eget orci.
%
% Lorem ipsum dolor sit amet, consectetuer adipiscing elit. Maecenas fermentum, elit non lobortis cursus, orci velit suscipit est, id mollis turpis mi eget orci.
%
% \section*{Première section de l'introduction}
% \addcontentsline{toc}{section}{Première section de l'intro}
%
% Lorem ipsum dolor sit amet, «~consectetuer~» adipiscing elit. Maecenas fermentum, elit non lobortis cursus, orci velit suscipit est, id mollis turpis mi eget orci.
%
% \textbf{Une boite magique : }
%
% \boitemagique{Titre de la boite}{
% Praesent placerat, ante at venenatis pretium, diam turpis faucibus arcu, nec vehicula quam lorem ut leo. Sed facilisis, augue in pharetra dapibus, ligula justo accumsan massa, eu suscipit felis ipsum eget enim.
% }
%
% Laoreet iaculis, nonummy eget, massa. Phasellus ullamcorper commodo velit. Class aptent taciti sociosqu ad litora torquent per «~conubia nostra~», per inceptos hymenaeos. Phasellus est. Maecenas felis augue, gravida quis, porta adipiscing, iaculis vitae, felis.
%
% \textbf{Une boite simple : }
% \boitesimple{Mauris lorem quam, tristique sollicitudin egestas sed, sodales vel leo. In hac habitasse platea dictumst. Lorem ipsum dolor sit amet, consectetur adipiscing elit. Sed sed lorem lacus, at venenatis elit. Pellentesque nisl arcu, blandit ac eleifend non, sodales a quam.}
%
% Laoreet iaculis, nonummy eget, massa. Phasellus ullamcorper commodo velit.
%
% This thesis is organized as follows:
%
% \begin{itemize}
%     \item \autoref{chap:1}
%     \item \autoref{chap:2}
%     \item \autoref{chap:conclusions}
% \end{itemize}
%
% % In this manuscript I'd like to cite \cite{remo1,remo2}.
%
% % \addcontentsline{toc}{section}{Bibliography}
% % \putbib[./Introduction/Intro-Biblio.bib]
% \end{bibunit}
