\markboth{}{}
% Plus petite marge du bas pour la quatrième de couverture
% Shorter bottom margin for the back cover
\newgeometry{inner=30mm,outer=20mm,top=40mm,bottom=20mm}

%insertion de l'image de fond du dos (resume)
%background image for resume (back)
\backcoverheader

% Switch font style to back cover style
\selectfontbackcover{ % Font style change is limited to this page using braces, just in case

\titleFR{Apprentissage profond pour l'altimétrie satellitaire océanique : sp\'{e}cificit\'{e}s et implications pratiques.}

\keywordsFR{Apprentissage profond, Altimétrie, SWOT}

\abstractFR{Cette thèse explore comment les avancées en apprentissage profond peuvent aider à l'analyse des mesures satellitaires de la hauteur de surface de la mer (SSH). Les altimètres actuels fournissent des données échantillonnées de manière irrégulière limitant l'observation des processus les plus fins. Repousser cette limite améliorerait nos capacités de surveillance du climat. D'excitantes opportunités ont émergées avec la mission SWOT. Les approches d'apprentissage ont démontré des capacités remarquables dans de nombreux domaines. Cette thèse aborde les considérations spécifiques de l'application de l'apprentissage profond aux données altimétriques en trois parties.

Premièrement, à travers l'étalonnage du capteur KaRIn, nous démontrons comment des connaissances spécifiques du domaine peuvent être intégrées dans les cadres d'apprentissage profond. Deuxièmement, nous abordons la rareté des données de vérité terrain lors de l'apprentissage de méthodes  d'interpolation de données altimétriques. Nous illustrons comment les simulations de modèles océaniques et de systèmes d'observation peuvent surmonter ce défi en fournissant des environnements d'entraînement supervisés qui se généralisent aux données réelles.Enfin, notre troisième contribution traite des défis rencontrés pour combler le fossé entre les communautés "océan" et "apprentissage profond". Nous décrivons comment nous avons abordé ces aspects lors du développement du projet OceanBench.}



\titleEN{Deep Learning for ocean satellite altimetry : specificities and
practical implications}

\keywordsEN{Deep Learning, Altimetry, SWOT}

\abstractEN{This thesis explores how advancements in deep learning can aid in the analysis of satellite measurements of sea surface height (SSH). Current altimeters provide data sampled in an irregular manner, limiting the observation of finer processes. Pushing this limit would enhance our climate monitoring capabilities. Exciting opportunities have emerged with the SWOT mission. Learning approaches have shown remarkable capabilities in many areas. This thesis addresses the specific considerations of applying deep learning to altimetry data in three parts.

First, through the calibration of the KaRIn sensor, we demonstrate how specific domain knowledge can be integrated into deep learning frameworks. Second, we address the scarcity of ground truth data when learning altimetry data interpolation methods. We illustrate how ocean model simulations and observation systems can overcome this challenge by providing supervised training environments that generalize to real data. Lastly, our third contribution discusses the challenges faced in bridging the gap between the "ocean" and "deep learning" communities. We describe how we approached these aspects during the development of the OceanBench project.}

}

% Rétablit les marges d'origines
% Restore original margin settings
\restoregeometry
