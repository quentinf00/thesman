%This file is the main file where the final document is generated

\documentclass{these-ubl} 
% % if you need to pass options to natbib, use, e.g.:
%     \PassOptionsToPackage{numbers, compress}{natbib}
% before loading neurips_data_2023

% ready for submission
% \usepackage{neurips_data_2023}
% \usepackage[preprint]{./00_Oceanbench/neurips_data_2023}

% to compile a preprint version, add the [preprint] option, e.g.:
%     \usepackage[preprint]{neurips_data_2023}
% This will indicate that the work is currently under review.

% to compile a camera-ready version, add the [final] option, e.g.:
%     \usepackage[final]{neurips_data_2023}

% to avoid loading the natbib package, add option nonatbib:
%    \usepackage[nonatbib]{neurips_data_2023}

% Submissions to the datasets and benchmarks are typically non anonymous,
% but anonymous submissions are allowed. If you feel that you must submit 
% anonymously, you can compile an anonymous version by adding the [anonymous] 
% option, e.g.:
%     \usepackage[anonymous]{neurips_data_2023}
% This will hide all author names.

\usepackage{color}
\usepackage{soul}
\usepackage[dvipsnames]{xcolor}
% \usepackage[normalem]{ulem}
\newcommand{\todo}[1]{\textcolor{BrickRed}{[\textbf{TODO}]} }
\newcommand{\tocite}[1]{\textcolor{Plum}{[\textbf{CITE}: #1]}}
\newcommand{\hlc}[2][yellow]{{%
    \colorlet{foo}{#1}%
    \sethlcolor{foo}\hl{#2}}%
}
\newcommand{\tofix}[1]{\hlc[red!20]{#1}}

\usepackage[utf8]{inputenc} % allow utf-8 input
\usepackage[T1]{fontenc}    % use 8-bit T1 fonts
\usepackage{hyperref}       % hyperlinks
\usepackage{url}            % simple URL typesetting
\usepackage{booktabs}       % professional-quality tables
\usepackage{nicefrac}       % compact symbols for 1/2, etc.
\usepackage{microtype}      % microtypography
\usepackage{xcolor}         % colors
\usepackage{multirow}
\usepackage{graphicx}
%% MATH PACKAGES
\usepackage{amsfonts}       % blackboard math symbols
\usepackage{amsmath}
\usepackage{nicefrac} 
\usepackage{lscape}


\usepackage{listings}
% \usepackage{minted}
% \usepackage[cachedir=.]{minted}
% \usepackage{frozencache}
% \usepackage[frozencache=true,cachedir=minted-cache]{minted} 
\usepackage[finalizecache=true,cachedir=minted-cache]{minted}
% \usepackage[cachedir=.]{minted}

% \usepackage{appendix}

% here is a macro expanding to the name of the language
% (handy if you decide to change it further down the road)

% \newcommand\YAMLcolonstyle{\color{red}\mdseries}
\newcommand\YAMLkeystyle{\color{black}\bfseries}
\newcommand\YAMLvaluestyle{\color{blue}\mdseries}

\makeatletter

% here is a macro expanding to the name of the language
% (handy if you decide to change it further down the road)
\newcommand\language@yaml{yaml}

\expandafter\expandafter\expandafter\lstdefinelanguage
\expandafter{\language@yaml}
{
  keywords={true,false,null,y,n},
  keywordstyle=\color{darkgray}\bfseries,
  basicstyle=\YAMLkeystyle,                                 % assuming a key comes first
  sensitive=false,
  comment=[l]{\#},
  morecomment=[s]{/*}{*/},
  commentstyle=\color{purple}\ttfamily,
  stringstyle=\YAMLvaluestyle\ttfamily,
  moredelim=[l][\color{orange}]{\&},
  moredelim=[l][\color{magenta}]{*},
  moredelim=**[il][\YAMLcolonstyle{:}\YAMLvaluestyle]{:},   % switch to value style at :
  morestring=[b]',
  morestring=[b]",
  literate =    {---}{{\ProcessThreeDashes}}3
                {>}{{\textcolor{red}\textgreater}}1     
                {|}{{\textcolor{red}\textbar}}1 
                {\ -\ }{{\mdseries\ -\ }}3,
}

% switch to key style at EOL
\lst@AddToHook{EveryLine}{\ifx\lst@language\language@yaml\YAMLkeystyle\fi}
\makeatother

\newcommand\ProcessThreeDashes{\llap{\color{cyan}\mdseries-{-}-}}


\definecolor{codegreen}{rgb}{0,0.6,0}
\definecolor{codegray}{rgb}{0.5,0.5,0.5}
\definecolor{codepurple}{rgb}{0.58,0,0.82}
\definecolor{backcolour}{rgb}{0.95,0.95,0.92}

\lstdefinestyle{pythonstyle}{
    % backgroundcolor=\color{backcolour},   
    commentstyle=\color{codegreen},
    keywordstyle=\color{magenta},
    numberstyle=\tiny\color{codegray},
    stringstyle=\color{codepurple},
    basicstyle=\ttfamily\footnotesize,
    breakatwhitespace=false,         
    breaklines=true,                 
    captionpos=b,                    
    keepspaces=true,                 
    numbers=left,                    
    numbersep=5pt,                  
    showspaces=false,                
    showstringspaces=false,
    showtabs=false,                  
    tabsize=2
}

\lstset{style=pythonstyle}
% \usepackage{apacite}
%\RequirePackage{natbib}
% \let\cite\shortcite %xx So get et al. with three authors the first time.
% %\let\citep\cite %xx A natbib command.
% \let\citet\shortciteA %xx Ditto.
% \usepackage{url}

% \bibliographystyle{apacite}

\usepackage{color}
\usepackage{soul}
% \usepackage[normalem]{ulem}
\newcommand{\hlc}[2][yellow]{{%
    \colorlet{foo}{#1}%
    \sethlcolor{foo}\hl{#2}}%
}
\newcommand{\tofix}[1]{\hlc[red!20]{#1}}

\usepackage[utf8]{inputenc} % allow utf-8 input
\usepackage[T1]{fontenc}    % use 8-bit T1 fonts
\usepackage{hyperref}       % hyperlinks
\usepackage{url}            % simple URL typesetting
\usepackage{booktabs}       % professional-quality tables
\usepackage{nicefrac}       % compact symbols for 1/2, etc.
\usepackage{microtype}      % microtypography
\usepackage{xcolor}         % colors
\usepackage{multirow}

\usepackage{nicefrac} 
\usepackage{lscape}


\usepackage{listings}
% \usepackage{minted}
% \usepackage[cachedir=.]{minted}
% \usepackage{frozencache}
% \usepackage[frozencache=true,cachedir=minted-cache]{minted} 
\usepackage[finalizecache=true,cachedir=minted-cache]{minted}

\usepackage{bibunits}
\usepackage{amsmath}
\usepackage{amsfonts} 
\usepackage[caption=false,font=normalsize,labelfont=sf,textfont=sf]{subfig}
\usepackage{textcomp}
\usepackage{stfloats}
\usepackage{url} %this package should fix any errors with URLs in refs.
\usepackage{lineno}
\usepackage{booktabs}
\usepackage{graphicx}
\usepackage{subcaption}
\usepackage{float}
\geometry{margin=4.0cm}

\input{./Couverture-these/pagedegarde} % page de garde UR1

% Select the content language following this line
\selectlanguage{english}

%input acknowledgement chapter 
\clearemptydoublepage
\chapter*{Acknowledgement}


The work presented in this manuscript was supported by the French National Research Agency (ANR) Melody and OceaniX and CNES, through projects number ANR-17- CE01-0009-01, ANR-19-CE46-0011 and ANR-19-CHIA-0016); by the French National Space Agency (CNES) through the SWOT Science Team program (projects MIDAS and DIEGO) and the OSTST program (project DUACS-HR); by the French National Centre for Scientific Research (CNRS) through the LEFE-MANU program (project IA-OAC). This project also received funding from the European Union’s Horizon Europe research and innovation programme under the grant No 101093293 (EDITO-Model Lab project). This project benefited from HPC and GPU computing resources from GENCI-IDRIS (Grant 2021-101030).


%Inut resume en Francais
\clearemptydoublepage
\begin{bibunit}[IEEEtran.bst]

\chapter*{Résumé en français}
Cette thèse explore comment les avancées en apprentissage profond peuvent aider à l'analyse des mesures satellitaires de la hauteur de surface de la mer (SSH). Les altimètres actuels fournissent des données échantillonnées de manière irrégulière avec une faible couverture spatiale, limitant ainsi l'observation de processus prenant places aux petites échelles. Repousser cette limite améliorerait notre connaissance des dynamique de la surface de l'océan ainsi que nos capacités de surveillance du climat. De nouvelles opportunités pour renforcer nos capacités d'observation ont émergé avec le déploiement du capteur KaRIn lors de la mission SWOT.

Ces dernières années, les approches d'apprentissage ont démontré des capacités remarquables en vision par ordinateur et en traitement du langage naturel. Contrairement à de telles tâches, les problèmes d'observations océaniques peuvent impliquer de forts a priori physiques ainsi que peu de données annotées. Ce travail aborde les considérations spécifiques de l'application de l'apprentissage profond aux données altimétriques en trois parties.

Premièrement, à travers l'étalonnage du capteur KaRIn, nous démontrons comment des connaissances spécifiques du domaine peuvent être intégrées dans les cadres d'apprentissage profond. Nous montrons spécifiquement comment le budget d'erreur spectral de la mission SWOT peut informer la conception d'architectures neuronales.

Deuxièmement, nous abordons la rareté des données de vérité terrain lors de la conception de méthodes neuronales d'interpolation de données altimétriques. Nous illustrons comment les simulations de modèles océaniques et de systèmes d'observation peuvent surmonter ce défi en fournissant des environnements d'entraînement supervisés qui se généralisent aux données réelles.

Enfin, notre troisième contribution traite des défis rencontrés pour combler le fossé entre les communautés "océan" et "apprentissage profond". Une collaboration efficace nécessite que des experts de l'océan définissent les défis d'intérêt ainsi que les critères adéquats pour évaluer les solutions. Le praticien en apprentissage automatique nécessite de son côté les outils nécessaires pour accéder et manipuler les différentes données pertinentes.  Nous décrivons comment nous avons abordé ces aspects lors du développement du projet OceanBench.

% \section*{Motivations}

% \section*{Objectifs}

% \section*{Contributions}

% \section*{Contenu du manuscrit}

% In this manuscript I'd like to cite \cite{remo1,remo2}.

% \putbib[./Resume-Francais/Res-Biblio.bib]
\end{bibunit}

%This command will generate the front cover
\clearemptydoublepage
\frontmatter 
\renewcommand{\contentsname}{Table of Contents}
\setcounter{tocdepth}{5}
\tableofcontents %sommaire %table of content

\clearemptydoublepage
\chapter*{List of acronyms}
\addcontentsline{toc}{chapter}{List of acronyms}
\chaptermark{List of acronyms}


\begin{tabular}{ll}
4DVAR & Four Dimensional Variational Data Assimilation\\
3DVAR & Three Dimensional Variational Data Assimilation\\
BFN & Back-and-Forth Nudging\\
CV  & Computer Vision\\
DL & Deep Learning \\
DUACS & Data Unification and Altimeter Combination System\\
DYMOST & Dynamic Interpolation Ocean Science Topography\\
KaRIn & Ka-band Radar Interferometer\\
KF & Kalman Filter \\
L4 & Level 4\\
MDT & Mean Dynamic Topography\\
MIOST & Multiscale Interpolation Ocean Science Topography\\
ML & Machine Learning\\
NEMO & Nucleus for European Modelling of the Ocean\\
NLP &  Natural Language Processing\\
nRMSE & normalized Root Mean Squared Error\\
OSE  & Observing System Experiment\\
OSSE  & Observing System Simulation Experiment\\
PSD & Power Spectrum Density\\
QG & Quasi-Geostrophic\\
RMSE & Root Mean Squared Error\\
SLA & Sea Level Anomaly\\
SSH & Sea Surface Height\\
SST & Sea Surface Temperature\\
SWOT &  Surface Water Ocean Topography \\


% AAM & Active Appearance Models \\
% ANN & Artificial Neural Networks \\
% APSS & Association of the Psychophysiological Study of Sleep \\ 
% AR & Auto Regressive \\
% AS & Active Sleep \\
% AU & Action Unit \\
% AUC & Area Under Curve \\
% B\&W & Black and White \\
% CCHS & Congenital Central Hypoventilation Syndrome \\
% CNN & Convolution Neural Networks \\
% CNS & Central Nervous System \\
% CP & Cerebral Palsy \\
% CU & Cry Unit \\
% CWT & Continuous Wavelet Transform \\
% D & Drowsiness \\
% DAN & Douleur Aig{\"u}e du Nouveau-n{\'e} \\
% DCT & Discrete Cosinus Transform \\
%
\end{tabular}
%
% \begin{tabular}{ll}
% KLT & Kanade-Lucas-Tomasi \\
% KNN & K-Nearest Neighbors \\
% LDA & Linear Discriminant Analysis \\
% LOOCV & Leave-one-out cross-validation \\
% LPCCs & Linear Prediction Cepstral Coefficients \\
% LR & Logistic Regression \\
% LTAS & Long Time Average Spectrum \\
% MFCCs & Mel Frequency Cepstral Coefficients \\
% MLE & Maximum Likelihood Estimation \\
% MLP & Multi-Layer Perceptron \\
% NBAS & Neonatal Behavioral Assessment Scale \\
% NFCS & Neonatal Facial Coding System \\
% NICU & Neonatal Intensive Care Units\\
% NIDCAP & Newborn Individualized Developmental Care and Assessment Program  \\
% NIR & Near-InfraRed \\
% NLEO & Non Linear Energy Operator \\
% NUC & Next Unit of Computing \\
% OSA & Obstructive Sleep Apnea \\
% PCA & Principal Component Analysis \\
% PMA & PostMenstrual Age \\
%
%
% \end{tabular}
%



\listoffigures
\addcontentsline{toc}{chapter}{List of figures}

\listoftables
\addcontentsline{toc}{chapter}{List of tables} 

\clearemptydoublepage
\begin{bibunit}[IEEEtran.bst]

\chapter*{Introduction}
\addcontentsline{toc}{chapter}{Introduction}
\chaptermark{Introduction}

  \section{Motivation}
Understanding and anticipating the ramifications of climate change represents a pressing challenge of our era.
Enhancing our knowledge of Earth's systems is a key factor in confronting this challenge and observation data are key resources to this end.
  % Given that the primary source of factual information about the Earth system is observational data,
  % improving our ability to exploit these data could lead to better monitoring and understanding of our planet.
  This thesis is situated within the context of the Surface Water Ocean topography (SWOT)\cite{KaRInSWOTCharacteristics} mission, which presents opportunities for enhancing our observational capabilities of the oceans.
  We focus on the development of methods to exploit satellite observations of the ocean surface height for improving our knowledge of ocean surface dynamics. 
  More specifically we're asking how advances in deep learning research can be beneficial to ocean altimetry analysis.
  Deep learning research provides a rapidly evolving set of tools that have been successfully applied to a wide range of domain, surpassing existing methods and succeeding in previously unsolved problems.
  
  In order to study the potentials of deep learning for tackling ocean observation problems, we'll introduce the necessary methodological components involved when addressing an observation problem by walking through the toy example of a thermometer graduation procedure.
  We will explicit the similarities between this example and the altimetry challenges studied in this thesis.
 This simplified problem will help illustrate and contextualize the complementary roles of data and domain knowledge when addressing this class of problem. 
  We will then describe how the tools brought by the deep learning field fit in this methodological framework and consider the opportunities and challenges that arise when applying them to ocean altimetry analysis.

  Finally we'll outline the structure of this manuscript. We'll present the altimetry use-cases and the deep learning methodological aspect considered in the second and third chapter. Finally, we'll introduce the scientific contribution of the OceanBench project which aims at facilitating collaboration between the ocean altimetry and deep learning communities.
  


  \section{Toy example: Graduating a thermometer}


\subsection{Estimating temperature from observations}
\begin{figure}[h]
    \centering
        \includegraphics[clip, width=3cm]{Introduction/pics/therm_pb.png}
    \caption{Thermometer Graduation problem illustration. Given a simple liquid based thermometer, we aim at finding the matching between height of the liquid within the glass tube and temperature.}
    \label{fig:therm_calib}
\end{figure}

Let us consider a standard liquid based thermometer that consists of a liquid in a glass tube.
When interested in knowing the temperature, we observe the level of a thermometer.
In order to do so, someone had to graduate the thermometer. 
This seemingly simple action can be detailed in a two-step process, which involves the construction of a theoretical model and its calibration using real-world data.

The first step involves compiling theories and assumptions to construct a model linking the observed level and the actual temperature.
In this instance, based on our knowledge of fluid dilation in response to temperature, assuming the diameter of the tube is constant with height, we can propose the model that the level is linearly correlated with the temperature.
This model introduces two parameters: the slope and offset of our linear model that need to be ascertained.

The second step involves determining these parameters. This step requires some calibration data as inputs. They are traditionally obtained by immersing the thermometer in icing and boiling water to acquire the levels corresponding to 0°C and 100°C.
  Using those data points, a linear system can then be used to solve for the parameters. Which finally gives use our level-to-temperature relationship.


  The model we chose can have more or less independant parameters that need to be calibrated depending on the assumptions that were made.
  Interestingly, this introduce a relationship between the assumptions made and the amount of data required for calibration. For instance, a model with fewer assumptions demands more data. If we were to abandon the assumption of the thermometer tube's constant diameter, we would need to incorporate a parameterization of the tube diameter in our model. This addition creates more parameters and consequently demands additional data for calibration. Conversely, having access to more data can allow us to work with fewer assumptions. Suppose we possess a well-calibrated thermometer that can provide unlimited data points. In that case, we could reduce our assumptions to a minimum and rely heavily on empirical evidence, marking each thermometer graduation using data directly from our well-calibrated thermometer.

With these carefully calibrated graduations now etched onto our thermometer, we can use the liquid level as a convenient stand-in for the temperature. However, an important question remains: How can we assess the accuracy of our newly graduated thermometer?

\begin{figure}[h]
\centering
\begin{tabular}{ccc}

    \includegraphics[clip, width=3cm, width=3cm]{Introduction/pics/therm_theroy.png} &   \includegraphics[clip, width=5cm]{Introduction/pics/therm_model.png}  &   \\   Assumptions & Model&\\ 
     \includegraphics[clip, width=4cm, height=4cm, trim={2cm 1cm 2cm 2cm}]{Introduction/pics/therm_obs.png} &    \includegraphics[clip, width=5cm]{Introduction/pics/therm_calib.png}   &  \\    Data & Calibration&\\
\end{tabular}

    \caption{Mapping thermometer level to temperature. The first step consists in compiling theoretical knowledge to determine a model of the level to temperature relationship. This model define the set of candidate graduations. The second step consists in leveraging data to chose the best candidate graduation through some calibration procedure.}
    \label{fig:therm_mapping}
\end{figure}

\subsection{Evaluation}

Without evaluation the use of our calibrated instrument would solely rely in the faith given to our mapping above.
However one may prefer quantifying the thermometers quality through metrics.
In our case the most intuitive metric for characterizing our thermometer's quality is be the precision of the graduations.
Each tick of our thermometer has a precision value, different aggregations of the individual values also constitute different metrics (bias versus variance for example)
In order to properly evaluate our calibrated instrument, we need to test it in conditions corresponding to its intended use. (testing it domestic thermometer 5 kilometers underwater would not give a relevant evaluation).

 To do so, let's explicit some assumptions made on what we expect from our thermometer.
  For example that it needs to "be accurate to the half of degree", "have response time under 10 minutes", "work between -30°C and 200°C" "work at a reasonable atmospheric pressure" etc...

Then we need data to measure the precision of our thermometer in a way that is representative of how we want our thermometer to behave. Using a trustworthy reference like a third-party well-graduated thermometer, we could compare the measurements of the reference with the one given by our solution.
  An example evaluation procedure could be to confront the measurements of the two instruments at different temperatures such as: in a freezer, in a fridge, at ambient room temperature and in an oven.

Using the procedure above, we can compute our precision metrics and assess if the quality of our thermometer is acceptable.
This exercise, raises some critical points about evaluation. The process relies on two components that require a deep understanding of the thermometer's intended use: a suitable choice of metrics and representative data.
If the metrics do not align with the intended use of the thermometer, the evaluation will be flawed. Similarly, if the data are not representative of the thermometer's intended use, the evaluation will also be flawed.
 Furthermore, the reliability of the reference thermometer is pivotal. If the reference thermometer is not well-graduated, the best of metrics will not be able to correctly evaluate our thermometer. 

It's also crucial to differentiate between calibration data, which is used to determine the graduation, and evaluation data, which is used to assess the graduation's quality.
A well-functioning thermometer should provide accurate temperature readings even for levels it wasn't calibrated on. Thus, evaluation data should differ from calibration data.
If we only measure precision at 0°C and 100°C, a thermometer that perfectly fits the calibration data would receive the highest metric, whatever the other graduation indicates.

\begin{figure}
\includegraphics[clip, width=6cm]{Introduction/pics/errors.png}  
    \centering
    \caption{Evaluation and errors. Given some evaluation data and choice of metric, we can compute the errors associated with our graduation. $T^{\circ}_{calib}$ and $T^{\circ}_{ref}$ are respectively the temperatures given by our graduation and a reference well graduated thermometer}
    \label{fig:errors}
\end{figure}



 \subsection{Sources of errors}


\begin{figure}
\begin{tabular}{c|c|c}
     \hspace{-.15\linewidth}\includegraphics[width=.4\linewidth]{Introduction/pics/model_err_w_source.png}  &
     \includegraphics[width=.4\linewidth]{Introduction/pics/data_err_w_source.png} &
     \includegraphics[width=.4\linewidth]{Introduction/pics/optim_err_w_source.png} \\
     \hspace{-.15\linewidth}Model &  Data &  Algorithm \\
\end{tabular}
    \centering
    \caption{Illustration of the different sources of errors for the thermometer graduation. From left to right: model errors result from erroneous assumptions about the system. Data errors result from inaccuracies in the calibration data and algorithmic errors result from a failure of the calibration procedure to select the best candidate from the model.}
    \label{fig:err_sources}
\end{figure}
Given an evaluation procedure, the errors are the differences to the reference and can be attributed to three sources: the model, the data and the calibration algorithm.
 The model is a source of error if the assumptions made were inaccurate. For example if the diameter of the tube is not constant with height the linear relationship between level and temperature is not verified and will induce errors when interpreting the level.

 Even with perfect assumptions, noisy data can introduce errors in the calibration. If we interpreted our 0°C and 100°C in icing and boiling water at the top of a mountain with lower atmospheric pressure, we will have calibrated our parameters with erroneous measurements and the subsequent graduation of our thermometer will be inaccurate.

 Finally even with perfect assumptions and perfect data, the calibration algorithm used to find the solution's parameters can be a source of errors if it fails to find the optimal parameters. For example if we solve for the parameters with a gradient descent method, using a step size too big will prevent finding the exact parameters which will also results errors in the subsequent measurements.

In order to develop a graduation procedure, we need to take those sources of error into account. The graduation procedure choice will not only depend on the level-temperature relationship but on the whole relationship between calibration data to the final calibrated thermometer. We therefore need to incorporate in our reasoning how the calibration data was acquired, what is the best model to map the level to the temperature, and what is the best algorithm to find the optimal parameters of the model.


This example allows us to formulate a generic methodological framework.

\subsection{Methodological framework}
In the process of finding a level-temperature relationship, we chose a model, a calibration algorithm, and had access to calibration data. Additionally, to evaluate our solution, we defined a metric and had access to evaluation data. Interestingly, these components can be specified at a higher level for finding and evaluating the graduation procedure itself, essentially creating a meta-level or "second order" problem.

The \textbf{Model}, in this second order scenario, combines different assumptions to determine the set of potential graduation procedures. 

The \textbf{Calibration Algorithm} is used to select the best graduation procedure. This could be as straightforward as testing different combinations and choosing the most effective one, or it could involve complex numerical optimization procedures to determine higher-level parameters.

The \textbf{Calibration Data}, at the second order, consists of graduation tasks with a method to assess the performance of candidate procedures. This allows the algorithm to select the best solution.

The \textbf{Evaluation Metric} should reflect the intended use of the graduation procedure, including the range of thermometers we plan to use this procedure for. A useful metric might be the precision of all the thermometers we aim to graduate using the proposed solution.

The \textbf{Evaluation Data} should be representative of the variety of intended uses. This means it should contain graduation tasks for a range of thermometers of interest. Additionally, we need a reference for these tasks to measure the precision of our solution.

By leveraging these five components, we can select the best calibration procedure, quantify its quality using the evaluation data, and use it to calibrate new thermometers with confidence in the resulting calibrated instrument. This parallels the problems of "Finding the level-temperature mapping" (which we refer to as the first order problem) and "Finding the graduation procedure" (the second order problem) and offers insights on where general purpose methodological tools can find applications.

Note that second order metrics can extend beyond the scope of the first order problem. These metrics could encompass aspects such as robustness to noise or the computational complexity of the calibration procedure. This means our evaluation of a calibration procedure not only includes how well it measures temperature, but also how well it handles uncertainties or computational burdens.

A second order solution takes first order calibration data as inputs, which contain observations of a specific thermometer and their corresponding temperatures, and outputs a first order solution: a tailored graduation for the thermometer represented in the data.

The second order problem also involves making decisions on parameters to select the best solution, which can take various forms. For instance, these parameters can be discrete choices between different assumptions, like whether to consider the thermometer's tube diameter as constant or not. The parameters can also denote choices between different first order algorithms like choosing a direct linear system inversion or a iterative optimization procedure. Lastly, these second order parameters can be constants in the level-temperature mappings or parameters of an optimization procedure, like step size. This shows that the parameters in the second order problem have a broad range of applicability, affecting both the details of the calibration procedure and how the procedure is chosen.

Finally, a critical note is that the data used to evaluate a solution at the second order level should still be separate from the calibration data. This principle holds true for the same reasons it applies to the first order problem - using distinct data sets helps to ensure that our solutions generalize well beyond the specific scenarios they were trained on.

\section{From toy example to ocean altimetry scenarii}
\subsection{Introducing space and time}
Our previous example implicitly solved the estimation problem of the liquid temperature within the thermometer at a single point in space and time.
However, we can extend the problem formulation to estimate a quantity over a spatial and temporal domain.
Given some observations $y$ defined on a spatio-temporal domain $\Omega_y$ we want to estimate a quantity of interest $u$ on a domain $\Omega_u$. We are therefore looking for a mapping $f$ that estimate $u$ from $y$
The process of determining $f$ can be detailed in two steps, first determining the set $\cal{F}$ such that $f \in \cal{F}$ by making some assumptions on $f$. Then determining the calibration procedure $c$ that will use the calibration data $\cal{D}$ to select $f$ from $\cal{F}$
The evaluation of the solution rests on the choice of metrics $m$ and evaluation data $\cal{E}$

Solving this general problem requires considering the sampling pattern of the observations with respect to the target estimation domain.
Incomplete coverage will require the \textbf{model} to account for temporal and spatial relationship between $y$ values and $u$ as well as the spatio-temporal structure $u$.
These additional assumptions will require suitable \textbf{Calibration data and algorithm} and \textbf{Evaluation data and metrics} to be calibrated and evaluated.



\subsection{Satellite altimetry}


In recent decades, satellite NADIR altimeters have greatly improved our observational capabilities enabled by providing a global coverage of the SSH.

However due to the scarce and irregular sampling of altimeter constellations, current operational products do not resolve processes at mid-latitudes with horizontal scales smaller than 150 km\cite{ballarottaResolutionsOceanAltimetry2019}.
These processes linked to the mesoscale and submesoscale dynamics of the ocean surface play an important role in the heat redistribution within the ocean, which has implications for climate monitoring.
The recent deployment of the novel Ka-band Radar Interferometer (KaRIn) sensor during the SWOT satellite mission\cite{KaRInSWOTCharacteristics} provides opportunities to address this gap.
This new sensor will provide unprecedented two dimensional images of the ocean surface topography but will also introduce calibration challenges\cite{EmpiricalCrossCalibrationCoherent} due to previously unseen errors.

The estimation of the sea surface height (SSH) given satellite altimetry data enters nicely in the methodological presented above.

  \begin{figure}
      \centering
            \includegraphics[width=\linewidth]{Introduction/pics/calib_task.png}    
      \caption{SWOT calibration. The left part illustrate the observed domain in red while the right part indicates the domain on which we aim at estimating the SSH.}
      \label{fig:calibration_task}
  \end{figure}
The calibration problem considered in this thesis consists in estimating the SSH measured by the KaRIN instrument by removing  correlated error signals, using calibrated Nadir observations. The target estimation domain here is fully observed, however, some observations contains errors originating from the instrument acquisition process. The \textbf{model} for this problem includes assumptions about the processes that produce the errors. Note that some assumptions about the spatio-temporal structure of the SSH are also required here to relate the SSH on the KaRIN instrument to surrounding  NADIR altimer measurements. The mapping problem below isolate this challenge more specifically.

\begin{figure}
    \centering
          \includegraphics[width=\linewidth]{Introduction/pics/mapping_task.png}
    \caption{Nadir Altimetry mapping. The left part illustrate the observed domain in red while the right part indicates the domain on which we aim at estimating the SSH.}
    \label{fig:mapping_task}
\end{figure}
The altimetry mapping problem focuses on the spatial and temporal interpolation of NADIR altimetry data. Considering the observation as direct measurements of SSH, we aim at estimating daily maps over a delimited domain. The \textbf{model} for such a problem needs to take into account the dynamical structure of the ocean surface topography.
This manuscript, therefore, aims to explore the application of deep learning to these two observation problems. The first is estimating SSH from noisy SWOT observations, and the second is inferring the complete SSH field from partial measurements which we detail in the following two sections.
We give an overview of the next chapter the different \textbf{models} and \textbf{calibration procedure} that have been developped for tackling these challenges.
  


In order to solve and evaluate solutions to these problems \textbf{Calibration and Evaluation data} are necessary. However, the SSH we aim to estimate is unknown on the target domain.
Two separate experimental setups are used to adress this issue.

Observing System Experiments (OSE) \cite{hamonImpactMultipleAltimeter2019} constitute a framework for working directly with observations for calibrating and evaluating new methods.
For altimetry mapping for example, some satellite observations may be reserved for the interpolation process while others are employed to calibrate and evaluate the resulting map.

Observing System Simulation Experiments (OSSE)\cite{verrierAssessingImpactMultiple2017} use ocean models as well as simulated observing systems to create an controlled environment where simulated ocean quantities are known. For SWOT calibration, this includes simulating processes of the satellite movement such as roll oscillation that are the source of error signals\cite{EmpiricalCrossCalibrationCoherent}.

This peculiar data context is a critical factor when developing data centric methods like deep learning.

% 

\section{Deep learning: opportunities and challenges}

\subsection{Success stories in computer vision (CV) and natural language processing (NLP)}

In regard to the framework described above, deep learning brings forth \textbf{models}, such as neural networks, that are predicated on very weak assumptions. Their strength lies in the fact that, given sufficient parameters, they can approximate any function\cite{hornikMultilayerFeedforwardNetworks1989}. This leads to deep learning models defining vast parameter-space, consequently requiring substantial datasets and sophisticated optimization procedures to identify a good solution. These optimization procedures are the \textbf{calibration algorithms}.

Deep learning \textbf{models} and \textbf{calibration algorithms} have advanced in tandem over the last decades.
Innovations in model architectures such as ResNet\cite{heDeepResidualLearning2016}, batch normalization\cite{ioffeBatchNormalizationAccelerating2015}, and in optimization procedures like Stochastic Gradient Descent (SGD)\cite{summaLargeScaleMachineLearning2011}, Adam\cite{kingmaAdamMethodStochastic2017}, and various learning rate schedules have consistently improved the calibration large models, therefore enabling the use of even larger neural networks. 

However, the fact that deep learning models can in theory approximate any function introduces a peculiar consideration which is that fitting exactly the calibration data gives you no guarantee on how the model will behave on unseen data.
Addressing this problem have motivated many innovations in regularization, architectures, initialization schemes and data augmentation techniques.
It has also standardized the practice of splitting the \textbf{calibration data} in two sets: \textbf{training} and \textbf{validation}.
The training set is used by the optimization procedure to search for the parameters whereas the validation set is used to assess the generalization on "unseen" data.

When looking at different application domain, the track records of deep learning in CV\cite{chaiDeepLearningComputer2021} and NLP\cite{brownLanguageModelsAre2020} are especially impressive. 
The tools brought by deep learning have managed to solve tasks that were previously unsolved.
Indeed the universality of deep learning models have brought a big leap forward in tasks such as natural image recognition or natural language generation which are very hard to model using theoretical knowledge.

Presented as such deep learning seems to provide universal tools,  
however we would like to stress two factors that seem of great importance when looking at the contributions of deep learning in specific fields.
The first factor is quality and availability of data.
Indeed the creation of large, curated datasets, like ImageNet\cite{dengImageNetLargescaleHierarchical2009} in computer vision or ThePile\cite{gaoPile800GBDataset2020} in natural language processing have shown to dramatically expedite the development of novel approaches. 
The second factor is the design of informed architectural patterns that are particularly suited to the domain, leading to performance breakthroughs.
Examples of these include convolution techniques\cite{lecunHandwrittenDigitRecognition1989} and U-Net architectures\cite{ronnebergerUNetConvolutionalNetworks2015} in computer vision, and attention mechanisms\cite{vaswaniAttentionAllYou} in natural language processing.


\subsection{Deep learning for ocean altimetry}
For the mapping and calibration challenges at hand in this thesis, the use of deep learning advances presents both opportunities and challenges.
In deep learning, existing architectures used for video inpainting\cite{kimDeepVideoInpainting} and image denoising\cite{tianDeepLearningImage2020} solve tasks that are formally similar to altimetry mapping and calibration.
The universality of such models can potentially capture processes relating to ocean dynamics and observations that are hard formulate formally and improve on existing state of the art.

The altimetry data however presents some specific characteristics compared to natural images and video.
The sampling pattern of altimeters is particularly sparse and irregular which may not be adapted for classical CV architectures.
Furthermore ocean surface dynamics involve processes at different scales and the less well resolved that operate at finer scales (mesoscale, submesoscale) and that are of interests in this thesis constitute a small proportion of the whole SSH signal.
Classical methods in altimetry adress this challenge by leveraging theoretical and statistical knowledge of ocean processes and observation data.
The first question we ask is: \textit{How specific altimetry knowledge can be integrated in deep learning methodology?}

Another peculiar aspect of the altimetry context for using deep learning concerns the available calibration and evaluation data.
Indeed at the beginning of my thesis, the 4dVarNet neural mapping scheme has shown to surpass operational products for a usecase in the gulfstream area\cite{fabletENDTOENDPHYSICSINFORMEDREPRESENTATION2021}. However, this results had been calibrated and evaluated in an OSSE setup. 
The second scientific question tackled in this manuscript is: \textit{How can deep learning approaches overcome the lack of true SSH and be calibrated for use on real altimetry?}


These studies demonstrate how applying deep learning to ocean observation sciences such as altimetry is fully dependent on domain expertise for both the data and evaluation contexts.
The development and evaluation of methods like the 4dVarNet relied heavily on prior efforts to standardized OSSE and OSE usecases in the form of data challenges\cite{ballarottaOceandatachallenges2020a_SSH_mapping_NATL60Material2020,ballarottaOceandatachallenges2021a_SSH_mapping_OSEMaterial2021}. These usecases provided data as well as metrics relevant for ocean physicists. 
This raises the third question of this research: \textit{What is needed for a wide-spread adoption of deep learning tools  in ocean observation sciences?}




\section{Thesis objectives and outline}

The following chapters of this thesis are organized as follows:

\textbf{Chapter 2} describes the main assumptions and methods that are formulated in current methods for altimetry mapping and calibration. This chapter aims at describing the existing \textbf{models} and \textbf{calibration algorithm} available in altimetry analysis as well as the related work in deep learning. A more detailed decription of the neural network-based 4DVarNet framework and its applications will be presented as the contributions of this thesis make extensive use of this architecture.

\textbf{Chapter 2} propose a deep learning architecture for the calibration of correlated errors in SWOT data.
From an applicative standpoint, the flexibility of deep learning methodology opens the potential for capturing signals that are tricky to explicitly parameterize.
From a methodological perspective, this study shows how deep learning architectures can be tailored with assumptions on the instrument and its measurements.
More specifically, this is done by showing how the spectral specifities of the errors can be leveraged to design a efficient neural calibration scheme.
This study bypasses the challenges brought by the lack of ground-truthed dataset.
It takes place in a simulated setup with the SSH and error signals are simulated and therefore known.


\textbf{Chapter 3} tackles more specifically the data availability problem. It studies how neural mapping schemes can be applied to real data despite the lack of reference dataset.
It looks more specifically at the 4DVarNet framework which has been demonstrated in a simulated setup\cite{fabletENDTOENDPHYSICSINFORMEDREPRESENTATION2021} using simulated SSH for training and evaluation.
This chapter look at the performance on real data of deep learning models trained on simulated data.
It shows how the extensive physical knowledge of the ocean dynamics can be leveraged to palliate the lack of ground-truthed dataset in altimetry through the use of numerical simulations for training.

\textbf{Chapter 4} considers the obstacles for better synergies between the ocean altimetry and deep learning community.
Both fields are well established with accumulated knowledge, conventions, and best practices. 
As described in this chapter, deep learning brings powerful models and algorithms.
However the calibration and evaluation data as well as the metrics of an altimetry product can only be sensibly designed by an domain expert. 
We propose OceanBench, an interface in the form of a software suite of tools.
Oceanbench aims at empowering domain experts to easily design altimetry problems of interests and qualifying them with relevant metrics. 
It then provides machine learning practicionners access to the necessary data as well as suited utilities for learning

\textbf{Chapter 5} discusses and concludes on the research presented in this manuscript. We summarize the main objectives and results in previous chapters as well as proposing some future avenues of research.


\addcontentsline{toc}{section}{Bibliography}
\putbib[./Introduction/Intro-Biblio.bib]
\end{bibunit}


%\shorttableofcontents{Sommaire}{0}

\clearemptydoublepage
\mainmatter % Ne pas oublier, avec \frontmatter et \backmatter
	           %this command inputs \frontmatter and \backmatter as a cover in the front and the back

%Input your chapter 1

% \clearemptydoublepage
% \begin{bibunit}[IEEEtran.bst]

\clearemptydoublepage
\chapter{Problem Formulation and Ontology of Approaches}
\label{chap:1}


\section{The Thermometer Calibration Example}

Journey from an observation problem to a learning problem.
In order to contextualize the work in this thesis, I aim to introduce and explicit the different sources of decision and assumption making when solving an observation problem. This will naturally introduce the concepts used when working with learning-based approaches. Furthermore it will provide a useful perspective for interpreting the research by pointing where learning approaches make different assumptions compared to more traditional approaches.
In order to elucidate the proposed ontology, let's delve into the process of calibrating a thermometer. This example was selected due to its relevance to our study and because it is easier to reason about the underlying physical processes and quantities than in the case of satellite altimetry. As a starting point we can state the problem of calibrating a thermometer as: given an ungraduated thermometer, how can we interpret the level of the liquid as a temperature?

\subsection{Calibrating a thermometer}

The overarching goal of this calibration is to be able to know the temperature at some place by putting the thermometer in a specific location, observe the height of the liquid, and infer the temperature at the point of measurement.

The first step involves accumulating theories and assumptions to construct a model linking the observed level and the actual temperature. For instance, based on our knowledge of fluid dilation in response to temperature, assuming the diameter of the tube is constant with height, we can posit that the level is linearly correlated with the temperature. This model introduces two parameters: the slope and offset of our linear model that need to be ascertained.

The second step involves determining these parameters. For this purpose, we traditionally immerse the thermometer in icing and boiling water to acquire the levels corresponding to 0°C and 100°C. A simple linear system can then be used to solve for the parameters.
This second step rely on data consisting of input-output pairs of the function we're looking for. 

A few notes on this example:
\begin{itemize}
\item The calibration procedure consisted in two steps that respectively rely on conceptual knowledge and data;
\item The two steps are coupled
\item If we loosen the assumption about the tube's constant diameter, we need to incorporate a parameterization of the tube diameter into the model, adding more parameters and necessitating additional data for calibration;
\item If we have a well calibrated thermometer that provides us as many data as we want, we could make very little assumptions and just mark each graduation using data from the calibrated thermometer.
\item By adding the knowledge that the thermometer is in boiling water, our mapping is reduced to a constant function returning 100°C by convention.
  \end{itemize}
% use a report the data points in the graduationswe could limit our assumptions to stationarity (a level that corresponded to a temperature in the past implies the same level corresponds to the same temperature) and smoothness (levels that are close to each other correspond to temperatures that are close to each other), and our temperature estimate could simply be the nearest observation from the data points (which would be a model without parameters);

We now have graduations on our thermometer and can use the level as a proxy for the temperature without further thought!... Although how do we know if our calibrated thermometer is good ? 


\subsection{Evaluation}

Let's define "evaluation" as quantifying quality through metrics.

In our case the most intuitive metric for characterizing our thermometer's quality would be the accuracy of the temperature it gives. However this is by no means obvious, some situation may put greater importance on the speed of the thermometer or the range at which it's functional. 
  Furthermore, in order to properly evaluate our calibrated instrument, we need to test it in conditions corresponding to its intended use, (indeed for a domestic thermometer, testing it it on Mars or 5000 meter underwater would not provide helpful information).

 In order to clarify its intended use, we need to explicit some silent assumptions made on what we would consider a good thermometer.
  For example that it needs to "be accurate to the half of degree", "have response time under 10 minutes", "work between -30°C and 200°C" "work at a reasonnable athmospheric pressure" etc...

Then, using a trustworthy reference like a third-party well-calibrated thermometer, we could compare the measurements of the reference with the one given by our solution.
  An example evaluation procedure could be to confront the measurements of the two instruments at different temperatures such as: in a freezer, in a fridge, at ambiant room temperature and in an oven.

Some remarks about the evaluation:
\begin{itemize}
\item Data is needed for computing metrics
\item The evaluation depend on the metric chosen and the way we compute it.
\item Defining relevant metrics requires intimate knowledge of the intended goal of the instrument.
\item Different metric can produce different rankings, therefore the evaluation is relative to the metric's choice
\item Both the single accuracy in the oven and the mean or standard deviations of the different measured accuracies can be considered as metrics
\item An evaluation can use multiple metrics, therefore no ranking between two methods is guaranteed
\item If the metrics' choice and computation are not suited to the intended use of the instrument, the evaluation will be flawed.
\item If the reference thermometer is biased (not well calibrated) a good metric will not define a thermometer of quality
\item The criterium on the range of the thermometer implies that the thermometer needs to give a temperature even for levels it were not calibrated on (since only finite number of observations were used to calibrate it).
\item If the evaluation only measured the accuracy at the observed temperature 0°C and 100°C, any procedures that fit the observations would get the highest metric even if all other graduations were non-sense.
\end{itemize}

 \subsection{Sources of errors}

Given an evaluation procedure, the errors are the gap to the reference and can be attributed to three sources.
 The model is a source of error if the assumptions made were inacurate. For example if the diameter of the tube is not constant with height the linear correlation between level and temperature is not fallacious and will induce errors when interpreting the level.
 Even with perfect assumptions, the noisy data can introduce errors in the calibration. If we interpreted our 0°C and 100°C in icing and boiling water at the top of a montain with lower athmospheric pressure, we will have calibrated our parameters with erroneous measurements and the subsequent graduation of our thermometer will be inaccurate.
 Finally even with perfect assumption and perfect data, the procedure used to find the solution's parameters can be a source of errors if it fails to find the optimal parameters. For example if we solve for the parameters using a gradient descent method, using a step size too big will prevent finding the exact parameters which will also generate errors in the subsequent measurements.


Once we have an evaluation of our instrument, the next natural step is to aim at reducing the errors. Doing this we're not just thinking about the level-temperature relationship but about the whole method used to determine this relationship. We need to incroporate in our reasoning the how the calibration method (assumptions, data, optimization) relates to the errors we measure. We can state this as a higher level problem: "How to find the best thermometer calibration procedure?"

\subsection{Finding the best thermometer calibration procedure}
In this section, we propose a parallel between the two problems of "knowing the temperature when looking at the thermometer" and  "knowing the calibration procedure". we respectively call first order and second order what refers to the former and latter.

  The second order objective is to find a function that takes in a set of level-temperature observations, and returns the level to temperature mapping (the first order solution)

In order to find such a procedure, we can apply the same two steps as before.

  First, how to model the link from the input observations and the output mapping. This would include (first order) assumptions about the relationship between level and temperature as well as assumptions about the type of errors in the observations and the optimization procedure. These different assumptions are compiled into a set of candidate calibration procedures characterized by some configuration (the second order parameters).

  The second step consists then in determining the optimal configuration among the different candidates.
  When considering the first order problem, the data used to determine the parameter were levels with corresponding temperature. These data are  the inputs and outputs of the function we are trying to determine.
  If we reason about the inputs and outputs of our calibration procedure, the inputs consists in a set of level-temperature pairs  and the output are calibrated thermometer.
  Indeed 
  
  In order to do so, we need data points as well as an evaluation procedure.


A few notes on this example:
\begin{itemize}
\item The same two steps applied for the first and second order problem: defining candidates using theory and finding the optimal solution using data
\item A second order solution is a function that takes as inputs first order data and outputs first order solution
\item Second order parameters can be discrete choices like different first order assumptions 'considering the diameter is constant or not'
\item Second order parameters can be discrete choices between two different optimization procedure
\item Second order parameters can be constants in the level-temperature mappings
\item Second order parameters can be parameters of an optimization procedure like step size
\item The first order assumptions are about the physics of the system 
\item The second order assumptions includes considerations about the methods through data (errors) and optimization
\end{itemize}

We now have a great calibration procedure!... Or do we?  Wait what is a "great" calibration procedure?

\subsection{Evaluation of a calibration procedure}

A first question we need to ask to evaluate the procedure in itsefl is: what would be the desired domain (i.e. range) of a calibration procedure ? Or put differentely, on what calibration problems do we intend to use it.
We therefore need to explicit what kind of thermometer we want the calibration procedure to be applied to, and what observations we expect to have.
In our example, I implicitely wanted the calibration procedure to work on any standard thermometer, having at least two level-temperature observations without noise.

Now we need to determine a metric and an evaluation procedure.

Having determined a satisfying metric for evaluating a thermometer, a natural metric for the procedure is to compute the calibration metric on results of the procedure
The evaluation procedure can then be to use a few different thermometers a verify that the caliration works well on them.


A few remarks here:
\begin{itemize}
\item other metrics can relate to second order aspects of the problem, for example the robustness to noise or the computational complexity.
\item Different calibration tasks need not be on multiple thermometers, but could the same thermometer with different observation data points
\item if we tuned our procedure and computed our metric on a single thermometer, the metric would not informed us if the calibration worked on different thermometer.
\end{itemize}


Using this evaluation procedure, we could quantify the second order errors that could be classified in the same categories.
The model errors now includes the repercussions of wrong assumptions and parameterizations of errors in the data and optimization dynamics.
Data errors would for example be due an inaccurate reference thermometer used tune the second order parameters.
And optimization errors denote the gap between the solution found compared to the best possible candidate when evaluated on second step's data.


\subsection{Introducing space}
  As mentioned in the notes ultimate objective (zero order) of the calibration is to know the temperature at a single place and time given by the location of the thermometer. 
  This is a particular case  -- related to the calibration task  -- of the classes of problem we're interested in.
  The more generic class of problems would be knowing the temperature field over a spatio-temporal domain.

We could then update the associated hierarchy of problems as:
  - Zero order: Know the temperature given a location and a time within a spatio-temporal frame, 
  - First order: Find the field of temperature given observations of thermometers (potentially at different places and times)
  - Second order: Determine a procedure that can map a set of observation to the temperature field

And the conceptual blocks introduce above apply in a similar manner.


  As mentioned in the notes ultimate objective (zero order) of the calibration is to know the temperature at a single place and time given by the location of the thermometer. 
  This is a particular case  -- related to the calibration task  -- of the generic classes of problem we're interested in.
  The more associated generic class of problems would be knowing the temperature field over a spatio-temporal domain.

We could then update the associated hierarchy of problems as:
  - Zero order: Know the temperature given a location and a time within a spatio-temporal frame, 
  - First order: Find the field of temperature given observations of thermometer levels (potentially at different places and times)
  - Second order: Determine a procedure that can map a set of observation to the temperature field

And the conceptual blocks introduce above apply in a similar manner.


\subsection{In a nutshell}

A necessary step before developing a solution to a problem is to be able to evaluate it, this requires informed knowledge and assumptions on the intended use of the solution as well as a choice of metric computation and the associated data.

Then to actually develop the solution, three components are needed: a model of the relationship between input and output of the solution as well algorithm to chose the model configuration given some data.


To solve the calibration problem we need to come up with a calibration procedure that will take level-temperature observations to produce the calibrated thermometer.

To evaluate the calibration procedure we evaluate the calibrated thermometers over a range of representative tasks.
To evaluate a calibrated thermometer, we compare the temperatures given by the thermometer over a range of representative situations.

To develop the calibration procedure we need three components:
The data: an ensemble of calibration task that we can evaluate (of calibrated thermometers)
The algorithm that choses the best calibration procedures 
The model that is the set of candidates calibration procedures that differ by one or more component:
  The model which is the set of candidate of level-temperature mapping
  The algorithm that choses the best mapping given some level-temperature observations







Solving an observation problem can be structured in two levels:
- Finding the best estimate of the temperature when looking at a thermometer
- Finding the best way to convert observations 
- Linking the observation to the quantity of interest

\subsection{Domain experts versus deep learning approaches}
  We group what we describe above in the domain experts category of methods, with a tendency 




\section{Generic formulation}

  The zero order objective of the class of problem we're interested in is to find a function $f_0=\hat{u}$  that approximates a quantity of interest $u(t, x)$ on a domain $\Gamma_u$ with values in $\mathbb{R}^d_u$. 
  In geoscience we can generally consider $\Gamma$ to be spatio-temporal, and define the spatial and temporal domains  as {\Omega = {x, (t,x) \in \Gamma} and {\Tau = {t, (t,x) \in \Gamma}

   $u$ is defined 
  conditioned on a set of observations $\cal{D}_0$.

  The first order objective is therefore to formulates $f_1$ that outputs $f_0$ from  a set of observations $\cal{D}_0$
  with $\cal{D}_0 = \{(p_1, y(p_1)), ..., (p_{N_0}, y(p_{N_0})) \}$ with  $p_i \in \Gamma_y$ and $y: \Gamma_y \to \mathbb{R}^d_y$.

  The second order objective is therefore to formulates $f_2$ that outputs $f_1$ from a dataset $\cal{D}_1$ .
  with $\cal{D}_1 = \{(cal{D}_0^1, Y^1),... (cal{D}_0^{N_1}, Y^2) \}$ with  $cal{D}_0^{i}$ different sets of observations and $Y^i$ additional data that can be used to evaluate first order solutions.

  We note $\cal{F}_1$  the set of candidates $f_1$ and $\theta_1 \in \mathbb{R}^p_1$ the parameters of $f_1$
  We note $\cal{F}_2$  the set of candidates $f_2$ and $\theta_2 \in \mathbb{R}^p_2$ the parameters of $f_2$



% We could then state that a second order metric is can be composed of first order metrics.
%
% Some 
%
% The domain 
%
% We can then 
%
% We can use analog steps as for the evaluation of the calibrated thermometer to reason about the assessment of the quality of the calibration procedure.
%
% 1) What assumptions did we make about what is a good calibration procedure ?
% 2) What metric would be a good way a quantifying them ?
% 3) How can we compute this metric in a representative way ?
%
%
% A first idea to assess if our calibration procedure is good, would be to evaluate our thermometer once calibrated with it.
% However we would then be faced with a dilemna, 
% The issue encountered is that if we use the same evaluation procedure we 
% The most straightforward idea for a good k
%
%  Indeed the best possible calibration procedure will certainly produce a weakly calibrated thermometer if the input observations associate random levels with random temperatures.
%
%
%
% Those are the three sources of error from our calibration procedure, however note that in the absence of perfect reference we cannot directly quantify them.
%  Then as is often the case, some proxy is used as another source of error is even trickier and comes from the gap betkno In the absence of such reference we could 
% A reference 
% We would like to clarify that the evaluation procedure needs to be done on data that were not used 
%  Evaluating a second order solution on a data point uis
% Evaluating a second order solution only makes sense if the evalua
% We first want to point out the importance of the scope of evaluation, the second order problem only makes sense
%
% \subsection{Introducing time}
% The response time of the thermometer is the duration before the temperature indicated by the level reflects the temperature of the location it's in. This is related to the fact that the level is actually related to the temperature of the liquid which will take some time to adjusts to the location temperature.
% One could aim at using the thermometer to estimate the instantaneous temperature.
% However, in order to do so multiple recent observations at the same location  would be needed to take into account the dynamics.
% The mapping between recent observations and instantaneous temperature would include a parameterization of the diffusion process.
% The evaluation should also be consciensiously chosen to measure the dynamical aspects.
%
% Note that this could be treated as an end to calibration problem if we consider the levels as the thermometer as inputs, or as a separate problemif we consider the thermometer already graduated. This would impact the different hypothesis made.
%
%
%
%
% %###################
%
%
% \begin{itemize}
%   \item The optimization procedure also relies on some assumptions and parameters
%   \item We can then think of the tuning of hyper-parameters as an a higher order problem
%   \item The solution of this problem takes in a set of data points and produces a mapping between levels and temperatures.
%   \item Such solution could be applied to a range of different thermometer given that the set of available data points meets the requirements
%   \item The accuracy of a calibrated thermometer would depend on the validity of the assumptions made
% \end{itemize}
%
% \section{Problem Formulation Bis}
% \label{sec:chap1_problem_formbis}
%
%   The end objective (zero order) of the class of problem we're interested in is to find a function $f_0=\hat{u}$  that approximates a quantity of interest $u(t, x)$ conditioned on a set of observations $\cal{D}_0$.
%    $u$ is defined on a domain $\Gamma_u$ with values in $\mathbb{R}^d_u$.
%   In geoscience we can generally consider $\Gamma$ to be spatio-temporal, and define the spatial and temporal domains  as {\Omega = {x, (t,x) \in \Gamma} and {\Tau = {t, (t,x) \in \Gamma}
%   A set of observations is in the form $\cal{D}_0 = \{(p_0, y(p_0)), ..., (p_{N_y}, y(p_{N_y})) \}$ with  $p_i \in \Gamma_y$ and $y(p_i) \in \mathbb{R}^d_y$.
%
%
%   Therefore, $f_0 \in \cal{F}_0 = { h \to \alpha h + \beta , (\alpha, \beta) \in |R^2 }$.
%
%   More generally, $f_0 \in \cal{F}_0 = \{ f: y; z \to f(y; z), z \in |R^{N_0} \}$ with f the modelisation of the first order problem parameterizes by $z$.
%
%
%   Solving for $f_0$ consists in two steps: 
%   Defining $\cal{F}_0$  the set of candidate $f_0$ and $\theta_0 \in \mathbb{R}^p_0$ the parameters of $f_0$
%   Defining the procedure $f_1$ that outputs $f_0$ from $\cal{D}_0$
%
%
%   The first order objective is therefore to formulates $f_1$ that outputs $f_0$ from $\cal{D}_0$
%   We note $\cal{F}_1$  the set of candidate $f_1$ and $\theta_1 \in \mathbb{R}^p_1$ the parameters of $f_1$
%
%   We can similarly introduce a second order objective which is to find $f_2$ that outputs $f_1$ from  $\cal{D}_1$ which groups multiple observation sets 
%   We note $\cal{F}_1$  the set of candidate $f_1$ and $\theta_1 \in \mathbb{R}^p_1$ the parameters of $f_1$
%
% All along this section we will illustrate our ontology of problems and methods using an example of thermometer calibration. We chose this use case due to its relevance to our work given that it's about sensor calibration and because the underlying physical processes are much simpler that the ones involved in the earth system wich makes for a comprehensive and useful usecase.
%
% In the case of our thermometer the quantity of interest is the temperature  $T$ which is a scalar field and $\Omega_u$ is the thermometer location  $x_t$ at time $t_t$. 
%   We want to estimate it given a single observation of the level of the thermometer $h_t$ at the same point: $\cal{D}_0 = \{((x_t, t_t), h_t$.
%
%   In the case of our thermometer, $f_1: \{(t_t, x_t, h_t)\} -> \{f_0(t_t, x_t)\}$ can be reduced  to finding the relationship between the level $h_t$ and the temperature $T_t$ of the thermometer.
%
%
%   In order to solve the first order problem, the first step involves compiling our theoretical knowledge and making assumption to come up with a model. For example, given our understanding of fluid dilation in response to temperature, under the assumption that the diameter of the tube is constant with height, can assume that level is linearly correlated with the temperature. Therefore, $f_1 \in \cal{F}_1 = { h \to \alpha h + \beta , (\alpha, \beta)=\theta_1 \in |R^2 }$.
%     A second necessary step is to determine the values of $\alpha$ and $\beta$. To this end we require a set of data points $\cal{D}_1 = \{\cal{D}_0^1=(h^1, T^1), ..., \cal{D}_0^{N_1}=(h^{N_1}, T^{N_1})\}$ to calibrate our model, traditionally obtained by placing the thermometer in icy and boiling water to get the levels corresponding to 0°C and 100°C. We can then solve for the parameters with a simple linear system.
%
%
%
%
%   The second order objective is then to formulates $f_2$ that outputs $f_1$ from  $\cal{D}_1$
%
%
%   We can now formulate the second order problem associated with 
%   The 
%    To solve this 
%   The end objective of the class of problem we're interested in is to find a function $f_0$  that approximates a quantity of interest $u(t, x)$ on a domain $\Gamma_u$ with values in $\mathbb{R}^d_u$.
%   In order to estimate find $f_0$, we have observation data $\cal{D}_0 = \{(p_0, y(p_0)), ..., (p_{N_y}, y(p_{N_y})) \}$ with  $p_i \in \Gamma_y$ and $y(p_i) \in \mathbb{R}^d_y$.
%   In geoscience we can generally consider $\Gamma$ to be spatio-temporal, and define the spatial and temporal domains  as {\Omega = {x, (t,x) \in \Gamma} and {\Tau = {t, (t,x) \in \Gamma}
%
%
% \section{Problem Formulation}
% \label{sec:chap1_problem_form}
%
%
%
% The broadest formulation the problems we're interested in is finding a function $f$ which maps the available inputs $y$ to the desired outputs $u$.
%
%
%
%
%
% All along this section we will illustrate our ontology of problems and methods using an example of thermometer calibration. We chose this use case due to its relevance to our work given that it's about sensor calibration and because the underlying physical processes are much simpler that the ones involved in the earth system.
%
% This formulation encompasses different order of problem that we detail below.
%
% \subsection{First order: Calibrating the thermometer}
%   \label{ssec:firstorder}
% The first order of problem would be to find the mapping $f_0$ between a given level $h$ of the thermometer and the associated temperature $T$ of the liquid within.
%
% A first step would involve compiling our theoretical knowledge and making assumption to come up with a model. For example, given our understanding of fluid dilation in response to temperature, under the assumption that the diameter of the tube is constant with height, if we don't know the volume of liquid and the dilation rate we can model an affine  that the level is linearly correlated with the temperature. Therefore, $f_0 \in \cal{F}_0 = { h \to \alpha h + \beta , (\alpha, \beta) \in |R^2 }$.
%
%   More generally, $f_0 \in \cal{F}_0 = \{ f: y; z \to f(y; z), z \in |R^{N_0} \}$ with f the modelisation of the first order problem parameterizes by $z$.
%
%   A second necessary step is to determine the values of $\alpha$ and $\beta$. To this end we require a set of data points $\cal{D}_0 = \{(h^1, T^1), ..., (h^{d_0}, T^{d_0})\}$ to calibrate our model, traditionally obtained by placing the thermometer in icy and boiling water to get the levels corresponding to 0°C and 100°C. We can then solve for the parameters with a simple linear system.
%
%
%   The first and second step are coupled. Indeed, the more parameters $f_0$ depends on, the more data points will be needed in $\cal{D}_0$
%   For example if some assumptions are considered too restrictive like "the diameter of the thermometer is constant with height", they can be relaxed.
%   However this expands the class of functions $F_0$, by requiring the incorporation of a model of the diameter in function of height, which will introduce new parameters. Let's assume that the diameter is linear per part for every $K$mm section, and the corresponding parameters to find would be the $N_k$ value of the diameter every $K$mm: $(d_1, ... d_{N_k}) \in |R^{N_k}$ 
%
%   In this new formulation, note that chosing $d_i$ values determi
%   In order to find adequate $d_i$ values, more data points would be needed in $\cal{D}_0$
%
% \subsection{Second order: Finding a thermometer calibration procedure}
%   The resolution of a first order problem can be stated as the following: How to find the mapping $f_1$ between a set of data points $\cal{D}_0$ and $f_0$.
%   We refer to this formulation as a second order problem.
%
%   The second order solution $f_1$, is a procedure that outputs the best first order solution given a set of data points $\cal{D}_0$
%   This procedure is an algorithm chosen in regard of some assumptions on the level of noise in the input $\cal{D}_0$ and the landscape of the search space $F_0$. These assumptions define the classes of functions $\cal{F}_1 = \{f: \cal{D}_0; \theta \to f_0, \theta \in |R^{N_1}\}$ conditionned on $\theta$ which are the constant parameters across first order instances.
%   In the above example $f_1$ could be the resolution of a simple linear system without any parameters, but could also be a gradient descent procedure relying on a definition of an objective function and with parameters such as the step size. 
%   More generally, when focusing on iterative optimization procedure such as gradient descent based algorithms, the parameters \theta can be characterize initialization schemes, the step computation or regularization parameters that reduce the search space $F_0$.
%
%
%   % Note that regularization can be loosely defined as assumptions used to reduce the search space $\cal{F}_0$ across all tasks such that $f_0 \in \cal{F}_0 = \{ f: y; z \to f(y; z), z \in \cal{Z}(\theta) \subset |R^{N_0} \}$.
%   % An example of regularization assumption could be a smoothness a priori on the diameter of the thermometer conditionned on a parameter $\lambda$ such that the parameters $(d_1, ... d_{N_k}) \in |R^{N_k}$ are constrained by  $\forall i \in [|1, N_k-1|], d_i - d_{i+1} < \lambda$ effectively reducing the search space.
%
%   The parameters $\theta$ are then tuned using a dataset $\cal{D}_1 = \{\cal{D}_0^1, ..., \cal{D}_0^{d_1}\}$ comprised of multiple instances of first order problems.
%
%  For example one can use the different examples in  $\cal{D}_1$ 
%
% %   We when modeling The function $f_1$ also depends on parameters that are 
% %   Additionally, for second order problems, the solution $f_1$ performs the search for the parameters of $f_0$ i.e. an optimization procedure. This procedure also depends on assumptions such as the level of noise in the input $\cal{D}_0$, and the landscape of the search space $F_0$. This will define parameters to find for $f_1$. Such parameters are tuned using a dataset $\cal{D}_1 = \{\cal{D}_0^1, ..., \cal{D}_0^{d_1}\}$ comprised of multiple first order problems.
% %
% %
% % Finally some parameters of the modelisation of first order problem can be unknown but common across problems and therefore are parameters of 
%
%
% \subsection{Second order ++: Finding a generic procedure to solve a class of problem}
%
% In the above second order problem, note that the class of function $F_0$ considered needs now to apply to every thermometers we want to calibrate using $f_1$.
%
%   This can motivate the elaboration of more generic modelisation $f_0$ with higher dimensional $cal{\F}_0$.
%   However this requires more data.
%   The two sources of data $\cal{D}_0$ and $\cal{D}_1$ are complimentary to this end.
%   More data in $\cal{D}_1$ allow for more tuning more parameters generic across and therefore reducing the search space of $cal{\F}_0$ during the second order resolution.
%   whereas a larger $\cal{D}_0$ can be used to search higher dimensional $cal{\F}_0$ during the first order resolution.
%
%   In the thermometer example above assuming the determination of the diameter every $K$mm requires too much datapoints compare to what we expect to have in $\cal{D}_0$, but that we have a many examples of thermometers in $\cal{D}_1$.
%   We can learn the distribution 
%   We could then introduce regularization parameters to be fitted across different first order instances and that would couple the $d_i$ together such as reducing the search space of $F_0$.
%   For example coupling the 
%   degrees of freedom that can be solved for each $\cal{D}_0$
%
%
% In the previous example, we saw an example of broadening the space of possible first order solution by relaxing some assumptions and then searching this larger space relying on more data.
% Going even further we could aim at finding a generic approach for finding calibration operator of differnt kind of sensors or even generic 
%
%
%
%
%   The model of the thermometer made for the first order problem needs to be questioned and 
%   When solving a second order problem, the assumptions made to 
%   $f_1$ then incorporate the search for $f_0$'s parameters.
%
%
%
% Given a non graduated thermometer we want to find 
%
%
% In geoscience, $u$ and $y$ can generally be defined as $D_u$ and $D_y$ dimensional vector fields defined on spatio-temporal domains $\Omega_u$ and $\Omega_y$. Here, we consider scalar fields and discrete fields as particular cases of this generic formulation. The different problems can be characterized by the types of quantities $u$ and $y$ and the spatio-temporal domains $\Omega_u$ and $\Omega_y$ on which they are defined. Figure \ref{fig:planet_drawings} illustrates how our two use cases of sensor calibration and SSH mapping easily fit into such formulation. Additionally, Figure \ref{fig:task_ontology} shows how tasks such as calibration, mapping, and forecasting can differ through the domain of definition of $u$ and $y$.
%
% \begin{figure}[htbp]
% \begin{center}
% \includegraphics[width=0.8\linewidth]{Chapitre1/Ch1-Figures/Cal_drawing.png} 
% \includegraphics[width=0.8\linewidth]{Chapitre1/Ch1-Figures/Mapping_drawing.png} 
% \end{center}
% \caption[Swot calibration and altimetry mapping problem illustration]
% {\footnotesize The calibration problem (top row) consists in finding the mapping $f$ that estimates the observed SSH $u$ from the SWOT satellite given the actual noisy measurement and ancillary calibrated measures ($y$).
% The mapping task (bottom row) consist in finding an operator $f$ that maps partial measurements of the SSH $y$ to a map of SSH $u$}
% \label{fig:planet_drawings}
% \end{figure}
%
%
% % \begin{figure}[htbp]
% % \begin{center}
% % \begin{tabular}[c]
% % \includegraphics[width=0.8\linewidth]{Chapitre1/Ch1-Figures/Cal_drawing.png} \\
% % \includegraphics[width=0.8\linewidth]{Chapitre1/Ch1-Figures/Mapping_drawing.png} \\
% % \end{tabular}
% % \end{center}
% % \caption[Swot calibration and altimetry mapping problem illustration]
% % {\footnotesize The calibration problem (top row) consists in finding the mapping $f$ that estimates the observed SSH $u$ from the SWOT satellite given the actual noisy measurement and ancillary calibrated measures ($y$).
% % The mapping task (bottom row) consist in finding an operator $f$ that maps partial measurements of the SSH $y$ to a map of SSH $u$}
% % \label{fig:planet_drawings}
% % \end{figure}
%
% \begin{figure}[htbp]
% \begin{center}
% \includegraphics[width=0.8\linewidth]{Chapitre1/Ch1-Figures/Task_ontology.png}
% \end{center}
% \caption[Task characterization through the domains $\Omega_u$ and $\Omega_y$ of $u$ and $y$]
% {\footnotesize Using the perspective provided by the problem definition, we can easily categorize earth observation problems.
% The calibration consist of estimating the field $u$ on a subset of the observation domain, the mapping consist in estimating $u$ on the same temporal domain but extending the spatial domain.
% And finally forecast can considered as wanting to estimate a quantity on an unobserved future domain.}
% \label{fig:task_ontology}
% \end{figure}
%
% \section{Method Ontology}
% We aim to characterize and organize the various methods used to tackle the class of problem introduced in \ref{sec:chap1_problem_form}. We propose that all methods can be decomposed into the following two steps:
% \begin{itemize}
% \item Step 1: Define the set $\cal{F}$ of possible $f$ using theoretical knowledge and making assumptions about the problem (conceptual models)
% \item Step 2: Search $\cal{F}$ for an optimal $f$ using factual knowledge (data) and making assumptions about the data
% % \item The first and second steps respectively rely on conceptual knowledge and available data;
% % \item If the data had included the volume of the liquid, the tube's diameter, and the dilation rate of the liquid, we might have needed theoretical knowledge about the volume of a tube to construct a relationship between the level and the temperature;
% % \item If only the diameter were unknown in a similar situation, a single data point would have been sufficient to calibrate the model;
% \end{itemize}
%
%   To illustrate our point let's consider the simple problem of thermometer calibration.
%
%
%
%
%
% To illustrate our point, consider a simple example of building a thermometer by placing a liquid in a tube and wanting to interpret the level of the liquid as a temperature. According to our previous notations, $y$ is the level of the liquid and $u$ is the temperature of the liquid inside, and we seek to find the mapping $f$ between the two.
%
% Step 1 involves compiling our theoretical knowledge on the problem to define the class of function. Given our understanding of fluid dilation in response to temperature, under the assumption that the diameter of the tube is constant with height, we can state that the level is linearly correlated with the temperature. Therefore, $f$ will be part of $\cal{F} = { y: \alpha y + \beta , (\alpha, \beta) \in |R^2 }$.
%
% In Step 2, to find $\alpha$ and $\beta$, we require two data points to calibrate our model, traditionally obtained by placing the thermometer in icy and boiling water at 1 bar of pressure to get the levels corresponding to 0°C and 100°C. This method relies on strong theoretical foundations and assumptions to reduce the dimensionality of the search space $\cal{F}$, thus facilitating the parameter search with relatively few data points.
%
% However, if we clearly see that the diameter of our thermometer is not constant, the model needs to incorporate that the evolution of the temperature depends on the diameter at each height. This expands the class of functions, necessitating the incorporation of a model of the evolution of the diameter in function of the height, which will introduce new parameters. We could assume that the diameter is linear for every 5mm section, and the corresponding parameters to search would be the value of the diameter every 5mm.
%
% To estimate these new parameters, we need more data which could be direct measures of the diameter or measures of temperature every 5mm. We could directly model $f$ as linear per part, thereby reducing the number of parameters to estimate (no more $\alpha$ and $\beta$). If we have measurements of the temperature, this also alleviates the need to explicitly model the relationship between diameter and temperature.
%
%
%
%
%
%
% % Tasks
% % Simple exemple
% % Calibration and mapping example
%
% % Tasks of interests can be summed up as finding f
% % Finding f takes two steps: defining the set of possible fs, searching the set for the best f
% % Formulating the sets of F requires theoritical knowledge
% % Searching the sets of F requires data
%
% % From theory to sets of 
% %   inverse problems state x
% %   data assimilation: dynamical model
% %   spatio temporal correlation: Covariance model
% %   deep learning
% %   locality: convolution
% %   temporal dependence RNN LSTM
%
%
% Lorem ipsum dolor sit amet, consectetuer adipiscing elit. Maecenas fermentum, elit non lobortis cursus, orci velit suscipit est, id mollis turpis mi eget orci.
%
% \section{Première section du chapitre}
%
% Lorem ipsum dolor sit amet, consectetuer adipiscing elit. Maecenas fermentum, elit non lobortis cursus, orci velit suscipit est, id mollis turpis mi eget orci.
%
% \subsection{Première sous-section}
%
% Lorem ipsum dolor sit amet, consectetuer adipiscing elit. Maecenas fermentum, elit non lobortis cursus, orci velit suscipit est, id mollis turpis mi eget orci.
%
% Voir figure \ref{fig:mafigure2}.
%
%
% \begin{figure}[htbp]
%    \begin{center}
%       \includegraphics[width=0.8\linewidth]{Chapitre1/Ch1-Figures/comparison.png}
%    \end{center}
%    \caption[titre court pour la liste des figures]
%    {\footnotesize Titre plus long avec des explications.}
%    \label{fig:mafigure2}
% \end{figure}
%
% \subsection{Deuxième sous-section}
%
% The calibration procedure described above followed an intuitive flow of framing the problem using assumptions on the physics and solving it using data.
% The calibration procedure described above followed an intuitive flow of framing the problem using assumptions on the physics and solving it using data.
%
% The above section detailed how to find a function that maps an observation of the level of the thermometer to the temperature.
% We saw how this function is the result of an procedure that uses some assumptions about the problem and takes as input a set of data points.
%
% We propose here a parallel between the calibration problem as stated above and the optimization problem of infering optimal parameters from data-points.
%
% The first step 
%
% Lorem ipsum dolor sit amet, consectetuer adipiscing elit. Maecenas fermentum, elit non lobortis cursus, orci velit suscipit est, id mollis turpis mi eget orci.
%
% \section{Conclusion du premier chapitre}
%
% Lorem ipsum dolor sit amet, consectetuer adipiscing elit. Maecenas fermentum, elit non lobortis cursus, orci velit suscipit est, id mollis turpis mi eget orci.
%
% In this manuscript I'd like to cite \cite{remo3,remo4}.

\addcontentsline{toc}{section}{Bibliography}
\putbib[./Chapitre1/Ch1-Biblio.bib]
\end{bibunit}

\clearemptydoublepage
\begin{bibunit}[IEEEtran.bst]

  \chapter*{First and second order modeling for altimetry problems}
\addcontentsline{toc}{chapter}{First and second order modeling for altimetry problems}
  \chaptermark{First and second order modeling for altimetry problems}
 
 Exsisting approaches for modeling alitmetry problems.
 Goal estimate the SSH on a continuous spatio-temporal domain $\Omega_u$
 
 
  \section{Priors: Chosing a model of the sea surface height (SSH)}
A first step is to compile our theoritical knowledge into some assumptions 
We differentiate between two ways of specifying our prior knowledge.

The first is the choice of representation of the SSH field we want to estimate. This define the space of all the possible SSH states.
The second way is to define some kind of distribution over the possible states.


 
  \subsection{State representation}
Chosing a state representation is equivalent to determining the quantities that characterizes the state and the relationship between the state values and the SSH field.


  Looking at existing methods, the ssh field can be characterized through values sampled on a grid of the domain. These values can be directly SSH (OI), Kalman Smoother, or some scale components of the SSH (4darnet) or even values within another basis like MIOST which uses the wavelet transform.
  The choices like the mesh, basis or SSH decomposition used are a way to dimension the state space informed by  prior knowledge on the SSH estimation problem.
Given a discrete representation of the SSH, the SSH estimation can be made on the whole domain using an interpolation scheme.

The strong contraint four dimensional variational data assimilation 4DVAR method characterizes the SSH field over a period through the initial conditions. A numerical integration scheme of a dynamical model is then used to infer the SSH over the whole temporal horizon. Note that the initial condition representation is usually also made on some grid. 

Deep learning has introduced another way of representing the SSH through Neural fields, this consists of describing the SSH with the parameters of a coordinate based neural network. The neural network can then be used to output the SSH value for given any coordinate of the domain.


Finally the state can also contain ancillary quantities that are linked to the SSH. In the GLORYS data assimilation product, the ocean model used NEMO considers the state of the ocean beyond the SSH.
In the case of the SWOT calibration, the estimating the SSH is equivalent to estimating the error signals. Operational method approach the problem as chose a state representation of the different error signals making assumptions on the processes generating them.

\subsection{Prior costs}
The representation of SSH we chose defined the space of all possible states. Additional constraints can be added to better characterize the assumptions on the SSH by specifying which states are more likely than others. This is the prior on the states distribution.
In Optimal interpolation and strong 4dVar this is done by defining a first guess (or background state) and its error covariance matrix.
In Kalman filtering, 3DVar or weak4DVar, the prior likelihood is computed with errors covariance with regard to the trajectory of a dynamical model.

In this regard deep learning introduce energy-based model to this end. In 4DVarNet, the prior over the state space is formulated using an auto-encoder $\phi$  as $\| x - \phi(x)\|$



We summarize in the table below the different first order \textbf{models} of different 
  \begin{table}
\begin{tabular}{|l|l|l|l|}
\hline
Method & state repr & state $\to$ ssh & prior cost \\
\hline
OI & Grid $x$ & interp & $\|x - x_b\|_B$ \\
\hline
s4DVar & Grid t0 $x_0$ & dyn model  ++ & $\|x_0 - x_b\|_B$ \\
\hline
w4DVar & Grid & interp & $\|x - x_b\|_B + \sum\|x_{k+1} - \mathcal{M}_{k\to k+1}(x_k)\|_{Q_k}$ \\
\hline
MIOST & wavelet $x$ + & wavelet transform + & $\|x - x_b\|_{\Gamma Q\Gamma}$ \\
\hline
4DVarNet & 2 scale Grid $x =(u_{ss}, u_{ls})$ & $u = u_{ss} + u_{ls}$, interp & $\|z - \Phi_{NN}(z)\|_B$ \\
\hline
Nerf & NN Params & NN inference +++ & None \\
\hline
Direct NN inversion & Grid & interp & None \\
\hline
\end{tabular}
\end{table}

% | Method              | state repr                         |          state -> ssh           |                            prior cost                             |
% | ------------------- |:---------------------------------- |:-------------------------------:|:-----------------------------------------------------------------:|
% | OI                  | Grid $x$                           |             interp              |                          $\|x - x_b\|_B$                          |
% | s4DVar              | Grid t0 $x_0$                      |          dyn model  ++          |                         $\|x_0 - x_b\|_B$                         |
% | w4DVar              | Grid                               |             interp              | $\|x - x_b\|_B + \sum\|x_{k+1} - \cal{M}_{k\to k+1}(x_k)\|_{Q_k}$ |
% | MIOST               | wavelet $x$    +                   |      wavelet transform   +      |                  $\|x - x_b\|_{\Gamma Q\Gamma}$                   |
% | 4DVarNet            | 2 scale Grid $x =(u_{ss}, u_{ls})$ | $u = u_{ss} + u_{ls}$,   interp |                     $\|z - \Phi_{NN}(z)\|_B$                      |
% | Nerf                | NN Params                          |        NN inference +++         |                               None                                |
% | Direct NN inversion | Grid                               |             interp              |                               None                                |



\section{Solvers: Estimating the state given some observations}

Once all prior assumptions about the SSH field have been made, the next choices concern the \textbf{calibration procedure} used to estimate the state given some observations.

This requires formulating about the relationship from the observations to the state and chosing an estimation procedure.

A class of method start by defining an observation operator $H$ that describe how to go from state to observations, finding $f$ consist in finding the inverse of this operator $H$. This class of method is named "inverse problem"
This problem is usually ill posed, without a unique solution.

$y = H(x) = \cal{H}(x) + \epsilon$
After making assumptions on $\epsilon$ such as unbiased gaussian noise.
The field of data assimilation in geoscience propose a variety of methods to solve inverse problems.
Kalman filters use a bayesian based statistical formulation. They solve for the posterior estimate given the observation. Optimal interpolation can also use such formulation to estimate the SSH. 
But also variational methods that formulate the estimation as a minimization problem, the objective to minimize is called variational cost and usually include a observation term and a regularization term making use of the prior cost.
The minimization procedure usually involves some kind of iterative gradient based algorithm.
In previous work using neural fields, the state is fitted only on the observation cost.


Deep learning also opened the way to directly model the inversion process with a neural network as done in \cite{}. We call this approach direct inversion.



The different procedures are summarized in the table below

\begin{table}
  \begin{tabular}{|c|c|c|}
\hline
Methods & Estimation & $O(y, x)$ \\
\hline
Nerf & $\theta = \arg\min(\mathcal{L})$ & $\mathcal{L}$ \\
\hline
Var & $x = \arg\min(\alpha R(z) + \beta O(y, x))$ & $\|y - Hx\|$ \\
\hline
OI, Kalman & $x = x_b + K(y - Hx_b)$ & $\|y - Hx\|$ \\
\hline
Direct Inv & $x = \phi(y)$ & \\
\hline
\end{tabular}

\end{table}

% | Methods    |         estimation         |   O(y, x)   |
% | ---------- |:--------------------------:|:------------:|
% | Nerf       | $\theta = argmin(\cal{L})$ |  $\cal{L}$   |
% | Var        |      $ x = argmin(\alpha R(z) + \beta O(y, x))$      |    $\|y - Hx\|$          |
% | OI, Kalman | $x = x_b + K(y - Hx_b)$ | $\|y - Hx\|$ |
% | Direct Inv |          $x = \phi(y)$          |              |
%



For nadir altimetry, thanks to the calibration  most methods usually make the assumption of unbiased gaussian noise. 
$\| y - \cal{H}(x)\|_R$ 


\section{Second order Modeling and estimation}
All those choices of prior on the ssh field, observation cost, estimation procedure introduced new factors that need to be determined.
Those factors include the background field of data assimilation schemes, as well as the error covariances for the observation and background. They also include the choices numerical schemes parameter for variational optimization procedure or numerical model integrations. Finally those factors also include the neural network parameters.

The process estimation of those quantities unfold in a similar manner as the estimation of the SSH.
It relies on data that include numerical model outputs and historical observations.

Given the theoritical knowledge at hand, assumptions are made about what distribution is reasonable to model the errors and to how to model the covariances between the errors, what numerical schemes should be considered for integrating the dynamical models, what constitutes a good first guess for data assimilation schemes or what neural architecture is suited for the direct inversion problem.

Those assumptions characterize the quantities that need to be estimated and we can differentiate multiple ways that are used to determined them.

A combination of different methods are used to estimate those quantities.

Cross validation consists in using part of the calibration data to evaluate the choice, 
The choice can then be made "randomly", through some statistics on the calibration data aor through some optimization procedure (bayesian opt or gradient descent...)

In the case of direct inversion, the search for the neural parameters are equivalent the state search of the nerf method

Note that in some sense second order calibration data also contain information that we would like to infuse to our method, so solving the second order problem is the data centric part of the methodology. 
The information stored in available data can be used to tune the prior models, or the state estimation procedure.
Neural Direct inversion approach virtually has no prior models (except from the state formulation) and all info contained in the data goes only into the estimation procedure.

\section{A closer look on the 4dVarNet}
In this section we look in closer detail at an architecture that is used throughout this thesis.
The 4dvarnet framework was a novel and promising  formulation at the time of my thesis. In \cite{}, it showed some strong performances when evaluated on simulated SSH Gulf stream study.
The simulated SSH used was from the NATL60 simulation and the altimetry configurations considered were 4 nadir altimeters with and without SWOT observations
Early version also considered an OI-based product as observations to the mapping problem
We detail below the assumptions made and the different components used.


\subsection{State formulation}
The 4dvarnet framework uses a grid representation of the SSH with a few different variations have been experimented in different work.
some work represent the SSH directly as scalar
other decompose each ssh value in a large and small scale components
finally another formulation consists in introducing latent values in addition to the two scale components 


\subsection{Prior cost}
In the 4dvarnet the prior cost is formulated using a neural network.
Given a nn $\phi$, $R(x)=\|x - \phi(x)\|$.
Different choices of $\phi$ have been used in different works.
Some use simple or multiscale bilinear blocks
Some tried with Unets
on simplified lorenz system, the pde of the system has been tested therefore being a 4dvar formulation



\subsection{Observation operator}
Previous work make a no noise assumptions and link directly the observed values to the coresponding grid values
Some work also introduce a multimodal verion making use of sea surface temperature observations that are linked to the state using neural network formulation.


\subsection{State Estimation procedure}
In this work different approaches were considered for  estimating the state.
A fixed point algorithm analog to EM were used by maximizing the obs likelihood (clipping the obs to the state) then making a forward pass with the $\phi$
Other approaches used the variational formulation of minimizing a combination of observation and prior cost


\subsection{Learning: Estimation procedure}
Apart from the cross validation and exploration of different architectures and configurations, the actual parameter values of both the neural solver and the neural prior are trained using a standard deep learning otpimization procedure Adam. There parameter are tune to minimize the mean squared error of the SSH reconstruction as well as  the reconstruction of the gradient. In order to further guide the weights of the neural prior, a term in the training loss is added to nudge the estimated states have low autoencoder loss.

%   \section{First order: Estimating an SSH field}
%
%   First order calibration data is a sample of altimetry observations corresponding to a single estimation task.
%   The estimation can be of a map and the data would be the surrounding nadir observations.
%   The estimation could also be of the calibrated SSH, and then the data would be the concerned nadir track as well as the surrounding nadir observation.
%
%   \subsection{Assumptions about the SSH field: Model and model state}
%   % maybe start with the state, ssh, constraints on state
%   The kind of assumptions we make about the SSH field will result in two components.
%   The definition of a state which will be a set of values to be determined for the SSH estimation as well as some computational procedure to infer the SSH values on the domain from the state.
%   Each choice of model introduce parameters that need to be tuned prior to state inference
%
%   Model -> state -> parameters -> inference
%   Dynamical models of the ocean -> initial conditions (weak constraints: initial condition per subsegment) -> integration grid and step -> integration scheme
%   Grid -> SSH grid values -> resolution -> interpolation
%   Covariance model of errors wrt a first guess -> First guess errors -> addition / interpolation
%   Neural network -> parameters -> architecture -> nn inference
%   Composite signal with additive errors -> error parameters -> 
%   Dimensionality reduction -> basis components
%   Auto-Encoder -> grid -> interpolation (similar to denoising)
%
%
%   \subsection{State estimation procedure}
%     Depending on the prior assumptions, different approaches exists to perform the actual state estimation from the altimetry observations.
%     Each procedure also introduce parameters that need to be tuned for inference.
%
%     cross-validation leaving some calibration data out, trying different values and checking which ones works best.
%     bayesian estimation: kalman gains of kalman filters  and optimal interpolation computation in observation error modeling
%     Variational methods: data assimilation or optimal interpolation solving a minimization problem.  Minimization procedure (iterative step), observation cost
%     Neural network training: Stochastic gradient descent, learning to learn algorithm,  (J em)
%     Neural network inference (for grid values, for covariance matrix) (Manuch)
%
%
%   \section{Second order problem: tuning the parameters of SSH model and estimation procedure}
%   In order to solve the SSH estimation tasks, the parameters introduced in the model and estimation procedure need to be determined.
%
%
%   The calibration data for this second order data represent historical observations as well as potential numerical simulation
%   Some parameters can be estimated through statistical analysis of historical data when available.
%   the first guess can be determined through historical average
%   noise levels can be informed by historical data and tuned through different 
%
%   Others need to be calibrated on similar tasks that can be evaluated on OSE or OSSE setup.
%   Neural network parameters can be trained
%   Sensible grid and integration schemes of dynamical model can be chosen
%   Covariance matrices 
%
%
%   Note that the second order calibration can introduce hyper parameters than themselves need to be determined,
%   they are usually found through trial and error in a cross validation  fashion.
%
%
% \section{A closer look at the 4dVarNet prospect: a hybrid approach}
%   \subsection{overview}
% This thesis is extensively on prior work 
% This 4dVarNet makes an interesting combination of classical and deep learning based methods.
% The first order assumptions made on the SSH field is that the field should be a fixed point of a certain neural network phi.
% the state is represented as a spatial temporal grid, first application decompose the ssh value in a large scale and small scale component.
% this phi can be though as analoguous as the integration of the dynamical model in strong 4DVAR, in which want the estimated SSH to be the integration of the 
% in order to find the state that satisfy the prior for given observations, a variational formulation is employed,
%   meaning that the estimation is done trhough the minimization of a quantity.
%   This minimization is done using a neural based gradient descent initially developped for meta learning tasks involving a recurrent neural network
%   The second order parameters therefore consists in the neural network parameters of the phi operator as well as the parameters of the recurrent neural network.
%
% \subsection{Existing results}




  
%   \chapter*{Model driven, data-driven and deep learning for altimetry analysis.}
% \addcontentsline{toc}{chapter}{Deep Learning, inverse problems and altimetry}
%   \chaptermark{Model driven, data-driven and deep learning for altimetry analysis.}
%
%
%
%
% As presented in the previous chapter, the \textbf{model} and \textbf{calibration data} are inter-dependent. The number of parameters of the model will impose constraints on the quantity calibration data and reciproquely the available data will constrain the kind of model that can be considered.
% This coupling introduce a possible distinction between model-driven and data-driven approaches. We use this distinction to organize the overview of the existing altimetry methods in the first two sections.
% Deep learning can be viewed as data-driven but introduce specific considerations that are presented in a third section.
%
%
%   \section{Model driven}
% We designate by model-driven the class of methods that are predicated on domain specific assumptions.
% \subsection{Data assimilation for altimetry: mapping}
% The ocean being a dynamical system, we look more in detail about the methods making use of physical assumptions of the system in the form of dynamical models.
%
% We will first detail here more precisely the data assimilation methods that leverage dynamical knowledge of ocean processes for altimetry mapping.
% In geoscience, data assimilation refer to the estimation of a state $X$ from observation data $y$ using a dynamical model $M$.
%
% In practice, data assimilation work in altimetry rely on \textbf{models} that vary greatly in complexity.
%   The reanalysis GLORYS12 rely on the full-fledged Ocean General Circulation Model (OGCM) NEMO\cite{} which solves the primitive equations and models the sea-ice.
%   Other works rely on simplified Quasi-Geostrophic dynamics.
%
%
%   Given a dynamical model, different formulation and algorithms are used to assimilate the observation data.
%   We detail three methods which are Kalman filtering\cite{}, Variational data assimilation\cite{} and back and forth nudging\cite{}.
% Kalman filtering is a sequential assimilation method that requires a linearization of the dynamical model and the observation model and assume gaussian noise in the observation and the model.
%   This principle is at the base of the SEEK formulation used for state of the art operational oceanography like product like Glorys.
%   Variational data assimilation (VarDA) formulates the problem as a minimization problem in which the state minimizes an observation discrepency cost combined with a regularization cost involving a dynamical integration of the state.
%   The minimization of the variational cost then involves an iterative optimization procedure like a gradient descent.
%   Flavours of VarDA go from 3DVAR, 4DVAR, weak4DVAR. 3D-Var is also used for biais correction in GLORYS12 product\cite{}. Variational approaches do not require the model to be linear but the optimization procedure can be computationally expensive by requiring multiple integrations of the dynamical model.
%   Back and forth nudging (BFN) is an approach that has been succesfully used to assimilate altimetry data with a QG model\cite{}. It can be seen like an hybrid method between kalman filtering and VarDA and consists in iteratively integrating the model forward and backward in time while adding a term to the dynamical model that nudges the integration towards observed values. 
%
% \subsection{Systematic error modeling for SWOT calibration}
% For the SWOT calibration of correlated errors, fewer studies exists. However envisionned operational approaches also rely on models, but instead of modeling ocean processes, they model the processes behind the error signals.
% Once the different processes are modeled, calibration data is used to estimate the parameters of the error processes.
%
% \section{Data driven}
% Data driven aims at making the minimal assumptions given the available data, the problem can then be seen as an interpolation problem. 
% Optimal interpolation is the main method used in altimetry. The model characterizes  the spatial and temporal decorrelation rate through a covariance model.
% parameters covaraince model as well as decorrelation factor.
% covariance can depend on place and time
%   MIOST solve in reduced wavelet basis, the choice of basis adds additional parameters.
%
%
% \section{deep learning}
% Deep learning models are data driven but require potentially even less
%   deep learning models for computer vision are mostly based on convolution filters and non linearities.
%   deep learning calibration algorithms include stateful gradient descent with adaptive step size and second order term estimation that can themselfes be parameterized with neural networks.
%
%   Application of deep learning for altimetry has known a significant boom during the course of this PhD.
%   Traditional CV architectures have been tested on QG simulation with toy spatial and temporal interpolation tasks.
%   the 4dvarnet inspired from variational data assimilation tested in  OSSE with a SOTA simulation.
%   by the end of the thesis, ConvLSTM have been trained on real altimetry data.
%
%   For the calibration of swot, studies exist to remove the KaRIN noise but not for the correlated errors.
%
%





% We aim here at providing a more detailed overview of the state of the art methods for tackling the mapping and calibration challenges adressed in following chapters.
% First we introduce the generic class of inverse problems.
%
%
%
% Denoising. Reconstructing the trajectory of a dynamical system from observation.
%
% One characteristic of such problem is that they are usually ill-posed, in the sense that the solution may not be unique.
% Different approaches for solving these inverse problems rely on different way to model the sytem and computational methods to estimate the parameters of the system from the data.
% Therefore inverse problem solving methods usually rely on injecting prior knowledge about the system in the model.
%
% We'll first look in detail at two different method
% Tasks such as altimetry mapping and SWOT calibration have both been adressed before.
%   In this chapter we review the formalism and methods that exists for solving such problems.
%   Both task can be formulated as inverse problems, and this chapter is organized as follows:
%   First part will look 
%   Given some data $y$ that result from a process $\cal{F}$ applied to some state $x$, the task of determining $x$ from $y$ can 
% Estimating an underlying state from a 
% The task we introduced can be seen as inverse problems and existing 
% We have introduced in the previous chapter the different components to consider when addressing observation problems such as altimetry mapping and calibration. In this chapter, we'll paint the landscape of the different approaches that have been developped in order to contextualize where our research fit in.
%
%
%
%
% Observation tasks such as altimetry mapping and sensor calibration can be seen as inverse problems. 
% Inverse problems broadly encompasses the tasks of estimating states parameters of a system from data produced by that system.
% These problems are characterized by their ill-posedness. 
% Over the years different class of computational methods have been developped to solve these problems. Recently deep learning has also been increasingly used for such problems.
% We detail in the first section of this chapter how the altimetry usecases considered fit in the inverse problem formulation justifying therefore the relevance of this category of methods.
% In the second section we present two state of the art domain approaches that are especially relevant for our case.
% In the third section, we present the deep learning methods 
%   Finally we'll describe the 4dvarnet, a neural scheme inspired by variational data assimilation
%
%   \section{Altimetry Usecases as inverse problems}
%   \subsection{Notation for inverse problems}
%
%   \subsection{Mapping methods}
%   \subsection{Calibration Methods}
%
%   \section{Domain methods for Inverse Problems}
%   \subsection{Model driven: Dynamical prior and data assimilation}
%   \subsection{Data driven: Statistical prior and optimal interpolation}
%   \subsection{Experimental methods }
%
%
%   \section{Deep learning for inverse problems}
%   \subsection{Deep prior and neural radiance fields}
%   \subsection{Direct inversion}
%   \subsection{Physics informed deep learning}
%
%
%   \section{An hybrid method: the 4dVarNet}
%
%
%
%
%   \begin{itemize}
%     \item OI
%     \item MIOST
%     \item DYMOST
%     \item DA (Kalman filters, 4DVAR)
%     \item convlstm
%     \item Dincae
%     \item 4dVarNet
%   \end{itemize}
% \section{Models}
%   \subsection{Physics}
%   \begin{itemize}
%     \item Ocean physics BFN, GLORYS 
%     \item Calibration roll estimation
%   \end{itemize}
%   \subsection{Statistics}
%   \begin{itemize}
%     \item OI
%   \end{itemize}
%   \subsection{Deep learning}
%   \begin{itemize}
%     \item 
%   \end{itemize}
% \section{Data}
%   \subsection{Observations}
%   \begin{itemize}
%     \item 
%   \end{itemize}
%   \subsection{Numerical model simulations}
%   \begin{itemize}
%     \item 
%   \end{itemize}
% \section{Algorithm}
%   \subsection{Statistics}
%   \begin{itemize}
%     \item 
%   \end{itemize}
%   \subsection{Data Assimilation}
%   \begin{itemize}
%     \item 
%   \end{itemize}
%   \subsection{Iterative Gradient based}
%   \begin{itemize}
%     \item 
%   \end{itemize}
% \section{Evaluation}
%   \subsection{OSSE}
%   \subsection{OSE}
%
%
%
%
% Section 1: Altimetry Usecases as Inverse Problems
% Introduction
%
% Altimetry, particularly when it involves oceanic applications, often deals with indirect measurements. Essentially, we have observable data—like sea surface height—from which we aim to estimate underlying physical states or parameters, such as current velocities or sea bed topology. This task fits squarely within the framework of what are called 'inverse problems'.
% Notation for Inverse Problems
%
% Before we delve into the specifics, let's set some simple notation to help us along the way:
%
%     yy: Observed data (e.g., sea surface height)
%     xx: Parameters or states to be estimated (e.g., ocean currents)
%     FF: Forward model that connects xx to yy, F(x)=yF(x)=y
%
% The goal of an inverse problem is to find xx given yy and FF.
% Mapping Methods
%
% Ocean altimetry mapping aims to derive high-resolution ocean surface topography or currents from relatively sparse and irregularly distributed satellite altimetry data. These methods take the observable—sea surface height (yy)—and use it to estimate underlying physical states like ocean currents (xx) using a forward model FF that incorporates equations of fluid dynamics and other physical laws. These are quintessential examples of inverse problems.
%
% In the literature, techniques like Optimal Interpolation and Kalman Filtering have been extensively used for this task. These methods come with their own assumptions and limitations, such as requiring the error statistics to be Gaussian or assuming linearity in the forward model FF.
% Calibration Methods
%
% Calibration in the context of altimetry involves adjusting sensor parameters to ensure that the measurements are accurate and reliable. Here, the observed data (yy) could be the raw readings from the altimeter, and the states or parameters (xx) would be the calibration factors. The forward model FF in this case would describe how the calibrated sensor should behave under ideal conditions.
%
% For example, one might have a mathematical model FF that predicts sensor readings based on laboratory conditions and known physical laws. The inverse problem then is to adjust xx (calibration parameters) such that F(x)F(x) closely matches yy (actual sensor readings).
%
%
% Section 2: Domain Methods for Inverse Problems
% Introduction
%
% The landscape of computational methods for solving inverse problems in altimetry is quite diverse. Broadly, these methods can be classified into three categories: model-driven, data-driven, and experimental methods. Each has its advantages and limitations, and the choice often depends on the specific use-case, data availability, and computational resources. This section aims to provide an overview of these classes of methods, particularly in the context of ocean altimetry.
% Model-Driven: Dynamical Prior and Data Assimilation
% Definition
%
% Model-driven methods often employ a priori knowledge of the physical laws governing the system. In oceanography, this could involve fluid dynamics, gravitational forces, and thermodynamics to make educated estimations. Data assimilation techniques, such as the Kalman filter, are typical examples.
% Pros and Cons
%
%
% Data-Driven: Statistical Prior and Optimal Interpolation
% Data-driven methods rely on statistical models to solve inverse problems. Rather than using detailed physics-based models, these methods use statistical approaches to approximate the relationship between observed data and underlying states. Optimal Interpolation is a commonly used technique.
%
%
% Experimental methods:
% BFNQG
% MIOST
% DYMOST

\end{bibunit}


\clearemptydoublepage
% \documentclass[lettersize,journal]{IEEEtran}
% \usepackage{amsmath,amsfonts}
% \usepackage{algorithmic}
% \usepackage{array}
% \usepackage[caption=false,font=normalsize,labelfont=sf,textfont=sf]{subfig}
% \usepackage{textcomp}
% \usepackage{stfloats}
% \usepackage{url}
% \usepackage{verbatim}
% \usepackage{graphicx}
% \usepackage{booktabs}
% \usepackage{pgfplots}
% \usepgfplotslibrary{external} 
% \usepackage{layouts}
% \hyphenation{op-tical net-works semi-conduc-tor IEEE-Xplore}
% \def\BibTeX{{\rm B\kern-.05em{\sc i\kern-.025em b}\kern-.08em
%     T\kern-.1667em\lower.7ex\hbox{E}\kern-.125emX}}
% \usepackage{balance}
% \begin{document}
% \title{Scale-aware neural calibration for wide swath altimetry observations}

% \author{Quentin Febvre, ~\IEEEmembership{Student,~IEEE,} Clément Ubelmann, Julien Le Sommer, Ronan Fablet%
% \thanks{Quentin Febvre and Ronan Fablet are with IMT Atlantique and UMR CNRS Lab-STICC, INRIA team Odyssey, Brest, France. Email: quentin.febvre@imt-atlantique.fr, ronan.fablet@imt-atlantique.fr}%
% \thanks{Clément Ubelmann is with Datlas, Greno, France. Email: clement.ubelmann@datlas.fr}
% \thanks{Julien Le Sommer is with MEOM group at Université Grenoble-Alpes, CNRS, IRD, Grenoble, France. Email: julien.lesommer@univ-grenoble-alpes.fr}}

% \markboth{Transactions on geoscience and remote sensing}%
% {Scale aware deep learning for wide-swath altimetry calibration}


% \maketitle
\begin{bibunit}[IEEEtran.bst]

\clearemptydoublepage
% \chapter{Scale-aware neural calibration for wide swath altimetry observations}
  \chapter*{Scale-aware neural calibration for wide swath altimetry observations}
\addcontentsline{toc}{chapter}{Scale-aware neural calibration for wide swath altimetry observations}
  \chaptermark{Scale-aware neural calibration for wide swath altimetry observations}

  
% \begin{abstract}
% 	Sea surface height (SSH) is a key geophysical parameter for monitoring and studying meso-scale surface ocean dynamics. For several decades, the mapping of SSH products at regional and global scales has relied on nadir satellite altimeters, which provide one-dimensional-only along-track satellite observations of the SSH.  
% 	The Surface Water and Ocean Topography (SWOT) mission  deploys a new sensor that acquires for the first time wide-swath two-dimensional observations of the SSH. This provides new means to observe the ocean at previously unresolved spatial scales. A critical challenge for the exploiting of SWOT data is the separation of the SSH from other signals present in the observations. In this paper, we propose a novel learning-based approach for this SWOT calibration problem. It benefits from calibrated nadir altimetry products and a scale-space decomposition adapted to the structure of the different processes in play in the SWOT's swath geometry.
% 	In a supervised setting, our method reaches the state-of-the-art residual error of $\approx$ 1.4cm while proposing a correction on the entire spectral from 10km to 1000km and using weaker constraints on the modeled error signal.
% \end{abstract}

% \begin{IEEEkeywords}
% Deep Learning, Altimetry, Calibration, SWOT.
% \end{IEEEkeywords}


\section{Preface}
This chapter

\section{Introduction}

Nadir altimeter satellites provide invaluable direct measurements of the sea surface height (SSH) to monitor sea surface dynamics. 
They have played a key role in better understanding ocean circulation and improving climate monitoring. Altimeter-derived SSH data are also of key interest for offshore activities, marine pollution monitoring or maritime traffic routing among others. 

However due to the sparse and irregular sampling associated with nadir altimeter constellations, a wide range of ocean processes from the mesoscale to the submesoscale range remains unresolved, typically for horizontal scales below 150 kilometers and time scales below 10 days. 
The recently launched SWOT mission, with its Ka-band radar interferometer (KaRIn) sensor, provides for the first time higher-resolution and two-dimensional snapshots of the SSH. Once this data is adequately processed, it will likely strongly impact our ability to observe and study upper ocean dynamics \cite{Peral_Esteban-Fernandez_2018}. 

\begin{figure}[!t]
    \begin{center}
        \includegraphics[width=\linewidth]{00_Calib/gridded_sensors.png}
    \end{center}
    \caption{\textbf{Observing System Simulation Experiment Cross-Calibration data:} \textit{Top left:} Sea surface height (SSH) on October 26th 2012 from NATL60 simulation dataset. \textit{Top right:} Calibrated NADIR pseudo-observations sampled using realistic orbits from the SSH, they are used to compute the gridded product for the cross-calibration.\textit{Bottom-left:} NADIR + noise-free-KaRIn pseudo-observations, the  2{\sc d} sampled SSH is the target of the cross-calibration.\textit{Bottom-right:} NADIR + noisy-KaRIn pseudo-observations, simulated errors added to the swath SSH constitute the uncalibrated input of the cross-calibration problem}
\label{c3fig:gridded}
\end{figure}

As reported in Figure \ref{c3fig:gridded}, KaRIn data will be affected by instrument and geophysical errors  \cite{ubelmann_swot_nodate} and their exploitation requires to develop robust calibration schemes. We illustrate in Fig.\ref{c3fig:filtered_swath_uncal_comp} the two main error sources: instrument errors, especially roll errors, are expected to cause the dominant large-scale signal in both across-swath and along-swath directions; and geophysical errors, in particular due to wet-troposphere-induced delays\footnote{We refer the reader to Section \ref{c3subsec:altimetry} for the description of these error signals in raw KaRIn observations.}.
The amplitude of these errors typically range from a few centimeters to a few meters in simulation, when the variability of the SSH for scales below 150km typically amounts to centimeters (See Fig. \ref{c3fig:gridded_impact}). This makes SWOT calibration a particularly challenging task in terms of signal-to-noise ratio. State-of-the calibration schemes \cite{Dibarboure_Ubelmann_Flamant_Briol_Peral_Bracher_Vergara_Faugere_Soulat_Picot_2022} rely on explicit spectral priors to address the calibration problem. 
The underlying hypotheses that one can linearly separate the SSH and the different error components may however impede the performance of such calibration methods. Here, we propose a novel learning-based approach. 
We leverage the computational efficiency of deep learning schemes with a scale-space decomposition \cite{Witkin_1984} adapted to the geometry of KaRIn observations. 

Our main contributions are as follows:
\begin{itemize}
\item{We state the cross-calibration of KaRIn altimetry observations as a learning problem using both raw KaRIn altimetry data and a gridded altimetry product as inputs to the neural network.}
\item{Our neural network architecture applies a scale-space decomposition scheme in the geometry of the KaRIn swath to improve the separability of the SSH and of the errors.}
\item{Numerical experiments using an Observing System Simulation Experiment (OSSE) demonstrate the relevance of the proposed approach and highlight the impact of the quality of the gridded altimetry product to retrieve finer-scale patterns in the calibrated KaRIn observations.}
\end{itemize}
This paper is organized as follows. Section \ref{c3sec:background} provides some background on related work. We introduce the considered data and case-study in Section \ref{c3sec:case_study}. 
Section \ref{c3sec:method} presents our method and we report numerical experiments in Section \ref{c3sec:results}. 
Section \ref{c3sec:conclusion} discusses further our main contributions.

\section{Background}
\label{c3sec:background}
\noindent
\begin{figure*}[!t]%
   \centering
    \subfloat[$f$]{{\includegraphics[width=.49\textwidth]{00_Calib/swath_err_details} }}%
    \subfloat[$\mathcal{G}_{200km}(f) - \mathcal{G}_{10km}(f)$]{{\includegraphics[width=.49\textwidth]{00_Calib/swath_err_f10_200_details} }}%
    \caption{\textbf{1000km segment of KaRIn observation components in swath geometry:}\textit{(a)} Looking at the three signals we see that the large scale instrument errors (middle) are predominant compared to the SSH (top) and geophysical error (bottom). \textit{(b)} Looking at the along-track scales between 10km and 200km, we note that the SSH is dominant w.r.t the error signals.}%
    \label{c3fig:filtered_swath_uncal_comp}%
\end{figure*}
\subsection{Satellite altimeters}
In this paper, we address the cross-calibration of KaRIn observations, meaning that the proposed calibration scheme relies on external calibrated data. More specifically, we consider a constellation of  4 nadir satellite altimeters. We recall that nadir altimeters provide can provide calibrated measurements of the SSH for medium to large scales along 1{\sc d} profiles corresponding to satellites' orbiting paths. Over the last decades the constellation counted typically from 4 to 7 satellites.

By contrast, according to the mission's error budget specification \cite{Peral_Esteban-Fernandez_2018} the KaRIn instrument samples a two-dimensional swath of approximately 120km-wide with a 2km$\times$2km pixel resolution everywhere over the ocean.


In Figure \ref{c3fig:gridded}, we report simulated altimetry observations for both nadir altimeters and KaRIn along with the reference SSH issued from a numerical simulation dataset (see  the Section \ref{c3sec:case_study} for details). As an illustration of the complexity of calibration problem, the error signals completely occlude the SSH signal in the uncalibrated KaRIn observation.
Figure \ref{c3fig:filtered_swath_uncal_comp} illustrates further this point in the swath geometry. When focusing to along-track scales between 10km and 200km, the SSH signal becomes the main signal (Fig. \ref{c3fig:filtered_swath_uncal_comp}). This supports
both to consider a scale-space decomposition and to 
investigate a cross-calibration approach with
the exploitation of nadir-altimeter-derived altimetry products, which typically resolve horizontal scales above 100 km.


\subsection{Interpolation of satellite-derived altimetry data}
\label{c3subsec:interpolation}

As mentioned previously, flying nadir altimeter constellations naturally advocate for considering the resulting interpolated SSH products as auxiliary data of interest to address the calibration of KaRIn observations. 

Regarding operational SSH products, we may distinguish the optimally-interpolated altimetry-derived product (DUACS) \cite{taburet_duacs_2019} and reanalysis products using ocean general circulation models to assimilate various observation datasets, including satellite altimetry and satellite-derived sea surface temperature data \cite{glorys_rea_2021}. Both types of products typically retrieve SSH dynamics on a global scale for horizontal scales above 150km and 10 days. 

Recently, a renewed interest has emerged in interpolation methods for ocean remote sensing data \cite{beauchamp_intercomparison_2020}\cite{fablet_end2end_2021}. Especially, deep learning schemes have emerged as appealing approaches to make the most of available observation datasets. Recent benchmarking experiments \cite{osse_data_challenge} point out significant potential gains compared with the above-mentioned operational products.

Here, we aim at investigating the extent to which the quality of L4 nadir-altimetry-derived SSH products may impact the calibration of KaRIn observations.



\subsection{Scale-space theory}
\label{c3subsec:scalespace}

The scale-space theory provides a mathematically-sound framework to decompose 2{\sc d} signals at different spatial scales \cite{Witkin_1984}. Gaussian scale-space methods are among the most widely used. They rely on applying Gaussian blur transformations with different standard deviations. This approach has been widely used in low-level image processing tasks  \cite{lindeberg1996edge,Lindeberg_2015}. 
Recent studies have used the scale-space theory in deep learning architectures \cite{Pintea_Tomen_Goes_Loog_van_Gemert_2021,Worrall_Welling_2019}. These neural networks better deal with multi-scale patterns in the data. 
Here, we draw inspiration from the scale-space framework to address the KaRIn calibration problem.
We design a scale-aware decomposition scheme as part of our learning approach with a view to 
better accounting for the different characteristic scales of the signals in play.


\subsection{Deep Learning for earth observation}
\label{c3subsec:dl}
\noindent
Convolutional neural networks are among the state-of-the-art neural architectures for image processing applications, including
%have been widely considered the standard architecture for image processing in the deep learning field for a variety of task such as 
image classification\cite{lecun98,resnet2016}, image in-painting\cite{liu_image_2018}, object detection\cite{yolo2016} and more.
They have also led to breakthroughs in remote sensing problems such as SAR image segmentation\cite{colin2021,colin_2022}, altimetry data interpolation \cite{fablet_joint_2021} and even sensor calibration \cite{li_convolutional_2022}.
The problem of multi-scale processing in neural networks has traditionally been tackled through the use of pooling layers in architectures such as the UNet \cite{Ronneberger_Fischer_Brox_2015}. As shown in the reported numerical experiments, these architectures do not reach a state-of-the-art performance for our KaRIn calibration problem. This advocates for the design of neural architectures better accounting for the key features of KaRIn observations.


\section{Data and Case-study}
\label{c3sec:case_study}
In this paper, we run an Observing System Simulation Experiment (OSSE), meaning that we rely on simulated data to apply and evaluate the proposed neural approach.
In this section, we present the different datasets considered in this OSSE.



\subsection{NATL60}
The simulation of the sea surface height field is taken from the NATL60 \cite{ajayi_spatial_2020} run of the NEMO ocean model. This simulation spans one year and covers the North Atlantic basin with a 1/60° spatial resolution. We more specifically use the data from a 12°$\times$12° domain over the Gulfstream ranging from the longitudes -66° to -54° and latitudes 32° to 44°.


\subsection{Nadir observations}
\label{c3subsec:altimetry}


In order to generate realistic nadir-altimeter pseudo-observations, we consider the real orbits of the years 2012 and 2013 of the four missions Topex-Poseidon, Jason 1, Geosat Follow-On, Envisat, as well as the 21-day cycle phase SWOT orbit from the SWOT simulator \cite{ubelmann_swot_nodate} project. The sampling of the nadir-altimeter pseudo-observations relies on the interpolation of the hourly SSH fields of the NATL60 run at the orbit coordinates. We consider nearest-neighbor interpolation in time and a bilinear interpolation in space.


\subsection{KaRIn observations}
The SWOT simulator also generates the swath coordinates on each side of the SWOT nadir. The swath spans from 10km to 60km off nadir with a 2km by 2km resolution. The SSH is then sampled on those coordinates the same way as the nadir observations.
Additionally, we also use the SWOT simulator in its "baseline" configuration to generate observation errors. 
Our simulation includes the systematic instrument errors with the roll, phase, timing and baseline dilation signals. Those signals have time-varying constant, linear or quadratic shape in the across track dimension. 
We also consider the geophysical error with the wet troposphere residual error as implemented in the simulator. 
We refer the reader to \cite{ubelmann_swot_nodate}
for a detailed presentation of the SWOT simulator.

\subsection{Gridded Altimetry Products}
\label{c3subsec:mapping}
\noindent
As explained in section \ref{c3subsec:interpolation}, we make use of interpolated SSH products based on
nadir altimetry data as inputs for our cross-calibration method.
We consider two interpolation schemes in our study:
\begin{itemize}
    \item the operational state-of-the-art based on optimal interpolation as implemented in the DUACS product \cite{taburet_duacs_2019}.
    \item a state-of-the-art neural interpolation scheme, referred to as 4DVarNet \cite{fablet_joint_2021}. This method is based on a trainable adaptation of the 4DVAR \cite{carrassi_data_2018} variational data assimilation method, and out-performs concurrent approaches in the considered OSSE setup \cite{osse_data_challenge}. We consider two 4DVarNet interpolation configurations, one using only nadir altimetry data \cite{Beauchamp_Febvre_Georgenthum_Fablet_2022}, one using jointly nadir altimetry and sea surface temperature data \cite{Fablet_Febvre_Chapron_2022}. We also include the latter as it significantly improves the reconstruction of the SSH at finer scales.
    
\end{itemize}

\section{Proposed Methodology}
\label{c3sec:method}
\noindent


This section presents the proposed methodology for the cross-calibration of raw KaRIn observations.
We design trainable neural architectures that take as inputs the uncalibrated KaRIn observations and the nadir-altimeter-derived gridded SSH products interpolated on the KaRIn swath. We train these architectures in a supervised manner on the reconstruction of the SSH on the KaRIn swath.
We first present an overview of the proposed neural architectures (Section \ref{c3subsec:neural_arch}). We then detail two specific components, namely the  scale-space decomposition block (Section \ref{c3subsec:scale_decomp}) and the swath-mixing layers (Section \ref{c3subsec:mixing}).

\subsection{Proposed neural architecture}
\label{c3subsec:neural_arch}
\noindent
\begin{figure*}
    \begin{center}
	    \includegraphics[width=\textwidth]{00_Calib/CalDiag2.png}
    \end{center}
    \caption{\textbf{Overview of the proposed architecture:} From left to right: The first step interpolates the nadir-based gridded product onto the swath segment. Afterwards, both the nadir-based gridded product and KaRIn observation undergo the scale-space decomposition scheme outlined in \ref{c3subsec:scale_decomp}. The scale components are stacked as channels and processed through the neural network. The blue color of the "Split Conv" indicates that each side of the swath is processed independently by the convolution layer whereas the orange coloring of the "Swath Mix" layer tells that the whole data is processed jointly (more details in \ref{c3subsec:mixing}). The final convolution computes a correction to be added to the gridded product for computing the calibrated KaRIn data}
    \label{c3fig:arch}	
\end{figure*}
The overall architecture considered is shown in figure \ref{c3fig:arch}. The scale-space decomposition block first decomposed independently the input L4 SSH products and KaRIn observations
into $N_s$-scale tensors, which we concatenate as the channel dimension.
This results into a tensor of shape $(2N_s, N_{al}, N_{ac})$ where $N_{al}$ and $N_{ac}$ are respectively the along track and across track sizes of the input swath section. The scale-space decomposition step is described in section \ref{c3subsec:scale_decomp}
A linear 2{\sc d} convolution layer follows. 
The data is then processed by a series of residual convolutional blocks composed of a convolution layer, a ReLU non-linearity \cite{Nair_Hinton_2019}, a skip connection and a swath-mixing layer as described in \ref{c3subsec:mixing}. A last linear convolution layer outputs a residual field, which we sum with the input gridded L4 SSH product to produce the calibrated KaRIn observation.
The interested reader can refer to our implementation\footnote{\url{https://github.com/CIA-Oceanix/4dvarnet-core/releases/tag/tgrs-calcnn-2023}}.



 
\subsection{Scale-space decomposition}
\label{c3subsec:scale_decomp}


We exploit a Gaussian scale-space to compute a scale-space decomposition of the fields provided as inputs to our neural architecture. For given scales $\sigma_1$ and $\sigma_2$, we extract the associated signal as the difference between filtered versions of the input signal using two Gaussian filters with standard deviation $\sigma_1$ and $\sigma_2$. We consider one-dimensional filters for the along-track direction. Formally, denoting 
 $\cal{G_{\sigma}}$ the 1-dimensional Gaussian blur operator with standard deviation $\sigma$ in the along track dimension, the considered scale-space decomposition of a signal $f$ given a sequence of increasing scales $[\sigma_1, \sigma_1, ..., \sigma_S]$ computes the following $S+1$ components: $[\mathcal{G}_{\sigma_1}(f), \mathcal{G}_{\sigma_2}(f) - \mathcal{G}_{\sigma_1}(f),...,\mathcal{G}_{\sigma_S}(f) - \mathcal{G}_{\sigma_{S-1}}(f), f - \mathcal{G}_{\sigma_S}]$
These different components are then considered as channels for the convolutionnal networks. In our experiments, we consider 20 scales in the along-track direction evenly spaced from 8km to 160km. We discuss in section \ref{c3subsec:decomp_sens} how sensitive the proposed method is to the parameterization of the decomposition.
To account for scale-dependent energy levels in the computed scale-space representation (see fig. \ref{c3fig:var_in_out}), we introduce a batch normalization layer \cite{Ioffe_Szegedy_2015}. It re-scales each component to centered and unit-variance variables.
We illustrate in Fig.\ref{c3fig:var_in_out} the impact of the batch normalization step on the relative variance of the signal of each scale of the decomposition.

One may regard the proposed scale-scale decomposition as a convolutional block. Learning such a decomposition from data would however require very large convolutional filters, which does not seem 
realistic, or a deeper architecture with pooling layers that would require very efficient optimization given the quantity of data available. 
\begin{figure}[!t]% 
    \centering
    {\includegraphics[width=\linewidth]{00_Calib/var_rescale_obs} }%
    \caption{\textbf{Explained variance of scale components before and after re-scaling:} Each bar indicates how much each scale component of the uncalibrated KaRIn contributes to the total variance of the signal, we can see that before re-scaling (blue) there is four orders of magnitude between largest scale and the others. The learnt re-scaling allows for scale component to be spread within a single order of magnitude (orange), which is more suited to the downstream neural architecture.}%
    \label{c3fig:var_in_out}%
\end{figure}


\subsection{Swath mixer block}
\label{c3subsec:mixing}

As observed in Figure \ref{c3fig:filtered_swath_uncal_comp}, the swath observed from the KaRIn sensor is not contiguous in the across-track dimension. The observation errors are however clearly correlated between the two sides of the swath. To exploit these correlations in our architecture, we design a swath-mixer block with two specific layers. 

To avoid convolution kernels to mix information from the two sides of the swath which could result in some unwanted side effects, each side is processed separately by each convolution layer noted "Split Conv" in Fig. \ref{c3fig:arch}. Additionally, each convolution layer input is padded so that the height and width of the input remain unchanged throughout the network.

Besides, to combine relevant features from the two sides of the swath, we introduce a layer denoted as "Swath-mix" in Fig. \ref{c3fig:arch}. It implements a convolution layer after transposing the across-track dimension as a channel dimension. This idea of a mixer layer has been used in architectures such as the MLP-Mixer \cite{mlpmixer}, in which it has been shown to help with the expressiveness of neural networks.


We analyse in section \ref{c3subsec:ablation} the contribution of the mixing layer.

\section{Experimental results}
\label{c3sec:results}

\subsection{Setup}
\noindent
The results of this section have been computed using the one year ocean simulation NATL60, over the 12°x12° domain over the Gulfstream. The model evaluation is done on forty days in the inner 10°x10° region. The training of the mapping and calibration models are done on the remaining days.
The experimental setup used is the same as in \cite{osse_data_challenge}
The base configuration for our architecture uses three convolutional blocks with 128 channels as presented in Figure \ref{c3fig:arch}.
The supervised training loss is a weighted mean of the mean square errors for the reconstruction of the SSH, its gradient and its Laplacian.
The default scale-space decomposition used is made of twenty 8 kilometers band.
The calibration model is trained for 250 epochs with a annealing triangular cyclical learning rate \cite{Smith_2017}.

\subsection{Benchmarking experiments}
\label{c3subsec:main_res}
\noindent

\begin{table}[t]
\begin{center}
\begin{tabular}{lrr}
\toprule
 & RMSE (m) & RMSE $|| \nabla_{ssh} ||$ \\
\midrule
CalCNN & 1.39e-02 & 6.46e-03 \\
UNet & 2.34e-02 & 1.07e-02 \\
4DVarNet-5nad & 2.17e-02 & 9.57e-03 \\
\bottomrule
\end{tabular}

\end{center}
\caption{Residual error of the benchmarked calibration frameworks}
\label{c3table:main}
\end{table}

We summarize our benchmarking experiments in Table \ref{c3table:main}.
We compare our approach, referred to as CalCNN, with a standard UNet \cite{Ronneberger_Fischer_Brox_2015} architecture. The latter uses as inputs the gridded altimetry product and the uncalibrated KaRIn observation stacked together as a 2{\sc d} field with 2 channels. We consider the same training configuration for this UNet model as for the CalCNN.
As baseline, we also consider the reconstruction performance for the KaRIn SSH issued from the 4DVarNet method using nadir-altimeter-only data. 
We evaluate all methods according to the following two metrics, the root mean squared error (RMSE) of the SSH field, and the RMSE of the amplitude of the gradients of the SSH field. 
Whereas the UNet fails to produce a better estimate than the nadir-only interpolation baseline, our CalCNN improves the estimation of the SSH and its gradient by over 35\% and brings the residual error below 1.4cm (Table \ref{c3table:main}).

In Figure \ref{c3fig:err_scales}, we further decompose the calibration error of the CalCNN w.r.t. the spatial scale using 1-dimensional Gaussian blurs as introduced in \ref{c3subsec:scale_decomp}. We draw a comparison with the 4DVarNet interpolation baseline and observation errors. The CalCNN reaches a lower error than both KaRIn observations and the interpolation baseline across all scales. At larger scales the error gets closer to the latter as instrument errors dominate the large-scale components of KaRIn observations. 
Interestingly, at scales lower than 10km, we still retrieve some improvement even though the observation error is quite high.
This can be explained by the fact that the high frequency errors on the KaRIn observations is easily separable from the underlying SSH signal.
Between 10-100km, our method successfully exploits the lower observation errors to improve the interpolation baseline.



\begin{figure}[!t]
    \centering
    \includegraphics[width=\linewidth]
    {00_Calib/norm_cumsum_highpass_errors_1.png}%
    \caption{{\bf Observation and reconstruction error for the SSH at different spatial scales:}  The figure shows the relative error w.r.t to the SSH at different along-track scales for the inputs (Uncalibrated KaRIn in orange and nadir based interpolation in blue) and output (calibrated KaRIn in green) of our method. The x axis indicates the standard deviation of the Gaussian blur that was used to remove the high scale components of the different signals. We can see the expected trend of the interpolation error that is concentratedat fine scales. The uncalibrated KaRIn  error on the other hand is lower than the interpolation only in the 10km-100km range. We see the calibrated output of our method achieves lower error across all scales.}
    \label{c3fig:err_scales}%
\end{figure}


\subsection{Ablation Study}
\label{c3subsec:ablation}
\noindent

\begin{table}[t]
\begin{center}
\begin{tabular}{lrr}
\toprule
 & RMSE (m) & RMSE $|| \nabla_{ssh} ||$ \\
xp &  &  \\
\midrule
CalCNN & 1.39e-02 & 6.46e-03 \\
CalCNN w/o skip connection & 2.17e-02 & 9.57e-03 \\
CalCNN w/o gridded product & 1.70e-01 & 2.47e-02 \\
CalCNN w/o scale decomposition & 2.17e-02 & 9.58e-03 \\
CalCNN w/o mixing layer & 1.94e-02 & 9.60e-03 \\
\bottomrule
\end{tabular}

\end{center}
\label{c3table:ablation}
\caption{Ablation results}
\end{table}


In this section we analyse further the contribution of the different components of our neural architecture. 
In Table \ref{c3table:ablation}, we report 
the performance metrics of the considered ablation study with the following models:
a model without skip connections, one without a gridded product as input, one without the scale-space decomposition scheme (Sec. \ref{c3subsec:scale_decomp}) and one without the swath-mixer layers (Sec.\ref{c3subsec:mixing}. Overall, these four models lead to a significantly lower performance. 
The largest impact comes from the ablation of the nadir-altimetry-only gridded product which provides large-scale information about the SSH. It leads to a loss which amount to an order of magnitude in the calibration errors.
Moreover, we can see that without the skip connections or scale decomposition, we fail to improve on the L4 gridded product.
Finally, we can note that we still get a ~10\% reduction of the RMSE w.r.t the L4 product without the mixing layer, however sharing the information between each side of the swath improves this gain three fold.

% \begin{tabular}{llrrrrrrrrr}
\toprule
{}            xp &       rmse &   grad\_rmse &   spat\_res\_mean &  spat\_res\_std & \\
\midrule
     base &    0.013922 &    0.006465 &          44.991787 &     14.161194 &   \\
tgrs\_no\_decomp & 0.0216827 &   0.00958416 &  68.506 &  22.4284 \\
   no\_res &   0.021685 &    0.009566 &          68.684462 &     22.501356 &   \\
   no\_mix &   0.019368 &    0.009605 &          57.610785 &     16.674918 &   \\
   no\_xb &   0.170286 &    0.024656 &         121.735632 &     43.637274 &   \\
\bottomrule
\end{tabular}


In Table \ref{c3table:size}, we show the sensitivity to the size of the network for the same training configuration. We compare the base architecture 3x128 (3 convolution blocks with 128 channels) with a linear operator, as well as a smaller network 1x32 and a bigger one 5x512. The linear version fails to extract geophysical information from the uncalibrated information. This further points out how challenging the considered calibration task is. Interestingly, our architecture leads to a similar performance for different complexity levels. The smaller and larger architectures leads to a slight increase in the residual error but the smaller model shows a slight improvement in the gradient reconstruction and spatial resolution.
Overall, these results support the robustness of the proposed learning-based approaches and the conclusions we raise in section \ref{c3subsec:main_res} are not very sensitive to the hyper-parameters of our network architecture.
\begin{table}[t]
\begin{center}
\begin{tabular}{lrr}
\toprule
 & RMSE (m) & RMSE $|| \nabla_{ssh} ||$ \\
xp &  &  \\
\midrule
128x3 (Ref) & 1.39e-02 & 6.46e-03 \\
Linear & 2.13e-02 & 1.02e-02 \\
32x1 & 1.44e-02 & 6.22e-03 \\
512x5 & 1.49e-02 & 7.19e-03 \\
\bottomrule
\end{tabular}

\end{center}
\caption{Impact of network size}
\label{c3table:size}
\end{table}

\subsection{Gridded product sensitivity} 
\label{c3subsec:gridded_sens}
\noindent

We analyze further how the quality of nadir-altimetry-only gridded product affects the calibration performance.
In Figure \ref{c3fig:gridded_impact}, we display the improvement in the RMSE of the SSH on the swath and of the gradients of the SSH obtained by our CalCNN for the three gridded products introduced in Sec. \ref{c3subsec:interpolation}.

For all three interpolated products, the proposed calibration method improves the reconstruction of the SSH for the KaRIn swath from the joint analysis of the interpolation product and raw KaRIn observations. We report the larger improvement for DUACS product. This relates to the spectral overlap between the SSH information of the uncalibrated KaRIn and SWOT's NADIR. The associated calibration performance remains however significantly worse than that of the two 4DVarNet products, which may relate to the worse interpolation performance of DUACS product \cite{fablet_end2end_2021,osse_data_challenge}. 
When comparing the impact of the two 4DVarNet products, the results are more nuanced. The 4DVarNet-SST product leads to better metrics. The difference of RMSE is greatly reduced after calibraFKation whereas the gap in RMSE of the gradients is conserved.
This could be interpreted as the gain of RMSE we get from using the SST can be obtained from the uncalibrated KaRIn. However some of the gradients we reconstruct through the SST are not easily extracted from the observations.
Overall this shows interesting relations between the redundant information in the uncalibrated KaRIn and the interpolated products.

\begin{figure}
    \begin{center}
        % \input{gridded_impact.pgf}
        \includegraphics{00_Calib/gridded_impact.png}
    \end{center}
    \caption{{\bf Impact of the nadir-based gridded product on the CalCNN output:} The figure shows the RMSE and the RMSE of the $|| \nabla_{ssh} ||$ of the calibrated observation (stars) and their associated nadir-based gridded products (squares). The improvement brought by the CalCNN is illustrated by the arrows. This improvement can be interpreted as the relevant information extracted from the uncalibrated KaRIn by the CalCNN. Note that the biggest relative improvement concerns the DUACS gridded product (blue) which doesn't uses the SWOT's nadir altimeter.}
    \label{c3fig:gridded_impact}
\end{figure}

\subsection{Sensitivity to the scale-space decomposition}
\label{c3subsec:decomp_sens}
\noindent
In Table \ref{c3table:scale_dec}, we display the calibration metrics for different scale-space decompositions. We vary the number of scales considered and the spacing between two consecutive scales. When considering the same scale range from 8km to 160km, we retrieve the best performance with 20 scales. But, even with only 5 scales evenly separated by 32 km, the performance decreases only by 3\%.
By contrast, when considering a scale separation of 8km but varying the number of scales, we note a more significant drop of performance (about 10\% in the residual RMSE). This suggests a greater sensitivity to the span of the scale-space decomposition than to the number and spacing of the components. 
However we still achieve less than 1.6cm residual error for any of the considered variations which is still a competitive calibration outcome.

\begin{table}[!t]
\begin{center}
	\begin{tabular}{llrr}
\toprule
 &  & RMSE (m) & RMSE $|| \nabla_{ssh} ||$ \\
$N_{band}$ & $\delta_{band}$ &  &  \\
\midrule
20 & 8 & 1.39e-02 & 6.46e-03 \\
40 & 4 & 1.44e-02 & 6.64e-03 \\
10 & 16 & 1.48e-02 & 6.75e-03 \\
5 & 32 & 1.41e-02 & 6.65e-03 \\
10 & 8 & 1.56e-02 & 6.81e-03 \\
40 & 8 & 1.54e-02 & 6.88e-03 \\
\bottomrule
\end{tabular}

\end{center}
\caption{Calibration metrics in function of the scale decomposition}
\label{c3table:scale_dec}
\end{table}

\section{Conclusion}
\label{c3sec:conclusion}
\noindent
We have proposed in this chapter a neural calibration approach which combines a scale-space decomposition of KaRIn observations and a convolutional architecture. This approach proves to be robust with a 
residual error below 1.5cm which can be compared with the 2cm residual error of the expected operational approaches performance although demonstrated globally using a different ocean simulation \cite{Dibarboure_Ubelmann_Flamant_Briol_Peral_Bracher_Vergara_Faugere_Soulat_Picot_2022}. While we can reach a satisfactory calibration performance using the operational nadir altimetry mapping product, our experiments highlight the potential benefit of ongoing effort on neural SSH interpolation schemes to further improve the retrieval of finer-scale features from KaRIn observations.
This naturally advocates for future work exploring jointly calibration and mapping problems for nadir and wide-swath altimeters, possibly combining our deep learning approach and variational mapping formulations introduced in \cite{Febvre_Fablet_Sommer_Ubelmann_2022}.
The generalization to real signals of our calibration operator trained on simulated data is obviously a key challenge to be investigated.

\addcontentsline{toc}{section}{Bibliography}

% \begin{thebibliography}{biblio}
% \bibliographystyle{IEEEbib}
% \bibliography{biblio}
% \end{thebibliography}
\begin{thebibliography}{10}

\bibitem{Peral_Esteban-Fernandez_2018}
E.~Peral et~al.,
\newblock ``Swot mission performance and error budget,''
\newblock in {\em IGARSS 2018 - 2018 IEEE International Geoscience and Remote
  Sensing Symposium}, Jul 2018, p. 8625–8628.

\bibitem{ubelmann_swot_nodate}
C.~Ubelmann, et~al.,
\newblock ``{SWOT} {Simulator} documentation,'' .

\bibitem{Dibarboure_Ubelmann_Flamant_Briol_Peral_Bracher_Vergara_Faugere_Soulat_Picot_2022}
G.~Dibarboure, et~al.,
\newblock ``Data-driven calibration algorithm and pre-launch performance
  simulations for the swot mission,''
\newblock {\em Remote Sensing}, vol. 14, no. 2323, pp. 6070, Jan 2022.

\bibitem{Witkin_1984}
A.~Witkin,
\newblock ``Scale-space filtering: A new approach to multi-scale description,''
\newblock in {\em ICASSP ’84. IEEE International Conference on Acoustics,
  Speech, and Signal Processing}, Mar 1984, vol.~9, p. 150–153.

\bibitem{taburet_duacs_2019}
G.~Taburet, et~al.,
\newblock ``{DUACS} {DT2018}: 25 years of reprocessed sea level altimetry
  products,''
\newblock {\em Ocean Science}, vol. 15, no. 5, pp. 1207--1224, 2019.

\bibitem{glorys_rea_2021}
J-M.~Lellouche, et~al.,
\newblock ``The copernicus global 1/12° oceanic and sea ice glorys12
  reanalysis,''
\newblock {\em Frontiers in Earth Science}, vol. 9, pp. 585, 2021.

\bibitem{beauchamp_intercomparison_2020}
M.~Beauchamp, et~al.,
\newblock ``Intercomparison of {Data}-{Driven} and {Learning}-{Based}
  {Interpolations} of {Along}-{Track} {Nadir} and {Wide}-{Swath} {SWOT}
  {Altimetry} {Observations},''
\newblock {\em Remote Sensing}, vol. 12, no. 22, 2020.

\bibitem{fablet_end2end_2021}
R.~Fablet, et~al.,
\newblock ``{END}-{TO}-{END} {PHYSICS}-{INFORMED} {REPRESENTATION} {LEARNING}
  {FOR} {SATELLITE} {OCEAN} {REMOTE} {SENSING} {DATA}: {APPLICATIONS} {TO}
  {SATELLITE} {ALTIMETRY} {AND} {SEA} {SURFACE} {CURRENTS},''
\newblock {\em ISPRS Annals of the Photogrammetry, Remote Sensing and Spatial
  Information Sciences}, vol. V-3-2021, pp. 295--302, 2021.

\bibitem{osse_data_challenge}
M.~Ballarota, et~al.,
\newblock ``ocean-data-challenges/2020a\_ssh\_mapping\_natl60: Material for ssh
  mapping data challenge,'' Sep 2020.

\bibitem{lindeberg1996edge}
T.~Lindeberg,
\newblock ``Edge detection and ridge detection with automatic scale
  selection,''
\newblock in {\em IEEE CVPR}. IEEE, 1996, pp. 465--470.

\bibitem{Lindeberg_2015}
T.~Lindeberg,
\newblock ``Image matching using generalized scale-space interest points,''
\newblock {\em Journal of Mathematical Imaging and Vision}, vol. 52, no. 1, pp.
  3–36, May 2015.

\bibitem{Pintea_Tomen_Goes_Loog_van_Gemert_2021}
S.~L. Pintea, et~al.,
\newblock ``Resolution learning in deep convolutional networks using
  scale-space theory,''
\newblock {\em IEEE Transactions on Image Processing}, vol. 30, pp.
  8342–8353, Jan 2021.

\bibitem{Worrall_Welling_2019}
D.~Worrall et~al.,
\newblock ``Deep scale-spaces: Equivariance over scale,''
\newblock in {\em Advances in Neural Information Processing Systems}. 2019,
  vol.~32, Curran Associates, Inc.

\bibitem{lecun98}
Y.~Lecun, et~al.,
\newblock ``Gradient-based learning applied to document recognition,''
\newblock {\em Proceedings of the IEEE}, vol. 86, no. 11, pp. 2278–2324, Nov
  1998.

\bibitem{resnet2016}
K.~He, et~al.,
\newblock ``Deep residual learning for image recognition,''
\newblock 2016, p. 770–778.

\bibitem{liu_image_2018}
G.~Liu, et~al.,
\newblock ``Image inpainting for irregular holes using partial convolutions,''
\newblock in {\em Proceedings of the {European} {Conference} on {Computer}
  {Vision} ({ECCV})}, 2018, pp. 85--100.

\bibitem{yolo2016}
J.~Redmon, et~al.,
\newblock ``You only look once: Unified, real-time object detection,''
\newblock 2016, p. 779–788.

\bibitem{colin2021}
A.~Colin, et~al.,
\newblock ``Segmentation of sentinel-1 sar images over the ocean, preliminary
  methods and assessments,''
\newblock in {\em 2021 IEEE International Geoscience and Remote Sensing
  Symposium IGARSS}, Jul 2021, p. 4067–4070.

\bibitem{colin_2022}
A.~Colin, et~al.,
\newblock ``Semantic segmentation of metoceanic processes using sar
  observations and deep learning,''
\newblock {\em Remote Sensing}, vol. 14, no. 44, pp. 851, Jan 2022.

\bibitem{fablet_joint_2021}
R.~Fablet, et~al.,
\newblock ``Joint {Interpolation} and {Representation} {Learning} for
  {Irregularly} {Sampled} {Satellite}-{Derived} {Geophysical} {Fields},''
\newblock {\em Frontiers in Applied Mathematics and Statistics}, vol. 7, 2021.

\bibitem{li_convolutional_2022}
X.~Li, et~al.,
\newblock ``A {Convolutional} {Neural} {Network}-{Based} {Relative}
  {Radiometric} {Calibration} {Method},''
\newblock {\em IEEE Transactions on Geoscience and Remote Sensing}, vol. 60,
  pp. 1--11, 2022,
\newblock Conference Name: IEEE Transactions on Geoscience and Remote Sensing.

\bibitem{Ronneberger_Fischer_Brox_2015}
O.~Ronneberger, et~al.,
\newblock ``U-net: Convolutional networks for biomedical image segmentation,''
\newblock , no. arXiv:1505.04597, May 2015,
\newblock arXiv:1505.04597 [cs].

\bibitem{ajayi_spatial_2020}
A.~Ajayi, et~al.,
\newblock ``Spatial and {Temporal} {Variability} of the {North} {Atlantic}
  {Eddy} {Field} {From} {Two} {Kilometric}-{Resolution} {Ocean} {Models},''
\newblock {\em Journal of Geophysical Research: Oceans}, 125, 5.
  e2019JC015827, 2020.

\bibitem{carrassi_data_2018}
A.~Carrassi, et~al.,
\newblock ``Data assimilation in the geosciences: {An} overview of methods,
  issues, and perspectives,''
\newblock {\em Wiley Interdisciplinary Reviews: Climate Change}, vol. 9, no. 5,
  pp. e535, Sept. 2018,
\newblock Publisher: Wiley.

\bibitem{Beauchamp_Febvre_Georgenthum_Fablet_2022}
M.~Beauchamp, et~al.,
\newblock ``4dvarnet-ssh: end-to-end learning of variational interpolation
  schemes for nadir and wide-swath satellite altimetry,''
\newblock {\em Geoscientific Model Development Discussions}, vol. 2022, pp.
  1–37, 2022.

\bibitem{Fablet_Febvre_Chapron_2022}
R.~Fablet, et~al.,
\newblock ``Multimodal 4dvarnets for the reconstruction of sea surface dynamics
  from sst-ssh synergies,''
\newblock , arXiv:2207.01372, Jul 2022.

\bibitem{Nair_Hinton_2019}
V.~Nair et~al.,
\newblock ``Rectified linear units improve restricted boltzmann machines,''
\newblock Jul 2019.

\bibitem{Ioffe_Szegedy_2015}
S.~Ioffe et~al.,
\newblock ``Batch normalization: Accelerating deep network training by reducing
  internal covariate shift,''
\newblock in {\em Proceedings of the 32nd International Conference on Machine
  Learning}. Jun 2015, p. 448–456, PMLR.

\bibitem{mlpmixer}
I.~O. Tolstikhin, et~al.,
\newblock ``Mlp-mixer: An all-mlp architecture for vision,''
\newblock in {\em Advances in Neural Information Processing Systems}. 2021,
  vol.~34, p. 24261–24272, Curran Associates, Inc.

\bibitem{Smith_2017}
L.~N. Smith,
\newblock ``Cyclical learning rates for training neural networks,''
\newblock in {\em 2017 IEEE Winter Conference on Applications of Computer
  Vision (WACV)}, Mar 2017, p. 464–472.

\bibitem{Febvre_Fablet_Sommer_Ubelmann_2022}
Q.~Febvre, et~al.,
\newblock ``Joint calibration and mapping of satellite altimetry data using
  trainable variational models,''
\newblock in {\em IEEE ICASSP 2022 - 2022}, May 2022, p. 1536–1540.

\end{thebibliography}


\end{bibunit}
%\begin{IEEEbiographynophoto}{Quentin Febvre}
%Quentin Febvre is a 2nd year PhD student at IMT Atlantique Brest
%\end{IEEEbiographynophoto}

% \begin{IEEEbiography}[{\includegraphics[width=1in,height=1.25in,clip,keepaspectratio]{photo_qfebvre.jpg}}]{Quentin Febvre}
% recived the graduate degree from the École Supérieure d'électricité, France and  Master of Science degree in Computer Science from Tsinghua University, China in 2015. He then worked as a data scientist for an international advertising company in 2016 before joining Theodo Group in Paris where he worked on web development, data engineering and computer vision applications using data driven approaches. In 2020, he started as a PhD student in the Mathematical and Electrical engineering department at IMT Atlantique Bretagne-Pays de la Loire. His research focuses on the benefits deep learning approaches can bring to ocean observation systems, and more specifically on altimetry data analysis.
% %\begin{IEEEbiographynophoto}{Ronan Fablet}

% \end{IEEEbiography}

% \end{document}




\clearemptydoublepage
% %%%%%%%%%%%%%%%%%%%%%%%%%%%%%%%%%%%%%%%%%%%%%%%%%%%%%%%%%%%%%%%%%%%%%%%%%%%%
% % AGUJournalTemplate.tex: this template file is for articles formatted with LaTeX
% %
% % This file includes commands and instructions
% % given in the order necessary to produce a final output that will
% % satisfy AGU requirements, including customized APA reference formatting.
% %
% % You may copy this file and give it your
% % article name, and enter your text.
% %
% % guidelines and troubleshooting are here: 

% %% To submit your paper:
% \documentclass[draft]{00_Simulearning/agujournal2019}
% \usepackage{url} %this package should fix any errors with URLs in refs.
% \usepackage{lineno}
% \usepackage{booktabs}
% \usepackage{graphicx}
% \usepackage{subcaption}
% \usepackage{float}

% % \usepackage[inline]{trackchanges} %for better track changes. finalnew option will compile document with changes incorporated.
% \usepackage{soul}
% % \linenumbers
% %%%%%%%
% % As of 2018 we recommend use of the TrackChanges package to mark revisions.
% % The trackchanges package adds five new LaTeX commands:
% %
% %  \note[editor]{The note}
% %  \annote[editor]{Text to annotate}{The note}
% %  \add[editor]{Text to add}
% %  \remove[editor]{Text to remove}
% %  \change[editor]{Text to remove}{Text to add}
% %
% % complete documentation is here: http://trackchanges.sourceforge.net/
% %%%%%%%

% \draftfalse

% %% Enter journal name below.
% %% Choose from this list of Journals:
% %
% % JGR: Atmospheres
% % JGR: Biogeosciences
% % JGR: Earth Surface
% % JGR: Oceans
% % JGR: Planets
% % JGR: Solid Earth
% % JGR: Space Physics
% % Global Biogeochemical Cycles
% % Geophysical Research Letters
% % Paleoceanography and Paleoclimatology
% % Radio Science
% % Reviews of Geophysics
% % Tectonics
% % Space Weather
% % Water Resources Research
% % Geochemistry, Geophysics, Geosystems
% % Journal of Advances in Modeling Earth Systems (JAMES)
% % Earth's Future
% % Earth and Space Science
% % Geohealth
% %
% % ie, \journalname{Water Resources Research}

% \journalname{Journal of Advances in Modeling Earth Systems (JAMES)}


% \begin{document}

%%%%%%%%%%%%%%%%%%%%%%%%%%%%%%%%%%%%%%%%%%%%%%%
%  TITLE
%
% (A title should be specific, informative, and brief. Use
% abbreviations only if they are defined in the abstract. Titles that
% start with general keywords then specific terms are optimized in
% searches)
%
%%%%%%%%%%%%%%%%%%%%%%%%%%%%%%%%%%%%%%%%%%%%%%%

% Example: \title{This is a test title}

% \title{Training neural mapping schemes for satellite altimetry with simulation data}
% % \title{OSSE-trained neural mapping of real satellite altimetry data}
% % \title{Enhancing Satellite Altimetry Mapping through Neural Architectures Trained with Ocean Model Data}
% \authors{Q. Febvre\affil{1,4}, J. Le Sommer\affil{2}, C. Ubelmann\affil{3}, R. Fablet\affil{1,4}}

% \affiliation{1}{IMT Atlantique, Lab-STICC, Brest, France}
% \affiliation{2}{Université Grenoble Alpes, CNRS, IRD, Grenoble, France}
% \affiliation{3}{Datlas, Grenoble, France}
% \affiliation{4}{INRIA Team ODYSSEY}
% \correspondingauthor{Quentin Febvre}{quentin.febvre@imt-atlantique.fr}
\begin{bibunit}[IEEEtran.bst]

\chapter*{Training neural mapping schemes for satellite altimetry with simulation data}
\addcontentsline{toc}{chapter}{Training neural mapping schemes for satellite altimetry with simulation data}
\chaptermark{Training neural mapping schemes for satellite altimetry with simulation data}



% \begin{keypoints}

% \item We propose to train neural mapping schemes for real altimeter data from ocean simulation data.
% \item  The trained neural schemes significantly outperform the operational mapping of real altimetry data for a Gulf Stream case-study.
% \item More realistic simulation datasets improve the performance of the trained neural mapping with a 20\% improvement in the spatial  scales.
% \end{keypoints}


% \begin{abstract}
%     Satellite altimetry combined with data assimilation and optimal interpolation schemes have deeply renewed our ability to monitor sea surface dynamics. Recently, deep learning (DL) schemes have emerged as appealing solutions to address space-time interpolation problems. The scarcity of real altimetry dataset, in terms of space-time coverage of the sea surface, however impedes the training of state-of-the-art neural schemes on real-world case-studies. Here, we leverage both simulations of ocean dynamics and satellite altimeters to train simulation-based neural mapping schemes for the sea surface height and demonstrate their performance for real altimetry datasets. We analyze further how the ocean simulation dataset used during the training phase impacts this performance. This experimental analysis covers both the resolution from eddy-present configurations to eddy-rich ones, forced simulations vs. reanalyses using data assimilation and tide-free vs. tide-resolving simulations. Our benchmarking framework focuses on a Gulf Stream region for a realistic 5-altimeter constellation using NEMO ocean simulations and 4DVarNet mapping schemes. All simulation-based 4DVarNets outperform the operational observation-driven and reanalysis products, namely DUACS and GLORYS. The more realistic the ocean simulation dataset used during the training phase, the better the mapping. The best 4DVarNet mapping was trained from an eddy-rich and tide-free simulation datasets. It improves the resolved longitudinal scale from 151 kilometers for DUACS and 241 kilometers for GLORYS to 98 kilometers and reduces the root mean squared error (RMSE) by 23\% and 61\%. These results open research avenues for new synergies between ocean modelling and ocean observation using learning-based approaches.
  

% \end{abstract}

% \section*{Plain Language Summary}

% % https://www.agu.org/Share-and-Advocate/Share/Community/Plain-language-summary
% For an artificial intelligence (AI) to learn, one need to describe a task using data and an evaluation procedure. 
% Here we aim at constructing images related to the ocean surface currents. The satellite data we use provide images of the ocean surface with a lot of missing data (around 95\% of missing pixels for a given day), and we aim at finding the values of the missing pixels.
% Because we don't know the full image, it is challenging to train an AI on this task using only the satellite data.
% However, today's physical knowledge makes it possible to numerically simulate oceans on big computers. For these simulated oceans, we have access to the gap-free image, so we can train AI models by first hiding some pixels and checking if the model fill the gaps with the correct values.
% Here, we explore under which conditions AIs trained on simulated oceans are useful for the real ocean.
% We show that today's simulated oceans work well for training an AI on this task and that training on more realistic simulated oceans improve the performance of the AI!


\section{Introduction}



  Satellite altimeters have brought a great leap forward in the observation of sea surface height on a global scale since the 80's. 

  
  Altimetry data have greatly contributed to the monitoring and understanding of key processes such as the sea-level rise and the role of mesoscale dynamics.
    The scarce and irregular sampling of the measurements presents a challenge for training deep neural networks.
  The retrieval of mesoscale-to-submesoscale sea surface dynamics for horizontal scales smaller than 150 km however remains a challenge for operational systems based on optimal interpolation \cite{taburetDUACSDT2018252019} and data assimilation \cite{jean-michelCopernicusGlobal122021} schemes. This has motivated a wealth of research to develop novel mapping schemes \cite{ballarottaDynamicMappingAlongTrack2020,ubelmannReconstructingOceanSurface2021,guillouMappingAltimetryForthcoming2021}.

  In this context, data-driven and learning-based approaches \cite{alveraazcarateReconstructionIncompleteOceanographic2005,barthDINCAEMultivariateConvolutional2022,lguensatAnalogDataAssimilation2017,fabletENDTOENDPHYSICSINFORMEDREPRESENTATION2021,martinSynthesizingSeaSurface2023} appear as appealing alternatives to make the most of the available observation and simulation datasets. Especially, Observing System Simulation Experiments (OSSE) have stressed the potential of neural schemes trained through supervised learning for the mapping of satellite-derived altimetry data \cite{fabletENDTOENDPHYSICSINFORMEDREPRESENTATION2021,beauchamp4DVarNetSSHEndtoendLearning2023}. 
  Their applicability to real datasets has yet to be assessed and recent studies have rather explored learning strategies from real gappy multi-year altimetry datasets \cite{martinSynthesizingSeaSurface2023}. Despite promising results, schemes trained with unsupervised strategies do not reach the relative improvement of the operational processing suggested by OSSE-based studies.
  
Here, we go beyond using OSSEs as benchmarking-only testbeds. We explore their use for the training of neural mapping schemes and address the space-time interpolation of real satellite altimetry observations. Through numerical experiments on a Gulf Stream case-study with a 5-nadir altimeter constellation, our main contributions are three-fold. We demonstrate the relevance of the simulation-based learning of neural mapping schemes and their generalization performance for real nadir altimetry data. We benchmark the proposed approach with state-of-the-art operational products as well as neural schemes trained from real altimetry datasets. We also assess how the characteristics of the training datasets, especially in terms of resolved ocean processes, drives the mapping performance.
To ensure the reproducibility of our results, our code is made available through an open source license along with the considered datasets and the trained models \cite{febvreCodeDataRelease2023a}.

%benchmark neural schemes trained with the proposed OSSE-based strategy using different simulation datasets and we assess the impact of the characteristics of these datasets on the mapping performance.
%    Demonstrate numerically that today's numerical simulations of the ocean are a valuable resource for training machine learning models to be applied on real data, with added benefit of more realistic synthetic data}


    
The content of this paper is organized as follows. Section \ref{sec:background} offers background information on related work, Section \ref{sec:method} presents our method, Section \ref{sec:results} reports our numerical experiments, and Section \ref{sec:discussion} elaborates on our main contributions.




\section{Background}
\label{sec:background}
\subsection{Gridded satellite altimetry products}
\label{ssec:interpolation}
The ability to produce gridded maps from scattered along-track nadir altimeter measurements of sea surface height is key to the exploitation of altimeter data in operational services and science studies \cite{abdallaAltimetryFutureBuilding2021}.
As detailed below, we can distinguish three categories of approaches to produce such maps: reanalysis products \cite{jean-michelCopernicusGlobal122021} using data assimilation schemes, observation-based products \cite{taburetDUACSDT2018252019} and learning-based approaches \cite{fabletENDTOENDPHYSICSINFORMEDREPRESENTATION2021}.

Reanalysis products such as the GLORYS12 reanalysis \cite{jean-michelCopernicusGlobal122021} leverage the full expressiveness of state-of-the-art ocean models. They aim at retrieving ocean state trajectories close to observed quantities through data assimilation methods including among others Kalman filters and variational schemes \cite{carrassiDataAssimilationGeosciences2018}. Such reanalyses usually exploit satellite-derived and in situ data sources. For instance, GLORYS12 reanalysis assimilates satellite altimetry data, but also satellite-derived observations of the sea surface temperature, sea-ice concentration as well as in situ ARGO data  \cite{wongArgoData19992020}.


The second category involves observation-based products. In contrast to reanalyses, they only rely on altimetry data and address a space-time interpolation problem. They usually rely on simplifying assumptions on sea surface dynamics. In this category, optimal-interpolation-based product DUACS (Data Unification and Altimeter Combination System) \cite{taburetDUACSDT2018252019} exploits a covariance-based prior, while recent studies involve quasi-geostrophic dynamics to guide the interpolation scheme \cite{guillouMappingAltimetryForthcoming2021,ballarottaDynamicMappingAlongTrack2020}.

Data-driven and learning-based approaches form a third category of SSH mapping schemes. 
Similarly to observation-based methods, they are framed as interpolation schemes.
Especially deep learning schemes have gained some attention. Recent studies have explored different neural architectures both for real and OSSE altimetry datasets \cite{archambaultMultimodalUnsupervisedSpatioTemporal2023,beauchampDatadrivenLearningbasedInterpolations2021,martinSynthesizingSeaSurface2023}. These studies investigate both different training strategies as well as different neural architectures from off-the-shelf computer vision ones such as convolutional LSTMs and UNets \cite{ronnebergerUNetConvolutionalNetworks2015} to data-assimilation-inspired ones \cite{beauchampDatadrivenLearningbasedInterpolations2021, fabletLearningVariationalData2021}.



\subsection{Ocean Modeling and OSSE}
\label{ssec:oceanmodeling}
Advances in modeling and simulating ocean physics have largely contributed to a better understanding of the processes involved in the earth system and to the development of operational oceanography \cite{bernardImpactPartialSteps2006,ajayiSpatialTemporalVariability2020}. 
High-resolution simulations used in Observing System Simulation Experiments (OSSE) also provide a great test-bed for the design and evaluation of new of ocean observation systems \cite{benkiranAssessingImpactAssimilation2021}.
The availability of numerical model outputs enables the computation of interpretable metrics directly on the quantities of interest. This avoids challenges met when working solely with observation data that may be incomplete, noisy or indirectly related to the desired quantity.
For example, in the case of the recently launched SWOT mission, OSSEs combined ocean and instrument simulations to address calibration issues and interpolation performance for SWOT altimetry data  \cite{dibarboureDataDrivenCalibrationAlgorithm2022}.
Such OSSEs have also promoted novel developments for the interpolation of satellite altimetry such as the BFN-QG and 4DVarNet schemes \cite{guillouMappingAltimetryForthcoming2021,beauchamp4DVarNetSSHEndtoendLearning2023}. 

In OSSE settings, we can train learning-based mapping schemes in a supervised manner using 
model outputs as the "ground truth" during the training phase.
Nonetheless, these training methods cannot be straightforwardly applied to Observing System Experiments (OSEs) due to a lack of comprehensive groundtruthed observation datasets.
Applied machine learning practitioners often grapple with insufficient amount of labelled data during the training of supervised learning schemes, as the collection of large annotated datasets for a specific task can be costly or unattainable.
Proposed solutions includes the exploitation of large existing datasets (such as ImageNet \cite{dengImageNetLargescaleHierarchical2009}) to train general purpose models \cite{heDeepResidualLearning2016}. Another approach involves the generation of synthetic datasets to facilitate the creation of groundtruthed samples \cite{gomezgonzalezVIPVortexImage2017,dosovitskiyFlowNetLearningOptical2015}. OSSEs, which combine ocean model outputs and observing system simulators \cite{boukabaraCommunityGlobalObserving2018}, can deliver such large synthetic groundtruthed datasets. We propose to investigate how OSSE-based training strategies apply to the analysis of real satellite altimetry datasets. Recent results of SSH super-resolution model trained on simulation datasets and evaluated on real ones \cite{buongiornonardelliSuperResolvingOceanDynamics2022} support the relevance of such strategies.

%We propose that ocean model output can be considered as a large synthetic annotated dataset for a variety of task. Recent work have used simulation data to train super-resolution model of the SSH \cite{buongiornonardelliSuperResolvingOceanDynamics2022} later evaluated on observation data. Our work aims to further investigate the potential of such training schemes. 



\subsection{Physics-aware deep-learning}
\label{ssec:deeplearning}
In the last decades, DL advances combined with the rise in computational resources and amount of data have shown the power of extracting knowledge from data in domains ranging from computer vision to language processing \cite{lecunDeepLearning2015}. 
Yet, despite to the universality of DL architectures \cite{hornikMultilayerFeedforwardNetworks1989}, a central challenge persists in learning from data: the generalization performance beyond the distribution of the training data. To tackle this problem, the literature includes a variety of strategies such as data augmentation \cite{shortenSurveyImageData2019} and regularization techniques, including dropout layers \cite{srivastavaDropoutSimpleWay2014} and weight decay schemes \cite{krizhevskyImageNetClassificationDeep2012}. This is of critical importance for physical systems, where models trained on past data will be challenged when the system evolves and reaches dynamics absent from the training data. We can see evidence of this shortcoming in the instability challenges faced by neural closures for climate models \cite{brenowitzInterpretingStabilizingMachineLearning2020}. 

There have been a variety of approaches to harness physical priors within learning schemes to address this issue. Some injects trainable components in classical integration schemes of physical models \cite{yinAugmentingPhysicalModels2021b}, others leverage physical priors within their learning setups which can been used in the training objective \cite{raissiPhysicsinformedNeuralNetworks2019,greydanusHamiltonianNeuralNetworks2019}, as well as in the architecture \cite{li2020fourier,Wang2020TF}. 
However most of these works have focused on relatively simple physical models and it remains challenging to combine current state-of-the-art ocean models with such methods. Obstacles include the complexity and cost of running the physical models, the differences in programming tools and the computing infrastructures used in each domain, as well as the availability of automatic differentiation tools for state-of-the-art ocean models.

The proposed simulation-based training strategy offers another way to benefit from the advances in high-resolution ocean modeling in the design of deep neural models for ocean reanalysis problems. 


\section{Method}
\label{sec:method}


\subsection{Overview}
\label{ssec:overview}

We designate our approach as "simulation-based", it consists in leveraging ocean models and simulations of observing systems to design supervised training environments.
In this section, we describe the proposed method for assessing the potential of simulation-based neural schemes for the mapping real altimetry tracks. We describe the architecture considered in our study, as well as the different datasets used for training purposes. We also detail our simulation-based training setup and the proposed evaluation framework on real altimetry.  

\begin{figure}[ht]
    \centering
    \includegraphics[width=\textwidth]{./00_Simulearning/standalone_figures/schema_method.png}
    \caption{\textbf{Overview of the experimental setup}. On the left side we display the simulation-based training strategy based on an ocean simulation which will be used for 1) generating synthetic observation and 2) computing the training objective of the neural mapping scheme. On the right side we show the evaluation principle of splitting the available satellite observations to evaluate the method on data that were not used for the inference.}
    \label{fig:method}
\end{figure}


\subsection{Neural mapping scheme}
\label{ssec:4DVarNet}

The neural mapping scheme considered for this study is the 4DVarNet framework\cite{fabletENDTOENDPHYSICSINFORMEDREPRESENTATION2021}. 
We choose this scheme due to the performance shown in the OSSE setup.
Previous studies \cite{beauchamp4DVarNetSSHEndtoendLearning2023} report significant better performance than the DUACS product \cite{taburetDUACSDT2018252019} in the targeted Gulf stream region. %It therefore constitute a good candidate for transfer learning.
4DVarNet relies on a variational data assimilation formulation.
The reconstruction results from the minimization of a variational cost. This cost encapsulates a data fidelity term and a regularization term.
It exploits a prior on the space-time dynamics through a convolutional neural network inspired from \cite{fabletBilinearResidualNeural2018}, and an iterative gradient-based minimization based on a recurrent neural network as introduced for meta-learning purposes \cite{andrychowiczLearningLearnGradient}. The overall architecture and components are similar to those presented in existing work \cite{beauchamp4DVarNetSSHEndtoendLearning2023}. We adapt some implementation details based on cross-validation experiments to improve the performance and reduce the training time. We refer the reader to the code for more details \cite{febvreCodeDataRelease2023a}.



\subsection{SSH Data}
\label{ssec:data}

\begin{table}[h]

\begin{tabular}{ll||cccc}
\toprule
{} & {}& Resolution & Reanalysis & Tide & DAC  \\
\midrule
NATL60 &\cite{ajayiSpatialTemporalVariability2020}               &      1/60$^\circ$ &               No &            No &                   No  \\
eNATL60-t &\cite{brodeauOceannextENATL60Material2020}         &      1/60$^\circ$ &               No &           Yes &                  Yes  \\
eNATL60-0 &\cite{brodeauOceannextENATL60Material2020}         &      1/60$^\circ$ &               No &            No &                  Yes  \\
GLORYS12-r &\cite{jean-michelCopernicusGlobal122021} &      1/12$^\circ$ &              Yes &            No &                   No  \\
GLORYS12-f &\cite{jean-michelCopernicusGlobal122021}   &      1/12$^\circ$ &               No &            No &                   No  \\
ORCA025& \cite{bernardImpactPartialSteps2006}             &       1/4$^\circ$ &               No &            No &                   No  \\
\bottomrule
\end{tabular}
\caption{\textbf{Summary table of the different synthetic SSH fields used for training}. The last column indicate whether the Dynamic Atmospheric Correction was applied on the synthetic SSH. It justify the presence of both eNATL60-0 and NATL60 to isolate the impacts of resolution and tide.}
\label{tab:data}
\end{table}

We use numerical simulations of ocean general circulation models (OGCM) to build our reference SSH datasets. Such simulations involve a multitude of decisions that affect the resulting simulated SSH. Here we consider NEMO (Nucleus for European Modelling of the Ocean) \cite{gurvanNEMOOceanEngine2022} which is among the state-of-the art OGCM in operational oceanography \cite{ajayiSpatialTemporalVariability2020} as well as in climate studies \cite{voldoireCNRMCM5GlobalClimate2013}. The selected SSH datasets reported in Table \ref{tab:data} focus on three main aspects: the added-value of high-resolution eddy-rich simulations, the impact of reanalysis datasets and the relevance of tide-resolving simulations.

% /ocean model with different configurations. We selected six distinct datasets to examine the influence of three factors on the dynamical structures manifested in the simulations.

In order to evaluate the impact of eddy-rich simulations, we consider NATL60, GLORYS12-f and ORCA025 free runs, respectively with a horizontal grid resolution of 1/60$^\circ$, 1/12$^\circ$, and 1/4$^\circ$. Finer grids allow for more processes to be simulated. We therefore expect higher-resolution simulations to exhibit structures closer to the real ocean and the associated trained deep learning model to perform better.
Regarding the impact of reanalysis data, we compare numerical experiments with the GLORYS12-r reanalysis and the associated free run GLORYS12-f. This reanalysis dataset relies on the assimilation of temperature, sea level and sea ice concentration observations. 
%Another way to nudge ocean simulations to more realistic dynamics is the assimilation of observation data. To evaluate this aspect we compare the GLORYS12-r reanalysis and its associated the free run GLORYS12-f. Measurements of temperature, sea level, sea ice concentration and salinity have been assimilated during the reanalysis. This comparison will also give indication on the impact of using true observation during the training.
Besides, the recent eNATL60 twin simulations eNATL60-t and eNATL60-0 allow us to evaluate the impact of tide-resolving simulations.
We summarize in Table \ref{tab:data} the characteristics of the different datasets. 

\subsection{OSSE-based training setup}
\label{ssec:training}
We sketch the proposed OSSE-based training setup on the left side of the Figure \ref{fig:method}.
In order to fairly evaluate the datasets' quality as a training resource, we standardize the training procedure.
We regrid all simulations to the same resolution (1/20°) and we use daily-averaged SSH fields as training targets. We generate noise-free pseudo-observations by sampling values of the daily-averaged fields corresponding to realistic orbits of a 5 altimeter-constellation. We train all models from a one-year dataset in a Gulfstream domain from (66°W, 32°N) to (54°W, 44°N) in which we keep the same two months for validation. The hyper-parameters of the model and training procedure such as the number of epoch, learning rate scheduler are the same for all the experiments. The detailed configuration can be found by the reader in the available implementation. As training objective, we combine the mean square errors for the SSH fields and the amplitude of the gradients as well as a regularization loss for the prior model.

\subsection{OSE-based evaluation setup}
\label{ssec:eval}
As sketched on the right side of the Figure \ref{fig:method},
the evaluation setup relies on real altimetry data from the constellation of 6 satellites from 2017 (SARAL/Altika, Jason 2, Jason 3, Sentinel 3A, Haiyang-2A and Cryosat-2 ).
We apply the standardized setup presented in a data-challenge \url{https://github.com/ocean-data-challenges/2021a_SSH_mapping_OSE}.
We use the data from the first five satellites as inputs for the mapping and the last one (Cryosat-2) for computing the performance metrics. We compute these metrics in the along-track geometry. % on which the gridded product is interpolated on the altimetry tracks. 
The evaluation domain spans from (65°W, 33°N) to (55°W, 43°N)  and the evaluation period from January 1$^{st}$ to December 31$^{st}$ 2017.  Given $\eta_{c2}$ and $\hat{\eta}$ the measured SSH and the reconstructed SSH respectively, we compute the following two metrics:
\begin{itemize}
    \item $\mu_{ssh}$ is a score based on the normalized root mean squared (nRMSE) error  computed as $1 - \displaystyle\frac{RMS(\hat{\eta} - \eta_{c2})}{RMS(\eta_{c2})}$
    \item $\lambda_x$ is the wavelength at which the power spectrum density (PSD) score  $1 - \displaystyle \frac{PSD(\hat{\eta} - \eta_{c2})}{PSD(\eta_{c2})}$ crosses the $0.5$ threshold, which characterize the scales resolved by the reconstruction (the error below that wavelength makes up for more than half of the total signal)
\end{itemize}

In Table \ref{tab:res}, we also consider the root mean square error (RMSE) as well as the nRMSE score of the sea level anomaly $\mu_{sla}$ obtained by subtracting the mean dynamic topography to the SSH. Lastly, we assess the performance degradation resulting from the transition from simulated to real data by quantifying the improvement relative to DUACS in the resolved scale $\lambda_x$ on our OSE setup as well as on
the OSSE benchmarking setup proposed in related studies \cite{guillouMappingAltimetryForthcoming2021}. This benchmarking setup relies on NATL60-CJM165 OSSE dataset. We refer the reader to \url{https://github.com/ocean-data-challenges/2020a_SSH_mapping\_NATL60} for a detailed description of this experimental setup.




\begin{figure}[H]
\small

\begin{center}

\includegraphics[width=\linewidth]{./00_Simulearning/standalone_figures/maps.pdf}

% \caption{Row I - Isotrophic PSD. Row 2 - Isotrophic PSD Score}
\caption{\textbf{Samples 
Kinetic energy and relative vorticity of the training and reconstruction data of January 6$^th$ }.
The reconstructed year is 2017 while the training year vary depending on the simulation.
The first two columns (a) and (b) show the training data while columns (c) and (d) show the associated 4DVarNet reconstruction.
The kinetic energy is displayed in columns ((a) and (c)) and the relative vorticity normalized by the local Coriolis parameter in columns ((b) and (d)).
Each row shows the experiment using respectively: ORCA025 (I), GLORYS12-f (II), GLORYS12-r (III), NATL60 (IV), eNATL60-t (V) and eNATL60-0 (VI)}
\vspace{-5mm}
\label{fig:maps}
\end{center}
\end{figure}



\section{Results}
\label{sec:results}

This section details our numerical experiments for the considered real altimetry case-study for a Gulf Stream region as described in Section \ref{ssec:eval}. We first report the benchmarking experiments to assess the performance of the proposed learning-based strategy with respect to (w.r.t.) state-of-the-art mapping schemes. We then analyse how the characteristics of the training datasets drive the mapping performance. 

\subsection{Benchmarking against the state of the art}
\label{ssec:benchmarks}

We report in Table \ref{tab:bench} the performance metrics of state-of-the-art approaches including both operational observation products \cite{taburetDUACSDT2018252019,ubelmannReconstructingOceanSurface2021}, deep-learning-based schemes trained on observation data \cite{archambaultMultimodalUnsupervisedSpatioTemporal2023,martinSynthesizingSeaSurface2023} as well as methods using explicitly a model-based prior on sea surface dynamics \cite{guillouMappingAltimetryForthcoming2021,ballarottaDynamicMappingAlongTrack2020,jean-michelCopernicusGlobal122021}. We compare those methods with a 4DVarNet trained on eNATL60-0 OSSE dataset. The latter outperforms all other methods on the two metrics considered (22\% improvement in RMSE w.r.t. the DUACS product and 33\% improvement in the resolved scale). We report a significantly worse performance for GLORYS12 reanalysis. This illustrates the challenge of combining large ocean general circulation models and observation data for the mapping of the SSH.


The last column indicates that the 4DVarNet scheme leads to the best mapping scores for both the OSE and OSSE setups. For the latter, the reported improvement of 47\% is twice greater than the second best at 22\%. The performance of the 4DVarNet drops by 11\% when considering the former. By contrast, other methods do not show such differences between the OSE and OSSE case-studies. This suggests that the finer-scale structures that are well reconstructed in the OSSE setup are not as beneficial in the OSE setup. While one could question the representativeness of the OSSE datasets for the fine-scale patterns in the true ocean, real nadir altimetry data may also involve multiple processes which could impede the reconstruction and evaluation of horizontal scales below 100km.  



\begin{table}[h]
%\centering
\hspace{-10mm}\begin{tabular}{l||llll|rrrc}
\toprule
 & SSH  & Deep  & Calibrated on  & Physical  & rmse & $\mu_{ssh}$  & $\lambda_x$ & $1 - \frac{\lambda_x}{\lambda_{ref}}$ \\
 &  Only &  Learning &  data from &  Model &  (cm) &  () &  (km) & (\% ose, osse) \\
\midrule
(a) \textbf{4DVarNet} &  Yes & Yes & Simulation  & -- & \textbf{5.9}  & \textbf{0.91}  & \textbf{100} & \textbf{33}, \textbf{47} \\
(b) MUSTI & No &  Yes & Satellite  & -- & 6.3  & 0.90  & 112 & 26, 22 \\
(c) ConvLstm-SST & No &  Yes & Satellite  & -- & 6.7  & 0.90  & 108 & 28, -- \\
(d) ConvLstm &  Yes &  Yes & Satellite  & -- & 7.2  & 0.89  & 113 & 25, -- \\
(e) DYMOST&  Yes & No & Satellite  & QG & 6.7  & 0.90  & 131 & 13, 11 \\
(f) MIOST &  Yes & No & Satellite  & -- & 6.8  & 0.90  & 135 & 11, 10 \\
(g) BFN-QG &  Yes & No & Satellite  & QG & 7.6  & 0.89  & 122 & 19, 21 \\
(h) DUACS &  Yes & No & Satellite  & -- & 7.7  & 0.88  & 151 &  ~0,  0 \\
(i) GLORYS12 & No & No & Satellite  & NEMO & 15.1  & 0.77  & 241 & -60, -- \\
\bottomrule
\end{tabular}
\caption{ \textbf{SSH reconstruction performance of the benchmarked methods} (a) 4DVarNet from this study trained on eNATL60-0 (b) Archambault et al. (2023), (c and d)
ConvLstm-SST and ConvLstm from Martin et al. (2023), (e) DYMOST from Ballarotta
et al. (2020), (f) MIOST from Ubelmann et al. (2021), (g) BFN-QG from Guillou et
al. (2021), (h) DUACS from Taburet et al. (2019), (i) GLORYS12 from Lellouche et al.
(2021. The columns indicate from left to right: whether athe mapping schemes rely only on SSH data or also exploit additional data such as gap free SST products; if the method uses deep learning architectures; the data used to calibrate (or train) the mapping scheme; the numerical model of the ocean used for the mapping if any (QG stands for quasi-geostrophic); $\mu$ and $\lambda_x$ are the metrics as described in Section \ref{ssec:eval}}
\label{tab:bench}
\end{table}



\begin{figure}[H]
\small
\begin{center}
\includegraphics[width=\linewidth]{./00_Simulearning/standalone_figures/st_psd.png}
\end{center}

\caption{
\textbf{Space-time spectral densities of the training datasets (first row) and of their associated reconstruction (second row)}. Darker blue in the lower left corner indicates higher energy at larger wavelength and periods. The different SSH fields exhibit different energy cascades when moving to  finer temporal (upward) or spatial (rightward) scales.} \vspace{-5mm}
\label{fig:spacetime_psd}
\end{figure}






\begin{table}[H]
\centering
\begin{tabular}{l||rrrrc}
\toprule
Training Data & RMSE  & $\mu_{ssh}$  & $\mu_{sla}$ & $\lambda_x$ & $1 - \frac{\lambda_x}{\lambda_{ref}}$ \\
 &  (cm) &   &   &  (km) & (\% ose, osse) \\
\midrule
NATL60 & \textbf{5.9}  & \textbf{0.91}  & \textbf{0.80}  & \textbf{98} & (\textbf{35}, --)\\
eNATL60-t & \textbf{5.9}  & \textbf{0.91}  & \textbf{0.80}  & 100 & (33, \textbf{48})\\
eNATL60-0 & \textbf{5.9}  & \textbf{0.91}  & \textbf{0.80}  & 100 & (33, 47)\\
GLORYS12-r & 6.3  & 0.90  & 0.78  & 106  & (30, 28)\\
GLORYS12-f & 6.7  & 0.90  & 0.77  & 119 & (21, 23)\\
ORCA025 & 7.1  & 0.89  & 0.76  & 126 & (17, 17)\\
\bottomrule
\end{tabular}

\caption{\textbf{Performance of 4DVarNet mapping schemes trained on different simulated datasets}. The first column shows the source of the training dataset as described in Table \ref{tab:data}; the subsequent columns indicate the reconstruction metrics described in Section \ref{ssec:eval}. Note that the NATL60 could not be evaluated on the OSSE setup since the evaluation data were used for validation during the training stage.}
\label{tab:res}
\end{table}

\subsection{Eddy-present datasets versus eddy-rich ones}
\label{ssec:resolution}


\begin{figure}[H]
\small
\includegraphics[width=\linewidth]{./00_Simulearning/standalone_figures/psd_res.png}
% \caption{Row I - Isotrophic PSD. Row 2 - Isotrophic PSD Score}
\caption{
\textbf{Spectral analysis of the training and reconstructed SSH datasets}. We display the PSD of the training dataset (left plot), reconstructed SSH field (center plot) as well as the associated PSD score (right plot)}\vspace{-5mm}
\label{fig:respsd}
\end{figure}


We analyse here in more detail the impact of the spatial resolution of the training dataset onto the reconstruction performance. In Table \ref{tab:res}, as expected, the higher resolution grid in the ocean run simulation leads to a better mapping with a 22\% improvement in $\lambda_x$  and a 17\% improvement in the RMSE score between the experiments with the coarsest (ORCA025) and finest (NATL60) resolutions.
We also observe qualitative differences in the relative vorticity fields in Figure \ref{fig:maps}.  Residual artifacts due to the altimetry tracks appear (60°W, 39°N) for the two lower-resolution training datasets. They are greatly diminished when considering the NATL60 dataset. 
Despite these differences, the reconstructed vorticity and kinetic energy fields in Figure \ref{fig:maps} look very similar for the different 4DVarNet schemes, whatever the training datasets. By contrast, the vorticity and kinetic energy fields in the training datasets clearly depict fewer fine-scale structures and weaker gradients for the lower-resolution simulation datasets, namely ORCA025 and GLORYS12-f.
%However we see that the reconstructed vorticity and kinetic energy of the ORCA025 experiment are quite similar to the one from the NATL60 experiment despite the ORCA025 training data clearly containing fewer fine scale structures an weaker gradients. 
These results support the generalization skills of 4DVarNet schemes to map real altimetry tracks despite being trained on SSH sensibly different from the reconstruction. 

We draw similar conclusions from the analysis of 
the spectral densities shown in Figure \ref{fig:respsd}. The differences in the energy distribution of the training data significantly reduce in the reconstructions. 4DVarNet schemes trained from higher-resolution datasets however result in more faithful reconstruction at all scales.
%, however the PSD score plot on the right shows that even if the energy levels are close in the reconstructions, training with finer resolution produces a more faithful reconstruction at all scales
The patterns observed for the temporal PSD are slightly different in Figure \ref{fig:spacetime_psd}. We do not observe the same homogenization as for the spatial PSD. Lower-resolution training datasets involve a significant drop 
of an order of magnitude for periods greater than 10 days and wavelength greater than 200km.
%Finally when looking at the space-time PSD in Figure \ref{fig:spacetime_psd}, we can note that even though the spatial spectra have been greatly homogenized, the temporal PSD still contains significant gaps of an order of magnitude for periods greater than 10 days and wavelength greater than 200km.

\subsection{Forced simulation datasets versus reanalysis ones}
\label{ssec:reanalysis}

\begin{figure}[h]
% \vspace{0.9cm}
\small
\begin{center}
    
\includegraphics[width=\linewidth]{./00_Simulearning/standalone_figures/psd_rea.png}
% \caption{Row I - Isotrophic PSD. Row 2 - Isotrophic PSD Score}
\caption{\textbf{Spectral impact of model reanalysis}. We display the PSD of the training dataset (left plot), reconstructed SSH field (center plot) as well as the associated PSD score (right plot)}
\vspace{-5mm}
\label{fig:reapsd}
\end{center}
\end{figure}
Looking in more specifically at the effect of ocean reanalysis between the two experiments GLORYS12-f and GLORYS12-r. We can first note the impact of observation data assimilation in Figure \ref{fig:spacetime_psd} where we see how the power spectrum of the reanalysis is significantly raised compared to the free run. The spectrum is closer to ones of the higher resolution simulations. Visually we also clearly see stronger gradients in the kinetic energy in Figure \ref{fig:maps}.

We can observe a similar behavior as in Section \ref{ssec:resolution} in Figure \ref{fig:reapsd} with the gap of in spectral density being diminished between the training and reconstruction data, and the PSD score indicating a lower energy of the error at all scales for the reanalysis-based experiment.

Quantitatively in Table \ref{tab:data} we see an improvement of 11\% in both the RMSE and the scale resolved, besides training on a reanalysis increase the relative gain w.r.t. DUACS significantly more on real data (+9\%) than on simulated data (+5\%) as we can see in the right most column. This suggests that the reanalysis dataset conveys information on real world observations which improves the generalization performance.

\subsection{Tide-free datasets versus tide-resolving ones}
\label{ssec:tide}
% \begin{figure}[h]
% \small
% \begin{center}
% \setlength{\tabcolsep}{1pt}
% \begin{tabular}{ccc}

% \hspace{3mm} Training PSD & 
% \hspace{3mm} Reconstruction PSD & 
% \hspace{3mm} PSD Score  \\

% %\vspace{-2mm}
% %%%%% Tide psd %%%%%%%%

% \includegraphics[width=0.31\textwidth]{figures/plots2/isotrop_psd_tide_train.png} &
% \includegraphics[width=0.31\textwidth]{figures/plots2/isotrop_psd_tide_rec.png} &
% \includegraphics[width=0.31\textwidth]{figures/plots2/tide_1d_psd_score.png}


% \end{tabular}
% \vspace{-3mm}
% % \caption{Row I - Isotrophic PSD. Row 2 - Isotrophic PSD Score}
% \caption{
% Impact of explicit tide modelling on the PSDs. We display the PSD of the training dataset after the prepocessing (left plot), the reconstructed SSH field (center plot) as well as the associated PSD score (right plot)}\vspace{-5mm}
% \label{fig:tidepsd}
% \end{center}
% \end{figure}

We assess here the impact of tide-resolving simulation used as training data. We use the twin eNATL60 runs eNATL60-t and eNATL60-0. Contrary to other runs, those simulations contain barometric and wind forcing, we therefore remove the Dynamic Atmospheric Correction \cite{carrereMajorImprovementAltimetry2016} from the SSH fields. Additionally since the barotropic tide signals are removed from real altimetry tracks prior to interpolation, we also remove the signal from the training data by subtracting the spatial mean over the training domain for each hourly snapshot before calculating the daily averages.  

Given those processing steps, the two training datasets  exhibit very similar wavenumber spectra as shown in Figures \ref{fig:spacetime_psd}. 
We also find that training on those two datasets produce little differences in the reconstructions both quantitatively  (see Table \ref{tab:res}) and qualitatively (Fig. \ref{fig:maps}). The resulting performance is comparable to that of the NATL60 experiment.
 
We identify two hypotheses for explaining why tide-resolving simulation do not lead to better mapping schemes:
\begin{itemize}
    \item The preprocessing applied on the training field remove the main tide signals. We therefore effectively measure the impact of tide modeling on other ocean processes that may be less significant;
    \item The evaluation procedure applied on altimetry tracks on which the barotropic tide has been filtered may not be interpretable enough to measure the reconstruction of residual tide signals. New instruments like the KaRIN deployed in the SWOT mission may provide new ways to better quantify those effects.   
\end{itemize}

These findings provide motivation for carefully considering the purpose of the learning-based model when making decisions about the training data. In our case, explicitly modeling tide processes that are removed from the observations in the evaluation setup added overheads in the computational cost of running the simulation as well as in the preprocessing of the training data. Additionally given the considered evaluation data and metrics, we were not able to quantify any significant differences between the two trained mapping schemes.


\section{Discussion}
\label{sec:discussion}
This study has been greatly facilitated by the standardized tasks and evaluation setups proposed in data-challenges \url{https://ocean-data-challenges.github.io/}. Data-challenges are used to specify a targeted problem of interest to domain experts through datasets and relevant evaluation metrics. This preliminary work have been instrumental in constituting the comprehensive benchmark and combining methods from different teams and institution around the world. Additionally, it also constitutes a strong basis for a trans-disciplinary collaboration between the ocean and machine learning research communities.

Moreover, the results presented in this study introduce a use of ocean simulations for developing altimetry products. This opens new ways for ocean physicist, modelers and operational oceanographers to collaborate. In order to assess the range of these new synergies, it would be interesting to explore if the approach proposed here of training neural schemes using simulation data would generalize to other tasks such as forecast or sensor calibration and to other quantities like surface temperature, currents, salinity or biochemical tracers.

If the simulation-based training approach introduced here is successfully extended to other ocean problems, one could envision training large foundation deep learning models \cite{brownLanguageModelsAre} capturing the inner structure of high resolution ocean simulations which could then be used in many downstream applications. This could be the way to capitalize on all the advancement in ocean modeling without having to run OGCM numerical simulation for each downstream products.    

Furthermore, we would like to highlight the cost consideration when running numerical simulation intended for training learning based schemes. Indeed given that the eNATL60 run took 2700x CPU hours and 350x memory compared to the ORCA025 run for a smaller domain, a trade-off arises between generating multiple "cheap" trajectories versus generating a single more realistic trajectory. 

To conclude, we have shown in this study that training machine learning models on simulations datasets leads good performance on real altimetry data mapping and outperforms current state of the art approaches. The model trained on NATL60 reduces the RMSE by 18\% compared neural schemes trained on observation data and improves the scales resolved by 33\% compared to the DUACS operational product. Even the coarsest simulation considered ORCA025 provides competitive results with current operational methods. We have shown that using a more realistic SSH fields using reanalysis or higher resolution simulations increases the performances of the trained model. This is an exciting result that shows the potential for training operational products from ocean simulations and how advances in ocean modeling in operational oceanography can be beneficial. The results shown here are limited to the interpolation problem on a regional domain but the robustness of the performance shown are encouraging for further developing these results using a larger domain.



\section*{Open Research Section}
The authors provide the training data, source code, reconstructed maps and trained model for each experiments of the manuscript at https://doi.org/10.5281/zenodo.8064114.



% \acknowledgments
% This work was supported by ANR Projects Melody and OceaniX and CNES. It benefited from HPC and GPU resources from GENCI-IDRIS (Grant 2020-101030) and Ifremer.


% \bibliography{}

% \addcontentsline{toc}{section}{Bibliography}
\putbib[./00_Simulearning/biblio.bib]
\end{bibunit}
% \end{document}







\clearemptydoublepage
% \documentclass{article}



% % if you need to pass options to natbib, use, e.g.:
%     \PassOptionsToPackage{numbers, compress}{natbib}
% before loading neurips_data_2023

% ready for submission
% \usepackage{neurips_data_2023}
% \usepackage[preprint]{./00_Oceanbench/neurips_data_2023}

% to compile a preprint version, add the [preprint] option, e.g.:
%     \usepackage[preprint]{neurips_data_2023}
% This will indicate that the work is currently under review.

% to compile a camera-ready version, add the [final] option, e.g.:
%     \usepackage[final]{neurips_data_2023}

% to avoid loading the natbib package, add option nonatbib:
%    \usepackage[nonatbib]{neurips_data_2023}

% Submissions to the datasets and benchmarks are typically non anonymous,
% but anonymous submissions are allowed. If you feel that you must submit 
% anonymously, you can compile an anonymous version by adding the [anonymous] 
% option, e.g.:
%     \usepackage[anonymous]{neurips_data_2023}
% This will hide all author names.

\usepackage{color}
\usepackage{soul}
\usepackage[dvipsnames]{xcolor}
% \usepackage[normalem]{ulem}
\newcommand{\todo}[1]{\textcolor{BrickRed}{[\textbf{TODO}]} }
\newcommand{\tocite}[1]{\textcolor{Plum}{[\textbf{CITE}: #1]}}
\newcommand{\hlc}[2][yellow]{{%
    \colorlet{foo}{#1}%
    \sethlcolor{foo}\hl{#2}}%
}
\newcommand{\tofix}[1]{\hlc[red!20]{#1}}

\usepackage[utf8]{inputenc} % allow utf-8 input
\usepackage[T1]{fontenc}    % use 8-bit T1 fonts
\usepackage{hyperref}       % hyperlinks
\usepackage{url}            % simple URL typesetting
\usepackage{booktabs}       % professional-quality tables
\usepackage{nicefrac}       % compact symbols for 1/2, etc.
\usepackage{microtype}      % microtypography
\usepackage{xcolor}         % colors
\usepackage{multirow}
\usepackage{graphicx}
%% MATH PACKAGES
\usepackage{amsfonts}       % blackboard math symbols
\usepackage{amsmath}
\usepackage{nicefrac} 
\usepackage{lscape}


\usepackage{listings}
% \usepackage{minted}
% \usepackage[cachedir=.]{minted}
% \usepackage{frozencache}
% \usepackage[frozencache=true,cachedir=minted-cache]{minted} 
\usepackage[finalizecache=true,cachedir=minted-cache]{minted}
% \usepackage[cachedir=.]{minted}

% \usepackage{appendix}

% here is a macro expanding to the name of the language
% (handy if you decide to change it further down the road)

% \newcommand\YAMLcolonstyle{\color{red}\mdseries}
\newcommand\YAMLkeystyle{\color{black}\bfseries}
\newcommand\YAMLvaluestyle{\color{blue}\mdseries}

\makeatletter

% here is a macro expanding to the name of the language
% (handy if you decide to change it further down the road)
\newcommand\language@yaml{yaml}

\expandafter\expandafter\expandafter\lstdefinelanguage
\expandafter{\language@yaml}
{
  keywords={true,false,null,y,n},
  keywordstyle=\color{darkgray}\bfseries,
  basicstyle=\YAMLkeystyle,                                 % assuming a key comes first
  sensitive=false,
  comment=[l]{\#},
  morecomment=[s]{/*}{*/},
  commentstyle=\color{purple}\ttfamily,
  stringstyle=\YAMLvaluestyle\ttfamily,
  moredelim=[l][\color{orange}]{\&},
  moredelim=[l][\color{magenta}]{*},
  moredelim=**[il][\YAMLcolonstyle{:}\YAMLvaluestyle]{:},   % switch to value style at :
  morestring=[b]',
  morestring=[b]",
  literate =    {---}{{\ProcessThreeDashes}}3
                {>}{{\textcolor{red}\textgreater}}1     
                {|}{{\textcolor{red}\textbar}}1 
                {\ -\ }{{\mdseries\ -\ }}3,
}

% switch to key style at EOL
\lst@AddToHook{EveryLine}{\ifx\lst@language\language@yaml\YAMLkeystyle\fi}
\makeatother

\newcommand\ProcessThreeDashes{\llap{\color{cyan}\mdseries-{-}-}}


\definecolor{codegreen}{rgb}{0,0.6,0}
\definecolor{codegray}{rgb}{0.5,0.5,0.5}
\definecolor{codepurple}{rgb}{0.58,0,0.82}
\definecolor{backcolour}{rgb}{0.95,0.95,0.92}

\lstdefinestyle{pythonstyle}{
    % backgroundcolor=\color{backcolour},   
    commentstyle=\color{codegreen},
    keywordstyle=\color{magenta},
    numberstyle=\tiny\color{codegray},
    stringstyle=\color{codepurple},
    basicstyle=\ttfamily\footnotesize,
    breakatwhitespace=false,         
    breaklines=true,                 
    captionpos=b,                    
    keepspaces=true,                 
    numbers=left,                    
    numbersep=5pt,                  
    showspaces=false,                
    showstringspaces=false,
    showtabs=false,                  
    tabsize=2
}

\lstset{style=pythonstyle}


% \title{\textsc{OceanBench}: \\ The Sea Surface Height Edition}

% % % The \author macro works with any number of authors. There are two commands
% used to separate the names and addresses of multiple authors: \And and \AND.
%
% Using \And between authors leaves it to LaTeX to determine where to break the
% lines. Using \AND forces a line break at that point. So, if LaTeX puts 3 of 4
% authors names on the first line, and the last on the second line, try using
% \AND instead of \And before the third author name.

\author{%
  J. Emmanuel Johnson$^*$\\
%   MEOM/Institut des Géosciences de l’Environnement, UGA\\
  % Univ. Grenoble Alpes, CNRS UMR IGE, Grenoble, France\\
  CNRS UMR IGE \\
  \texttt{johnsonj@univ-grenoble-alpes.fr}\\
  % examples of more authors
  \And
  Quentin Febvre$^*$\\
  IMT Atlantique\\
  \texttt{quentin.febvre@imt-atlantique.fr} \\
  \And
  Anastasia Gorbunova\\
  CNRS UMR IGE \\
  % \texttt{gorbunoa@univ-grenoble-alpes.fr} \\
  \And
  Sammy Metref\\
  DATLAS\\
  % \texttt{metrefs@univ-grenoble-alpes.fr}\\
  \And
  Maxime Ballarotta\\
  CLS\\
  % \texttt{mballarotta@groupcls.com} \\
  \And
  Julien Le Sommer\\
  CNRS UMR IGE\\
  % \texttt{julien.lesommer@univ-grenoble-alpes.fr} \\
  \And
  Ronan Fablet\\
  IMT Atlantique \\
  % \texttt{ronan.fablet@imt-atlantique.fr} \\
}


% \author{%
%   David S.~Hippocampus\thanks{Use footnote for providing further information
%     about author (webpage, alternative address)---\emph{not} for acknowledging
%     funding agencies.} \\
%   Department of Computer Science\\
%   Cranberry-Lemon University\\
%   Pittsburgh, PA 15213 \\
%   \texttt{hippo@cs.cranberry-lemon.edu} \\
%   % examples of more authors
%   % \And
%   % Coauthor \\
%   % Affiliation \\
%   % Address \\
%   % \texttt{email} \\
%   % \AND
%   % Coauthor \\
%   % Affiliation \\
%   % Address \\
%   % \texttt{email} \\
%   % \And
%   % Coauthor \\
%   % Affiliation \\
%   % Address \\
%   % \texttt{email} \\
%   % \And
%   % Coauthor \\
%   % Affiliation \\
%   % Address \\
%   % \texttt{email} \\
% }


% \begin{document}




% \maketitle
% \def\thefootnote{*}\footnotetext{These authors contributed equally to this work}\def\thefootnote{\arabic{footnote}}
\begin{bibunit}[IEEEtran.bst]

\chapter*{\textsc{OceanBench}: \\ The Sea Surface Height Edition}
\addcontentsline{toc}{chapter}{\textsc{OceanBench}: \\ The Sea Surface Height Edition}
\chaptermark{\textsc{OceanBench}: \\ The Sea Surface Height Edition}
%##################################
% KEYWORDS
% - Sea Surface Height
% - Interpolation
% - Inverse Problem
% - 
% Scientific Machine Learning
% Benchmark, Partial Differential Equations, PINN, FNO, U-Net, Inverse problem
% \linenumbers
% % \begin{abstract}
% The ocean is a key component of the climate system, profoundly influencing human activities. Consequently, a central question for oceanographers revolves around its observation and prediction at different timescales. Similar to various research fields, oceanography has recently witnessed the emergence of machine learning-based methods as an alternative to more domain-specific approaches. But, surprisingly, the adoption of machine learning methods in this field has been arguably slower compared to others. This can be attributed to the complexity of pre-processing tasks in the domain of geospatial data and to the specialized nature of evaluation metrics relevant to ocean science problems. In this context, we introduce \texttt{OceanBench}, a framework designed to lower the barrier to entry for ML researchers by standardizing and automating this process that comply with domain-expert standards.
% It seeks to maintains readability, robustness, and flexibility by providing a standardized framework to explicitly highlight the preprocessing step used, the parameters chosen and the sequence of operations performed. 
% It takes a step towards incorporate these domain-driven decisions to integrate well with existing ML pipelines.
% We demonstrate the usefulness of \texttt{OceanBench} on a series of SSH interpolation tasks with consistent dataflow pipelines, physically-inspired metrics, and interpretable visualizations.

% \end{abstract}

  % The abstract paragraph should be indented \nicefrac{1}{2}~inch (3~picas) on
  % both the left- and right-hand margins. Use 10~point type, with a vertical
  % spacing (leading) of 11~points.  The word \textbf{Abstract} must be centered,
  % bold, and in point size 12. Two line spaces precede the abstract. The abstract
  % must be limited to one paragraph.

\begin{abstract}

The ocean is a crucial component of the Earth's system. 
It profoundly influences human activities and plays a critical role in climate regulation. 
Our understanding has significantly improved over the last decades with the advent of satellite remote sensing data, allowing us to capture essential sea surface quantities over the globe, e.g., sea surface height (SSH). 
Despite their ever-increasing abundance, ocean satellite data presents challenges for information extraction due to their sparsity and irregular sampling, signal complexity, and noise. 
Machine learning (ML) techniques have demonstrated their capabilities in dealing with large-scale, complex signals. 
Therefore we see an opportunity for these ML models to harness the full extent of the information contained in ocean satellite data. 
However, data representation and relevant evaluation metrics can be \textit{the} defining factors when determining the success of applied ML. 
The processing steps from the raw observation data to a ML-ready state and from model outputs to interpretable quantities require domain expertise, which can be a significant barrier to entry for ML researchers. 
In addition, imposing fixed processing steps, like committing to specific variables, regions, and geometries, will narrow the scope of ML models and their potential impact on real-world applications. 
\textbf{OceanBench} is a unifying framework that provides standardized processing steps that comply with domain-expert standards. 
It is designed with a flexible and pedagogical abstraction: it a) provides plug-and-play data and pre-configured pipelines for ML researchers to benchmark their models w.r.t. ML and domain-related baselines and b) provides a transparent and configurable framework for researchers to customize and extend the pipeline for their tasks. 
In this work, we demonstrate the \texttt{OceanBench} framework through a first edition dedicated to SSH interpolation challenges. 
We provide datasets and ML-ready benchmarking pipelines for the long-standing problem of interpolating observations from simulated ocean satellite data, multi-modal and multi-sensor fusion issues, and transfer-learning to real ocean satellite observations. 
The \texttt{OceanBench} framework is available at~\href{https://github.com/jejjohnson/oceanbench}{github.com/jejjohnson/oceanbench} and the dataset registry is available at~\href{https://github.com/quentinf00/oceanbench-data-registry}{github.com/quentinf00/oceanbench-data-registry}.

\end{abstract}


%%%%%%%%%%%%%%%%%%%%%%%%%%%%%%%%%%%%%%%%%%%%%%%%%%%%%%%%%%%%%%%%%%%%%%%%%%%%%%%%%%%%%%%%%%%%%%%%%%%%%%%%%%%%%%%%%%%%%%%%%%%%%%%%%%%%%%%%%%%%%%%%%

% - We demonstrate how \texttt{OceanBench} can be used as an underlying framework with a series of SSH interpolation tasks which combines different variables, regions and geometries. 


% Q - make use of everything we mentioned above!
% E/Q - we want to showcase that OceanBench can be used for SSH interpolation 
% - we want to demonstrate how we can design different experiments/modifications with minimal code changes
% Q - simple tasks SSH -> SSH; extend it a) add domain, b) include other variables, c) switching to real data
% - enrich a simple experimental setups with additional variations

%%%%%%%%%%%%%%%%%%%%%%%%%%%%%%%%%%%%%%%%%%%%%%%%%%%%%%%%%%%%%%%%%%%%%%%%%%%%%%%%%%%%%%%%%%%%%%%%%%%%%%%%%%%%%%%%%%%%%%%%%%%%%%%%%%%%%%%%%%%%%%%%%



%%%%%%%%%%%%%%%%%%%%%%%%%%%%%%%%%%%%%%%%%%%%%%%%%%%%%%%%%%%%%%%%%%%%%%%%%%%%%%%%%%%%%%%%%%%%%%%%%%%%%%%%%%%%%%%%%%%%%%%%%%%%%%%%%%%%%%%%%%%%%%%%%

% - There is some work to be done to get the observation data to a ML-ready state.
% - Deciding the appropriate transformations requires domain-expertise which can be a large barrier to entry for ML researchers.

% - ML methods cannot be applied out of the box on raw observation data because .. blah blah

% - Despite the potential of machine learning to solve the issues that plague geosciences, the barrier to entry for machine learning (ML) researchers in ocean sciences is often hindered by the vastly different schemes for preprocessing techniques required to get the geo-centric data to a ML-ready state.
% \texttt{OceanBench} is a framework for co-designing machine learning-driven high level data products from ocean observations. 
%%%%%%%%%
% Why dont we just provide the processed data and call it a day...?
% Q - 1) non-standardized preprocessing steps; 2) regional differences
% 1 - multiple ways to take advantage of the Obs data (covariates, coordinate-based, gridded)
% Keeping the flexibilities allow for extensibility and multiple 
%%%%%%%%%%%%$
% 1 - OceanBench is a unifying framework that provides standardized processing steps that comply with domain-expert standards.
% 2 - It is designed for - flexible, padagological, abstract, :
% a - getting ML researchers quickly involved (data + preconfigured pipelines -> ML ready setups and test their models)
% - provides data and preconfigured pipelines for ML researchers to quickly experiment with different setups,
% - transparent and configurable for researchers to customize and extend the pipeline for their own objectives
% b - pedagological, unified setup to create and modified tasks (extensible)
% - flexibility
% 3 - We demonstrate ho
% %
% \texttt{OceanBench} is a framework that standardizing ... these processing steps that comply with domain-expert standards.
% %
% It is designed to lower the barrier to entry for ML researchers by providing preconfigured pipelines, ... "one-stop-shop for it all", and providing illustrative examples for different audiences.
% * 
% * provide a common interface for the processing steps
% * create a few ML tasks from raw data to ML ready using the above steps
% * showcase how one can orchestrate preconfigured pipelines to create a variety of ML tasks
% * preconfigured pipelines to illustrate example applications

% 1 - getting ML researchers quickly involved (data + preconfigured pipelines -> ML ready setups and test their models)
% 2 - pedagological, unified setup to create and modified tasks

% %%%%%
% \texttt{OceanBench} is a framework 
% - lower the barrier to entry
% - accessibility
% - one place to look to get all the stuff you want to do
% - preconfigured pipelines
% - jbooks with illustrated examples.
% - standardized preprocessing steps.
%
% It seeks to maintains readability, robustness, and flexibility by providing a standardized framework to explicitly highlight the preprocessing step used, the parameters chosen and the sequence of operations performed. 
%
% It takes a step towards incorporate these domain-driven decisions to integrate well with existing ML pipelines.
%
% We demonstrate the usefulness of \texttt{OceanBench} on a series of SSH interpolation tasks with 
% consistent dataflow pipelines, physically-inspired metrics, and interpretable visualizations.

The previous chapters have underscored the potential of learning-based methods for developing real-world altimetry solutions. The next chapter delves into what deep learning practitioners can contribute effectively to ocean observation science.

It is my strong belief that the adoption of deep learning approaches necessitates collaboration between the oceanography and machine learning communities. To ensure that these methods provide genuine value, evaluation criteria and metrics must be defined with domain expertise by ocean experts. The quality of SSH estimations, for instance, depends on factors such as geographical region, season, physical plausibility of derived quantities. The choice of using observational or simulated data for metric computation yields different assessments, and these distinctions should be carefully considered.

Evaluation criteria should remain adaptable, especially for deep learning methods, which may optimize training objectives by taking shortcuts (as per the "No Free Lunch" theorem). Consequently, the next chapter introduces a platform that facilitates an ongoing dialogue between deep learning experts and oceanography specialists to develop and assess these methods effectively.

On the data front, the selection of data used to train deep learning models is a critical consideration when shaping the final method. As demonstrated in previous chapters, both observational and simulation data can be instrumental in calibrating and evaluating the method. These data undergo domain-specific processing steps. Accessibility to the data and the relevant processing steps can significantly lower the entry barriers for aspiring machine learning practitioners.

These considerations provide the motivation for the work presented in the next chapter. Oceanbench is a suite of tools designed to load and process ocean data. The software is organized into a series of tasks, each characterized by evaluation data and metrics. Its purpose is to simplify the loading and formatting of data relevant for training neural schemes and to streamline the configuration of new evaluation setups for diverse tasks, regions, and datasets.

\section{Introduction}

The ocean is vital to the Earth's system~\cite{OCEANWARMING}. 
It plays a significant role in climate regulation regarding carbon~\cite{OCEANCARBONCYCLE} and heat uptake~\cite{OCEANHEATUPTAKE}. It is also a primary driver of human activities (e.g., maritime traffic and world trade, marine resources and services)~\cite{SSHOPERATIONAL, ML4OCN}. 
However, monitoring the ocean is a critical challenge: the ocean state can only partially be determined because most of the ocean consists of subsurface quantities that we cannot directly observe. 
Thus, to quantify even a fraction of the physical or biochemical ocean state, we must often rely only on surface quantities that we can monitor from space, drifting buoys, or autonomous devices.
Satellite remote sensing, in particular, is one of the most effective ways of measuring essential sea surface quantities~\cite{Altimetry} such as sea surface height (SSH)~\cite{DUACS}, sea surface temperature (SST)~\cite{OCEANSATELLITESST}, and ocean color ~\cite{OCEANSATELLITEOC}. 
While these variables characterize only a tiny portion of the ocean ecosystem, they present a gateway to many other derived physical quantities~\cite{ML4OCN}.


% satellite-derived SSH data allow us to monitor the sea level rise as well as critical physical and biogeochemical ocean processes,
Although we can access observable sea surface quantities, they can be irregularly and extremely sparsely sampled like the altimetry data considered in previous in this thesis~\cite{DUACS}. 
These sampling gaps make the characterization of ocean processes highly challenging for operational products and downstream tasks that depend on relevant gap-free variables. This has motivated a rich literature in geoscience over the last decades, mainly using geostatistical methods \cite{DUACS, MIOST} and model-driven data assimilation schemes~\cite{BFNQG, GLORYS12}. Despite significant progress, these schemes often need to improve their ability to leverage available observation datasets' potential fully. 
This has naturally advocated for exploring data-driven approaches like shallow ML schemes~\cite{DINEOF, DINEOF2, ANALOGDA2, ANALOGDA}. Very recently, as presented in previous chapters, deep learning schemes \cite{SSHInterpAttention, SSHInterpConvLSTM, SSHInterpUNet} have become appealing solutions to benefit from existing large-scale observation and simulation datasets and reach significant breakthroughs in the monitoring of upper ocean dynamics from scarcely and irregularly sampled observations. 
To ensure that these methods provide genuine value, evaluation criteria and metrics must be defined with domain expertise by ocean experts. The quality of SSH estimations, for instance, depends on factors such as geographical region, season, physical plausibility of derived quantities. The choice of using observational or simulated data for metric computation also yields different assessments


Furthermore, the heterogeneity and characteristics of the observation data present major challenges for effectively applying these methods beyond idealized case studies. 
A data source can have different variables, geometries, and noise levels, resulting in many domain-specific preprocessing procedures that can vastly change the solution outcome. Accessibility to the data and the relevant processing steps can significantly lower the entry barriers for aspiring machine learning practitioners.

 % Oceanbench is a suite of tools designed to load and process ocean data. The software is organized into a series of tasks, each characterized by evaluation data and metrics. Its purpose is to simplify the loading and formatting of data relevant for training neural schemes and to streamline the configuration of new evaluation setups for diverse tasks, regions, and datasets.

% So the entire ML pipeline requires a unified framework for dealing with heterogeneous data sources, different pre- and post-processing methodologies, and regionally-dependent evaluation procedures.

% The previous chapters have underscored the potential of learning-based methods for developing real-world altimetry solutions. The next chapter delves into what deep learning practitioners can contribute effectively to ocean observation science.

% It is my strong belief that the adoption of deep learning approaches necessitates collaboration between the oceanography and machine learning communities. .

% Evaluation criteria should remain adaptable, especially for deep learning methods, which may optimize training objectives by taking shortcuts (as per the "No Free Lunch" theorem). Consequently, the next chapter introduces a platform that facilitates an ongoing dialogue between deep learning experts and oceanography specialists to develop and assess these methods effectively.

% On the data front, the selection of data used to train deep learning models is a critical consideration when shaping the final method. As demonstrated in previous chapters, both observational and simulation data can be instrumental in calibrating and evaluating the method. These data undergo domain-specific processing steps. Accessibility to the data and the relevant processing steps can significantly lower the entry barriers for aspiring machine learning practitioners.




 %So the nature of the observations presents a key challenge for operational products and downstream tasks that depend on relevant gap-free variables.
%In addition, this sparsity 
%Although methods like smoothers~\tocite{OI, MIOST}, data assimilation~\tocite{BFN} and machine learning (ML)~\tocite{NerFs,recent-stuff, Ronan} have produced viable solutions to filling the gaps in ocean satellite, the heterogeneity and characteristics of the observation data present major challenges for effectively applying these methods. 
% No model nor observation systems captures all scales and processes.
% Operational methods like optimal interpolation and data assimilation ...
% Thus, there is a need for better model/data integration schemes that fully benefit from HR models and ever-increasing observation datasets.

% \textbf{Opportunity}.
%An end-to-end framework for piping data from its raw form to an ML-ready state and from model outputs to interpretable quantities is challenging. Designing an effective system that does this is the basis for many MLOPs tools and research~\tocite{MLOPs Research}.
These considerations provide the motivation for the work presented in this chapter. To address these challenges, we introduce \textbf{OceanBench}, a framework for co-designing machine-learning-driven high-level experiments from ocean observations. 
It consists of an end-to-end framework for piping data from its raw form to an ML-ready state and from model outputs to interpretable quantities. 
We regard \texttt{OceanBench} as a key facilitator for the uptake of MLOPs tools and research~\cite{MLOPS1,MLOPS2} for ocean-related datasets and case studies. This first edition provides datasets and ML-ready benchmarking pipelines for SSH interpolation problems, an essential topic for the space oceanography community, related to ML communities dealing with issues like in-painting~\cite{InPaintingSurvey}, denoising~\cite{DENOISESURVEY,DENOISESURVEY2}, and super-resolution~\cite{SuperResSurvey}. 
We expect \texttt{OceanBench} to facilitate new challenges to the applied machine learning community and contribute to meaningful ocean-relevant breakthroughs.
%
The remainder of the paper is organized as follows: in \S2, we outline some related work that was inspirational for this work; in \S3, we formally outline \texttt{OceanBench} by highlighting the target audience, code structure, and problem scope; in \S4, we outline the problem formulation of SSH interpolation and provide some insight into different tasks related to SSH interpolation where \texttt{OceanBench} could provide some helpful utility; and in \S5 we give some concluding remarks while also informally inviting other researchers to help fill in the gaps.

% \textbf{How We Measure SSH + Gap Problem}. 




% % \textbf{Data Deluge, Gaps, \& ML Integration}. 
% Like most geosciences, the ocean sciences also suffer from the data deluge that persists due to the large amounts of observed quantities being collected daily~\tocite{Gus}. 
% The increased availability and quantity of observations are beneficial because it offers the community more insight into the physical processes at a finer resolution.
% % Consequently, this can help oceanographers revolve a central question of observing and predicting the ocean state at different spatial and timescales. 
% The trade-off is that processing this data is very expensive.
% Furthermore, it is impossible to accurately measure the entire globe concurrently, resulting in data with many gaps.
% Many operational systems have historically used ad-hoc coordinate-based schemes to gap-fill the data based on covariance assumptions. 
% However, with the abundance of new data~\tocite{SWOT}, these covariance-based schemes are reaching their limit due to the computational and signal complexity.
% Machine learning (ML) has been recently introduced as a viable alternative to filling the gaps in data where many classes of methods have found success, including coordinate-based methods~\tocite{OI,MIOST,NerFs}, direct surrogate/hybrid methods~\tocite{BFN, recent-stuff}, and end-to-end learning schemes~\tocite{Ronan}. 
% Nevertheless, most of these methods have only been applied to regions, and more needs to be applied at a large scale that comes close to the operational settings.


% % \textbf{Logistics Problem}.
% % \textbf{The Problem with Observation Data}.
% % The community is quickly facing a logistics problem.
% Dealing with complex problems such as SSH interpolation requires dealing directly with the observation data, often involving aggregating observations from satellites, buoys, or autonomous devices.
% In all cases, we can only observe a tiny portion of the Earth at any given time, resulting in highly sparse data (often <5\% coverage).
% This problem is even more difficult because the data sources are often heterogeneous, with different variables, geometries, and noise levels.
% Furthermore, the evaluation procedure can be regionally dependent as the physical phenomena vary globally. 
% So the entire ML pipeline now requires a unified framework for dealing with heterogeneous data sources, different pre- and post-processing methodologies, and regional dependent evaluation procedures.

% To solve a difficult problem such as SSH interpolation requires a lot of interdisciplinary expertise including scientific, numerical and machine learning.
% To do effective ML research in this area, one needs a consistent \textit{pipeline} to funnel the data from its raw form through the transformations and to its evaluation state. 
% In standard ML research, we have copious amounts of \textit{datamodules} which encapsulate many underlying transformations procedures for many staple datasets, e.g. MNIST, CIFAR10, CELEB, etc.
% Even some weather and climate benchmark datasets have \cite{weatherbench,ClimateBench} access to gap-free, non-heterogenous reanalysis data.
% Despite the focus on the ML model in the current state of affairs, we have often found that the variables chosen and how they were preprocessed has the greatest effect on the final result.
% In previous benchmark settings~\citep{CHAOSBENCH,PDEBench,ClimateBench,weatherbench,ENS10Bench}, little attention was paid to the internals of the data preprocessing and this process was kept static.
% In geosciences, because we often deal with sparse, heterogeneous observations, we often need to decide and implement many crucial preprocessing transformations before we get to the modeling aspects.
% % including selecting regions, filtering, and calculating derived variables.
% As such, we should include the chosen variables and how they are preprocessed as hyperparameters to be optimized within the full ML pipeline.
% However, this complicates the entire pipeline because we have effectively compounded the complexity due to the many ways one can preprocess and modify the data in conjunction with decisions from the ML side like the architecture, the optimizer and the loss function.
% In addition, we may even have a large series of post-processing transformations which determine how we evaluate our models. Those transformations contain implementation details that may change between studies which make comparing and reproducing some results challenging.
% This results in a software engineering (SWE) roadblock which needs to be addressed should we wish to do effective research in applied ML research and this obstacle is especially prevalent in the geosciences.



% , through the transformations, into the model, and through the 
% Due to many of the many components needed to solve the problem it requires a pipeline. 
% For domain scientists, there is a lack of organization and ontology of previous ideas. 
% For machine learning researchers, ... 
% For downstream users, there is a lack of ontology of the current state of research which makes it difficult to readily apply these techniques for their problems.

% Nonetheless, most of this research has been done in parallel across disjoint sub-fields of research and to the best of our knowledge little to no work has been done on homogenising and integrating these distinct methods in a common framework



% In this paper, we wish to establish a well-defined software abstraction for dealing SSH interpolation using ML.
% Using ocean data is the primary goal which is a data assimilation problem. 
% However, there are many subsequent steps we need to improve. 1) We need gap-free data from the satellite observations, 2) we need surrogate or hybrid models to replace the expensive ocean models, and 3) we need end-to-end learning schemes to perform domain-specific tasks that incorporate good priors and the observations.

\section{Related Work}

Machine learning applied to geosciences is becoming increasingly popular, but there are few examples of transparent pipelines involving observation data. 
After a thorough literature review, we have divided the field into three camps of ML applications that pertain to this work: 1) toy simulation datasets, 2) reanalysis datasets, and 3) observation datasets. 
We outline the literature for each of the three categories below.

\textbf{Toy Simulation Data}. 
One set of benchmarks focuses on learning surrogate models for well-defined but chaotic dynamical systems in the form of ordinary differential equations (ODEs) and partial differential equations (PDEs) and there are freely available code bases which implement different ODEs/PDEs~\citep{CHAOSBENCH,PDEBench,pyQG,JAXCFD,NCARDART,NCARDARTSOFTWARE,VEROS,OCEANANIGANS}.
% For example, \texttt{Dyst} package~\cite{CHAOSBENCH} has many chaotic ODEs, and packages such as \texttt{PDEBench}~\cite{PDEBench}, pyQG~\citep{pyQG}, NCAR DART~\citep{NCARDART,NCARDARTSOFTWARE}, and jax-CFD~\citep{JAXCFD} have implementations of 2D/3D Spatial-Temporal PDEs similar to Navier-Stokes; 
This is a great testing ground for simple toy problems that better mimic the structures we see in real-world observations. 
% More ocean-related simulations include the pyQG package~\citep{pyQG}, the NCAR DART data assimilation testbed~\citep{NCARDART,NCARDARTSOFTWARE}.
Working with simulated data is excellent because it is logistically simple and allows users to test their ideas on toy problems without increasing the complexity when dealing with real-world data.
However, these are ultimately simple physical models that often do not reflect the authentic structures we see in real-world, observed data.

\textbf{Reanalysis Data}. 
This is assimilated data of real observations and model simulations. 
There are a few major platforms that host ocean reanalysis data like the Copernicus Marine Data Store~\citep{MDSOCEANPHYSICS,MDSBIOGEOCHEMICAL,MDSOCEANPHYSICSENS,MDSWAVES}, the Climate Data Store~\citep{CDSREANALYSISSST}, the BRAN2020 Model~\citep{DATABLUELINK}, and the NOAA platform~\citep{DATANCEP}. 
However, to our knowledge, there is no standard ML-specific ocean-related tasks to accompany the data. On the atmospheric side, platforms like \texttt{WeatherBench}~\cite{weatherbench}, \texttt{ClimateBench}~\cite{ClimateBench}, \texttt{ENS10}~\cite{ENS10Bench} were designed to assess short-term and medium-term forecasting using ML techniques with recent success of ML~\cite{GraphCast,FourCastNet}
% While the original papers featured straightforward methods, there has been swift subsequent development within the last few years~\cite{GraphCast,FourCastNet}. 
% Apart from industry momentum and investment, we attribute the recent adoption and success of ML to the problem's clarity, the original tasks' openness, and the software's ML compatibility. 
The clarity of the challenges set by the benchmark suites has inspired the idea of \texttt{OceanBench}, where we directly focus on problems dealing with ocean observation data.

\textbf{Observation Data}. 
These observation datasets (typically sparse) stem from satellite observations that measure surface variables or in-situ measurements that measure quantities within the water column. 
Some major platforms to host data include the Marine Data Store~\citep{MDSALONGTRACK,MDSINSITU}, the Climate Data Store~\citep{CDSOBSSST,CDSOBSSSTENS,CDSOBSOC}, ARGO~\citep{ARGO}, and the SOCAT platform~\citep{SOCAT}.
However, it is more difficult to assess the efficacy of operational ML methods that have been trained only on observation data and, to our knowledge, there is no coherent ML benchmarking system for ocean state estimation.
% In one community, there is the Surface Ocean CO$_2$ Atlas (SOCAT)~\cite{SOCAT} which is a community effort to aggregate all collocated observations (included SSH) which help predict the fugacity of carbon dioxide (fCO$_2$). 
% This has been a huge effort to provide a consistently updated suite of observations for some key variables that are important in biogeochemical processes.
% However, there is currently no coherent benchmarking system with standard metrics despite their being a wide range of new methods to try and tackle the interpolation problem.
% Most new work tends to use the observations provided with their own additional variable with no standard comparison framework other works.
% In a completely different community, 
There has been significant effort by the \textit{Ocean-Data-Challenge} Group\footnote{Ocean Data Challenge group: Freely associated scientist for oceanographic algorithm and product improvements (\href{https://ocean-data-challenges.github.io/}{ocean-data-challenges.github.io})} which provides an extensive suite of datasets and metrics for SSH interpolation.
% Their motivation is to investigate which methods could be employed for the upcoming SWOT mission~\cite{SWOT} which is highly-challenging for current operational interpolation techniques.
% where they provide over eight challenges of varying degrees of difficulty with completely open-source data, along with tutorials for metrics.
Their efforts heavily inspired our work, and we hope that \texttt{OceanBench} can build upon their work by adding cohesion and facilitating the ease of use for ML research and providing a high-level framework for providing ML-related data products.




\section{OceanBench} \label{sec:oceanbench_intro}

\subsection{Why OceanBench?} \label{sec:oceanbench_why}

There is a high barrier to entry in working with ocean observations for researchers in applied machine learning as there are many processing steps for both the observation data and the domain-specific evaluation procedures. 
\texttt{OceanBench} aims to lower the barrier to entry cost for ML researchers to make meaningful progress in the field of state prediction. 
We distribute a standardized, transparent, and flexible procedure for defining data and evaluation pipelines for data-intensive geoscience applications. 
Proposed examples and case studies provide a plug-and-play framework to benchmark novel ML schemes w.r.t.  state-of-the-art, domain-specific ML baselines. 
In addition, we adopt a pedagogical abstraction that allows users to customize and extend the pipelines for their specific tasks.
To our knowledge, no framework embeds processing steps for earth observation data in a manner compatible with MLOps abstractions and standards regarding reproducibility and evaluation procedures. 
Ultimately, we aim to facilitate the uptake of ML schemes to address ocean observation challenges and to bring new challenges to the ML community to extend additional ML tools and methods for irregularly-sampled and partially-observed high-dimensional space-time dynamics.
The abstractions proposed here apply beyond ocean sciences and SSH interpolation to other geosciences with similar tasks that intersect with machine learning.

% \begin{itemize}
%     \item Establish a rigorous ontology of pre-, geo- and ML processing tools.
%     \item Consistent problems with detailed introductory tutorials
%     \item aggregate benchmarks with common tools
%     \item \textit{what i wish I had when i started this problem}
%     \item \textit{A lot of research gets lots in grad students laptops}
%     \item provide training, context and awareness to the different packages that exist to solve problems.
% \end{itemize}



\subsection{Code Structure} \label{sec:code_structure}

\texttt{OceanBench} is lightweight in terms of the core functionality.
We keep the code base simple and focus more on how the user can combine each piece.
We adopt a strict functional style because it is easier to maintain and combine sequential transformations. 
There are five features we would like to highlight about \texttt{OceanBench}: 1) Data availability and version control, 2) an agnostic suite of geoprocessing tools for \texttt{xarray} datasets that were aggregated from different sources,  3) Hydra integration to pipe sequential transformations, 4) a flexible multi-dimensional array generator from \texttt{xarray} datasets that are compatible with common deep learning (DL) frameworks, and 5) a JupyterBook~\cite{JupyterBook} that offers library tutorials and demonstrates use-cases.
In the following section, we highlight these components in more detail.

\textbf{Data Availability}. 
The most important aspect is the public availability of the datasets. 
We aggregate all pre-curated datasets from other sources, e.g. the \textit{Ocean-Data-Challenge}~\cite{DCOSEGULFSSH,DCOSSEGULFSSH}, and organize them to be publicly available from a single source~\footnote{Available at: \href{https://github.com/quentinf00/oceanbench-data-registry}{oceanbench-data-registry.github.com}}. 
We also offer a few derived datasets which can be used for demonstrations and evaluation. 
Data is never static in a pipeline setting, as one can have many derived datasets which stem from numerous preprocessing choices. 
In fact, in research, we often work with derived datasets that have already been through some preliminary preprocessing methods. 
To facilitate the ever-changing nature of data, we use the Data Version Control (\texttt{DVC}) tool~\cite{DVC}, which offers a git-like version control of the datasets.

\textbf{Geoprocessing Tools}. 
The core \texttt{OceanBench} library offers a suite of functions specific to processing geo-centric data. 
While a few particular functionalities vary from domain to domain, many operations are standard, e.g., data variable selections, filtering/smoothing, regridding, coordinate transformations, and standardization. 
We almost work exclusively with the \texttt{xarray}~\cite{XARRAY} framework because it is a coordinate-aware, flexible data structure. 
In addition, the geoscience community has an extensive suite of specialized packages that operate in the \texttt{xarray} framework to accomplish many different tasks. 
% A non-exhaustive, highlighted list of packages that are compatible with the \texttt{xarray} data structure include: \texttt{xgcm} for non-uniform grid operators, \texttt{pint} for unit-aware operators, \texttt{metpy} for gradient-based calculations, \texttt{pyinterp} and \texttt{xesmf} for advanced regridding methods, \texttt{xrft} and \texttt{xwavelet} for power spectrum analysis, \texttt{gcm-filters} for advanced filtering methods, and the \texttt{dask} suite for scaling everything to multi-core/node HPC environments. 
Almost all \texttt{OceanBench} toolsets are exclusively within the \texttt{xarray} framework to maintain compatibility with a large suite of tools already available from the community.


% As an ML practicioner navigating this sea of libraries can be intimidating as they require different input formats and domain specific parameters. Oceanbench aims at simplifying the use of these libraries by providing the necessary formatting tools and default usage of the different utilities.  

% There are many packages that exist independently which accomplish many of the specific tasks that most geoscientists need. 

\textbf{Hydra Integration}. 
As discussed above, many specific packages accomplish many different tasks. 
However, what needs to be added is the flexibility to mix and match these operations as the users see fit. 
\texttt{Hydra}~\cite{Hydra} provides a configurable way to aggregate and \textit{pipe} many sequential operations together. 
It also maintains readability, robustness, and flexibility through the use of \texttt{.yaml} files which explicitly highlights the function used, the function parameters chosen, and the sequence of operations performed. 
In the ML software stack, \texttt{Hydra} is often used to manage the model, optimizer, and loss configurations which helps the user experiment with different options. 
We apply this same concept in preprocessing, geoprocessing, and evaluation steps, often more important than the model configuration in geoscience-related tasks.  

\texttt{XRPatcher}~\footnote{Available at: \href{https://github.com/jejjohnson/xrpatcher/}{github.com/jejjohnson/xrpatcher}}. 
Every machine learning pipeline will inevitably require moving data from the geo-specific data structure to a multi-dimensional array easily digestible for ML models. 
A rather underrated, yet critical, feature of ML frameworks such as \texttt{PyTorch}~\cite{PYTORCH} (\texttt{Lightning}~\cite{LIGHTNING}) and \texttt{TensorFlow}~\cite{TENSORFLOW} (\texttt{Keras}~\cite{KERAS}) is the abstraction of the dataset, dataloader, datamodules, and data pipelines. 
In applied ML in geosciences, the data pipelines are often more important than the actual model~\cite{DATA4ML}. 
The user can control the \textit{patch}-size and the \textit{stride}-step, which can generate arbitrary coordinate-aware items directly from the \texttt{xarray} data structure. 
In addition, \texttt{XRPatcher} provides a way to reconstruct the fields from an arbitrary patch configuration.
This robust reconstruction step is convenient to extend the ML inference step where one can reconstruct entire fields of arbitrary dimensions beyond the training configuration, e.g., to account for the border effects within the field (see appendix~\ref{sec:xrpatcher}) or to reconstruct quantities in specific regions or globally. 

\textbf{JupyterBook}.
% ~\footnote{Available at: \href{https://jejjohnson.github.io/oceanbench/content/overview.html}{jejjohnson.github.io/oceanbench}}. 
Building a set of tools is relatively straightforward; however, ensuring that it sees a broader adoption across a multi-disciplinary community is much more challenging. 
We invested heavily in showing use cases that appeal to different users with the \texttt{JupyterBook} platform~\cite{JupyterBook}. 
Code with context is imperative for domain and ML experts as we need to explain and justify each component and give many examples of how they can be used in other situations. 
% \texttt{Hydra} is an effective tool, but it has a steep learning curve, and the tutorials are very computer science-oriented, and there are many domain scientists (and seasoned ML researchers) who will. 
Thus, we have paid special attention to providing an extensive suite of tutorials, and we also highlight use cases for how one can effectively use the tools. 


\subsection{Problem Scope} \label{sec:problem_scope}

% As outlined in the introduction, we are interested in the state estimation problem for important variables like SSH and SST. We use the data assimilation perspective whereby state estimation is defined as the combination of observations, constraints and initial state~\citep{DAGEOSCIENCE}. Tackling the issue of state estimation directly is very challenging due to the uncertainty in the measurements, the prior constraints, and the initial state. In addition, there are many logistical constraints that further complicate the problem including the very high-dimensional, spatiotemporal state space, the multi-scale complexity of the dynamics, and the extreme missing data.




There are many problems that are of great interest the ocean community~\citep{ML4DA} but we limit the scope to state estimation problems~\citep{DAGEOSCIENCE}. Under this scope, there are research questions that are relevant to operational centers which are responsible for generating the vast majority of global ocean state maps~\citep{MDSOCEANPHYSICS,MDSOCEANPHYSICSENS,MDSBIOGEOCHEMICAL, MDSWAVES} that are subsequently used for many downstream tasks~\citep{ML4OCN}. For example: how can we effectively use heterogeneous observations to predict the ocean state on the sea surface~\citep{BFNQG,NERFSSSH,MIOST,4DVARNETSST,4DVARNETSWOT,OCEANSATELLITESST}; how can we incorporate prior physics knowledge into our predictions of ocean state trajectories~\citep{BFNQG,ML4DA,ML4OCN}; and how can we use the current ocean state at time $T$ to predict the future ocean state at time $T+\tau$~\citep{METNET2,weatherbench,FORECASTSSCGP}.
In the same vain, there are more research questions that are of interest to the academic modeling community. For example: is simulated or reanalysis data more effective for learning ML emulators that replace expensive ocean models~\citep{MLSUBGRID,MLCLOSURE}; what metrics are more effective for assessing our ability to mimic ocean dynamics~\citep{SSTFLOWANOMALY,MLMETRICSINVARIANCE}; and how much model error can we characterize when learning from observations~\citep{MLMODELERR,MLMODELERR2}. 

We have cited many potential applications of how ML can be applied to tackle the state estimation problem. 
However, to our knowledge there is no publicly available, standardized benchmark system that is caters to ML-research standards.
We believe that, irrespective of the questions posed above and the data we access, there are many logistical similarities for each of the problem formulations where we can start to set standards for a subset of tasks like interpolation or forecasting. 
On the front-end, we need a way to select regions, periods, variables, and a valid train-test split (see sec. ~\ref{sec:hydra_recipe_task}). 
On the back-end, we need a way to transform the predictions into more meaningful variables with appropriate metrics for validation (see sec. ~\ref{sec:hydra_geoprocess_task} and ~\ref{sec:hydra_evaluation_task}).
\texttt{OceanBench} was designed to be an agnostic tool that is extensible to the types of datasets, processing techniques and metrics needed for working with a specific class of Ocean-related datasets. 
We strongly feel that a suite like this is the first step in designing task-specific benchmarks within the ocean community that is compatible with ML standards. 
In the remainder of the paper, we will demonstrate how \texttt{OceanBench} can be configured to facilite a ML-ready data challenge involving our first edition to demonstrate \texttt{OceanBench}'s applicability: sea surface height interpolation.

\section{\textit{Sea Surface Height Edition}}\label{sec:interp_challenge}

Sea surface height (SSH) is one of the most critical, observable quantities when determining the ocean state. 
It is widely used to study ocean dynamics and the adverse impact on global climate and human activities~\cite{SSHMESOSCALE}. 
SSH enables us to track phenomena such as currents and eddies~\cite{SSHMESOSCALE,SSHMESOSCALE2,SSHMESOSCALE3}, which leads to a better quantification of the transport of energy, heat, and salt. 
In addition, SSH helps us quantify sea level rise at regional and global scales~\cite{SSHSEALEVEL,OCEANSEALEVEL}, which is used for operational monitoring of the marine environment~\cite{SSHOPERATIONAL}. 
Furthermore, SSH characterization provides a plethora of data products that downstream tasks can use for many other applications~\cite{SSH3DCIRCULATION, 3DQGOC}.
%
Due to the irregular sampling delivered by satellite altimeter, state-of-the-art operational methods using optimal interpolation schemes~\cite{DUACS, MIOST} or model-driven data assimilation~\cite{DINEOF, DINEOF2, ANALOGDA, ANALOGDA2} fail to fully retrieve SSH dynamics at fine scales below 100-200km on a global or regional scale, so improving the space-time resolution of SSH fields has been a critical challenge in ocean science. 
Beyond some technological developments~\cite{SWOT}, recent studies support the critical role of ML-based schemes in overcoming the current limitations of the operational systems~\cite{4DVARNETSWOT, BFNQG, SSHInterpAttention} .  
%
The rest of this section gives an overview of the general problem definition for SSH interpolation, followed by a brief ontology for ML approaches to address the problem. 
We also give an overview of some experimental designs and datasets with a demonstration of metrics and plots generated by the \texttt{OceanBench} platform. 



\subsection*{Problem Definition}\label{sec:prob_definition}

We are dealing with satellite observations, so we are interested in the domain across the Earth's surface. 
Let us define the Earth's domain by some spatial coordinates, $\mathbf{x} = [\text{Longitude},\text{Latitude}]^\top \in\mathbb{R}^{D_s}$, and temporal coordinates, $t=[\text{Time}]\in\mathbb{R}^+$, where $D_s$ is the dimensionality of the coordinate vector.  
We can define some spatial (sub-)domain, $\Omega\subseteq\mathbb{R}^{D_s}$, and a temporal (sub-)domain, $\mathcal{T}\subseteq\mathbb{R}^+$. 
This domain could be the entire globe for 10 years or a small region within the North Atlantic for 1 year.
\begin{align}  \label{eq:spatiotemporal_coords}
    \text{Spatial Coordinates}: && \mathbf{x} &\in \Omega \subseteq \mathbb{R}^{D_s}\\ 
    \text{Temporal Coordinates}: && t &\in \mathcal{T} \subseteq \mathbb{R}^+.
\end{align}
In this case $D_s=2$ because we only have a two coordinates, however we can do some coordinate transformations like spherical to Cartesian. Likewise, we can do some coordinate transformation for the temporal coordinates like cyclic transformations or sinusoidal embeddings~\cite{ATTENTION}. We have two fields of interest from these spatiotemporal coordinates: the state and the observations.
\begin{align} \label{eq:state_obs}
    \text{State}: && \boldsymbol{u}(\mathbf{x},t) &: \Omega\times\mathcal{T}\rightarrow\mathbb{R}^{D_u} \\
    \text{Observations}: && \boldsymbol{y}_{obs}(\mathbf{x},t) &: \Omega\times\mathcal{T}\rightarrow\mathbb{R}^{D_{obs}}
\end{align}
The state domain, $u\in\mathcal{U}$, is a scalar or vector-valued field of size $D_u$ which is typically the quantity of interest and the observation domain, $y_{obs}\in\mathcal{Y}_{obs}$, is the observable quantity which is also a scalar or vector-valued field of size $D_{obs}$. Now, we make the assumption that we have an operator $\mathcal{H}$ that transforms the field from the state space, $\boldsymbol{u}$, to the observation space, $\boldsymbol{y}_{obs}$.
\begin{align} \label{eq:prob_definition}
    \boldsymbol{y}_{obs}(\mathbf{x},t) = \mathcal{H}\left(\boldsymbol{u}(\mathbf{x},t), t, \boldsymbol{\varepsilon}, \boldsymbol{\mu}\right) 
\end{align}
This equation is the continuous function defined over the entire spatiotemporal domain.  
The operator, $\mathcal{H}(\cdot)$, is flexible and problem dependent.
For example, in a some discretized setting there are 0's wherever there are no observations, and 1's wherever there are observations, and in other discretized settings it takes a weighted average of the neighboring pixels.
We also include a generic noise function, $\boldsymbol{\varepsilon}(\mathbf{x},t)$.
This could stem from a distribution, it could stationary noise operator, $\boldsymbol{\varepsilon}(\mathbf{x})$, or it could be constant in space but vary with Time, $\boldsymbol{\varepsilon}(t)$. 
We also include a control parameter, $\boldsymbol{\mu}$, representing any external factors or latent variables that could connect the state vector to the observation vector, e.g., sea surface temperature.
%
%###########################################################################################
%
% CAN BE FRAMED AS INVERSE PROBLEMS
%
Our quantity of interest is SSH, $\eta$, a scalar-valued field defined everywhere on the domain. In our application, we assume that the SSH we observe from satellite altimeters, $\eta_{obs}$, is the same as the SSH state, except it could be missing for some coordinates due to incomplete coverage from the satellite. So our transformation is defined as follows:
\begin{align} \label{eq:ssh_field_continuous}
\boldsymbol{\eta}_{obs}(\mathbf{x},t) &= \mathcal{H}\left(\boldsymbol{\eta}(\mathbf{x},t), t, \boldsymbol{\varepsilon}, \boldsymbol{\mu}\right)
% , \hspace{10mm}
% \mathbf{x} \in \Omega \subseteq \mathbb{R}^{D_s}, \hspace{10mm} 
% t \in \mathcal{T} \subseteq \mathbb{R}^+.
\end{align}
In practice, the satellite providers have a reasonable estimation of the amount of structured noise level we can expect from the satellite altimetry data; however, unresolved noise could still be present. 
% Although we do not explicitly specify the control parameter, $\boldsymbol{\mu}$, we leave it into the equation to account for any other state parameters not accounted for in our model. 
Finally, we are interested in finding some model, $\mathcal{M}$, that maps the SSH we observe to the true SSH given by
\begin{align} \label{eq:interp_problem}
    \mathcal{M} &: \boldsymbol{\eta}_{obs}(\mathbf{x}, t, \boldsymbol{\mu}) \rightarrow \boldsymbol{\eta}(\mathbf{x},t),
%     , \hspace{10mm}
% \mathbf{x} \in \Omega \subseteq \mathbb{R}^{D_s}, \hspace{10mm} 
% t \in \mathcal{T} \subseteq \mathbb{R}^+.
\end{align}
which is essentially an inverse problem that maps the observations to the state.
One could think of it as trying to find the inverse operator, $\mathcal{M}=\mathcal{H}^{-1}$, but this could be some other arbitrary operator.  
%
\subsection*{Machine Learning Model Ontology} \label{sec:ml_ontology_mini}

In general, we are interested in finding some parameterized operator, $\mathcal{M}_{\boldsymbol{\theta}}$, that maps the incomplete SSH field to the complete SSH field
\begin{align} \label{eq:ml_interp_problem}
    \mathcal{M}_{\boldsymbol{\theta}} &: \boldsymbol{\eta}_{obs}(\mathbf{x}, t, \boldsymbol{\mu}) \rightarrow \boldsymbol{\eta}(\mathbf{x},t),
\end{align}
whereby we learn the parameters from data.
%
The two main tasks we can define from this problem setup are 1) interpolation and 2) extrapolation.
We define \textit{interpolation} as the case when the boundaries of the inferred state domain lie within a predefined shape for the boundaries of the spatiotemporal observation domain. 
For example, the shape of the spatial domain could be a line, box, or sphere, and the shape of the temporal domain could be a positive real number line.
We define \textit{extrapolation} as the case where the boundaries of the inferred state domain are outside the boundaries of the spatiotemporal observation domain. 
In this case, the inferred state domain could be outside of either domain or both. 
A prevalent specific case of extrapolation is \textit{hindcasting} or \textit{forecasting}, where the inferred state domain lies within the spatial observation domain's boundaries but outside of the temporal observation domain's.
In the rest of this paper, we will look exclusively at the interpolation problem. 
However, we refer the reader to appendix~\ref{sec:other_tasks} for a more detailed look at other subtasks that can arise.

From a ML point of view, we can explore various ways to define the operator in equation~\eqref{eq:interp_problem}. 
We may distinguish three main categories: (i) coordinate-based methods that learn a parameterized continuous function to map the domain coordinates to the scalar values, (ii) the explicit mapping of the state from the observation, (iii) implicit methods defined as the solution of an optimization problem. 
The first category comprises of kriging approaches, which have been used operationally with historical success~\cite{KRIGINGREVIEW,DUACS}. Beyond such covariance-based approaches, recent contributions explore more complex trainable functional models~\cite{GPsBIGDATA}, basis functions~\cite{MIOST}, and neural networks~\cite{NERFSSSH}. 
The second category of schemes bypasses the physical modeling aspect and amortizes the prediction directly using state-of-the-art neural architectures such as UNets and ConvLSTMs~\cite{SSHInterpAttention, SSHInterpConvLSTM, SSHInterpUNet}. 
This category may straightforwardly benefit from available auxiliary observations~\citep{CDSOBSSST,CDSOBSSSTENS,CDSOBSOC} 
% (including available operational gap-free SST products~\citep{CDSOBSSST,CDSOBSSSTENS} and other sea surface quantities~\citep{CDSOBSOC})
to state the interpolation problem as a super-resolution~\cite{SuperResSurvey} or image-to-image translation problem~\cite{IMAGE2IMAGETRANSLATION, IMAGE2IMAGETRANSLATION2}. 
The third category relates to inverse problem formulations and associated deep learning schemes, for example deep unfolding methods and plug-and-play priors~\cite{DEEPUNFOLDING}. 
Interestingly, recent contributions explore novel neural schemes which combine data assimilation formulations~\cite{DAGEOSCIENCE} and learned optimizer strategies~\cite{4DVARNETSWOT,4DVARNETSST}.
We provide a more detailed ontology of methods used for interpolation problems in appendix~\ref{sec:ml_ontology}. 
We consider at least one baseline approach from each category for each data challenge described in section~\ref{sec:data_challenges}. 
While all these methods have pros and cons, we expect the OceanBench platform to showcase to new experimental evidence and understanding regarding their applicability to SSH interpolation problems.
 

\subsection*{Experimental Design} \label{sec:experimental_design}

\begin{table}[ht]
\caption{This table gives a brief overview of the datasets provided to complete the data challenges listed in~\ref{sec:data_challenges} and~\ref{sec:data_challenges_extended}. Note that the OSSE datasets are all gridded products whereas the OSE NADIR is an alongtrack product. See figure~\ref{fig:oceanbench_maps} for an example of the OSSE NEMO Simulations for SSH and SST and pseudo-observations for NADIR \& SWOT.}
\label{tb:datasets}
\centering
\begin{tabular}{lcccc}
 \toprule
 & OSSE & OSSE NADIR + SWOT & OSSE SST & OSE NADIR  \\ \midrule
 Data Type & Simulations & 
Pseudo-Observations & 
 Simulations & Observations \\
Source     & 
NEMO~\citep{NEMOAJAYI2020} & 
NEMO~\citep{NEMOAJAYI2020} &
NEMO~\citep{NEMOAJAYI2020}
% \multicolumn{3}{c}{NEMO GCM\citep{NEMOAJAYI2020}}  
& Altimetry~\citep{MDSALONGTRACK} \\
Region & 
GulfStream & GulfStream & GulfStream & GulfStream \\
Domain Size &
% ($L_x\times L_y$) 
$10\times 10^\circ$ &
$10\times 10^\circ$ &
$10\times 10^\circ$ &
$10\times 10^\circ$
\\
Longitude Extent &
$[-65^\circ, -55^\circ]$ & 
$[-65^\circ, -55^\circ]$ &
$[-65^\circ, -55^\circ]$ &
$[-65^\circ, -55^\circ]$ \\
Latitude Extent &
$[33^\circ, 43^\circ]$ &
$[33^\circ, 43^\circ]$ &
$[33^\circ, 43^\circ]$ &
$[33^\circ, 43^\circ]$ \\
Resolution &
% ($\Delta_x\times \Delta_y$) 
$0.05^\circ\times 0.05^\circ$ &
$0.05^\circ\times 0.05^\circ$ &
$0.05^\circ\times 0.05^\circ$ &
$7$ km \\
Grid Size &
$200\times 200$ & $200\times 200$ & $200\times 200$ & N/A \\
Num Datapoints &
$\sim$14.6M & $\sim$14.6M & $\sim$14.6M & $\sim$1.6M \\
Period Start & 2012-10-01 & 2012-10-01 & 2012-10-01 & 2016-12-01 \\
Period End & 2013-09-30 & 2013-09-30 & 2013-09-30 & 2018-01-31 \\
Frequency  & Daily & Daily & Daily & 1 Hz \\
\bottomrule
\end{tabular}
\end{table}

The availability of multi-year simulation and observation datasets naturally advocates for the design of synthetic (or twin) experiments, referred to as observing system simulation experiments (OSSE), and of real-world experiments, referred to as observing system experiments (OSE).
We outline these two experimental setups below.

\textbf{Observing System Simulation Experiments (OSSE)}. A staple and groundtruthed experimental setup uses a reference simulation dataset to simulate the conditions we can expect from actual satellite observations. 
This setup allows researchers and operational centers to create a fully-fledged pipeline that mirrors the real-world experimental setting.
An ocean model simulation is deployed over a specified spatial domain and period, and a satellite observation simulator is deployed to simulate satellite observations over the same domain and period. 
This OSSE setup has primarily been considered for performance evaluation, as one can assess a reconstruction performance over the entire space-time domain. It also provides the basis for the implementation of classic supervised learning strategies~\cite{SSHInterpUNet,SSHInterpConvLSTM,SSHInterpAttention}.
The domain expert can vary the experimental conditions depending on the research question. 
For example, one could specify a region based on the expected dynamical regime~\cite{DCOSSEGULFSSH} or add a certain noise level to the observation tracks based on the satellite specifications.
The biggest downside to OSSE experiments is that we train models exclusively with ocean simulations which could produce models that fail to generalize to the actual ocean state. 
Furthermore, the simulations are often quite expensive, which prevents the community from having high spatial resolution over very long periods, which would be essential to capture as many dynamical regimes as possible.


\begin{figure}[t!]
\small
\begin{center}
\setlength{\tabcolsep}{1pt}
\begin{tabular}{ccc}
NADIR Altimetry Tracks & 
SWOT Altimetry Tracks &
Sea Surface Temperature \\
\includegraphics[width=42.5mm, height=30mm]{00_Oceanbench/content/figures/maps/sla/dc20a_ssh_anomaly_nadir4_20121027.png} 
% \includegraphics[bb=0 0 4 3]{content/figures/maps/sla/dc20a_ssh_anomaly_nadir4_20121027.png} 
&
\includegraphics[width=42.5mm, height=30mm]{00_Oceanbench/content/figures/maps/sla/dc20a_ssh_anomaly_swot1nadir5_20121027.png} &
\includegraphics[width=4.25cm,height=3cm]{00_Oceanbench/content/figures/maps/sst/dc20a_nemo_sst.png}
\end{tabular}
\begin{tabular}{cccc}
\hspace{3mm} NEMO Simulation & 
\hspace{3mm} MIOST & 
\hspace{3mm} BFNQG & 
4DVarNet \\
\vspace{-2mm}
%%%%% SEA LEVEL ANOMALY %%%%%%%%
\includegraphics[trim={0 0 42mm 0},clip, width=3.20cm,height=3cm]{00_Oceanbench/content/figures/maps/sla/dc20a_nemo_sla.png} &
\includegraphics[trim={0 0 42mm 0},clip, width=3.2cm,height=3cm]{00_Oceanbench/content/figures/maps/sla/dc20a_miost_sla.png} &
\includegraphics[trim={0 0 42mm 0},clip, width=3.2cm,height=3cm]{00_Oceanbench/content/figures/maps/sla/dc20a_bfnqg_sla.png} &
\includegraphics[width=4.0cm,height=3cm]{00_Oceanbench/content/figures/maps/sla/dc20a_4dvarnet_sla.png} \\
\vspace{-2mm}
%%%%% KINETIC ENERGY %%%%%%%%
\includegraphics[trim={0 0 42mm 0},clip, width=3.20cm,height=3cm]{00_Oceanbench/content/figures/maps/ke/dc20a/nadir4/dc20a_nemo_ke.png} &
\includegraphics[trim={0 0 42mm 0},clip, width=3.2cm,height=3cm]{00_Oceanbench/content/figures/maps/ke/dc20a/nadir4/dc20a_miost_ke.png} &
\includegraphics[trim={0 0 42mm 0},clip, width=3.2cm,height=3cm]{00_Oceanbench/content/figures/maps/ke/dc20a/nadir4/dc20a_bfnqg_ke.png} &
\includegraphics[width=4.0cm,height=3cm]{00_Oceanbench/content/figures/maps/ke/dc20a/nadir4/dc20a_4dvarnet_ke.png}  \\
\vspace{-2mm}
%%%%% RELATIVE VORTICITY %%%%%%%%
\includegraphics[trim={0 0 42mm 0},clip, width=3.20cm,height=3cm]{00_Oceanbench/content/figures/maps/rvort/dc20a/nadir4/dc20a_nemo_vort_r.png} &
\includegraphics[trim={0 0 42mm 0},clip, width=3.2cm,height=3cm]{00_Oceanbench/content/figures/maps/rvort/dc20a/nadir4/dc20a_miost_vort_r.png} &
\includegraphics[trim={0 0 42mm 0},clip, width=3.2cm,height=3cm]{00_Oceanbench/content/figures/maps/rvort/dc20a/nadir4/dc20a_bfnqg_vort_r.png} &
\includegraphics[width=4.0cm,height=3cm]{00_Oceanbench/content/figures/maps/rvort/dc20a/nadir4/dc20a_4dvarnet_vort_r.png}  \\
%%%%% STRAIN %%%%%%%%
\includegraphics[trim={0 0 38mm 0},clip, width=3.20cm,height=3cm]{00_Oceanbench/content/figures/maps/strain/dc20a/nadir4/dc20a_nemo_strain.png} &
\includegraphics[trim={0 0 38mm 0},clip, width=3.2cm,height=3cm]{00_Oceanbench/content/figures/maps/strain/dc20a/nadir4/dc20a_miost_strain.png} &
\includegraphics[trim={0 0 38mm 0},clip, width=3.2cm,height=3cm]{00_Oceanbench/content/figures/maps/strain/dc20a/nadir4/dc20a_bfnqg_strain.png} &
\includegraphics[width=4.0cm,height=3cm]{00_Oceanbench/content/figures/maps/strain/dc20a/nadir4/dc20a_4dvarnet_strain.png}  \\
% \vspace{-2mm}
(a) & (b) & (c) & (d)
\end{tabular}
\vspace{-3mm}
% \caption{Row I - Isotrophic PSD. Row 2 - Isotrophic PSD Score}
\caption{
A snapshot at $27^{th}$ October, 2012 of the sea level anomaly (SLA) from the NEMO simulation for the OSSE experiment outlined in section~\ref{sec:experimental_design}. 
The top row showcases the aggregated NADIR altimetry tracks and the aggregated SWOT altimetry tracks (12 hours before and 12 hours after) as well as the SST from the NEMO simulation.
Each subsequent row showcases the following physical variables found in appendix~\ref{sec:physical_variables}: (a) Sea Level Anomaly, (b) Kinetic Energy, (c) Relative Vorticity, and (d) Strain. 
Each column in the subsequent rows showcase the following reconstructed field from the NEMO simulation found in columrn (a): (b) MIOST~\cite{MIOST}, (c) BFN-QG~\cite{BFNQG}, and (d) 4DVarNet~\cite{4DVARNETSWOT}.}
\label{fig:oceanbench_maps}
\vspace{-5mm}
\end{center}
\end{figure}


% \subsubsection{Observing System Experiments (OSE)} \label{sec:ose}

\textbf{Observing System Experiments (OSE)}. As more observations have become available over the past few decades, we can also design experiments using real data. 
This involves aggregating as many observations from real ocean altimetry satellites as possible with some specific independent subset left out for evaluation purposes.
A major downside to OSE experiments is that the sparsity and spatial coverage of the observations narrow the possible scope of performance metrics and make it very challenging to learn directly from observation datasets. 
The current standard altimetry data are high resolution but cover a tiny area. 
As such, it can only inform fine-scale SSH patterns in the along-track satellite direction and cannot explicitly reveal two-dimensional patterns. 
Despite these drawbacks, it provides a quantitative evaluation of the generalizability of the ML methods concerning the true ocean state.
%and so it fails to capture many of the dynamical regimes we are interested in, i.e. mesoscale and sub-mesoscale processes. 
%However, it is still advantageous (and preferable) to include these experiments because these reflect the true ocean state and will help with the generalizability of the ML methods.

\begin{figure}[t!]
\small
\begin{center}
\setlength{\tabcolsep}{2pt}
\begin{tabular}{ccc}
% $\mathcal X$ & $\hat \z = \bG_\theta(\x)$ & $\x = \bG_\theta^{-1} (\hat \z)$\\[0mm]
% NATL60&
% \multicolumn{2}{c}{\includegraphics[width=6.25cm,height=4.5cm]{content/figures/exp_natl60/psd_st/osse_2020a_psd_natl60}} 
% \\
\includegraphics[width=3.75cm,height=3.25cm]{00_Oceanbench/content/figures/stats/nrmse_space.png} &
\includegraphics[width=4.25cm,height=3.5cm]{00_Oceanbench/content/figures/psd_isotropic/dc20a/nadir4/dc20a_psd_iso_ssh.png} &
\includegraphics[width=4.25cm,height=3.5cm]{00_Oceanbench/content/figures/psd_isotropic/dc20a/nadir4/dc20a_psd_score_iso_ssh.png} 
\\
(a) Normalized RMSE &
(b) Isotropic Power Spectrum &
(c) Isotropic Power Spectrum Score
\end{tabular}
\begin{tabular}{cccc}
\includegraphics[trim={0 0 0mm 0},clip, width=4.20cm,height=3cm]{00_Oceanbench/content/figures/psd_spacetime/dc20a/nadir4/dc20a_psd_spacetime_nemo_nadir4_ssh.png}  &
\includegraphics[trim={20mm 0 34mm 0},clip, width=2.9cm,height=3cm]{00_Oceanbench/content/figures/psd_spacetime/dc20a/nadir4/dc20a_psd_spacetime_score_miost_nadir4_ssh.png} &
\includegraphics[trim={20mm 0 34mm 0},clip, width=2.9cm,height=3cm]{00_Oceanbench/content/figures/psd_spacetime/dc20a/nadir4/dc20a_psd_spacetime_score_bfnqg_nadir4_ssh.png} &
\includegraphics[trim={20mm 0 0 0},clip, width=3.5cm,height=3cm]{00_Oceanbench/content/figures/psd_spacetime/dc20a/nadir4/dc20a_psd_spacetime_score_4dvarnet_nadir4_ssh.png} \\
(d) NEMO Simulation &
(e) MIOST &
(f) BFN-QG &
(g) 4DVarNet
\end{tabular}
% % \vspace{-4mm}
% \caption{Row I - Isotrophic PSD. Row 2 - Isotrophic PSD Score}
\caption{This figure showcases some statistics for evaluation of the SSH field reconstructions for the OSSE NADIR experiment outlined in section~\ref{sec:interp_challenge}. Subfigure (a) showcases the normalized root mean squared error (nRMSE), (b) showcases the isotropic power spectrum decomposition (PSD), (c) showcases isotropic PSD scores.
The bottom row showcases the space-time PSD for the NEMO simulation (subfigure (d)) and the PSD scores for three reconstruction models: (e) the MIOST model~\cite{MIOST}, (f) the BFN-QG model~\cite{BFNQG}, and (g) the 4DVarNet model~\cite{4DVARNETSWOT}.
}
% \vspace{-5mm}
\label{fig:oceanbench_psd}
\end{center}
\end{figure}


%
\subsection*{Data Challenges} \label{sec:data_challenges}



We rely on existing OSSE and OSE experiments for SSH interpolation designed by domain experts~\cite{DCOSEGULFSSH,DCOSSEGULFSSH} and recast them into \texttt{OceanBench} framework to deliver a ML-ready benchmarking suites. 
The selected data challenges  for this first edition address SSH interpolation for a 1000km$\times$1000km Gulfstream region. We briefly outline them below.
%The \textit{Ocean-Data-Challenge} group has a wide-range of different OSSE and OSE experiments involving SSH interpolation in particular~\cite{DCOSEGULFSSH,DCOSSEGULFSSH}. 
%For this paper, we will focus on a subset and demonstrate some evaluation steps generated from the \texttt{OceanBench} framework.
%We outline an experimental setup SSH reconstruction over the Gulfstream with three OSSE's configurations and one OSE configuration in the next section. For more detailed information about the experimental setups for each configuration, see section~\ref{sec:data_challenges_extended} in the appendix.

% \begin{itemize}
%     \item 
\textbf{Experiment I (\textit{OSSE NADIR})} addresses SSH interpolation using NADIR altimetry tracks which are very fine, thin ocean satellite observations (see Figure~\ref{fig:oceanbench_maps}). It relies on an OSSE using high-resolution ($1/60^\circ$ resolution) ocean simulations generated by the NEMO model over one year with a whole field every day. 

%The observation data uses NADIR altimetry tracks which are very fine, thin ocean satellite observations (see Figure~\ref{fig:oceanbench_maps} (a) \& (b)). 
\textbf{Experiment II (\textit{OSSE SWOT})} addresses SSH interpolation using jointly NADIR and SWOT altimetry data where we complement the \textbf{OSSE NADIR} configuration with simulated SWOT observations.
SWOT is a new satellite altimetry mission with a much higher spatial coverage but a much lower temporal resolution as illustrated in Figure~\ref{fig:oceanbench_maps}.
The higher spatial resolution allows us to see structures at a smaller resolution but at the cost of a massive influx of observations (over $\times$100).

\textbf{Experiment III (\textit{OSSE SST})} addresses SSH interpolation using altimetry and SST satellite data jointly. We complement the \textbf{OSSE SWOT} challenge with simulated SST observations. 
Satellite-derived SST observations are more abundantly available in natural operational settings than SSH at a finer resolution, and structures have visible similarities~\cite{SWOT,BFNQG}.
So this challenge allows for methods to take advantage of multi-modal learning~\cite{4DVARNETSST,SSHInterpAttention}.

\textbf{Experiment IV (\textit{OSE NADIR})} addresses SSH interpolation for real NADIR altimetry data. 
In contrast to the three OSSE data challenges, it only looks at actual observations aggregated from the currently available ocean altimetry data from actual satellites. 
It involves a similar space-time sampling as Experiment (\textbf{OSSE NADIR}) to evaluate the generalization of ML methods trained in Experiment I to real altimetry data. 
The training problem's complexity increases significantly due to the reference dataset's sparsity compared with the \textbf{OSSE NADIR} dataset. 
One may also explore transfer learning or fine-tuning strategies from the available OSSE dataset. 


% \begin{itemize}
%     \item Experiment I (\textbf{OSSE NADIR}) addresses SSH interpolation using NADIR altimetry tracks which are very fine, thin ocean satellite observations (see Figure~\ref{fig:oceanbench_maps}). It relies on an OSSE using anocean simulations generated by the NEMO model, more precisely a high-resolution simulation, $1/60^\circ$ resolution, over one year with a whole field every day. 
% %The observation data uses NADIR altimetry tracks which are very fine, thin ocean satellite observations (see Figure~\ref{fig:oceanbench_maps} (a) \& (b)). 
% \item Experiment II (\textbf{OSSE SWOT}) addresses SSH interpolation using jointly NADIR and SWOT altimetry data. We complement the \textbf{OSSE NADIR} configuration with simulated SWOT observations.  
% SWOT is a new satellite altimetry mission with  much higher spatial coverage but a much lower temporal resolution as illustrated in Figure~\ref{fig:oceanbench_maps}.
% The higher spatial resolution allows us to see structures at a smaller resolution but at the cost of a massive influx of observations (over $\times 100$).
% \item Experiment III (\textbf{OSSE SST}) addresses SSH interpolation using 
% jointly altimetry and SST satellite data. We complement the \textbf{OSSE SWOT} challenge with simulated SST observations. 
% Satellite-derived SST observations are more abundantly available in natural operational settings than SSH at a finer resolution, and structures have visible similarities \cite{}.
% So this challenge allows for methods to take advantage of multi-modal learning \cite{}.
% \item Experiment IV \textbf{OSE NADIR} addresses SSH interpolation for real NADIR altimetry data. In contrast to the three OSSE data challenges, it only looks at actual observations aggregated from the currently available ocean altimetry data from actual satellites. It involves a similar space-time sampling as Experiment (\textbf{OSSE NADIR}) to evaluate the generalization of ML methods trained in Experiment I to real altimetry data. Besides, one may also explore learning or fine-tuning strategies from the available OSE dataset. We may point out that the complexity of the training problem due to the sparsity of refence dataset compared with \textbf{OSSE NADIR} dataset.
% \end{itemize}




\subsection*{\texttt{OceanBench} Pipelines}

\begin{table}[h]
\caption{This table highlights some of the results for the \textbf{OSSE NADIR} experiment outlined in section~\ref{sec:data_challenges} and appendix~\ref{sec:data_challenges_extended}.
% and the OSE experiment outlined in section~\ref{sec:ose}~\tocite{}. 
% For more results regarding the SWOT data, please see section~\ref{sec:other_tasks}. 
This table highlights the performance statistically in the real and spectral space; the normalized RMSE score for the real space and the minimum spatial and temporal scales resolved in the spectral domain. 
For more information about the class of models displayed and class of metrics, see appendix~\ref{sec:ml_ontology} and appendix~\ref{sec:metrics} respectively. We only showcase the model performance on the alongtrack NADIR data available. For the extended table for each of the challenges, see Table~\ref{tb:exp-results-mega}.}
\label{tb:oceanbench_results}
\centering
\begin{tabular}{lllcccc}
 \toprule
% Experiment & Configuration & Method & nRMSE & Resolved Scale [km]    \\ \midrule
% \multirow{2}{*}{Experiment} & \multirow{2}{*}{Algorithm} & \multirow{2}{*}{Algorithm Class} & \multirow{2}{*}{nRMSE} & \multicolumn{2}{c}{Effective Resolution} \\ 
% &  &   &  & Wavelength [km]  & Period [days]      \\ \midrule
% \multirow{2}{*}{Experiment} & \multirow{2}{*}{Algorithm} & \multirow{2}{*}{Algorithm Class} & \multirow{2}{*}{nRMSE} & \multicolumn{2}{c}{Effective Resolution} \\ 
Experiment &  Algorithm &   Algorithm Class &  nRMSE Score & $\lambda_{\mathbf{x}}$ [km]  & $\lambda_{t}$ [days]      \\ \midrule
\multicolumn{1}{l}{OSSE NADIR}     &  OI~\cite{DUACS} &  Coordinate-Based & 0.92 $\pm$ 0.01 & 175 & 10.8 \\
\multicolumn{1}{l}{OSSE NADIR}     &  MIOST~\cite{MIOST} &  Coordinate-Based  & 0.93 $\pm$ 0.01 & 157 & 10.1 \\
\multicolumn{1}{l}{OSSE NADIR}     &  BFNQG~\cite{BFNQG} &  Hybrid Model   & 0.93 $\pm$ 0.01 & 139 & 10.6 \\
OSSE NADIR &  4DVarNet~\cite{4DVARNETSWOT} &  Bi-Level Opt.  & 0.95 $\pm$ 0.01 & 117 & 7.7 \\
\bottomrule
\end{tabular}
\end{table}



For the four data challenges presented in the previous section, we used \texttt{OceanBench} pipelines to deliver a ML-ready benchmarking framework.
We used the \texttt{hydra} and the geoprocessing tools outlined in section~\ref{sec:code_structure} with specialized routines for regridding the ocean satellite data to a uniformly gridded product and vice versa when necessary. 
Appendix~\ref{sec:hydra_recipes} showcases an example of the hydra integration for the preprocessing pipeline. 
A key feature is the creation of a custom patcher for the appropriate geophysical variables using our \texttt{XRPatcher} tool, which is later integrated into custom datasets and dataloaders for the appropriate model architecture, e.g., coordinate-based or grid-based. 
We provide an example snippet of how this can be done easily in section~\ref{sec:xrpatcher}.
\texttt{OceanBench} also features some tools specific to the analysis of SSH. 
For example, physically-interpretable variables like geostrophic currents and relative vorticity, which can be derived from first-order and second-order derivatives of the SSH, are essential for assessing the quality of the reconstructions generated by the models. 
Figure~\ref{fig:oceanbench_maps} showcases some fields of the most common physical variables used in the oceanography literature for the SSH-based analysis of sea surface dynamics. For more details regarding the nature of the physical variables, see appendix~\ref{sec:physical_variables}.



Regarding the evaluation framework, we include domain-relevant performance metrics beyond the standard ML loss and accuracy functions. They account for the sampling patterns of the evaluation data. Spectral analytics are widely used in geoscience~\cite{BFNQG}, and here, we consider spectral scores computed as the minimum spatial and temporal scales resolved by the reconstruction methods proposed in~\cite{BFNQG}.
For example, figure~\ref{fig:oceanbench_psd} showcases how \texttt{OceanBench} generated the isotropic power spectrum and score and the space-time power spectrum decomposition and score.
Table~\ref{tb:oceanbench_results} outlines some standard and domain-specific scores for the experiments outlined in section~\ref{sec:experimental_design}.
We give a more detailed description of the rationale and construction of the power-spectrum-specific metrics in appendix~\ref{sec:metrics}. In terms of baselines, we report for each data challenge the performance of at least one approach for each of the category outlined in Section \ref{sec:ml_ontology_mini}.


% \begin{table}[h]
% \caption{This table highlights some of the results for the \textbf{OSSE NADIR} experiment outlined in section~\ref{sec:data_challenges} and appendix~\ref{sec:data_challenges_extended}.
% % and the OSE experiment outlined in section~\ref{sec:ose}~\tocite{}. 
% % For more results regarding the SWOT data, please see section~\ref{sec:other_tasks}. 
% This table highlights the performance statistically in the real and spectral space; the normalized RMSE for the real space and the minimum spatial and temporal scales resolved in the spectral domain. 
% For more information about the class of models displayed and class of metrics, see appendix~\ref{sec:ml_ontology} and appendix~\ref{sec:metrics} respectively. We only showcase the model performance on the alongtrack NADIR data available. For the extended table for each of the challenges, see Table~\ref{tb:exp-results-mega}.}
% \label{tb:oceanbench_results}
% \centering
% \begin{tabular}{lllcccc}
%  \toprule
% % Experiment & Configuration & Method & nRMSE & Resolved Scale [km]    \\ \midrule
% % \multirow{2}{*}{Experiment} & \multirow{2}{*}{Algorithm} & \multirow{2}{*}{Algorithm Class} & \multirow{2}{*}{nRMSE} & \multicolumn{2}{c}{Effective Resolution} \\ 
% % &  &   &  & Wavelength [km]  & Period [days]      \\ \midrule
% % \multirow{2}{*}{Experiment} & \multirow{2}{*}{Algorithm} & \multirow{2}{*}{Algorithm Class} & \multirow{2}{*}{nRMSE} & \multicolumn{2}{c}{Effective Resolution} \\ 
% Experiment &  Algorithm &   Algorithm Class &  nRMSE & $\lambda_{\mathbf{x}}$ [km]  & $\lambda_{t}$ [days]      \\ \midrule
% \multicolumn{1}{l}{OSSE NADIR}     &  OI (app.~\ref{sec:oi}) &  Coordinate-Based & 0.91 $\pm$ 0.01 & 176 & 11.6\\
% \multicolumn{1}{l}{OSSE NADIR}     &  MIOST~\ &  Coordinate-Based  & 0.92 $\pm$ 0.01 & 157 & 10.3 \\
% % \multicolumn{1}{l}{OSSE Gulf}     &  NerF &  Coordinate-Based  & 0.92 $\pm$ 0.01 & ... &\\
% \multicolumn{1}{l}{OSSE NADIR}     &  BFNQG (app.~\ref{sec:bfn}) &  Hybrid Model   & 0.92 $\pm$ 0.01 & 139 & 10.6 \\
% OSSE NADIR &  4DVarNet (app.~\ref{sec:4dvarnet}) &  Bi-Level Opt.  & 0.95 $\pm$ 0.01 & 117 & 7.7 \\
% % \bottomrule
% % OSE Gulf     &  OI &  Coordinate-Based  & ... & ... &\\
% % \multicolumn{1}{l}{OSE Gulf}     &  MIOST &  Coordinate-Based  & ... & ... &\\
% % \multicolumn{1}{l}{OSE Gulf}     &  NerF &  Coordinate-Based  & ... & ... &\\
% % \multicolumn{1}{l}{OSE Gulf}     &  BFNQG &  Hybrid Model  & ... & ... &\\
% % \multicolumn{1}{l}{OSE Gulf}     &  4DVarNet &  Bi-Level Opt.  & ... & ... &\\
% \bottomrule
% \end{tabular}
% \end{table}


\section{Conclusions} \label{sec:conclusions}

The ocean community faces technological and algorithmic challenges to make the most of available observation and simulation datasets. 
In this context, recent studies evidence the critical role of ML schemes in reaching breakthroughs in our ability to monitor ocean dynamics for various space-time scales and processes. 
Nevertheless, domain-specific preprocessing steps and evaluation procedures slow down the uptake of ML toward real-world applications.

Through \texttt{OceanBench} framework, we embed domain-level requirements into the MLOPs considerations by building a flexible framework that adds this into the hyperparameter considerations for ML models. 
We proposed four challenges towards a ML-ready benchmarking suite for ocean observation challenges. 
We outlined the inner workings \texttt{OceanBench} and demonstrated its usefulness by recreating some preprocessing and analysis pipelines from a few data challenges involving SSH interpolation.
We firmly believe that the \texttt{OceanBench} platform is a crucial step to lowering the barrier of entry for new ML researchers interested in applying and developing their methods to relevant problems in the ocean sciences.



%% ACKNOWLEDGEMENTS
\newpage
% \input{./00_Oceanbench/content/acknowledgements}
% \section*{Checklist}

% %%% BEGIN INSTRUCTIONS %%%
% The checklist follows the references.  Please
% read the checklist guidelines carefully for information on how to answer these
% questions.  For each question, change the default \answerTODO{} to \answerYes{},
% \answerNo{}, or \answerNA{}.  You are strongly encouraged to include a {\bf
% justification to your answer}, either by referencing the appropriate section of
% your paper or providing a brief inline description.  For example:
% \begin{itemize}
%   \item Did you include the license to the code and datasets? \answerYes{See Section~\ref{gen_inst}.}
%   \item Did you include the license to the code and datasets? \answerNo{The code and the data are proprietary.}
%   \item Did you include the license to the code and datasets? \answerNA{}
% \end{itemize}
% Please do not modify the questions and only use the provided macros for your
% answers.  Note that the Checklist section does not count towards the page
% limit.  In your paper, please delete this instructions block and only keep the
% Checklist section heading above along with the questions/answers below.
% %%% END INSTRUCTIONS %%%

\begin{enumerate}

\item For all authors...
\begin{enumerate}
  \item Do the main claims made in the abstract and introduction accurately reflect the paper's contributions and scope?
    \answerYes{All the contributions listed in the abstract are elaborated in sections~\ref{sec:code_structure},~\ref{sec:data_challenges} and~\ref{sec:conclusions}}
  \item Did you describe the limitations of your work?
    \answerYes{See the last paragraph of section 5 and the appendix as well.}
  \item Did you discuss any potential negative societal impacts of your work?
    \answerYes{We do not believe that our work has any potential negative societal impacts directly as we do not deal with any confidential or private data. However, we do outline in the appendix how there may be some adverse effects related to downstream uses which could have some negative societal impacts.}
  \item Have you read the ethics review guidelines and ensured that your paper conforms to them?
    \answerYes{We do not include any confidential or private data. We only include numerical values which stem from general physical systems or machine learning models. We do not believe they hold any ethical issues. However, we do acknowledge that there would be environmental damage should users go forward and explore methods which obscenely high computing hours. This discussion outlined in the appendix.}
\end{enumerate}

\item If you are including theoretical results...
\begin{enumerate}
  \item Did you state the full set of assumptions of all theoretical results?
    \answerNA{We do not include any theoretical results.}
	\item Did you include complete proofs of all theoretical results?
    \answerNA{We do not include any theoretical results.}
\end{enumerate}

\item If you ran experiments (e.g. for benchmarks)...
\begin{enumerate}
  \item Did you include the code, data, and instructions needed to reproduce the main experimental results (either in the supplemental material or as a URL)?
    \answerYes{We include the parameters used to reproduce the dataset preprocessing and evaluation procedure in Appendix \ref{sec:data_challenges_extended} and instructions are given to download the data via~\href{https://github.com/quentinf00/oceanbench-data-registry}{https://github.com/quentinf00/oceanbench-data-registry} and rerun the evaluation procedure in our code repository which is available at~\href{https://github.com/jejjohnson/oceanbench}{https://github.com/jejjohnson/oceanbench}.}
  \item Did you specify all the training details (e.g., data splits, hyperparameters, how they were chosen)?
    \answerYes{We showcase all preprocessing steps necessary to reproduce the experimental configurations in Appendix~\ref{sec:data_challenges_extended} and the configuration files are available in our code repository at~\href{https://github.com/jejjohnson/oceanbench}{https://github.com/jejjohnson/oceanbench}. }
	\item Did you report error bars (e.g., with respect to the random seed after running experiments multiple times)?
    \answerNA{This is not applicable for this instantiation because we do not include any randomness within the experiment procedure nor the results.}
	\item Did you include the total amount of compute and the type of resources used (e.g., type of GPUs, internal cluster, or cloud provider)?
    \answerYes{We do not do any model training and leave it up the user for their local or cloud machine. However, we do provide the cloud provider for the data found the the data registry which can be found at~\href{https://github.com/quentinf00/oceanbench-data-registry}{https://github.com/quentinf00/oceanbench-data-registry}}
\end{enumerate}

\item If you are using existing assets (e.g., code, data, models) or curating/releasing new assets...
\begin{enumerate}
  \item If your work uses existing assets, did you cite the creators?
    \answerYes{We adopted the implementation of the preprocessing procedures and evaluation steps with some modifications. We give proper citation and credit to the authors as well as all other existing software packages included in this work.}
  \item Did you mention the license of the assets?
    \answerYes{The appropriate license notices are included in the source code files.}
  \item Did you include any new assets either in the supplemental material or as a URL?
    \answerYes{All the processing and evaluation scripts are included in the GitHub repository.}
  \item Did you discuss whether and how consent was obtained from people whose data you're using/curating?
    \answerYes{We only include data that is already publicly available. We also discussed with the original generators of the datasets and keep the appropriate licenses.}
  \item Did you discuss whether the data you are using/curating contains personally identifiable information or offensive content?
    \answerNA{We do not include any personal information or offensive content in our datasets.}
\end{enumerate}

\item If you used crowdsourcing or conducted research with human subjects...
\begin{enumerate}
  \item Did you include the full text of instructions given to participants and screenshots, if applicable?
    \answerNA{We do not use crowdsourcing and we do not conduct research with human subjects.}
  \item Did you describe any potential participant risks, with links to Institutional Review Board (IRB) approvals, if applicable?
    \answerNA{See the previous point.}
  \item Did you include the estimated hourly wage paid to participants and the total amount spent on participant compensation?
    \answerNA{See the previous point.}
\end{enumerate}

\end{enumerate}

% \bibliographystyle{plain}
% \nocite{*}
% % \documentclass{article}



% % if you need to pass options to natbib, use, e.g.:
%     \PassOptionsToPackage{numbers, compress}{natbib}
% before loading neurips_data_2023

% ready for submission
% \usepackage{neurips_data_2023}
% \usepackage[preprint]{./00_Oceanbench/neurips_data_2023}

% to compile a preprint version, add the [preprint] option, e.g.:
%     \usepackage[preprint]{neurips_data_2023}
% This will indicate that the work is currently under review.

% to compile a camera-ready version, add the [final] option, e.g.:
%     \usepackage[final]{neurips_data_2023}

% to avoid loading the natbib package, add option nonatbib:
%    \usepackage[nonatbib]{neurips_data_2023}

% Submissions to the datasets and benchmarks are typically non anonymous,
% but anonymous submissions are allowed. If you feel that you must submit 
% anonymously, you can compile an anonymous version by adding the [anonymous] 
% option, e.g.:
%     \usepackage[anonymous]{neurips_data_2023}
% This will hide all author names.

\usepackage{color}
\usepackage{soul}
\usepackage[dvipsnames]{xcolor}
% \usepackage[normalem]{ulem}
\newcommand{\todo}[1]{\textcolor{BrickRed}{[\textbf{TODO}]} }
\newcommand{\tocite}[1]{\textcolor{Plum}{[\textbf{CITE}: #1]}}
\newcommand{\hlc}[2][yellow]{{%
    \colorlet{foo}{#1}%
    \sethlcolor{foo}\hl{#2}}%
}
\newcommand{\tofix}[1]{\hlc[red!20]{#1}}

\usepackage[utf8]{inputenc} % allow utf-8 input
\usepackage[T1]{fontenc}    % use 8-bit T1 fonts
\usepackage{hyperref}       % hyperlinks
\usepackage{url}            % simple URL typesetting
\usepackage{booktabs}       % professional-quality tables
\usepackage{nicefrac}       % compact symbols for 1/2, etc.
\usepackage{microtype}      % microtypography
\usepackage{xcolor}         % colors
\usepackage{multirow}
\usepackage{graphicx}
%% MATH PACKAGES
\usepackage{amsfonts}       % blackboard math symbols
\usepackage{amsmath}
\usepackage{nicefrac} 
\usepackage{lscape}


\usepackage{listings}
% \usepackage{minted}
% \usepackage[cachedir=.]{minted}
% \usepackage{frozencache}
% \usepackage[frozencache=true,cachedir=minted-cache]{minted} 
\usepackage[finalizecache=true,cachedir=minted-cache]{minted}
% \usepackage[cachedir=.]{minted}

% \usepackage{appendix}

% here is a macro expanding to the name of the language
% (handy if you decide to change it further down the road)

% \newcommand\YAMLcolonstyle{\color{red}\mdseries}
\newcommand\YAMLkeystyle{\color{black}\bfseries}
\newcommand\YAMLvaluestyle{\color{blue}\mdseries}

\makeatletter

% here is a macro expanding to the name of the language
% (handy if you decide to change it further down the road)
\newcommand\language@yaml{yaml}

\expandafter\expandafter\expandafter\lstdefinelanguage
\expandafter{\language@yaml}
{
  keywords={true,false,null,y,n},
  keywordstyle=\color{darkgray}\bfseries,
  basicstyle=\YAMLkeystyle,                                 % assuming a key comes first
  sensitive=false,
  comment=[l]{\#},
  morecomment=[s]{/*}{*/},
  commentstyle=\color{purple}\ttfamily,
  stringstyle=\YAMLvaluestyle\ttfamily,
  moredelim=[l][\color{orange}]{\&},
  moredelim=[l][\color{magenta}]{*},
  moredelim=**[il][\YAMLcolonstyle{:}\YAMLvaluestyle]{:},   % switch to value style at :
  morestring=[b]',
  morestring=[b]",
  literate =    {---}{{\ProcessThreeDashes}}3
                {>}{{\textcolor{red}\textgreater}}1     
                {|}{{\textcolor{red}\textbar}}1 
                {\ -\ }{{\mdseries\ -\ }}3,
}

% switch to key style at EOL
\lst@AddToHook{EveryLine}{\ifx\lst@language\language@yaml\YAMLkeystyle\fi}
\makeatother

\newcommand\ProcessThreeDashes{\llap{\color{cyan}\mdseries-{-}-}}


\definecolor{codegreen}{rgb}{0,0.6,0}
\definecolor{codegray}{rgb}{0.5,0.5,0.5}
\definecolor{codepurple}{rgb}{0.58,0,0.82}
\definecolor{backcolour}{rgb}{0.95,0.95,0.92}

\lstdefinestyle{pythonstyle}{
    % backgroundcolor=\color{backcolour},   
    commentstyle=\color{codegreen},
    keywordstyle=\color{magenta},
    numberstyle=\tiny\color{codegray},
    stringstyle=\color{codepurple},
    basicstyle=\ttfamily\footnotesize,
    breakatwhitespace=false,         
    breaklines=true,                 
    captionpos=b,                    
    keepspaces=true,                 
    numbers=left,                    
    numbersep=5pt,                  
    showspaces=false,                
    showstringspaces=false,
    showtabs=false,                  
    tabsize=2
}

\lstset{style=pythonstyle}


% \title{\textsc{OceanBench}: \\ The Sea Surface Height Edition}

% % % The \author macro works with any number of authors. There are two commands
% used to separate the names and addresses of multiple authors: \And and \AND.
%
% Using \And between authors leaves it to LaTeX to determine where to break the
% lines. Using \AND forces a line break at that point. So, if LaTeX puts 3 of 4
% authors names on the first line, and the last on the second line, try using
% \AND instead of \And before the third author name.

\author{%
  J. Emmanuel Johnson$^*$\\
%   MEOM/Institut des Géosciences de l’Environnement, UGA\\
  % Univ. Grenoble Alpes, CNRS UMR IGE, Grenoble, France\\
  CNRS UMR IGE \\
  \texttt{johnsonj@univ-grenoble-alpes.fr}\\
  % examples of more authors
  \And
  Quentin Febvre$^*$\\
  IMT Atlantique\\
  \texttt{quentin.febvre@imt-atlantique.fr} \\
  \And
  Anastasia Gorbunova\\
  CNRS UMR IGE \\
  % \texttt{gorbunoa@univ-grenoble-alpes.fr} \\
  \And
  Sammy Metref\\
  DATLAS\\
  % \texttt{metrefs@univ-grenoble-alpes.fr}\\
  \And
  Maxime Ballarotta\\
  CLS\\
  % \texttt{mballarotta@groupcls.com} \\
  \And
  Julien Le Sommer\\
  CNRS UMR IGE\\
  % \texttt{julien.lesommer@univ-grenoble-alpes.fr} \\
  \And
  Ronan Fablet\\
  IMT Atlantique \\
  % \texttt{ronan.fablet@imt-atlantique.fr} \\
}


% \author{%
%   David S.~Hippocampus\thanks{Use footnote for providing further information
%     about author (webpage, alternative address)---\emph{not} for acknowledging
%     funding agencies.} \\
%   Department of Computer Science\\
%   Cranberry-Lemon University\\
%   Pittsburgh, PA 15213 \\
%   \texttt{hippo@cs.cranberry-lemon.edu} \\
%   % examples of more authors
%   % \And
%   % Coauthor \\
%   % Affiliation \\
%   % Address \\
%   % \texttt{email} \\
%   % \AND
%   % Coauthor \\
%   % Affiliation \\
%   % Address \\
%   % \texttt{email} \\
%   % \And
%   % Coauthor \\
%   % Affiliation \\
%   % Address \\
%   % \texttt{email} \\
%   % \And
%   % Coauthor \\
%   % Affiliation \\
%   % Address \\
%   % \texttt{email} \\
% }


% \begin{document}




% \maketitle
% \def\thefootnote{*}\footnotetext{These authors contributed equally to this work}\def\thefootnote{\arabic{footnote}}
\begin{bibunit}[IEEEtran.bst]

\chapter*{\textsc{OceanBench}: \\ The Sea Surface Height Edition}
\addcontentsline{toc}{chapter}{\textsc{OceanBench}: \\ The Sea Surface Height Edition}
\chaptermark{\textsc{OceanBench}: \\ The Sea Surface Height Edition}
%##################################
% KEYWORDS
% - Sea Surface Height
% - Interpolation
% - Inverse Problem
% - 
% Scientific Machine Learning
% Benchmark, Partial Differential Equations, PINN, FNO, U-Net, Inverse problem
% \linenumbers
% % \begin{abstract}
% The ocean is a key component of the climate system, profoundly influencing human activities. Consequently, a central question for oceanographers revolves around its observation and prediction at different timescales. Similar to various research fields, oceanography has recently witnessed the emergence of machine learning-based methods as an alternative to more domain-specific approaches. But, surprisingly, the adoption of machine learning methods in this field has been arguably slower compared to others. This can be attributed to the complexity of pre-processing tasks in the domain of geospatial data and to the specialized nature of evaluation metrics relevant to ocean science problems. In this context, we introduce \texttt{OceanBench}, a framework designed to lower the barrier to entry for ML researchers by standardizing and automating this process that comply with domain-expert standards.
% It seeks to maintains readability, robustness, and flexibility by providing a standardized framework to explicitly highlight the preprocessing step used, the parameters chosen and the sequence of operations performed. 
% It takes a step towards incorporate these domain-driven decisions to integrate well with existing ML pipelines.
% We demonstrate the usefulness of \texttt{OceanBench} on a series of SSH interpolation tasks with consistent dataflow pipelines, physically-inspired metrics, and interpretable visualizations.

% \end{abstract}

  % The abstract paragraph should be indented \nicefrac{1}{2}~inch (3~picas) on
  % both the left- and right-hand margins. Use 10~point type, with a vertical
  % spacing (leading) of 11~points.  The word \textbf{Abstract} must be centered,
  % bold, and in point size 12. Two line spaces precede the abstract. The abstract
  % must be limited to one paragraph.

\begin{abstract}

The ocean is a crucial component of the Earth's system. 
It profoundly influences human activities and plays a critical role in climate regulation. 
Our understanding has significantly improved over the last decades with the advent of satellite remote sensing data, allowing us to capture essential sea surface quantities over the globe, e.g., sea surface height (SSH). 
Despite their ever-increasing abundance, ocean satellite data presents challenges for information extraction due to their sparsity and irregular sampling, signal complexity, and noise. 
Machine learning (ML) techniques have demonstrated their capabilities in dealing with large-scale, complex signals. 
Therefore we see an opportunity for these ML models to harness the full extent of the information contained in ocean satellite data. 
However, data representation and relevant evaluation metrics can be \textit{the} defining factors when determining the success of applied ML. 
The processing steps from the raw observation data to a ML-ready state and from model outputs to interpretable quantities require domain expertise, which can be a significant barrier to entry for ML researchers. 
In addition, imposing fixed processing steps, like committing to specific variables, regions, and geometries, will narrow the scope of ML models and their potential impact on real-world applications. 
\textbf{OceanBench} is a unifying framework that provides standardized processing steps that comply with domain-expert standards. 
It is designed with a flexible and pedagogical abstraction: it a) provides plug-and-play data and pre-configured pipelines for ML researchers to benchmark their models w.r.t. ML and domain-related baselines and b) provides a transparent and configurable framework for researchers to customize and extend the pipeline for their tasks. 
In this work, we demonstrate the \texttt{OceanBench} framework through a first edition dedicated to SSH interpolation challenges. 
We provide datasets and ML-ready benchmarking pipelines for the long-standing problem of interpolating observations from simulated ocean satellite data, multi-modal and multi-sensor fusion issues, and transfer-learning to real ocean satellite observations. 
The \texttt{OceanBench} framework is available at~\href{https://github.com/jejjohnson/oceanbench}{github.com/jejjohnson/oceanbench} and the dataset registry is available at~\href{https://github.com/quentinf00/oceanbench-data-registry}{github.com/quentinf00/oceanbench-data-registry}.

\end{abstract}


%%%%%%%%%%%%%%%%%%%%%%%%%%%%%%%%%%%%%%%%%%%%%%%%%%%%%%%%%%%%%%%%%%%%%%%%%%%%%%%%%%%%%%%%%%%%%%%%%%%%%%%%%%%%%%%%%%%%%%%%%%%%%%%%%%%%%%%%%%%%%%%%%

% - We demonstrate how \texttt{OceanBench} can be used as an underlying framework with a series of SSH interpolation tasks which combines different variables, regions and geometries. 


% Q - make use of everything we mentioned above!
% E/Q - we want to showcase that OceanBench can be used for SSH interpolation 
% - we want to demonstrate how we can design different experiments/modifications with minimal code changes
% Q - simple tasks SSH -> SSH; extend it a) add domain, b) include other variables, c) switching to real data
% - enrich a simple experimental setups with additional variations

%%%%%%%%%%%%%%%%%%%%%%%%%%%%%%%%%%%%%%%%%%%%%%%%%%%%%%%%%%%%%%%%%%%%%%%%%%%%%%%%%%%%%%%%%%%%%%%%%%%%%%%%%%%%%%%%%%%%%%%%%%%%%%%%%%%%%%%%%%%%%%%%%



%%%%%%%%%%%%%%%%%%%%%%%%%%%%%%%%%%%%%%%%%%%%%%%%%%%%%%%%%%%%%%%%%%%%%%%%%%%%%%%%%%%%%%%%%%%%%%%%%%%%%%%%%%%%%%%%%%%%%%%%%%%%%%%%%%%%%%%%%%%%%%%%%

% - There is some work to be done to get the observation data to a ML-ready state.
% - Deciding the appropriate transformations requires domain-expertise which can be a large barrier to entry for ML researchers.

% - ML methods cannot be applied out of the box on raw observation data because .. blah blah

% - Despite the potential of machine learning to solve the issues that plague geosciences, the barrier to entry for machine learning (ML) researchers in ocean sciences is often hindered by the vastly different schemes for preprocessing techniques required to get the geo-centric data to a ML-ready state.
% \texttt{OceanBench} is a framework for co-designing machine learning-driven high level data products from ocean observations. 
%%%%%%%%%
% Why dont we just provide the processed data and call it a day...?
% Q - 1) non-standardized preprocessing steps; 2) regional differences
% 1 - multiple ways to take advantage of the Obs data (covariates, coordinate-based, gridded)
% Keeping the flexibilities allow for extensibility and multiple 
%%%%%%%%%%%%$
% 1 - OceanBench is a unifying framework that provides standardized processing steps that comply with domain-expert standards.
% 2 - It is designed for - flexible, padagological, abstract, :
% a - getting ML researchers quickly involved (data + preconfigured pipelines -> ML ready setups and test their models)
% - provides data and preconfigured pipelines for ML researchers to quickly experiment with different setups,
% - transparent and configurable for researchers to customize and extend the pipeline for their own objectives
% b - pedagological, unified setup to create and modified tasks (extensible)
% - flexibility
% 3 - We demonstrate ho
% %
% \texttt{OceanBench} is a framework that standardizing ... these processing steps that comply with domain-expert standards.
% %
% It is designed to lower the barrier to entry for ML researchers by providing preconfigured pipelines, ... "one-stop-shop for it all", and providing illustrative examples for different audiences.
% * 
% * provide a common interface for the processing steps
% * create a few ML tasks from raw data to ML ready using the above steps
% * showcase how one can orchestrate preconfigured pipelines to create a variety of ML tasks
% * preconfigured pipelines to illustrate example applications

% 1 - getting ML researchers quickly involved (data + preconfigured pipelines -> ML ready setups and test their models)
% 2 - pedagological, unified setup to create and modified tasks

% %%%%%
% \texttt{OceanBench} is a framework 
% - lower the barrier to entry
% - accessibility
% - one place to look to get all the stuff you want to do
% - preconfigured pipelines
% - jbooks with illustrated examples.
% - standardized preprocessing steps.
%
% It seeks to maintains readability, robustness, and flexibility by providing a standardized framework to explicitly highlight the preprocessing step used, the parameters chosen and the sequence of operations performed. 
%
% It takes a step towards incorporate these domain-driven decisions to integrate well with existing ML pipelines.
%
% We demonstrate the usefulness of \texttt{OceanBench} on a series of SSH interpolation tasks with 
% consistent dataflow pipelines, physically-inspired metrics, and interpretable visualizations.

The previous chapters have underscored the potential of learning-based methods for developing real-world altimetry solutions. The next chapter delves into what deep learning practitioners can contribute effectively to ocean observation science.

It is my strong belief that the adoption of deep learning approaches necessitates collaboration between the oceanography and machine learning communities. To ensure that these methods provide genuine value, evaluation criteria and metrics must be defined with domain expertise by ocean experts. The quality of SSH estimations, for instance, depends on factors such as geographical region, season, physical plausibility of derived quantities. The choice of using observational or simulated data for metric computation yields different assessments, and these distinctions should be carefully considered.

Evaluation criteria should remain adaptable, especially for deep learning methods, which may optimize training objectives by taking shortcuts (as per the "No Free Lunch" theorem). Consequently, the next chapter introduces a platform that facilitates an ongoing dialogue between deep learning experts and oceanography specialists to develop and assess these methods effectively.

On the data front, the selection of data used to train deep learning models is a critical consideration when shaping the final method. As demonstrated in previous chapters, both observational and simulation data can be instrumental in calibrating and evaluating the method. These data undergo domain-specific processing steps. Accessibility to the data and the relevant processing steps can significantly lower the entry barriers for aspiring machine learning practitioners.

These considerations provide the motivation for the work presented in the next chapter. Oceanbench is a suite of tools designed to load and process ocean data. The software is organized into a series of tasks, each characterized by evaluation data and metrics. Its purpose is to simplify the loading and formatting of data relevant for training neural schemes and to streamline the configuration of new evaluation setups for diverse tasks, regions, and datasets.

\section{Introduction}

The ocean is vital to the Earth's system~\cite{OCEANWARMING}. 
It plays a significant role in climate regulation regarding carbon~\cite{OCEANCARBONCYCLE} and heat uptake~\cite{OCEANHEATUPTAKE}. It is also a primary driver of human activities (e.g., maritime traffic and world trade, marine resources and services)~\cite{SSHOPERATIONAL, ML4OCN}. 
However, monitoring the ocean is a critical challenge: the ocean state can only partially be determined because most of the ocean consists of subsurface quantities that we cannot directly observe. 
Thus, to quantify even a fraction of the physical or biochemical ocean state, we must often rely only on surface quantities that we can monitor from space, drifting buoys, or autonomous devices.
Satellite remote sensing, in particular, is one of the most effective ways of measuring essential sea surface quantities~\cite{Altimetry} such as sea surface height (SSH)~\cite{DUACS}, sea surface temperature (SST)~\cite{OCEANSATELLITESST}, and ocean color ~\cite{OCEANSATELLITEOC}. 
While these variables characterize only a tiny portion of the ocean ecosystem, they present a gateway to many other derived physical quantities~\cite{ML4OCN}.


% satellite-derived SSH data allow us to monitor the sea level rise as well as critical physical and biogeochemical ocean processes,
Although we can access observable sea surface quantities, they can be irregularly and extremely sparsely sampled like the altimetry data considered in previous in this thesis~\cite{DUACS}. 
These sampling gaps make the characterization of ocean processes highly challenging for operational products and downstream tasks that depend on relevant gap-free variables. This has motivated a rich literature in geoscience over the last decades, mainly using geostatistical methods \cite{DUACS, MIOST} and model-driven data assimilation schemes~\cite{BFNQG, GLORYS12}. Despite significant progress, these schemes often need to improve their ability to leverage available observation datasets' potential fully. 
This has naturally advocated for exploring data-driven approaches like shallow ML schemes~\cite{DINEOF, DINEOF2, ANALOGDA2, ANALOGDA}. Very recently, as presented in previous chapters, deep learning schemes \cite{SSHInterpAttention, SSHInterpConvLSTM, SSHInterpUNet} have become appealing solutions to benefit from existing large-scale observation and simulation datasets and reach significant breakthroughs in the monitoring of upper ocean dynamics from scarcely and irregularly sampled observations. 
To ensure that these methods provide genuine value, evaluation criteria and metrics must be defined with domain expertise by ocean experts. The quality of SSH estimations, for instance, depends on factors such as geographical region, season, physical plausibility of derived quantities. The choice of using observational or simulated data for metric computation also yields different assessments


Furthermore, the heterogeneity and characteristics of the observation data present major challenges for effectively applying these methods beyond idealized case studies. 
A data source can have different variables, geometries, and noise levels, resulting in many domain-specific preprocessing procedures that can vastly change the solution outcome. Accessibility to the data and the relevant processing steps can significantly lower the entry barriers for aspiring machine learning practitioners.

 % Oceanbench is a suite of tools designed to load and process ocean data. The software is organized into a series of tasks, each characterized by evaluation data and metrics. Its purpose is to simplify the loading and formatting of data relevant for training neural schemes and to streamline the configuration of new evaluation setups for diverse tasks, regions, and datasets.

% So the entire ML pipeline requires a unified framework for dealing with heterogeneous data sources, different pre- and post-processing methodologies, and regionally-dependent evaluation procedures.

% The previous chapters have underscored the potential of learning-based methods for developing real-world altimetry solutions. The next chapter delves into what deep learning practitioners can contribute effectively to ocean observation science.

% It is my strong belief that the adoption of deep learning approaches necessitates collaboration between the oceanography and machine learning communities. .

% Evaluation criteria should remain adaptable, especially for deep learning methods, which may optimize training objectives by taking shortcuts (as per the "No Free Lunch" theorem). Consequently, the next chapter introduces a platform that facilitates an ongoing dialogue between deep learning experts and oceanography specialists to develop and assess these methods effectively.

% On the data front, the selection of data used to train deep learning models is a critical consideration when shaping the final method. As demonstrated in previous chapters, both observational and simulation data can be instrumental in calibrating and evaluating the method. These data undergo domain-specific processing steps. Accessibility to the data and the relevant processing steps can significantly lower the entry barriers for aspiring machine learning practitioners.




 %So the nature of the observations presents a key challenge for operational products and downstream tasks that depend on relevant gap-free variables.
%In addition, this sparsity 
%Although methods like smoothers~\tocite{OI, MIOST}, data assimilation~\tocite{BFN} and machine learning (ML)~\tocite{NerFs,recent-stuff, Ronan} have produced viable solutions to filling the gaps in ocean satellite, the heterogeneity and characteristics of the observation data present major challenges for effectively applying these methods. 
% No model nor observation systems captures all scales and processes.
% Operational methods like optimal interpolation and data assimilation ...
% Thus, there is a need for better model/data integration schemes that fully benefit from HR models and ever-increasing observation datasets.

% \textbf{Opportunity}.
%An end-to-end framework for piping data from its raw form to an ML-ready state and from model outputs to interpretable quantities is challenging. Designing an effective system that does this is the basis for many MLOPs tools and research~\tocite{MLOPs Research}.
These considerations provide the motivation for the work presented in this chapter. To address these challenges, we introduce \textbf{OceanBench}, a framework for co-designing machine-learning-driven high-level experiments from ocean observations. 
It consists of an end-to-end framework for piping data from its raw form to an ML-ready state and from model outputs to interpretable quantities. 
We regard \texttt{OceanBench} as a key facilitator for the uptake of MLOPs tools and research~\cite{MLOPS1,MLOPS2} for ocean-related datasets and case studies. This first edition provides datasets and ML-ready benchmarking pipelines for SSH interpolation problems, an essential topic for the space oceanography community, related to ML communities dealing with issues like in-painting~\cite{InPaintingSurvey}, denoising~\cite{DENOISESURVEY,DENOISESURVEY2}, and super-resolution~\cite{SuperResSurvey}. 
We expect \texttt{OceanBench} to facilitate new challenges to the applied machine learning community and contribute to meaningful ocean-relevant breakthroughs.
%
The remainder of the paper is organized as follows: in \S2, we outline some related work that was inspirational for this work; in \S3, we formally outline \texttt{OceanBench} by highlighting the target audience, code structure, and problem scope; in \S4, we outline the problem formulation of SSH interpolation and provide some insight into different tasks related to SSH interpolation where \texttt{OceanBench} could provide some helpful utility; and in \S5 we give some concluding remarks while also informally inviting other researchers to help fill in the gaps.

% \textbf{How We Measure SSH + Gap Problem}. 




% % \textbf{Data Deluge, Gaps, \& ML Integration}. 
% Like most geosciences, the ocean sciences also suffer from the data deluge that persists due to the large amounts of observed quantities being collected daily~\tocite{Gus}. 
% The increased availability and quantity of observations are beneficial because it offers the community more insight into the physical processes at a finer resolution.
% % Consequently, this can help oceanographers revolve a central question of observing and predicting the ocean state at different spatial and timescales. 
% The trade-off is that processing this data is very expensive.
% Furthermore, it is impossible to accurately measure the entire globe concurrently, resulting in data with many gaps.
% Many operational systems have historically used ad-hoc coordinate-based schemes to gap-fill the data based on covariance assumptions. 
% However, with the abundance of new data~\tocite{SWOT}, these covariance-based schemes are reaching their limit due to the computational and signal complexity.
% Machine learning (ML) has been recently introduced as a viable alternative to filling the gaps in data where many classes of methods have found success, including coordinate-based methods~\tocite{OI,MIOST,NerFs}, direct surrogate/hybrid methods~\tocite{BFN, recent-stuff}, and end-to-end learning schemes~\tocite{Ronan}. 
% Nevertheless, most of these methods have only been applied to regions, and more needs to be applied at a large scale that comes close to the operational settings.


% % \textbf{Logistics Problem}.
% % \textbf{The Problem with Observation Data}.
% % The community is quickly facing a logistics problem.
% Dealing with complex problems such as SSH interpolation requires dealing directly with the observation data, often involving aggregating observations from satellites, buoys, or autonomous devices.
% In all cases, we can only observe a tiny portion of the Earth at any given time, resulting in highly sparse data (often <5\% coverage).
% This problem is even more difficult because the data sources are often heterogeneous, with different variables, geometries, and noise levels.
% Furthermore, the evaluation procedure can be regionally dependent as the physical phenomena vary globally. 
% So the entire ML pipeline now requires a unified framework for dealing with heterogeneous data sources, different pre- and post-processing methodologies, and regional dependent evaluation procedures.

% To solve a difficult problem such as SSH interpolation requires a lot of interdisciplinary expertise including scientific, numerical and machine learning.
% To do effective ML research in this area, one needs a consistent \textit{pipeline} to funnel the data from its raw form through the transformations and to its evaluation state. 
% In standard ML research, we have copious amounts of \textit{datamodules} which encapsulate many underlying transformations procedures for many staple datasets, e.g. MNIST, CIFAR10, CELEB, etc.
% Even some weather and climate benchmark datasets have \cite{weatherbench,ClimateBench} access to gap-free, non-heterogenous reanalysis data.
% Despite the focus on the ML model in the current state of affairs, we have often found that the variables chosen and how they were preprocessed has the greatest effect on the final result.
% In previous benchmark settings~\citep{CHAOSBENCH,PDEBench,ClimateBench,weatherbench,ENS10Bench}, little attention was paid to the internals of the data preprocessing and this process was kept static.
% In geosciences, because we often deal with sparse, heterogeneous observations, we often need to decide and implement many crucial preprocessing transformations before we get to the modeling aspects.
% % including selecting regions, filtering, and calculating derived variables.
% As such, we should include the chosen variables and how they are preprocessed as hyperparameters to be optimized within the full ML pipeline.
% However, this complicates the entire pipeline because we have effectively compounded the complexity due to the many ways one can preprocess and modify the data in conjunction with decisions from the ML side like the architecture, the optimizer and the loss function.
% In addition, we may even have a large series of post-processing transformations which determine how we evaluate our models. Those transformations contain implementation details that may change between studies which make comparing and reproducing some results challenging.
% This results in a software engineering (SWE) roadblock which needs to be addressed should we wish to do effective research in applied ML research and this obstacle is especially prevalent in the geosciences.



% , through the transformations, into the model, and through the 
% Due to many of the many components needed to solve the problem it requires a pipeline. 
% For domain scientists, there is a lack of organization and ontology of previous ideas. 
% For machine learning researchers, ... 
% For downstream users, there is a lack of ontology of the current state of research which makes it difficult to readily apply these techniques for their problems.

% Nonetheless, most of this research has been done in parallel across disjoint sub-fields of research and to the best of our knowledge little to no work has been done on homogenising and integrating these distinct methods in a common framework



% In this paper, we wish to establish a well-defined software abstraction for dealing SSH interpolation using ML.
% Using ocean data is the primary goal which is a data assimilation problem. 
% However, there are many subsequent steps we need to improve. 1) We need gap-free data from the satellite observations, 2) we need surrogate or hybrid models to replace the expensive ocean models, and 3) we need end-to-end learning schemes to perform domain-specific tasks that incorporate good priors and the observations.

\section{Related Work}

Machine learning applied to geosciences is becoming increasingly popular, but there are few examples of transparent pipelines involving observation data. 
After a thorough literature review, we have divided the field into three camps of ML applications that pertain to this work: 1) toy simulation datasets, 2) reanalysis datasets, and 3) observation datasets. 
We outline the literature for each of the three categories below.

\textbf{Toy Simulation Data}. 
One set of benchmarks focuses on learning surrogate models for well-defined but chaotic dynamical systems in the form of ordinary differential equations (ODEs) and partial differential equations (PDEs) and there are freely available code bases which implement different ODEs/PDEs~\citep{CHAOSBENCH,PDEBench,pyQG,JAXCFD,NCARDART,NCARDARTSOFTWARE,VEROS,OCEANANIGANS}.
% For example, \texttt{Dyst} package~\cite{CHAOSBENCH} has many chaotic ODEs, and packages such as \texttt{PDEBench}~\cite{PDEBench}, pyQG~\citep{pyQG}, NCAR DART~\citep{NCARDART,NCARDARTSOFTWARE}, and jax-CFD~\citep{JAXCFD} have implementations of 2D/3D Spatial-Temporal PDEs similar to Navier-Stokes; 
This is a great testing ground for simple toy problems that better mimic the structures we see in real-world observations. 
% More ocean-related simulations include the pyQG package~\citep{pyQG}, the NCAR DART data assimilation testbed~\citep{NCARDART,NCARDARTSOFTWARE}.
Working with simulated data is excellent because it is logistically simple and allows users to test their ideas on toy problems without increasing the complexity when dealing with real-world data.
However, these are ultimately simple physical models that often do not reflect the authentic structures we see in real-world, observed data.

\textbf{Reanalysis Data}. 
This is assimilated data of real observations and model simulations. 
There are a few major platforms that host ocean reanalysis data like the Copernicus Marine Data Store~\citep{MDSOCEANPHYSICS,MDSBIOGEOCHEMICAL,MDSOCEANPHYSICSENS,MDSWAVES}, the Climate Data Store~\citep{CDSREANALYSISSST}, the BRAN2020 Model~\citep{DATABLUELINK}, and the NOAA platform~\citep{DATANCEP}. 
However, to our knowledge, there is no standard ML-specific ocean-related tasks to accompany the data. On the atmospheric side, platforms like \texttt{WeatherBench}~\cite{weatherbench}, \texttt{ClimateBench}~\cite{ClimateBench}, \texttt{ENS10}~\cite{ENS10Bench} were designed to assess short-term and medium-term forecasting using ML techniques with recent success of ML~\cite{GraphCast,FourCastNet}
% While the original papers featured straightforward methods, there has been swift subsequent development within the last few years~\cite{GraphCast,FourCastNet}. 
% Apart from industry momentum and investment, we attribute the recent adoption and success of ML to the problem's clarity, the original tasks' openness, and the software's ML compatibility. 
The clarity of the challenges set by the benchmark suites has inspired the idea of \texttt{OceanBench}, where we directly focus on problems dealing with ocean observation data.

\textbf{Observation Data}. 
These observation datasets (typically sparse) stem from satellite observations that measure surface variables or in-situ measurements that measure quantities within the water column. 
Some major platforms to host data include the Marine Data Store~\citep{MDSALONGTRACK,MDSINSITU}, the Climate Data Store~\citep{CDSOBSSST,CDSOBSSSTENS,CDSOBSOC}, ARGO~\citep{ARGO}, and the SOCAT platform~\citep{SOCAT}.
However, it is more difficult to assess the efficacy of operational ML methods that have been trained only on observation data and, to our knowledge, there is no coherent ML benchmarking system for ocean state estimation.
% In one community, there is the Surface Ocean CO$_2$ Atlas (SOCAT)~\cite{SOCAT} which is a community effort to aggregate all collocated observations (included SSH) which help predict the fugacity of carbon dioxide (fCO$_2$). 
% This has been a huge effort to provide a consistently updated suite of observations for some key variables that are important in biogeochemical processes.
% However, there is currently no coherent benchmarking system with standard metrics despite their being a wide range of new methods to try and tackle the interpolation problem.
% Most new work tends to use the observations provided with their own additional variable with no standard comparison framework other works.
% In a completely different community, 
There has been significant effort by the \textit{Ocean-Data-Challenge} Group\footnote{Ocean Data Challenge group: Freely associated scientist for oceanographic algorithm and product improvements (\href{https://ocean-data-challenges.github.io/}{ocean-data-challenges.github.io})} which provides an extensive suite of datasets and metrics for SSH interpolation.
% Their motivation is to investigate which methods could be employed for the upcoming SWOT mission~\cite{SWOT} which is highly-challenging for current operational interpolation techniques.
% where they provide over eight challenges of varying degrees of difficulty with completely open-source data, along with tutorials for metrics.
Their efforts heavily inspired our work, and we hope that \texttt{OceanBench} can build upon their work by adding cohesion and facilitating the ease of use for ML research and providing a high-level framework for providing ML-related data products.




\section{OceanBench} \label{sec:oceanbench_intro}

\subsection{Why OceanBench?} \label{sec:oceanbench_why}

There is a high barrier to entry in working with ocean observations for researchers in applied machine learning as there are many processing steps for both the observation data and the domain-specific evaluation procedures. 
\texttt{OceanBench} aims to lower the barrier to entry cost for ML researchers to make meaningful progress in the field of state prediction. 
We distribute a standardized, transparent, and flexible procedure for defining data and evaluation pipelines for data-intensive geoscience applications. 
Proposed examples and case studies provide a plug-and-play framework to benchmark novel ML schemes w.r.t.  state-of-the-art, domain-specific ML baselines. 
In addition, we adopt a pedagogical abstraction that allows users to customize and extend the pipelines for their specific tasks.
To our knowledge, no framework embeds processing steps for earth observation data in a manner compatible with MLOps abstractions and standards regarding reproducibility and evaluation procedures. 
Ultimately, we aim to facilitate the uptake of ML schemes to address ocean observation challenges and to bring new challenges to the ML community to extend additional ML tools and methods for irregularly-sampled and partially-observed high-dimensional space-time dynamics.
The abstractions proposed here apply beyond ocean sciences and SSH interpolation to other geosciences with similar tasks that intersect with machine learning.

% \begin{itemize}
%     \item Establish a rigorous ontology of pre-, geo- and ML processing tools.
%     \item Consistent problems with detailed introductory tutorials
%     \item aggregate benchmarks with common tools
%     \item \textit{what i wish I had when i started this problem}
%     \item \textit{A lot of research gets lots in grad students laptops}
%     \item provide training, context and awareness to the different packages that exist to solve problems.
% \end{itemize}



\subsection{Code Structure} \label{sec:code_structure}

\texttt{OceanBench} is lightweight in terms of the core functionality.
We keep the code base simple and focus more on how the user can combine each piece.
We adopt a strict functional style because it is easier to maintain and combine sequential transformations. 
There are five features we would like to highlight about \texttt{OceanBench}: 1) Data availability and version control, 2) an agnostic suite of geoprocessing tools for \texttt{xarray} datasets that were aggregated from different sources,  3) Hydra integration to pipe sequential transformations, 4) a flexible multi-dimensional array generator from \texttt{xarray} datasets that are compatible with common deep learning (DL) frameworks, and 5) a JupyterBook~\cite{JupyterBook} that offers library tutorials and demonstrates use-cases.
In the following section, we highlight these components in more detail.

\textbf{Data Availability}. 
The most important aspect is the public availability of the datasets. 
We aggregate all pre-curated datasets from other sources, e.g. the \textit{Ocean-Data-Challenge}~\cite{DCOSEGULFSSH,DCOSSEGULFSSH}, and organize them to be publicly available from a single source~\footnote{Available at: \href{https://github.com/quentinf00/oceanbench-data-registry}{oceanbench-data-registry.github.com}}. 
We also offer a few derived datasets which can be used for demonstrations and evaluation. 
Data is never static in a pipeline setting, as one can have many derived datasets which stem from numerous preprocessing choices. 
In fact, in research, we often work with derived datasets that have already been through some preliminary preprocessing methods. 
To facilitate the ever-changing nature of data, we use the Data Version Control (\texttt{DVC}) tool~\cite{DVC}, which offers a git-like version control of the datasets.

\textbf{Geoprocessing Tools}. 
The core \texttt{OceanBench} library offers a suite of functions specific to processing geo-centric data. 
While a few particular functionalities vary from domain to domain, many operations are standard, e.g., data variable selections, filtering/smoothing, regridding, coordinate transformations, and standardization. 
We almost work exclusively with the \texttt{xarray}~\cite{XARRAY} framework because it is a coordinate-aware, flexible data structure. 
In addition, the geoscience community has an extensive suite of specialized packages that operate in the \texttt{xarray} framework to accomplish many different tasks. 
% A non-exhaustive, highlighted list of packages that are compatible with the \texttt{xarray} data structure include: \texttt{xgcm} for non-uniform grid operators, \texttt{pint} for unit-aware operators, \texttt{metpy} for gradient-based calculations, \texttt{pyinterp} and \texttt{xesmf} for advanced regridding methods, \texttt{xrft} and \texttt{xwavelet} for power spectrum analysis, \texttt{gcm-filters} for advanced filtering methods, and the \texttt{dask} suite for scaling everything to multi-core/node HPC environments. 
Almost all \texttt{OceanBench} toolsets are exclusively within the \texttt{xarray} framework to maintain compatibility with a large suite of tools already available from the community.


% As an ML practicioner navigating this sea of libraries can be intimidating as they require different input formats and domain specific parameters. Oceanbench aims at simplifying the use of these libraries by providing the necessary formatting tools and default usage of the different utilities.  

% There are many packages that exist independently which accomplish many of the specific tasks that most geoscientists need. 

\textbf{Hydra Integration}. 
As discussed above, many specific packages accomplish many different tasks. 
However, what needs to be added is the flexibility to mix and match these operations as the users see fit. 
\texttt{Hydra}~\cite{Hydra} provides a configurable way to aggregate and \textit{pipe} many sequential operations together. 
It also maintains readability, robustness, and flexibility through the use of \texttt{.yaml} files which explicitly highlights the function used, the function parameters chosen, and the sequence of operations performed. 
In the ML software stack, \texttt{Hydra} is often used to manage the model, optimizer, and loss configurations which helps the user experiment with different options. 
We apply this same concept in preprocessing, geoprocessing, and evaluation steps, often more important than the model configuration in geoscience-related tasks.  

\texttt{XRPatcher}~\footnote{Available at: \href{https://github.com/jejjohnson/xrpatcher/}{github.com/jejjohnson/xrpatcher}}. 
Every machine learning pipeline will inevitably require moving data from the geo-specific data structure to a multi-dimensional array easily digestible for ML models. 
A rather underrated, yet critical, feature of ML frameworks such as \texttt{PyTorch}~\cite{PYTORCH} (\texttt{Lightning}~\cite{LIGHTNING}) and \texttt{TensorFlow}~\cite{TENSORFLOW} (\texttt{Keras}~\cite{KERAS}) is the abstraction of the dataset, dataloader, datamodules, and data pipelines. 
In applied ML in geosciences, the data pipelines are often more important than the actual model~\cite{DATA4ML}. 
The user can control the \textit{patch}-size and the \textit{stride}-step, which can generate arbitrary coordinate-aware items directly from the \texttt{xarray} data structure. 
In addition, \texttt{XRPatcher} provides a way to reconstruct the fields from an arbitrary patch configuration.
This robust reconstruction step is convenient to extend the ML inference step where one can reconstruct entire fields of arbitrary dimensions beyond the training configuration, e.g., to account for the border effects within the field (see appendix~\ref{sec:xrpatcher}) or to reconstruct quantities in specific regions or globally. 

\textbf{JupyterBook}.
% ~\footnote{Available at: \href{https://jejjohnson.github.io/oceanbench/content/overview.html}{jejjohnson.github.io/oceanbench}}. 
Building a set of tools is relatively straightforward; however, ensuring that it sees a broader adoption across a multi-disciplinary community is much more challenging. 
We invested heavily in showing use cases that appeal to different users with the \texttt{JupyterBook} platform~\cite{JupyterBook}. 
Code with context is imperative for domain and ML experts as we need to explain and justify each component and give many examples of how they can be used in other situations. 
% \texttt{Hydra} is an effective tool, but it has a steep learning curve, and the tutorials are very computer science-oriented, and there are many domain scientists (and seasoned ML researchers) who will. 
Thus, we have paid special attention to providing an extensive suite of tutorials, and we also highlight use cases for how one can effectively use the tools. 


\subsection{Problem Scope} \label{sec:problem_scope}

% As outlined in the introduction, we are interested in the state estimation problem for important variables like SSH and SST. We use the data assimilation perspective whereby state estimation is defined as the combination of observations, constraints and initial state~\citep{DAGEOSCIENCE}. Tackling the issue of state estimation directly is very challenging due to the uncertainty in the measurements, the prior constraints, and the initial state. In addition, there are many logistical constraints that further complicate the problem including the very high-dimensional, spatiotemporal state space, the multi-scale complexity of the dynamics, and the extreme missing data.




There are many problems that are of great interest the ocean community~\citep{ML4DA} but we limit the scope to state estimation problems~\citep{DAGEOSCIENCE}. Under this scope, there are research questions that are relevant to operational centers which are responsible for generating the vast majority of global ocean state maps~\citep{MDSOCEANPHYSICS,MDSOCEANPHYSICSENS,MDSBIOGEOCHEMICAL, MDSWAVES} that are subsequently used for many downstream tasks~\citep{ML4OCN}. For example: how can we effectively use heterogeneous observations to predict the ocean state on the sea surface~\citep{BFNQG,NERFSSSH,MIOST,4DVARNETSST,4DVARNETSWOT,OCEANSATELLITESST}; how can we incorporate prior physics knowledge into our predictions of ocean state trajectories~\citep{BFNQG,ML4DA,ML4OCN}; and how can we use the current ocean state at time $T$ to predict the future ocean state at time $T+\tau$~\citep{METNET2,weatherbench,FORECASTSSCGP}.
In the same vain, there are more research questions that are of interest to the academic modeling community. For example: is simulated or reanalysis data more effective for learning ML emulators that replace expensive ocean models~\citep{MLSUBGRID,MLCLOSURE}; what metrics are more effective for assessing our ability to mimic ocean dynamics~\citep{SSTFLOWANOMALY,MLMETRICSINVARIANCE}; and how much model error can we characterize when learning from observations~\citep{MLMODELERR,MLMODELERR2}. 

We have cited many potential applications of how ML can be applied to tackle the state estimation problem. 
However, to our knowledge there is no publicly available, standardized benchmark system that is caters to ML-research standards.
We believe that, irrespective of the questions posed above and the data we access, there are many logistical similarities for each of the problem formulations where we can start to set standards for a subset of tasks like interpolation or forecasting. 
On the front-end, we need a way to select regions, periods, variables, and a valid train-test split (see sec. ~\ref{sec:hydra_recipe_task}). 
On the back-end, we need a way to transform the predictions into more meaningful variables with appropriate metrics for validation (see sec. ~\ref{sec:hydra_geoprocess_task} and ~\ref{sec:hydra_evaluation_task}).
\texttt{OceanBench} was designed to be an agnostic tool that is extensible to the types of datasets, processing techniques and metrics needed for working with a specific class of Ocean-related datasets. 
We strongly feel that a suite like this is the first step in designing task-specific benchmarks within the ocean community that is compatible with ML standards. 
In the remainder of the paper, we will demonstrate how \texttt{OceanBench} can be configured to facilite a ML-ready data challenge involving our first edition to demonstrate \texttt{OceanBench}'s applicability: sea surface height interpolation.

\section{\textit{Sea Surface Height Edition}}\label{sec:interp_challenge}

Sea surface height (SSH) is one of the most critical, observable quantities when determining the ocean state. 
It is widely used to study ocean dynamics and the adverse impact on global climate and human activities~\cite{SSHMESOSCALE}. 
SSH enables us to track phenomena such as currents and eddies~\cite{SSHMESOSCALE,SSHMESOSCALE2,SSHMESOSCALE3}, which leads to a better quantification of the transport of energy, heat, and salt. 
In addition, SSH helps us quantify sea level rise at regional and global scales~\cite{SSHSEALEVEL,OCEANSEALEVEL}, which is used for operational monitoring of the marine environment~\cite{SSHOPERATIONAL}. 
Furthermore, SSH characterization provides a plethora of data products that downstream tasks can use for many other applications~\cite{SSH3DCIRCULATION, 3DQGOC}.
%
Due to the irregular sampling delivered by satellite altimeter, state-of-the-art operational methods using optimal interpolation schemes~\cite{DUACS, MIOST} or model-driven data assimilation~\cite{DINEOF, DINEOF2, ANALOGDA, ANALOGDA2} fail to fully retrieve SSH dynamics at fine scales below 100-200km on a global or regional scale, so improving the space-time resolution of SSH fields has been a critical challenge in ocean science. 
Beyond some technological developments~\cite{SWOT}, recent studies support the critical role of ML-based schemes in overcoming the current limitations of the operational systems~\cite{4DVARNETSWOT, BFNQG, SSHInterpAttention} .  
%
The rest of this section gives an overview of the general problem definition for SSH interpolation, followed by a brief ontology for ML approaches to address the problem. 
We also give an overview of some experimental designs and datasets with a demonstration of metrics and plots generated by the \texttt{OceanBench} platform. 



\subsection*{Problem Definition}\label{sec:prob_definition}

We are dealing with satellite observations, so we are interested in the domain across the Earth's surface. 
Let us define the Earth's domain by some spatial coordinates, $\mathbf{x} = [\text{Longitude},\text{Latitude}]^\top \in\mathbb{R}^{D_s}$, and temporal coordinates, $t=[\text{Time}]\in\mathbb{R}^+$, where $D_s$ is the dimensionality of the coordinate vector.  
We can define some spatial (sub-)domain, $\Omega\subseteq\mathbb{R}^{D_s}$, and a temporal (sub-)domain, $\mathcal{T}\subseteq\mathbb{R}^+$. 
This domain could be the entire globe for 10 years or a small region within the North Atlantic for 1 year.
\begin{align}  \label{eq:spatiotemporal_coords}
    \text{Spatial Coordinates}: && \mathbf{x} &\in \Omega \subseteq \mathbb{R}^{D_s}\\ 
    \text{Temporal Coordinates}: && t &\in \mathcal{T} \subseteq \mathbb{R}^+.
\end{align}
In this case $D_s=2$ because we only have a two coordinates, however we can do some coordinate transformations like spherical to Cartesian. Likewise, we can do some coordinate transformation for the temporal coordinates like cyclic transformations or sinusoidal embeddings~\cite{ATTENTION}. We have two fields of interest from these spatiotemporal coordinates: the state and the observations.
\begin{align} \label{eq:state_obs}
    \text{State}: && \boldsymbol{u}(\mathbf{x},t) &: \Omega\times\mathcal{T}\rightarrow\mathbb{R}^{D_u} \\
    \text{Observations}: && \boldsymbol{y}_{obs}(\mathbf{x},t) &: \Omega\times\mathcal{T}\rightarrow\mathbb{R}^{D_{obs}}
\end{align}
The state domain, $u\in\mathcal{U}$, is a scalar or vector-valued field of size $D_u$ which is typically the quantity of interest and the observation domain, $y_{obs}\in\mathcal{Y}_{obs}$, is the observable quantity which is also a scalar or vector-valued field of size $D_{obs}$. Now, we make the assumption that we have an operator $\mathcal{H}$ that transforms the field from the state space, $\boldsymbol{u}$, to the observation space, $\boldsymbol{y}_{obs}$.
\begin{align} \label{eq:prob_definition}
    \boldsymbol{y}_{obs}(\mathbf{x},t) = \mathcal{H}\left(\boldsymbol{u}(\mathbf{x},t), t, \boldsymbol{\varepsilon}, \boldsymbol{\mu}\right) 
\end{align}
This equation is the continuous function defined over the entire spatiotemporal domain.  
The operator, $\mathcal{H}(\cdot)$, is flexible and problem dependent.
For example, in a some discretized setting there are 0's wherever there are no observations, and 1's wherever there are observations, and in other discretized settings it takes a weighted average of the neighboring pixels.
We also include a generic noise function, $\boldsymbol{\varepsilon}(\mathbf{x},t)$.
This could stem from a distribution, it could stationary noise operator, $\boldsymbol{\varepsilon}(\mathbf{x})$, or it could be constant in space but vary with Time, $\boldsymbol{\varepsilon}(t)$. 
We also include a control parameter, $\boldsymbol{\mu}$, representing any external factors or latent variables that could connect the state vector to the observation vector, e.g., sea surface temperature.
%
%###########################################################################################
%
% CAN BE FRAMED AS INVERSE PROBLEMS
%
Our quantity of interest is SSH, $\eta$, a scalar-valued field defined everywhere on the domain. In our application, we assume that the SSH we observe from satellite altimeters, $\eta_{obs}$, is the same as the SSH state, except it could be missing for some coordinates due to incomplete coverage from the satellite. So our transformation is defined as follows:
\begin{align} \label{eq:ssh_field_continuous}
\boldsymbol{\eta}_{obs}(\mathbf{x},t) &= \mathcal{H}\left(\boldsymbol{\eta}(\mathbf{x},t), t, \boldsymbol{\varepsilon}, \boldsymbol{\mu}\right)
% , \hspace{10mm}
% \mathbf{x} \in \Omega \subseteq \mathbb{R}^{D_s}, \hspace{10mm} 
% t \in \mathcal{T} \subseteq \mathbb{R}^+.
\end{align}
In practice, the satellite providers have a reasonable estimation of the amount of structured noise level we can expect from the satellite altimetry data; however, unresolved noise could still be present. 
% Although we do not explicitly specify the control parameter, $\boldsymbol{\mu}$, we leave it into the equation to account for any other state parameters not accounted for in our model. 
Finally, we are interested in finding some model, $\mathcal{M}$, that maps the SSH we observe to the true SSH given by
\begin{align} \label{eq:interp_problem}
    \mathcal{M} &: \boldsymbol{\eta}_{obs}(\mathbf{x}, t, \boldsymbol{\mu}) \rightarrow \boldsymbol{\eta}(\mathbf{x},t),
%     , \hspace{10mm}
% \mathbf{x} \in \Omega \subseteq \mathbb{R}^{D_s}, \hspace{10mm} 
% t \in \mathcal{T} \subseteq \mathbb{R}^+.
\end{align}
which is essentially an inverse problem that maps the observations to the state.
One could think of it as trying to find the inverse operator, $\mathcal{M}=\mathcal{H}^{-1}$, but this could be some other arbitrary operator.  
%
\subsection*{Machine Learning Model Ontology} \label{sec:ml_ontology_mini}

In general, we are interested in finding some parameterized operator, $\mathcal{M}_{\boldsymbol{\theta}}$, that maps the incomplete SSH field to the complete SSH field
\begin{align} \label{eq:ml_interp_problem}
    \mathcal{M}_{\boldsymbol{\theta}} &: \boldsymbol{\eta}_{obs}(\mathbf{x}, t, \boldsymbol{\mu}) \rightarrow \boldsymbol{\eta}(\mathbf{x},t),
\end{align}
whereby we learn the parameters from data.
%
The two main tasks we can define from this problem setup are 1) interpolation and 2) extrapolation.
We define \textit{interpolation} as the case when the boundaries of the inferred state domain lie within a predefined shape for the boundaries of the spatiotemporal observation domain. 
For example, the shape of the spatial domain could be a line, box, or sphere, and the shape of the temporal domain could be a positive real number line.
We define \textit{extrapolation} as the case where the boundaries of the inferred state domain are outside the boundaries of the spatiotemporal observation domain. 
In this case, the inferred state domain could be outside of either domain or both. 
A prevalent specific case of extrapolation is \textit{hindcasting} or \textit{forecasting}, where the inferred state domain lies within the spatial observation domain's boundaries but outside of the temporal observation domain's.
In the rest of this paper, we will look exclusively at the interpolation problem. 
However, we refer the reader to appendix~\ref{sec:other_tasks} for a more detailed look at other subtasks that can arise.

From a ML point of view, we can explore various ways to define the operator in equation~\eqref{eq:interp_problem}. 
We may distinguish three main categories: (i) coordinate-based methods that learn a parameterized continuous function to map the domain coordinates to the scalar values, (ii) the explicit mapping of the state from the observation, (iii) implicit methods defined as the solution of an optimization problem. 
The first category comprises of kriging approaches, which have been used operationally with historical success~\cite{KRIGINGREVIEW,DUACS}. Beyond such covariance-based approaches, recent contributions explore more complex trainable functional models~\cite{GPsBIGDATA}, basis functions~\cite{MIOST}, and neural networks~\cite{NERFSSSH}. 
The second category of schemes bypasses the physical modeling aspect and amortizes the prediction directly using state-of-the-art neural architectures such as UNets and ConvLSTMs~\cite{SSHInterpAttention, SSHInterpConvLSTM, SSHInterpUNet}. 
This category may straightforwardly benefit from available auxiliary observations~\citep{CDSOBSSST,CDSOBSSSTENS,CDSOBSOC} 
% (including available operational gap-free SST products~\citep{CDSOBSSST,CDSOBSSSTENS} and other sea surface quantities~\citep{CDSOBSOC})
to state the interpolation problem as a super-resolution~\cite{SuperResSurvey} or image-to-image translation problem~\cite{IMAGE2IMAGETRANSLATION, IMAGE2IMAGETRANSLATION2}. 
The third category relates to inverse problem formulations and associated deep learning schemes, for example deep unfolding methods and plug-and-play priors~\cite{DEEPUNFOLDING}. 
Interestingly, recent contributions explore novel neural schemes which combine data assimilation formulations~\cite{DAGEOSCIENCE} and learned optimizer strategies~\cite{4DVARNETSWOT,4DVARNETSST}.
We provide a more detailed ontology of methods used for interpolation problems in appendix~\ref{sec:ml_ontology}. 
We consider at least one baseline approach from each category for each data challenge described in section~\ref{sec:data_challenges}. 
While all these methods have pros and cons, we expect the OceanBench platform to showcase to new experimental evidence and understanding regarding their applicability to SSH interpolation problems.
 

\subsection*{Experimental Design} \label{sec:experimental_design}

\begin{table}[ht]
\caption{This table gives a brief overview of the datasets provided to complete the data challenges listed in~\ref{sec:data_challenges} and~\ref{sec:data_challenges_extended}. Note that the OSSE datasets are all gridded products whereas the OSE NADIR is an alongtrack product. See figure~\ref{fig:oceanbench_maps} for an example of the OSSE NEMO Simulations for SSH and SST and pseudo-observations for NADIR \& SWOT.}
\label{tb:datasets}
\centering
\begin{tabular}{lcccc}
 \toprule
 & OSSE & OSSE NADIR + SWOT & OSSE SST & OSE NADIR  \\ \midrule
 Data Type & Simulations & 
Pseudo-Observations & 
 Simulations & Observations \\
Source     & 
NEMO~\citep{NEMOAJAYI2020} & 
NEMO~\citep{NEMOAJAYI2020} &
NEMO~\citep{NEMOAJAYI2020}
% \multicolumn{3}{c}{NEMO GCM\citep{NEMOAJAYI2020}}  
& Altimetry~\citep{MDSALONGTRACK} \\
Region & 
GulfStream & GulfStream & GulfStream & GulfStream \\
Domain Size &
% ($L_x\times L_y$) 
$10\times 10^\circ$ &
$10\times 10^\circ$ &
$10\times 10^\circ$ &
$10\times 10^\circ$
\\
Longitude Extent &
$[-65^\circ, -55^\circ]$ & 
$[-65^\circ, -55^\circ]$ &
$[-65^\circ, -55^\circ]$ &
$[-65^\circ, -55^\circ]$ \\
Latitude Extent &
$[33^\circ, 43^\circ]$ &
$[33^\circ, 43^\circ]$ &
$[33^\circ, 43^\circ]$ &
$[33^\circ, 43^\circ]$ \\
Resolution &
% ($\Delta_x\times \Delta_y$) 
$0.05^\circ\times 0.05^\circ$ &
$0.05^\circ\times 0.05^\circ$ &
$0.05^\circ\times 0.05^\circ$ &
$7$ km \\
Grid Size &
$200\times 200$ & $200\times 200$ & $200\times 200$ & N/A \\
Num Datapoints &
$\sim$14.6M & $\sim$14.6M & $\sim$14.6M & $\sim$1.6M \\
Period Start & 2012-10-01 & 2012-10-01 & 2012-10-01 & 2016-12-01 \\
Period End & 2013-09-30 & 2013-09-30 & 2013-09-30 & 2018-01-31 \\
Frequency  & Daily & Daily & Daily & 1 Hz \\
\bottomrule
\end{tabular}
\end{table}

The availability of multi-year simulation and observation datasets naturally advocates for the design of synthetic (or twin) experiments, referred to as observing system simulation experiments (OSSE), and of real-world experiments, referred to as observing system experiments (OSE).
We outline these two experimental setups below.

\textbf{Observing System Simulation Experiments (OSSE)}. A staple and groundtruthed experimental setup uses a reference simulation dataset to simulate the conditions we can expect from actual satellite observations. 
This setup allows researchers and operational centers to create a fully-fledged pipeline that mirrors the real-world experimental setting.
An ocean model simulation is deployed over a specified spatial domain and period, and a satellite observation simulator is deployed to simulate satellite observations over the same domain and period. 
This OSSE setup has primarily been considered for performance evaluation, as one can assess a reconstruction performance over the entire space-time domain. It also provides the basis for the implementation of classic supervised learning strategies~\cite{SSHInterpUNet,SSHInterpConvLSTM,SSHInterpAttention}.
The domain expert can vary the experimental conditions depending on the research question. 
For example, one could specify a region based on the expected dynamical regime~\cite{DCOSSEGULFSSH} or add a certain noise level to the observation tracks based on the satellite specifications.
The biggest downside to OSSE experiments is that we train models exclusively with ocean simulations which could produce models that fail to generalize to the actual ocean state. 
Furthermore, the simulations are often quite expensive, which prevents the community from having high spatial resolution over very long periods, which would be essential to capture as many dynamical regimes as possible.


\begin{figure}[t!]
\small
\begin{center}
\setlength{\tabcolsep}{1pt}
\begin{tabular}{ccc}
NADIR Altimetry Tracks & 
SWOT Altimetry Tracks &
Sea Surface Temperature \\
\includegraphics[width=42.5mm, height=30mm]{00_Oceanbench/content/figures/maps/sla/dc20a_ssh_anomaly_nadir4_20121027.png} 
% \includegraphics[bb=0 0 4 3]{content/figures/maps/sla/dc20a_ssh_anomaly_nadir4_20121027.png} 
&
\includegraphics[width=42.5mm, height=30mm]{00_Oceanbench/content/figures/maps/sla/dc20a_ssh_anomaly_swot1nadir5_20121027.png} &
\includegraphics[width=4.25cm,height=3cm]{00_Oceanbench/content/figures/maps/sst/dc20a_nemo_sst.png}
\end{tabular}
\begin{tabular}{cccc}
\hspace{3mm} NEMO Simulation & 
\hspace{3mm} MIOST & 
\hspace{3mm} BFNQG & 
4DVarNet \\
\vspace{-2mm}
%%%%% SEA LEVEL ANOMALY %%%%%%%%
\includegraphics[trim={0 0 42mm 0},clip, width=3.20cm,height=3cm]{00_Oceanbench/content/figures/maps/sla/dc20a_nemo_sla.png} &
\includegraphics[trim={0 0 42mm 0},clip, width=3.2cm,height=3cm]{00_Oceanbench/content/figures/maps/sla/dc20a_miost_sla.png} &
\includegraphics[trim={0 0 42mm 0},clip, width=3.2cm,height=3cm]{00_Oceanbench/content/figures/maps/sla/dc20a_bfnqg_sla.png} &
\includegraphics[width=4.0cm,height=3cm]{00_Oceanbench/content/figures/maps/sla/dc20a_4dvarnet_sla.png} \\
\vspace{-2mm}
%%%%% KINETIC ENERGY %%%%%%%%
\includegraphics[trim={0 0 42mm 0},clip, width=3.20cm,height=3cm]{00_Oceanbench/content/figures/maps/ke/dc20a/nadir4/dc20a_nemo_ke.png} &
\includegraphics[trim={0 0 42mm 0},clip, width=3.2cm,height=3cm]{00_Oceanbench/content/figures/maps/ke/dc20a/nadir4/dc20a_miost_ke.png} &
\includegraphics[trim={0 0 42mm 0},clip, width=3.2cm,height=3cm]{00_Oceanbench/content/figures/maps/ke/dc20a/nadir4/dc20a_bfnqg_ke.png} &
\includegraphics[width=4.0cm,height=3cm]{00_Oceanbench/content/figures/maps/ke/dc20a/nadir4/dc20a_4dvarnet_ke.png}  \\
\vspace{-2mm}
%%%%% RELATIVE VORTICITY %%%%%%%%
\includegraphics[trim={0 0 42mm 0},clip, width=3.20cm,height=3cm]{00_Oceanbench/content/figures/maps/rvort/dc20a/nadir4/dc20a_nemo_vort_r.png} &
\includegraphics[trim={0 0 42mm 0},clip, width=3.2cm,height=3cm]{00_Oceanbench/content/figures/maps/rvort/dc20a/nadir4/dc20a_miost_vort_r.png} &
\includegraphics[trim={0 0 42mm 0},clip, width=3.2cm,height=3cm]{00_Oceanbench/content/figures/maps/rvort/dc20a/nadir4/dc20a_bfnqg_vort_r.png} &
\includegraphics[width=4.0cm,height=3cm]{00_Oceanbench/content/figures/maps/rvort/dc20a/nadir4/dc20a_4dvarnet_vort_r.png}  \\
%%%%% STRAIN %%%%%%%%
\includegraphics[trim={0 0 38mm 0},clip, width=3.20cm,height=3cm]{00_Oceanbench/content/figures/maps/strain/dc20a/nadir4/dc20a_nemo_strain.png} &
\includegraphics[trim={0 0 38mm 0},clip, width=3.2cm,height=3cm]{00_Oceanbench/content/figures/maps/strain/dc20a/nadir4/dc20a_miost_strain.png} &
\includegraphics[trim={0 0 38mm 0},clip, width=3.2cm,height=3cm]{00_Oceanbench/content/figures/maps/strain/dc20a/nadir4/dc20a_bfnqg_strain.png} &
\includegraphics[width=4.0cm,height=3cm]{00_Oceanbench/content/figures/maps/strain/dc20a/nadir4/dc20a_4dvarnet_strain.png}  \\
% \vspace{-2mm}
(a) & (b) & (c) & (d)
\end{tabular}
\vspace{-3mm}
% \caption{Row I - Isotrophic PSD. Row 2 - Isotrophic PSD Score}
\caption{
A snapshot at $27^{th}$ October, 2012 of the sea level anomaly (SLA) from the NEMO simulation for the OSSE experiment outlined in section~\ref{sec:experimental_design}. 
The top row showcases the aggregated NADIR altimetry tracks and the aggregated SWOT altimetry tracks (12 hours before and 12 hours after) as well as the SST from the NEMO simulation.
Each subsequent row showcases the following physical variables found in appendix~\ref{sec:physical_variables}: (a) Sea Level Anomaly, (b) Kinetic Energy, (c) Relative Vorticity, and (d) Strain. 
Each column in the subsequent rows showcase the following reconstructed field from the NEMO simulation found in columrn (a): (b) MIOST~\cite{MIOST}, (c) BFN-QG~\cite{BFNQG}, and (d) 4DVarNet~\cite{4DVARNETSWOT}.}
\label{fig:oceanbench_maps}
\vspace{-5mm}
\end{center}
\end{figure}


% \subsubsection{Observing System Experiments (OSE)} \label{sec:ose}

\textbf{Observing System Experiments (OSE)}. As more observations have become available over the past few decades, we can also design experiments using real data. 
This involves aggregating as many observations from real ocean altimetry satellites as possible with some specific independent subset left out for evaluation purposes.
A major downside to OSE experiments is that the sparsity and spatial coverage of the observations narrow the possible scope of performance metrics and make it very challenging to learn directly from observation datasets. 
The current standard altimetry data are high resolution but cover a tiny area. 
As such, it can only inform fine-scale SSH patterns in the along-track satellite direction and cannot explicitly reveal two-dimensional patterns. 
Despite these drawbacks, it provides a quantitative evaluation of the generalizability of the ML methods concerning the true ocean state.
%and so it fails to capture many of the dynamical regimes we are interested in, i.e. mesoscale and sub-mesoscale processes. 
%However, it is still advantageous (and preferable) to include these experiments because these reflect the true ocean state and will help with the generalizability of the ML methods.

\begin{figure}[t!]
\small
\begin{center}
\setlength{\tabcolsep}{2pt}
\begin{tabular}{ccc}
% $\mathcal X$ & $\hat \z = \bG_\theta(\x)$ & $\x = \bG_\theta^{-1} (\hat \z)$\\[0mm]
% NATL60&
% \multicolumn{2}{c}{\includegraphics[width=6.25cm,height=4.5cm]{content/figures/exp_natl60/psd_st/osse_2020a_psd_natl60}} 
% \\
\includegraphics[width=3.75cm,height=3.25cm]{00_Oceanbench/content/figures/stats/nrmse_space.png} &
\includegraphics[width=4.25cm,height=3.5cm]{00_Oceanbench/content/figures/psd_isotropic/dc20a/nadir4/dc20a_psd_iso_ssh.png} &
\includegraphics[width=4.25cm,height=3.5cm]{00_Oceanbench/content/figures/psd_isotropic/dc20a/nadir4/dc20a_psd_score_iso_ssh.png} 
\\
(a) Normalized RMSE &
(b) Isotropic Power Spectrum &
(c) Isotropic Power Spectrum Score
\end{tabular}
\begin{tabular}{cccc}
\includegraphics[trim={0 0 0mm 0},clip, width=4.20cm,height=3cm]{00_Oceanbench/content/figures/psd_spacetime/dc20a/nadir4/dc20a_psd_spacetime_nemo_nadir4_ssh.png}  &
\includegraphics[trim={20mm 0 34mm 0},clip, width=2.9cm,height=3cm]{00_Oceanbench/content/figures/psd_spacetime/dc20a/nadir4/dc20a_psd_spacetime_score_miost_nadir4_ssh.png} &
\includegraphics[trim={20mm 0 34mm 0},clip, width=2.9cm,height=3cm]{00_Oceanbench/content/figures/psd_spacetime/dc20a/nadir4/dc20a_psd_spacetime_score_bfnqg_nadir4_ssh.png} &
\includegraphics[trim={20mm 0 0 0},clip, width=3.5cm,height=3cm]{00_Oceanbench/content/figures/psd_spacetime/dc20a/nadir4/dc20a_psd_spacetime_score_4dvarnet_nadir4_ssh.png} \\
(d) NEMO Simulation &
(e) MIOST &
(f) BFN-QG &
(g) 4DVarNet
\end{tabular}
% % \vspace{-4mm}
% \caption{Row I - Isotrophic PSD. Row 2 - Isotrophic PSD Score}
\caption{This figure showcases some statistics for evaluation of the SSH field reconstructions for the OSSE NADIR experiment outlined in section~\ref{sec:interp_challenge}. Subfigure (a) showcases the normalized root mean squared error (nRMSE), (b) showcases the isotropic power spectrum decomposition (PSD), (c) showcases isotropic PSD scores.
The bottom row showcases the space-time PSD for the NEMO simulation (subfigure (d)) and the PSD scores for three reconstruction models: (e) the MIOST model~\cite{MIOST}, (f) the BFN-QG model~\cite{BFNQG}, and (g) the 4DVarNet model~\cite{4DVARNETSWOT}.
}
% \vspace{-5mm}
\label{fig:oceanbench_psd}
\end{center}
\end{figure}


%
\subsection*{Data Challenges} \label{sec:data_challenges}



We rely on existing OSSE and OSE experiments for SSH interpolation designed by domain experts~\cite{DCOSEGULFSSH,DCOSSEGULFSSH} and recast them into \texttt{OceanBench} framework to deliver a ML-ready benchmarking suites. 
The selected data challenges  for this first edition address SSH interpolation for a 1000km$\times$1000km Gulfstream region. We briefly outline them below.
%The \textit{Ocean-Data-Challenge} group has a wide-range of different OSSE and OSE experiments involving SSH interpolation in particular~\cite{DCOSEGULFSSH,DCOSSEGULFSSH}. 
%For this paper, we will focus on a subset and demonstrate some evaluation steps generated from the \texttt{OceanBench} framework.
%We outline an experimental setup SSH reconstruction over the Gulfstream with three OSSE's configurations and one OSE configuration in the next section. For more detailed information about the experimental setups for each configuration, see section~\ref{sec:data_challenges_extended} in the appendix.

% \begin{itemize}
%     \item 
\textbf{Experiment I (\textit{OSSE NADIR})} addresses SSH interpolation using NADIR altimetry tracks which are very fine, thin ocean satellite observations (see Figure~\ref{fig:oceanbench_maps}). It relies on an OSSE using high-resolution ($1/60^\circ$ resolution) ocean simulations generated by the NEMO model over one year with a whole field every day. 

%The observation data uses NADIR altimetry tracks which are very fine, thin ocean satellite observations (see Figure~\ref{fig:oceanbench_maps} (a) \& (b)). 
\textbf{Experiment II (\textit{OSSE SWOT})} addresses SSH interpolation using jointly NADIR and SWOT altimetry data where we complement the \textbf{OSSE NADIR} configuration with simulated SWOT observations.
SWOT is a new satellite altimetry mission with a much higher spatial coverage but a much lower temporal resolution as illustrated in Figure~\ref{fig:oceanbench_maps}.
The higher spatial resolution allows us to see structures at a smaller resolution but at the cost of a massive influx of observations (over $\times$100).

\textbf{Experiment III (\textit{OSSE SST})} addresses SSH interpolation using altimetry and SST satellite data jointly. We complement the \textbf{OSSE SWOT} challenge with simulated SST observations. 
Satellite-derived SST observations are more abundantly available in natural operational settings than SSH at a finer resolution, and structures have visible similarities~\cite{SWOT,BFNQG}.
So this challenge allows for methods to take advantage of multi-modal learning~\cite{4DVARNETSST,SSHInterpAttention}.

\textbf{Experiment IV (\textit{OSE NADIR})} addresses SSH interpolation for real NADIR altimetry data. 
In contrast to the three OSSE data challenges, it only looks at actual observations aggregated from the currently available ocean altimetry data from actual satellites. 
It involves a similar space-time sampling as Experiment (\textbf{OSSE NADIR}) to evaluate the generalization of ML methods trained in Experiment I to real altimetry data. 
The training problem's complexity increases significantly due to the reference dataset's sparsity compared with the \textbf{OSSE NADIR} dataset. 
One may also explore transfer learning or fine-tuning strategies from the available OSSE dataset. 


% \begin{itemize}
%     \item Experiment I (\textbf{OSSE NADIR}) addresses SSH interpolation using NADIR altimetry tracks which are very fine, thin ocean satellite observations (see Figure~\ref{fig:oceanbench_maps}). It relies on an OSSE using anocean simulations generated by the NEMO model, more precisely a high-resolution simulation, $1/60^\circ$ resolution, over one year with a whole field every day. 
% %The observation data uses NADIR altimetry tracks which are very fine, thin ocean satellite observations (see Figure~\ref{fig:oceanbench_maps} (a) \& (b)). 
% \item Experiment II (\textbf{OSSE SWOT}) addresses SSH interpolation using jointly NADIR and SWOT altimetry data. We complement the \textbf{OSSE NADIR} configuration with simulated SWOT observations.  
% SWOT is a new satellite altimetry mission with  much higher spatial coverage but a much lower temporal resolution as illustrated in Figure~\ref{fig:oceanbench_maps}.
% The higher spatial resolution allows us to see structures at a smaller resolution but at the cost of a massive influx of observations (over $\times 100$).
% \item Experiment III (\textbf{OSSE SST}) addresses SSH interpolation using 
% jointly altimetry and SST satellite data. We complement the \textbf{OSSE SWOT} challenge with simulated SST observations. 
% Satellite-derived SST observations are more abundantly available in natural operational settings than SSH at a finer resolution, and structures have visible similarities \cite{}.
% So this challenge allows for methods to take advantage of multi-modal learning \cite{}.
% \item Experiment IV \textbf{OSE NADIR} addresses SSH interpolation for real NADIR altimetry data. In contrast to the three OSSE data challenges, it only looks at actual observations aggregated from the currently available ocean altimetry data from actual satellites. It involves a similar space-time sampling as Experiment (\textbf{OSSE NADIR}) to evaluate the generalization of ML methods trained in Experiment I to real altimetry data. Besides, one may also explore learning or fine-tuning strategies from the available OSE dataset. We may point out that the complexity of the training problem due to the sparsity of refence dataset compared with \textbf{OSSE NADIR} dataset.
% \end{itemize}




\subsection*{\texttt{OceanBench} Pipelines}

\begin{table}[h]
\caption{This table highlights some of the results for the \textbf{OSSE NADIR} experiment outlined in section~\ref{sec:data_challenges} and appendix~\ref{sec:data_challenges_extended}.
% and the OSE experiment outlined in section~\ref{sec:ose}~\tocite{}. 
% For more results regarding the SWOT data, please see section~\ref{sec:other_tasks}. 
This table highlights the performance statistically in the real and spectral space; the normalized RMSE score for the real space and the minimum spatial and temporal scales resolved in the spectral domain. 
For more information about the class of models displayed and class of metrics, see appendix~\ref{sec:ml_ontology} and appendix~\ref{sec:metrics} respectively. We only showcase the model performance on the alongtrack NADIR data available. For the extended table for each of the challenges, see Table~\ref{tb:exp-results-mega}.}
\label{tb:oceanbench_results}
\centering
\begin{tabular}{lllcccc}
 \toprule
% Experiment & Configuration & Method & nRMSE & Resolved Scale [km]    \\ \midrule
% \multirow{2}{*}{Experiment} & \multirow{2}{*}{Algorithm} & \multirow{2}{*}{Algorithm Class} & \multirow{2}{*}{nRMSE} & \multicolumn{2}{c}{Effective Resolution} \\ 
% &  &   &  & Wavelength [km]  & Period [days]      \\ \midrule
% \multirow{2}{*}{Experiment} & \multirow{2}{*}{Algorithm} & \multirow{2}{*}{Algorithm Class} & \multirow{2}{*}{nRMSE} & \multicolumn{2}{c}{Effective Resolution} \\ 
Experiment &  Algorithm &   Algorithm Class &  nRMSE Score & $\lambda_{\mathbf{x}}$ [km]  & $\lambda_{t}$ [days]      \\ \midrule
\multicolumn{1}{l}{OSSE NADIR}     &  OI~\cite{DUACS} &  Coordinate-Based & 0.92 $\pm$ 0.01 & 175 & 10.8 \\
\multicolumn{1}{l}{OSSE NADIR}     &  MIOST~\cite{MIOST} &  Coordinate-Based  & 0.93 $\pm$ 0.01 & 157 & 10.1 \\
\multicolumn{1}{l}{OSSE NADIR}     &  BFNQG~\cite{BFNQG} &  Hybrid Model   & 0.93 $\pm$ 0.01 & 139 & 10.6 \\
OSSE NADIR &  4DVarNet~\cite{4DVARNETSWOT} &  Bi-Level Opt.  & 0.95 $\pm$ 0.01 & 117 & 7.7 \\
\bottomrule
\end{tabular}
\end{table}



For the four data challenges presented in the previous section, we used \texttt{OceanBench} pipelines to deliver a ML-ready benchmarking framework.
We used the \texttt{hydra} and the geoprocessing tools outlined in section~\ref{sec:code_structure} with specialized routines for regridding the ocean satellite data to a uniformly gridded product and vice versa when necessary. 
Appendix~\ref{sec:hydra_recipes} showcases an example of the hydra integration for the preprocessing pipeline. 
A key feature is the creation of a custom patcher for the appropriate geophysical variables using our \texttt{XRPatcher} tool, which is later integrated into custom datasets and dataloaders for the appropriate model architecture, e.g., coordinate-based or grid-based. 
We provide an example snippet of how this can be done easily in section~\ref{sec:xrpatcher}.
\texttt{OceanBench} also features some tools specific to the analysis of SSH. 
For example, physically-interpretable variables like geostrophic currents and relative vorticity, which can be derived from first-order and second-order derivatives of the SSH, are essential for assessing the quality of the reconstructions generated by the models. 
Figure~\ref{fig:oceanbench_maps} showcases some fields of the most common physical variables used in the oceanography literature for the SSH-based analysis of sea surface dynamics. For more details regarding the nature of the physical variables, see appendix~\ref{sec:physical_variables}.



Regarding the evaluation framework, we include domain-relevant performance metrics beyond the standard ML loss and accuracy functions. They account for the sampling patterns of the evaluation data. Spectral analytics are widely used in geoscience~\cite{BFNQG}, and here, we consider spectral scores computed as the minimum spatial and temporal scales resolved by the reconstruction methods proposed in~\cite{BFNQG}.
For example, figure~\ref{fig:oceanbench_psd} showcases how \texttt{OceanBench} generated the isotropic power spectrum and score and the space-time power spectrum decomposition and score.
Table~\ref{tb:oceanbench_results} outlines some standard and domain-specific scores for the experiments outlined in section~\ref{sec:experimental_design}.
We give a more detailed description of the rationale and construction of the power-spectrum-specific metrics in appendix~\ref{sec:metrics}. In terms of baselines, we report for each data challenge the performance of at least one approach for each of the category outlined in Section \ref{sec:ml_ontology_mini}.


% \begin{table}[h]
% \caption{This table highlights some of the results for the \textbf{OSSE NADIR} experiment outlined in section~\ref{sec:data_challenges} and appendix~\ref{sec:data_challenges_extended}.
% % and the OSE experiment outlined in section~\ref{sec:ose}~\tocite{}. 
% % For more results regarding the SWOT data, please see section~\ref{sec:other_tasks}. 
% This table highlights the performance statistically in the real and spectral space; the normalized RMSE for the real space and the minimum spatial and temporal scales resolved in the spectral domain. 
% For more information about the class of models displayed and class of metrics, see appendix~\ref{sec:ml_ontology} and appendix~\ref{sec:metrics} respectively. We only showcase the model performance on the alongtrack NADIR data available. For the extended table for each of the challenges, see Table~\ref{tb:exp-results-mega}.}
% \label{tb:oceanbench_results}
% \centering
% \begin{tabular}{lllcccc}
%  \toprule
% % Experiment & Configuration & Method & nRMSE & Resolved Scale [km]    \\ \midrule
% % \multirow{2}{*}{Experiment} & \multirow{2}{*}{Algorithm} & \multirow{2}{*}{Algorithm Class} & \multirow{2}{*}{nRMSE} & \multicolumn{2}{c}{Effective Resolution} \\ 
% % &  &   &  & Wavelength [km]  & Period [days]      \\ \midrule
% % \multirow{2}{*}{Experiment} & \multirow{2}{*}{Algorithm} & \multirow{2}{*}{Algorithm Class} & \multirow{2}{*}{nRMSE} & \multicolumn{2}{c}{Effective Resolution} \\ 
% Experiment &  Algorithm &   Algorithm Class &  nRMSE & $\lambda_{\mathbf{x}}$ [km]  & $\lambda_{t}$ [days]      \\ \midrule
% \multicolumn{1}{l}{OSSE NADIR}     &  OI (app.~\ref{sec:oi}) &  Coordinate-Based & 0.91 $\pm$ 0.01 & 176 & 11.6\\
% \multicolumn{1}{l}{OSSE NADIR}     &  MIOST~\ &  Coordinate-Based  & 0.92 $\pm$ 0.01 & 157 & 10.3 \\
% % \multicolumn{1}{l}{OSSE Gulf}     &  NerF &  Coordinate-Based  & 0.92 $\pm$ 0.01 & ... &\\
% \multicolumn{1}{l}{OSSE NADIR}     &  BFNQG (app.~\ref{sec:bfn}) &  Hybrid Model   & 0.92 $\pm$ 0.01 & 139 & 10.6 \\
% OSSE NADIR &  4DVarNet (app.~\ref{sec:4dvarnet}) &  Bi-Level Opt.  & 0.95 $\pm$ 0.01 & 117 & 7.7 \\
% % \bottomrule
% % OSE Gulf     &  OI &  Coordinate-Based  & ... & ... &\\
% % \multicolumn{1}{l}{OSE Gulf}     &  MIOST &  Coordinate-Based  & ... & ... &\\
% % \multicolumn{1}{l}{OSE Gulf}     &  NerF &  Coordinate-Based  & ... & ... &\\
% % \multicolumn{1}{l}{OSE Gulf}     &  BFNQG &  Hybrid Model  & ... & ... &\\
% % \multicolumn{1}{l}{OSE Gulf}     &  4DVarNet &  Bi-Level Opt.  & ... & ... &\\
% \bottomrule
% \end{tabular}
% \end{table}


\section{Conclusions} \label{sec:conclusions}

The ocean community faces technological and algorithmic challenges to make the most of available observation and simulation datasets. 
In this context, recent studies evidence the critical role of ML schemes in reaching breakthroughs in our ability to monitor ocean dynamics for various space-time scales and processes. 
Nevertheless, domain-specific preprocessing steps and evaluation procedures slow down the uptake of ML toward real-world applications.

Through \texttt{OceanBench} framework, we embed domain-level requirements into the MLOPs considerations by building a flexible framework that adds this into the hyperparameter considerations for ML models. 
We proposed four challenges towards a ML-ready benchmarking suite for ocean observation challenges. 
We outlined the inner workings \texttt{OceanBench} and demonstrated its usefulness by recreating some preprocessing and analysis pipelines from a few data challenges involving SSH interpolation.
We firmly believe that the \texttt{OceanBench} platform is a crucial step to lowering the barrier of entry for new ML researchers interested in applying and developing their methods to relevant problems in the ocean sciences.



%% ACKNOWLEDGEMENTS
\newpage
% \input{./00_Oceanbench/content/acknowledgements}
% \section*{Checklist}

% %%% BEGIN INSTRUCTIONS %%%
% The checklist follows the references.  Please
% read the checklist guidelines carefully for information on how to answer these
% questions.  For each question, change the default \answerTODO{} to \answerYes{},
% \answerNo{}, or \answerNA{}.  You are strongly encouraged to include a {\bf
% justification to your answer}, either by referencing the appropriate section of
% your paper or providing a brief inline description.  For example:
% \begin{itemize}
%   \item Did you include the license to the code and datasets? \answerYes{See Section~\ref{gen_inst}.}
%   \item Did you include the license to the code and datasets? \answerNo{The code and the data are proprietary.}
%   \item Did you include the license to the code and datasets? \answerNA{}
% \end{itemize}
% Please do not modify the questions and only use the provided macros for your
% answers.  Note that the Checklist section does not count towards the page
% limit.  In your paper, please delete this instructions block and only keep the
% Checklist section heading above along with the questions/answers below.
% %%% END INSTRUCTIONS %%%

\begin{enumerate}

\item For all authors...
\begin{enumerate}
  \item Do the main claims made in the abstract and introduction accurately reflect the paper's contributions and scope?
    \answerYes{All the contributions listed in the abstract are elaborated in sections~\ref{sec:code_structure},~\ref{sec:data_challenges} and~\ref{sec:conclusions}}
  \item Did you describe the limitations of your work?
    \answerYes{See the last paragraph of section 5 and the appendix as well.}
  \item Did you discuss any potential negative societal impacts of your work?
    \answerYes{We do not believe that our work has any potential negative societal impacts directly as we do not deal with any confidential or private data. However, we do outline in the appendix how there may be some adverse effects related to downstream uses which could have some negative societal impacts.}
  \item Have you read the ethics review guidelines and ensured that your paper conforms to them?
    \answerYes{We do not include any confidential or private data. We only include numerical values which stem from general physical systems or machine learning models. We do not believe they hold any ethical issues. However, we do acknowledge that there would be environmental damage should users go forward and explore methods which obscenely high computing hours. This discussion outlined in the appendix.}
\end{enumerate}

\item If you are including theoretical results...
\begin{enumerate}
  \item Did you state the full set of assumptions of all theoretical results?
    \answerNA{We do not include any theoretical results.}
	\item Did you include complete proofs of all theoretical results?
    \answerNA{We do not include any theoretical results.}
\end{enumerate}

\item If you ran experiments (e.g. for benchmarks)...
\begin{enumerate}
  \item Did you include the code, data, and instructions needed to reproduce the main experimental results (either in the supplemental material or as a URL)?
    \answerYes{We include the parameters used to reproduce the dataset preprocessing and evaluation procedure in Appendix \ref{sec:data_challenges_extended} and instructions are given to download the data via~\href{https://github.com/quentinf00/oceanbench-data-registry}{https://github.com/quentinf00/oceanbench-data-registry} and rerun the evaluation procedure in our code repository which is available at~\href{https://github.com/jejjohnson/oceanbench}{https://github.com/jejjohnson/oceanbench}.}
  \item Did you specify all the training details (e.g., data splits, hyperparameters, how they were chosen)?
    \answerYes{We showcase all preprocessing steps necessary to reproduce the experimental configurations in Appendix~\ref{sec:data_challenges_extended} and the configuration files are available in our code repository at~\href{https://github.com/jejjohnson/oceanbench}{https://github.com/jejjohnson/oceanbench}. }
	\item Did you report error bars (e.g., with respect to the random seed after running experiments multiple times)?
    \answerNA{This is not applicable for this instantiation because we do not include any randomness within the experiment procedure nor the results.}
	\item Did you include the total amount of compute and the type of resources used (e.g., type of GPUs, internal cluster, or cloud provider)?
    \answerYes{We do not do any model training and leave it up the user for their local or cloud machine. However, we do provide the cloud provider for the data found the the data registry which can be found at~\href{https://github.com/quentinf00/oceanbench-data-registry}{https://github.com/quentinf00/oceanbench-data-registry}}
\end{enumerate}

\item If you are using existing assets (e.g., code, data, models) or curating/releasing new assets...
\begin{enumerate}
  \item If your work uses existing assets, did you cite the creators?
    \answerYes{We adopted the implementation of the preprocessing procedures and evaluation steps with some modifications. We give proper citation and credit to the authors as well as all other existing software packages included in this work.}
  \item Did you mention the license of the assets?
    \answerYes{The appropriate license notices are included in the source code files.}
  \item Did you include any new assets either in the supplemental material or as a URL?
    \answerYes{All the processing and evaluation scripts are included in the GitHub repository.}
  \item Did you discuss whether and how consent was obtained from people whose data you're using/curating?
    \answerYes{We only include data that is already publicly available. We also discussed with the original generators of the datasets and keep the appropriate licenses.}
  \item Did you discuss whether the data you are using/curating contains personally identifiable information or offensive content?
    \answerNA{We do not include any personal information or offensive content in our datasets.}
\end{enumerate}

\item If you used crowdsourcing or conducted research with human subjects...
\begin{enumerate}
  \item Did you include the full text of instructions given to participants and screenshots, if applicable?
    \answerNA{We do not use crowdsourcing and we do not conduct research with human subjects.}
  \item Did you describe any potential participant risks, with links to Institutional Review Board (IRB) approvals, if applicable?
    \answerNA{See the previous point.}
  \item Did you include the estimated hourly wage paid to participants and the total amount spent on participant compensation?
    \answerNA{See the previous point.}
\end{enumerate}

\end{enumerate}

% \bibliographystyle{plain}
% \nocite{*}
% % \documentclass{article}



% \input{content/preamble.tex}
% \input{content/custom}


% \title{\textsc{OceanBench}: \\ The Sea Surface Height Edition}

% % \input{content/authors.tex}

% \begin{document}




% \maketitle
% \def\thefootnote{*}\footnotetext{These authors contributed equally to this work}\def\thefootnote{\arabic{footnote}}
\begin{bibunit}[IEEEtran.bst]

\chapter*{\textsc{OceanBench}: \\ The Sea Surface Height Edition}
\addcontentsline{toc}{chapter}{\textsc{OceanBench}: \\ The Sea Surface Height Edition}
\chaptermark{\textsc{OceanBench}: \\ The Sea Surface Height Edition}
%##################################
% KEYWORDS
% - Sea Surface Height
% - Interpolation
% - Inverse Problem
% - 
% Scientific Machine Learning
% Benchmark, Partial Differential Equations, PINN, FNO, U-Net, Inverse problem
% \linenumbers
% \input{./00_Oceanbench/content/0_abstract}

The previous chapters have underscored the potential of learning-based methods for developing real-world altimetry solutions. The next chapter delves into what deep learning practitioners can contribute effectively to ocean observation science.

It is my strong belief that the adoption of deep learning approaches necessitates collaboration between the oceanography and machine learning communities. To ensure that these methods provide genuine value, evaluation criteria and metrics must be defined with domain expertise by ocean experts. The quality of SSH estimations, for instance, depends on factors such as geographical region, season, physical plausibility of derived quantities. The choice of using observational or simulated data for metric computation yields different assessments, and these distinctions should be carefully considered.

Evaluation criteria should remain adaptable, especially for deep learning methods, which may optimize training objectives by taking shortcuts (as per the "No Free Lunch" theorem). Consequently, the next chapter introduces a platform that facilitates an ongoing dialogue between deep learning experts and oceanography specialists to develop and assess these methods effectively.

On the data front, the selection of data used to train deep learning models is a critical consideration when shaping the final method. As demonstrated in previous chapters, both observational and simulation data can be instrumental in calibrating and evaluating the method. These data undergo domain-specific processing steps. Accessibility to the data and the relevant processing steps can significantly lower the entry barriers for aspiring machine learning practitioners.

These considerations provide the motivation for the work presented in the next chapter. Oceanbench is a suite of tools designed to load and process ocean data. The software is organized into a series of tasks, each characterized by evaluation data and metrics. Its purpose is to simplify the loading and formatting of data relevant for training neural schemes and to streamline the configuration of new evaluation setups for diverse tasks, regions, and datasets.

\input{./00_Oceanbench/content/1_motivation}
\input{./00_Oceanbench/content/2_related_work}
\input{./00_Oceanbench/content/3_oceanbench}
\input{./00_Oceanbench/content/4_tasks}
\input{./00_Oceanbench/content/5_conclusions}

%% ACKNOWLEDGEMENTS
\newpage
% \input{./00_Oceanbench/content/acknowledgements}
% \input{./00_Oceanbench/content/checklist}

% \bibliographystyle{plain}
% \nocite{*}
% \input{neurips_data_2023.bbl}
% \bibliography{
% content/biblio, 
% content/bibliographies/software, 
% content/bibliographies/machine_learning, 
% content/bibliographies/sea_surface_height,
% content/bibliographies/ocean,
% content/bibliographies/data,
% content/bibliographies/applications
% }

% \section*{References}

% References follow the acknowledgments. Use unnumbered first-level heading for
% the references. Any choice of citation style is acceptable as long as you are
% consistent. It is permissible to reduce the font size to \verb+small+ (9 point)
% when listing the references.
% Note that the Reference section does not count towards the page limit.
% \medskip

% {
% \small

% [1] Alexander, J.A.\ \& Mozer, M.C.\ (1995) Template-based algorithms for
% connectionist rule extraction. In G.\ Tesauro, D.S.\ Touretzky and T.K.\ Leen
% (eds.), {\it Advances in Neural Information Processing Systems 7},
% pp.\ 609--616. Cambridge, MA: MIT Press.

% [2] Bower, J.M.\ \& Beeman, D.\ (1995) {\it The Book of GENESIS: Exploring
%   Realistic Neural Models with the GEneral NEural SImulation System.}  New York:
% TELOS/Springer--Verlag.

% [3] Hasselmo, M.E., Schnell, E.\ \& Barkai, E.\ (1995) Dynamics of learning and
% recall at excitatory recurrent synapses and cholinergic modulation in rat
% hippocampal region CA3. {\it Journal of Neuroscience} {\bf 15}(7):5249-5262.
% }

%%%%%%%%%%%%%%%%%%%%%%%%%%%%%%%%%%%%%%%%%%%%%%%%%%%%%%%%%%%%
\newpage
% \appendix
% \section*{\textsc{OceanBench}: The Sea Surface Height Edition - Supplementary Material}
% % \newpage
% \input{content/appendix/data_challenges}
% \newpage
% \input{content/b_physical_variables}
% \newpage
% \input{content/c_metrics}
% \newpage
% \input{content/appendix/hydra}
% \newpage
% \input{content/a3_usecase_patcher}
% \newpage
% \input{content/e_other_tasks}
% \newpage
% \input{content/f_ml_methods}
% % \newpage
% % \input{content/g_userstories}
% \newpage
% \input{content/h_limitations}

\putbib[./00_Oceanbench/content/bibliographies/full.bib]
\end{bibunit}

% \bibliography{
% content/biblio, 
% content/bibliographies/software, 
% content/bibliographies/machine_learning, 
% content/bibliographies/sea_surface_height,
% content/bibliographies/ocean,
% content/bibliographies/data,
% content/bibliographies/applications
% }

% \section*{References}

% References follow the acknowledgments. Use unnumbered first-level heading for
% the references. Any choice of citation style is acceptable as long as you are
% consistent. It is permissible to reduce the font size to \verb+small+ (9 point)
% when listing the references.
% Note that the Reference section does not count towards the page limit.
% \medskip

% {
% \small

% [1] Alexander, J.A.\ \& Mozer, M.C.\ (1995) Template-based algorithms for
% connectionist rule extraction. In G.\ Tesauro, D.S.\ Touretzky and T.K.\ Leen
% (eds.), {\it Advances in Neural Information Processing Systems 7},
% pp.\ 609--616. Cambridge, MA: MIT Press.

% [2] Bower, J.M.\ \& Beeman, D.\ (1995) {\it The Book of GENESIS: Exploring
%   Realistic Neural Models with the GEneral NEural SImulation System.}  New York:
% TELOS/Springer--Verlag.

% [3] Hasselmo, M.E., Schnell, E.\ \& Barkai, E.\ (1995) Dynamics of learning and
% recall at excitatory recurrent synapses and cholinergic modulation in rat
% hippocampal region CA3. {\it Journal of Neuroscience} {\bf 15}(7):5249-5262.
% }

%%%%%%%%%%%%%%%%%%%%%%%%%%%%%%%%%%%%%%%%%%%%%%%%%%%%%%%%%%%%
\newpage
% \appendix
% \section*{\textsc{OceanBench}: The Sea Surface Height Edition - Supplementary Material}
% % \newpage
% \section{Data Challenges} \label{sec:data_challenges_extended}

In this section, we highlight some details that were omitted in section~\ref{sec:data_challenges}.
This includes details about the simulation type, the data structures, and the training/evaluation periods.

\subsection{OSSE NADIR}

The reference simulation is the \textit{NATL60} simulation based on the NEMO model~\cite{NEMOAJAYI2020}. 
This particular simulation was run over an entire year without any tidal forcing.
The simulation provides the outputs of SSH, SST, sea surface salinity (SSS) and the u,v velocities every 1 hour.
For the purposes of this data challenge, the spatial domain is over the Gulfstream with a spatial domain of $[-65^\circ, -55^\circ]$ longitude and $[33^\circ, 43^\circ]$ latitude.
The resolution of the original simulation is 1/60$^\circ$ resolution with hourly snapshots, and we consider a daily downsampled trajectory at 1/20$^\circ$ for the data challenge which results in a 365x200x200 spatio-temporal grid.
This simulation resolves finescale dynamical processes ($\sim$15km) which makes it a good test bed for creating an OSSE environment for mapping.
The SSH observations include simulations of ocean satellite NADIR tracks.
In particular, they are simulations of Topex-Poseidon, Jason 1, Geosat Follow-On, and Envisat.
There is no observation error considered within the challenge.
We use a the entire period from 2012-10-10 until 2013-09-30.
A training period is only from 2013-01-02 to 2013-09-30 where the users can use the reference simulation as well as all available simulated observations.
The evaluation period is from 2012-10-22 to 2012-12-02 (i.e. 41 days) which is considered decorrelated from the training period. 
During the evaluation period, the user cannot use the reference NATL60 simulation but they can use all available simulated observations. There is also a spin-up period allowance from 2012-10-01 where the user can also use all available simulated observations.

\subsection{OSSE SWOT \& OSSE SST}

For the OSSE SWOT and OSSE SST experiments, the reference simulation, domain, and evaluation period is the same as the OSSE NADIR experiment.
However, the OSSE SWOT includes simulated observations of the novel KaRIN sensor recently deployed during the SWOT mission, the pseudo-observations were generated using the SWOT simulator~\cite{SWOT}. 
This OSSE SST experiment allows the users to utilize the full fields of SST as inputs to help reconstruct the SSH field in conjunction with the NADIR and SWOT SSH observation.
Because the SST comes from the same NATL60 simulation, the geometry characteristics SST and SSH are exactly the same.

\subsection{OSE NADIR}

The OSE NADIR experiment only uses real observations aggregated from different altimeters. These SSH observations include observations from the SARAL/Altika, Jason 2, Jason 3, Sentinel 3A, Haiyang-2A and Cryosat-2 altimeters. The Cryosat-2 altimeter is used as the independent evaluation track used to assess the performance of the reconstructed SSH field.

\subsection{Results}

We use \texttt{OceanBench} to generate maps of relevant quantities from the 4DVarNet method~\cite{4DVARNETSWOT,4DVARNETSST}.
Figure~\ref{fig:oceanbench_maps_4dvarnet} showcases some demo maps for some key physical variables outlined in section~\ref{sec:physical_variables}.
We showcase the 4DVarNet method because it is the SOTA method that was applied to each of the data challenges.
We can see that the addition of more information, i.e. NADIR -> SWOT -> SST, results in maps look more similar to the NEMO simulation in the OSSE challenges.
It also produces sensible maps for the OSE challenge as well.

\texttt{OceanBench} also generated figure~\ref{fig:oceanbench_psd_4dvarnet} which shows plots of the PSD and PSD scores of SSH for the different challenges.
Again, as we increase the efficacy of the observations via SWOT and allow for more external factors like the SST, we get an improvement in the isotropic and spacetime PSD scores.
In addition, we see that the PSD plots for the OSE task look very similar to the OSE challenges. 

Lastly, we used \texttt{OceanBench} to generate a leaderboard of metrics for a diverse set of algorithms where the maps were available online.
Table~\ref{tb:exp-results-mega} displays all of the key metrics outlined in section~\ref{sec:metrics} including the normalized RMSE and various spectral scores which are appropriate for the challenge.
We see that as the complexity of the method increases, the metrics improve. 
In addition, the methods that involve end-to-end learning perform the best overall, i.e. 4DVarNet.

\begin{figure}[ht!]
\small
\begin{center}
\setlength{\tabcolsep}{1pt}
\begin{tabular}{cccc}
\hspace{3mm} Task OSSE & 
\hspace{3mm} Task OSSE & 
\hspace{2mm} Task OSSE & 
Task OSE \\
\hspace{3mm}  Nadir & 
\hspace{3mm} Nadir + SWOT & 
\hspace{2mm} Nadir + SST & 
Nadir \\
%\vspace{-2mm}
%%%%% SSH %%%%%%%%
\includegraphics[trim={0 13mm 22mm 0},clip, width=3.60cm,height=3.2cm]{content/figures/fourdvarnet_figs/osse_gf_nadir_ssh.png} &
\includegraphics[trim={13mm 13mm 22mm 0},clip, width=3.2cm,height=3.2cm]{content/figures/fourdvarnet_figs/osse_gf_nadirswot_ssh.png} &
\includegraphics[trim={13mm 13mm 22mm 0},clip, width=3.2cm,height=3.2cm]{content/figures/fourdvarnet_figs/osse_gf_nadir_sst_ssh.png} &
\includegraphics[trim={13mm 13mm 0 0},clip,width=4.0cm,height=3.2cm]{content/figures/fourdvarnet_figs/ose_gf_ssh.png} \\
%\vspace{3mm}
%%%%% KINETIC ENERGY %%%%%%%%
\includegraphics[trim={0 13mm 22mm 5mm}, clip, width=3.60cm,height=3cm]{content/figures/fourdvarnet_figs/osse_gf_nadir_ke.png} &
\includegraphics[trim={13mm 13mm 22mm 5mm},clip, width=3.2cm,height=3cm]{content/figures/fourdvarnet_figs/osse_gf_nadirswot_ke.png} &
\includegraphics[trim={13mm 13mm 22mm 5mm},clip, width=3.2cm,height=3cm]{content/figures/fourdvarnet_figs/osse_gf_nadir_sst_ke.png} &
\includegraphics[trim={13mm 13mm 0 5mm},clip,width=4cm,height=3cm]{content/figures/fourdvarnet_figs/ose_gf_ke.png} \\
%%%%% RELATIVE VORTICITY %%%%%%%%
\includegraphics[trim={0 13mm 21.2mm 5mm},clip, width=3.60cm,height=3cm]{content/figures/fourdvarnet_figs/osse_gf_nadir_vort_r.png} &
\includegraphics[trim={13mm 13mm 21.2mm 5mm},clip, width=3.2cm,height=3cm]{content/figures/fourdvarnet_figs/osse_gf_nadirswot_vort_r.png} &
\includegraphics[trim={13mm 13mm 21.2mm 5mm},clip, width=3.2cm,height=3cm]{content/figures/fourdvarnet_figs/osse_gf_nadir_sst_vort_r.png} &
\includegraphics[trim={13mm 13mm 0 5mm},clip,width=4.0cm,height=3cm]{content/figures/fourdvarnet_figs/ose_gf_vort_r.png} \\
%%%%% STRAIN %%%%%%%%
\includegraphics[trim={0 0 19mm 5mm},clip, width=3.60cm,height=3.4cm]{content/figures/fourdvarnet_figs/osse_gf_nadir_strain.png} &
\includegraphics[trim={13mm 0 19mm 5mm},clip, width=3.2cm,height=3.4cm]{content/figures/fourdvarnet_figs/osse_gf_nadirswot_strain.png} &
\includegraphics[trim={13mm 0 19mm 5mm},clip, width=3.2cm,height=3.4cm]{content/figures/fourdvarnet_figs/osse_gf_nadir_sst_strain.png} &
\includegraphics[trim={13mm 0 0 5mm},clip,width=4.0cm,height=3.4cm]{content/figures/fourdvarnet_figs/ose_gf_strain.png} \\
% \vspace{-2mm}
(a) & (b) & (c) & (d)
\end{tabular}
\vspace{-3mm}
% \caption{Row I - Isotrophic PSD. Row 2 - Isotrophic PSD Score}
\caption{
Reconstructed quantities by the 4dVarNet method for each of the four tasks.
Each row showcases the following physical variables found in appendix~\ref{sec:physical_variables}: (a) Sea Surface Height, (b) Kinetic Energy, (c) Relative Vorticity, and (d) Strain. 
Each column showcase the reconstructed from the tasks (a) OSSE using only Nadir tracks: (b) OSSE using Nadir tracks and SWOT swath, (c) Multimodal using Nadir tracks and sea surface temperature, and (d) Reconstruction using real nadir altimetry tracks.}
\vspace{-5mm}
\label{fig:oceanbench_maps_4dvarnet}
\end{center}
\end{figure}





% \include{content/appendix/fourdvarnet_maps}


\begin{table}[ht]
\caption{This table showcases all of the summary statistics for some methods for each of the data challenges listed in section~\ref{sec:data_challenges} and~\ref{sec:data_challenges_extended}. The summary statistics shown are the normalized RMSE and the effective resolution in the spectral domain. The spectral metrics for the effective resolution that were outlined in section~\ref{sec:metrics} are: i) $\lambda_a$ is the spatial score for the alongtrack PSD score, ii) $\lambda_r$ is the spatial score for the isotropic PSD, iii) $\lambda_x$ is the spatial score for space-time PSD score, and iv) $\lambda_t$ is the temporal score for the space-time PSD score.}
% \caption{This table highlights some of the results for the OSSE experiments outlined in section~\ref{sec:osse} and~\ref{sec:other_tasks}.

% This table highlights the performance statistically in the real and spectral space; the normalized RMSE for the real space and the minimum spatial and temporal scales resolved in the spectral domain. 
% For more information about the class of models displayed and class of metrics, see section~\ref{sec:ml_ontology} and section~\ref{sec:metrics} respectively.}
\label{tb:exp-results-mega}
\centering
\begin{tabular}{llcccccc}
 \toprule
% Experiment & Configuration & Method & nRMSE & Resolved Scale [km]    \\ \midrule
% \multirow{2}{*}{Experiment} & \multirow{2}{*}{Algorithm} & \multirow{2}{*}{Algorithm Class} & \multirow{2}{*}{nRMSE} & \multicolumn{2}{c}{Effective Resolution} \\ 
% &  &   &  & Wavelength [km]  & Period [days]      \\ \midrule
% \multirow{2}{*}{Experiment} & \multirow{2}{*}{Algorithm} & \multirow{2}{*}{Algorithm Class} & \multirow{2}{*}{nRMSE} & \multicolumn{2}{c}{Effective Resolution} \\ 
\multirow{2}{*}{Experiment} &  \multirow{2}{*}{Algorithm} &   \multirow{2}{*}{nRMSE} &
\multicolumn{4}{c}{Effective Resolution} \\
& & & $\lambda_{a}$ [km] & $\lambda_{r}$ [km]   &  $\lambda_{\mathbf{x}}$ [km]  &   $\lambda_{t}$ [days]      \\ \midrule
OSSE NADIR     &  OI & 0.92 & - & 123 & 174 & 10.8 \\
OSSE NADIR     &  MIOST &  0.93 & - & 100 & 157 & 10.1 \\
OSSE NADIR     &  BFNQG & 0.93 & - & 88 & 139 & 10.4 \\
OSSE NADIR &  4DVarNet &  \textbf{0.94} & - & \textbf{65} & \textbf{117} & \textbf{7.7} \\
\midrule
OSSE SWOT     &  OI & 0.92 & - & 106 & 139 & 11.7 \\
OSSE SWOT     &  MIOST &  0.94 & - & 88 & 131 & 10.1 \\
OSSE SWOT     &  BFNQG & 0.94 & - & 64 & 118 & 36.5 \\
OSSE SWOT &  4DVarNet &  \textbf{0.96} & - & \textbf{47} & \textbf{77} & \textbf{5.6} \\
\midrule
OSSE SST     &  Musti & 0.95 & - & 46 & 138 & 4.1 \\
OSSE SST &  4DVarNet &  \textbf{0.96} & - & \textbf{46} & \textbf{87} & \textbf{3.7} \\
\midrule
OSE NADIR     &  OI & 0.88 & 151 & - &  - &  -\\
OSE NADIR     &  MIOST &  0.90 & 135 & - &  - &  -\\
OSE NADIR     &  BFNQG & 0.88 & 122 & - & - &  -\\
OSE NADIR &  ConvLSTM &  0.89 & 113 &- &  - &  -\\
OSE NADIR &  4DVarNet & \textbf{0.91} & \textbf{98} & - &  -  &  -\\
\bottomrule
\end{tabular}
\end{table}

% \subsection{Simulated Altimetry Tracks} \label{sec:dc_osse_nadir}

% \textcolor{red}{
% The most commonly used SSH maps, the Developing Use of Altimetry for Climate Studies (DUACS) products, are derived from a statistical space–time interpolation of nadir altimeter observations. This intrinsically limits the effective resolution [as defined in Skamarock (2004), i.e., the fully resolved scales] of DUACS SSH maps to 150–200 km at middle latitudes (Ballarotta et al. 2019). The SSH mapping algorithm was developed by CNES and CLS in 1997, as part of the DUACS project, and has been continuously improved since then (Taburet et al. 2019). The DUACS products are now distributed by the Copernicus Marine Environment Monitoring Service (CMEMS). DUACS algorithm implements a statistical interpolation of SSH satellite data in space and time to produce global daily maps (Le Traon et al. 1998). The data are collected by a constellation of 2 to 4 nadir-looking altimeters (sometimes referred to as conventional altimeters), and characterized by large data gaps reaching 200 km in the zonal direction at the equator.
% }

% \subsection{Simulated SWOT Tracks} \label{sec:dc_osse_swot}


% \textcolor{red}{
% The Surface Water and Ocean Topography (SWOT; Fu et al. 2012; Morrow et al. 2019) altimetry mission, to be launched in early 2022, will open the way to SSH maps with resolution significantly higher than 150 km at midlatitudes, but this perspective entails a thorough revisit of the mapping algorithm. SWOT will considerably increase the measurement density at the surface of the oceans thanks to SSH measurements at a kilometric pixel resolution over a swath 120 km wide. On the swath, SWOT is expected to resolve scales down to 15 km at low latitude and 30–45 km at mid- and high latitudes (Wang et al. 2019). In its science phase, SWOT will have a 21 days repeat orbit, allowing an average revisit time of 11 days in most of the globe. Some of the dynamical processes observable by SWOT evolve over time scales on the order of 1 day, much shorter than the satellite revisit time. Consequently, the mapping method implemented in the current DUACS system will certainly not be sufficient to draw the maximum benefit from SWOT. A linear interpolation will filter most of the observed small-scale signals between two passes of the satellite, as anticipated by Gaultier et al. (2016).These authors advocate for using more advanced methods to build SSH maps.
% }

% \subsection{Multimodal with Sea Surface Temperature}  \label{sec:dc_osse_sst}


% \subsection{Real Altimetry Tracks}  \label{sec:dc_ose_nadir}
% \newpage
% \section{Physical Variables} \label{sec:physical_variables}

As alluded to in the main body of the paper, we have access to many physical quantities which can be derived from  sea surface height. 
This gives us a way to analyze how effective and trustworthy are our reconstructions. 
Many machine learning methods are unconstrained so they may provide solutions that are physically inconsistent and visualizing the field is a very easy eye test to assess the validity. 
In addition to post analysis, one could include some of these derived quantities maybe useful as additional inputs to the system and/or constraints to the loss function. 
Recall the spatiotemporal coordinates from equation~\ref{eq:spatiotemporal_coords}, 
we use the same coordinates for the subsequent physical quantities. \textbf{Sea Surface Height} is the deviation of the height of the ocean surface from the geoid of the Earth. We can define it as:
\begin{align}
	\text{Sea Surface Height }[m]:&& \quad
 \eta &= \boldsymbol{\eta}(\mathbf{x},t)&& \quad \Omega\times \mathcal{T}\rightarrow\mathbb{R} \label{eq:ssh}
\end{align}
This quantity is the actual value that is given from the satellite altimeters and is presented in the products for SSH maps~\cite{DUACS}. An example can be seen in the first row of figure~\ref{fig:oceanbench_maps_4dvarnet}.

\textbf{Sea Surface Anomaly} is the anomaly wrt to the spatial mean which is defined by
\begin{align}
	\text{Sea Level Anomaly }[m]:&& \quad
 \bar{\eta} &= \boldsymbol{\eta}(\mathbf{x},t) - \bar{\eta}(t) &&
 \quad \Omega\times \mathcal{T}\rightarrow\mathbb{R} \label{eq:sla}
\end{align}
where $\bar{\eta}(t)$ is the spatial average of the field at each time step.  
An example can be seen in the first row of figure~\ref{fig:oceanbench_maps}.

Another important quantity is the \textbf{geostrophic velocities} in the zonal and meridional directions. This is given by
\begin{align}
	\text{Zonal Velocity}[ms^{-2}]:&& \quad
 u &= -\frac{g}{f_0}\frac{\partial \eta}{\partial y} &&
 \quad \Omega\times \mathcal{T}\rightarrow\mathbb{R} \label{eq:u_vel} \\
	\text{Meridional Velocity}[ms^{-2}]:&& \quad
 v &= \frac{g}{f_0}\frac{\partial \eta}{\partial x} &&
 \quad \Omega\times \mathcal{T}\rightarrow\mathbb{R} \label{eq:v_vel}
\end{align}
where $g$ is the gravitational constant and $f_0$ is the mean Coriolis parameter. These quantities are important as they can be an related to the sea surface current. The geostrophic assumption is a very strong assumption however it can still be an important indicator variable. The \textbf{kinetic energy} is a way to summarize the (geostrophic) velocities as the total energy of the system. This is given by
\begin{equation} \label{eq:kineticenergy}
    KE = \frac{1}{2}\left(u^2 + v^2\right)
\end{equation}
An example can be seen in the second row of figure~\ref{fig:oceanbench_maps_4dvarnet}.

Another very important quantity is the \textit{vorticity} which measures the spin and rotation of a fluid. In geophysical fluid dynamics, we use the \textbf{relative vorticity} which is the vorticity observed within at rotating frame.
This is given by
\begin{equation} \label{eq:relvorticity}
    \zeta = \frac{\partial v}{\partial x} - \frac{\partial u}{\partial y}
\end{equation}
An example can be seen in the third row of figure~\ref{fig:oceanbench_maps_4dvarnet}.

% \subsection{Absolute Vorticity}

% \begin{equation} \label{eq:absvorticity}
%     |\zeta| = \frac{\partial v}{\partial x} + \frac{\partial u}{\partial y}
% \end{equation}

We can also use the \textbf{Enstrophy} to summarize the relative voriticty to measure the total contribution which is given by
\begin{equation} \label{eq:enstrophy}
    E = \frac{1}{2}\zeta^2
\end{equation}

The \textbf{Strain} is a measure of deformation of a fluid flow.

\begin{equation} \label{eq:strain}
    \sigma = \sqrt{\sigma_n^2 + \sigma_s^2}
\end{equation}

where $\sigma_n$ is the shear strain (aka the shearing deformation) and $\sigma_s$ is the normal strain (aka stretching deformation). An example can be seen in the fourth row of figure~\ref{fig:oceanbench_maps_4dvarnet}.

The \textbf{Okubo-Weiss Parameter} is high-order quantity which is a linear combination of the strain and the relative vorticity.

\begin{equation} \label{eq:okuboweiss}
    \sigma_{ow} = \sigma_n^2 + \sigma_s^2 - \zeta^2
\end{equation}

This quantity is often used as a threshold for determining the location of Eddies in sea surface height and sea surface current fields~\cite{OKUBO, WEISS, OKUBOWEISS}.

% \begin{table}[h!]
%   \caption{Table of nomanclature}
%   \label{sample-table}
%   \centering
%   \begin{tabular}{ccl}
%     \toprule
%     Symbol & Size & Description  \\
%     \midrule
%     $\mathbf{x}_s$ & $\mathbb{R}^{D_s}$ & Spatial Coordinates  \\
%     $t$ & $\mathbb{R}^{D_t}$ & Temporal Coordinates  \\
%    $\boldsymbol{f}(\mathbf{x}_s, t)$ & $\mathbb{R}^{D}$ & Spatiotemporal Field  \\
%    $\boldsymbol{y}_{obs}(\mathbf{x}_s, t)$ & $\mathbb{R}^{D_{obs}}$ & Spatiotemporal Observations  \\
%    $\eta$ & $\mathbb{R}$ & Sea Surface Height $[m]$ \\
%    $\bar{\eta}$ & $\mathbb{R}$ & Sea Surface Anomaly $[m]$ \\
%    $u$ & $\mathbb{R}$ & Zonal Velocity $[ms^{-2}]$ \\
%    $v$ & $\mathbb{R}$ & Meridional Velocity $[ms^{-2}]$ \\
%     \bottomrule
%   \end{tabular}
%   \label{tb:notation}
%  \end{table}


% \subsection{Coordinates}
% \textbf{SpatioTemporal Coordiantes}. We define some generic spatiotemporal coordinates.
% 
% We are dealing with satellite observations, so we are interested in the domain across the Earth's surface. 
% Let us define the Earth's domain by some spatial coordinates, $\mathbf{x} = [\text{Longitude},\text{Latitude}]^\top \in\mathbb{R}^{D_s}$, and temporal coordinates, $t=[\text{Time}]\in\mathbb{R}^+$, where $D_s$ is the dimensionality of the coordinate vector.  
% We can define some spatial (sub-)domain, $\Omega\subseteq\mathbb{R}^{D_s}$, and a temporal (sub-)domain, $\mathcal{T}\subseteq\mathbb{R}^+$. 
% This domain could be the entire globe for 10 years or a small region within the North Atlantic for 1 year.
%
%
% \begin{align} \label{eq:spatiotemporal_coords}
%     \text{Spatial Coordinates}: && \mathbf{x} &\in \Omega \subseteq \mathbb{R}^{D_s}\\ 
%     \text{Temporal Coordinates}: && t &\in \mathcal{T} \subseteq \mathbb{R}^{D_t}.
% \end{align}
% %
%
% In this case $D_s=2$ because we only have a two coordinates, however we can do some coordinate transformations like spherical to Cartesian. Likewise, we can do some coordinate transformation for the temporal coordinates like cyclic transformations or sinusoidal embeddings~\cite{ATTENTION}. We have two fields of interest from these spatiotemporal coordinates: the state and the observations.
% %
% %
% \begin{align} \label{eq:state_obs}
%     \text{State}: && \boldsymbol{u}(\mathbf{x},t) &: \Omega\times\mathcal{T}\rightarrow\mathbb{R}^{D_u} \\
%     \text{Observations}: && \boldsymbol{y}_{obs}(\mathbf{x},t) &: \Omega\times\mathcal{T}\rightarrow\mathbb{R}^{D_{obs}}
% \end{align}
% %
% %
% The state domain, $u\in\mathcal{U}$, is a scalar or vector-valued field of size $D_u$ which is typically the quantity of interest and the observation domain, $y_{obs}\in\mathcal{Y}_{obs}$, is the observable quantity which is also a scalar or vector-valued field of size $D_{obs}$. Now, we make the assumption that we have an operator $\mathcal{H}$ that transforms the field from the state space, $\boldsymbol{u}$, to the observation space, $\boldsymbol{y}_{obs}$.
% %
% %
% \begin{align} \label{eq:prob_definition}
%     \boldsymbol{y}_{obs}(\mathbf{x},t) = \mathcal{H}\left(\boldsymbol{u}(\mathbf{x},t), t, \boldsymbol{\varepsilon}, \boldsymbol{\mu}\right) 
% \end{align}
% %
% %


% % \subsection{Field}

% \textbf{Field}. We have the generic definition of a scalar or vector-valued field.

% \begin{align} \label{eq:field}
% \text{Field}:&& f&=\boldsymbol{f}(\mathbf{x},t), && \quad \Omega\times \mathcal{T}\rightarrow\mathbb{R}^{D}
% \end{align}
% \newpage
% \section{Metrics} \label{sec:metrics}

There are many metrics that are standard within the ML community but unconvincing for many parts the geoscience community. 
Specifically, many of these standard scores do not capture the important optimization criteria in the scientific machine learning tasks.
However, there is not consensus within domain-specific communities about the perfect metric which captures every aspect we are interested.
Therefore, we should have a variety of scores from different perspectives to really assess the pros and cons of each method we wish to evaluate thoroughly. 
Below, we outline two sets of scores we use within this framework: skill scores and spectral scores.

\subsection{Skill Scores}

We classify one set of metrics as \textit{skill scores}. 
These are globally averaged metrics which tend to operate within the real space.
Some examples include the root mean squared error (RMSE) and the normalized root mean squared (nRMSE) error.
The RMSE metric can also be calculated w.r.t. the spatial domain, temporal domain or both. 
For example, figure~\ref{fig:oceanbench_psd} showcases the nRMSE calculated only on the spatial domain and visualized for each time step.
%
\begin{align}
    \text{RMSE}: &&\text{RMSE}(\eta,\hat{\eta}) &= ||\eta - \hat{\eta}||_2 \label{eq:RMSE}\\
    % \text{RMSE}_t: &&\text{RMSE}_t(\eta,\hat{\eta}; t) &= ||\eta(t) - \hat{\eta}(t)||_2 \label{eq:RMSE_t}\\
    \text{nRMSE}: &&\text{nRMSE}(\eta,\hat{\eta}) &= 1 - \frac{\text{RMSE}(\eta - \hat{\eta})}{\text{RMSE}(\eta)} \label{eq:nRMSE}
\end{align}
%
However, we are not limited to just the standard MSE metrics.
We can easily incorporate more higher-order statistics like the Centered Kernel Alignment (CKA)~\cite{METRICSCKA} or information theory metrics like mutual information (MI)~\cite{METRICSITRBIG,METRICSITRBIG2}.
In addition, we could also utilize the same metrics in the frequency domain as is done in~\citep{PDEBench}.

\subsection{Spectral Scores}

Another class of scores that we use in \texttt{OceanBench} are the \textit{spectral scores}. These scores are calculated within the spectral space via the wavenumber power spectral density (PSD). 
This provides a spatial-scale-dependent metric which is useful for identifying the largest and smallest scales that were resolved by the reconstruction map. 
In general, we use these to measure the expected energy at different spatiotemporal scales and we can also construct custom score functions which gives us a summary statistic for how well we reconstructed certain scales.
%
\begin{align}
    \text{PSD}: &&\text{PSD}(\eta) &= \sum_{k_{min}}^{k_{max}}\|\mathcal{\mathcal{F}(\eta)}\|^2\label{psd}\\
    \text{PSD}_{score}: &&\text{PSD}_{score}(\eta,\hat{\eta}) &= 1 - \frac{\text{PSD}(\eta - \hat{\eta})}{\text{PSD}(\eta)} \label{eq:psd_score}
\end{align}
%
where $\mathcal{F}$ is the Fast Fourier Transformation (FFT). 
In our application, there are various ways to construct the PSD which depend on the FFT transformation.
We denote the \textit{space-time PSD} as $\lambda_\mathbf{x}$ which does the 2D FFT in the longitude and time direction, then takes the average over the latitude.
We denote the \textit{space-time PSD} as $\lambda_\mathbf{t}$ which does the 2D FFT in the longitude and latitude direction, then takes the average over the time.
We denote the \textit{isotropic PSD} as $\lambda_r$ which assumes a radial relationship in the spatial domain and then averages over the temporal domain.
Lastly, we denote the standard PSD score as $\lambda_a$ which is the 1D FFT over a prescribed distance along the satellite track; this is what is done for the OSE NADIR experiment.
We recognize that the FFT configurations are limited due to their global treatment of the spectral domain and we need more specialized metrics to handle the local scales.
This opens the door to new metrics that handle such cases such as the Wavelet transformation~\cite{METRICSWAVELET}.

\include{00_Oceanbench/content/appendix/fourdvarnet_psds}

% \begin{figure}[h]
% \small
% \begin{center}
% \setlength{\tabcolsep}{1pt}
% \begin{tabular}{cccc}
% \hspace{3mm} NEMO Simulation & 
% \hspace{3mm} MIOST & 
% \hspace{3mm} BFNQG & 
% 4DVarNet \\
% \vspace{-2mm}
% %%%%% SSH %%%%%%%%
% \includegraphics[trim={0 0 38mm 0},clip, width=3.20cm,height=3cm]{content/figures/psd_spacetime/dc20a/nadir4/dc20a_psd_spacetime_nemo_nadir4_ssh.png} &
% \includegraphics[trim={0 0 40mm 0},clip, width=3.2cm,height=3cm]{content/figures/psd_spacetime/dc20a/nadir4/dc20a_psd_spacetime_miost_nadir4_ssh.png} &
% \includegraphics[trim={0 0 38mm 0},clip, width=3.2cm,height=3cm]{content/figures/psd_spacetime/dc20a/nadir4/dc20a_psd_spacetime_bfnqg_nadir4_ssh.png} &
% \includegraphics[width=4.0cm,height=3cm]{content/figures/psd_spacetime/dc20a/nadir4/dc20a_psd_spacetime_4dvarnet_nadir4_ssh.png} \\
% \end{tabular}
% % \vspace{-3mm}
% % \caption{Row I - Isotrophic PSD. Row 2 - Isotrophic PSD Score}
% \caption{The space-time power spectrum decomposition.}
% % \vspace{-5mm}
% \label{fig:appendix_psd_spacetime_NADIR}
% \end{center}
% \end{figure}


% \begin{figure}[h]
% \small
% \begin{center}
% \setlength{\tabcolsep}{1pt}
% \begin{tabular}{cccc}
% \hspace{3mm} DUACS & 
% \hspace{3mm} MIOST & 
% \hspace{3mm} BFNQG & 
% 4DVarNet \\
% % \vspace{-2mm}
% %%%%% SSH %%%%%%%%
% \includegraphics[trim={0 0 38mm 0},clip, width=3.20cm,height=3cm]{content/figures/psd_spacetime/dc20a/nadir4/dc20a_psd_spacetime_score_duacs_nadir4_ssh.png} &
% \includegraphics[trim={0 0 40mm 0},clip, width=3.2cm,height=3cm]{content/figures/psd_spacetime/dc20a/nadir4/dc20a_psd_spacetime_score_miost_nadir4_ssh.png} &
% \includegraphics[trim={0 0 38mm 0},clip, width=3.2cm,height=3cm]{content/figures/psd_spacetime/dc20a/nadir4/dc20a_psd_spacetime_score_bfnqg_nadir4_ssh.png} &
% \includegraphics[width=4.0cm,height=3cm]{content/figures/psd_spacetime/dc20a/nadir4/dc20a_psd_spacetime_score_4dvarnet_nadir4_ssh.png} \\
% \end{tabular}
% % \caption{Row I - Isotrophic PSD. Row 2 - Isotrophic PSD Score}
% \caption{The space-time power spectrum score decomposition.}
% % \vspace{-5mm}
% \label{fig:appendix_psd_score_spacetime_NADIR}
% \end{center}
% \end{figure}
% \newpage
% \section{Use Case II: Hydra Recipes} \label{sec:hydra_recipes}

This framework has drastically reduced the overhead for the ML researcher while also enhancing the reprducibility and replicability of the preprocessing steps. In this section we showcase a few examples for how one can use oceanbench in conjunction with hydra to provide recipes for some standard processes.

\subsection{GeoProcessing Recipe}

In this example, we showcase how one can pipe a sequential transformation through the hydra framework. In this example, we open the dataset, validate the coordinates to comply to our standards, select the region of interest, subset the data, regrid the alongtrack data to a uniform grid, and save the data to a netcdf file. See the listing~\ref{hydraconfig:geoprocess} for more information.


\begin{listing}[ht!]
\begin{minted}[frame=lines]{yaml}
# Target Function to initialize
_target_: "oceanbench._src.dataset.pipe"
# netcdf file to be loaded
inp: "${data_directory}/nadir_tracks.nc"
# sequential transformations to be applied
fns:
    # Load Dataset
    - {_target_: "xarray.open_dataset", _partial_: True}
    # Validate LatLonTime Coordinates
    - {_target_: "oceanbench.validate_latlon", _partial_: True}
    - {_target_: "oceanbench.validate_time", _partial_: True}
    # Select Specific Region (Spatial | Temporal)
    - {_target_: "xarray.Dataset.sel", args: ${domain}, _partial_: True}
    # Take Subset of Data
    - {_target_: "oceanbench.subset", num_samples: 1500, _partial_: True}
    # Regridding (AlongTrack -> Uniform Grid)
    - {
        _target_: "oceanbench.regrid", 
        target_grid: ${grid.high_res}, 
        _partial_: True
      }
    # Save Dataset
    - {
        _target_: "xarray.Dataset.to_netcdf", 
        save_name: "demo.nc", 
        _partial_: True
      }
\end{minted}
\label{hydraconfig:geoprocess}
\caption{This is a \texttt{.yaml} which showcases how we can communicate with \texttt{Hydra} framework to list a predefined set of transformations to be \textit{piped} through sequentiall. In this example, we showcase some standard pre-processing strategies to be saved to another netcdf file.}
\end{listing}




\newpage
\subsection{Evaluation Recipe - OSSE}

In this example, we showcase how one can use hydra to do the evaluation procedure. This is the same evaluation procedure that is used to evaluate the effectiveness of the OSSE NADIR experiment. From code snippet~\ref{hydraconfig:geoprocess}, we see that we choose which target function to initialize and we choose the data directory where the \texttt{.netcdf} file is located. Then, we pipe some transformations for the \texttt{.netcdf} file: 1) validate the spatiotemporal coordinates, 2) we select the evaluation region, 3) we regrid it to the target get, 4) we fill in the nans with a Gauss-Seidel procedure, 5) we rescale the coordinates to be in meters and days, and 6) we perform the isotropic power spectrum transformation to get the effective resolution outlined in section~\ref{sec:metrics}.

\begin{listing}[ht!]
\begin{minted}[frame=lines]{yaml}
# Target Function to initialize
_target_: "oceanbench._src.dataset.pipe"
# netcdf file to be loaded
inp: "${data_directory}/ml_result.nc"
# sequential transformations to be applied
fns:
    # Load Dataset
    - {_target_: "xarray.open_dataset", _partial_: True}
    # Validate LatLonTime Coordinates
    - {_target_: "oceanbench.validate_latlon", _partial_: True}
    - {_target_: "oceanbench.validate_time", _partial_: True}
    # Select Specific Region (Spatial | Temporal)
    - {_target_: "xarray.Dataset.sel", args: ${domain}, _partial_: True}
    # Regridding (Uniform Grid -> Uniform Grid)
    - {_target_: "oceanbench.regrid", 
       target_grid: ${grid.reference}, _partial_: True}
    # Fill NANS (around the corners)
    - {_target_: "oceanbench.fill_nans", 
       method: "gauss_seidel", _partial_: True}
    # Coordinate Change (degree -> meters, ns -> days)
    - {_target_: "oceanbench.latlon_deg2m", _partial_: True}
    - {_target_: "oceanbench.time_rescale", 
       freq: 1, unit: "days", _partial_: True}
    # Calculate Isotropic Power Spectrum
    - {_target_: "oceanbench.power_spectrum_isotropic", 
       reference: ${grid.reference}, _partial_: True}
    # Calculate Resolved Spatial Scale
    - {_target_: "oceanbench.resolved_scale", _partial_: True}
    # Save Dataset
    - {_target_: "xarray.Dataset.to_netcdf", 
       save_name: "ml_result_psd.nc", _partial_: True}
\end{minted}
\label{hydraconfig:evaluation}
\caption{This is a \texttt{.yaml} which showcases how we can communicate with \texttt{Hydra} framework to list a predefined set of transformations to be \textit{piped} through sequentiall. In this example, we showcase some standard pre-processing strategies to be saved to another netcdf file.}
\end{listing}





% \newpage
% \section{Use Case III: XRPatcher}
\label{sec:xrpatcher}

There are many usecases for the \texttt{XRPatcher}. For example, we can do 1D Time chunking, 2D Spatial-Temporal Patches, or 3D Spatial-Temporal Cubes.


\begin{listing}[!ht]
\small
\begin{minted}[frame=lines]{python}
import xarray as xr
import torch
import itertools
from oceanbench import XRPatcher
# Easy Integration with PyTorch Datasets (and DataLoaders)
class XRTorchDataset(torch.utils.data.Dataset):
    def __init__(self, batcher: XRPatcher, item_postpro=None):
        self.batcher = batcher
        self.postpro = item_postpro
    def __getitem__(self, idx: int) -> torch.Tensor:
        item = self.batcher[idx].load().values
        if self.postpro:
            item = self.postpro(item)
        return item
    def reconstruct_from_batches(
            self, batches: list(torch.Tensor), **rec_kws
        ) -> xr.Dataset:
        return self.batcher.reconstruct(
            [*itertools.chain(*batches)], **rec_kws
        )
    def __len__(self) -> int:
        return len(self.batcher)
# load demo dataset
data = xr.tutorial.load_dataset("eraint_uvz")
# Instantiate the patching logic for training
patches = dict(longitude=30, latitude=30)
train_patcher = XRPatcher(
    da=data,
    patches=patches,
    strides=patches,        # No Overlap
    check_full_scan=True    # check no extra dimensions
)
# Instantiate the patching logic for testing
patches = dict(longitude=30, latitude=30)
strides = dict(longitude=5, latitude=5)
test_patcher = XRPatcher(
    da=data,
    patches=patches,
    strides=strides,        # Overlap
    check_full_scan=True    # check no extra dimensions
)
# instantiate PyTorch DataSet
train_ds = XRTorchDataset(train_patcher, item_postpro=TrainingItem._make)
test_ds = XRTorchDataset(test_patcher, item_postpro=TrainingItem._make)
# instantiate PyTorch DataLoader
train_dl = torch.utils.data.DataLoader(train_ds, batch_size=4, shuffle=False)
test_dl = torch.utils.data.DataLoader(test_ds, batch_size=4, shuffle=False)
\end{minted}
\label{listing:xrpatcher}
\caption{\texttt{XRPatcher} integration in Pytorch. We define a PyTorch dataset that handles the \texttt{XRPatcher}. We load an arbitrary dataset with \texttt{xarray}, then we instantiate the \texttt{XRPatcher} with the patching logic, then we instantiate the PyTorch dataset and dataloaders.}
\end{listing}





% \newpage
% \section{Additional Tasks}\label{sec:other_tasks}

In the main paper, we thoroughly outlined the interpolation task to showcase how \texttt{OceanBench} can be used to create automated pipelines for processing and evaluation procedures.
However, there are many other additional tasks that can make use of the \texttt{OceanBench} features. 

\textbf{Denoising}. A simpler problem for interpolation tasks is the denoising problem~\cite{DENOISESURVEY,DENOISESURVEY2}.
The SSH and SST measurements we obtain have inherent noise from the sensors.
A key problem is to calibrate the observations by separating the known noise patterns and the true signal.
There has already been a lot of work from the ML side ranging from amortized predictions~\cite{DENOISESWOT} to end-to-end learning schemes~\cite{DENOISESWOT2}.
Much of this work has been facilitated by the \textit{Ocean-Data-Challenge} group which have a few data challenges related to the denoising problem.
Just like \texttt{OceanBench} was able to create reproducible pipelines from the SSH interpolation challenge listed in section~\ref{sec:data_challenges}, we also believe that one could extend the denoising challenge in the same manner.

\textbf{Forecasting}. This is a special form of extrapolation whereby the temporal domain of the state variable is sufficiently outside of the domain of the observation domain. 
Many previous benchmarking suites already look at forecasting for weather~\cite{weatherbench} and climate~\cite{ClimateBench}.
However, in oceanography, it is also advantageous to do forecasting for problems involving currents~\cite{MLSSC,4DVarNetSSC} and eddies~\cite{OCEANEddyTracking,OKUBO,OKUBOWEISS}.
The \texttt{xrpatcher} will work out of the box for forecasting problems and contributions can be made to \texttt{OceanBench} to include some specific metrics for forecasting as were outlined in~\cite{weatherbench,ClimateBench,ENS10Bench}.

\textbf{Proxy Variables}. There are many other control variables that one could use to improve the interpolation or extrapolation task.
We mentioned SST in section~\ref{sec:data_challenges} because it is the most abundant observations available.
However, there are other important observed variables which could be useful, e.g. Ocean colour, Bigeochemical parameters, and atmospheric variables.
In many other downstream applications, the oceanography community often uses SSH and SST as proxy variables to predict important quantities related to the carbon uptake, e.g. SOCAT~\cite{SOCAT}.
It would be straightforward to include a specific variable (and the associated preprocessing operations) into \texttt{OceanBench}.

\textbf{Dimension Reduction}. We often have very resolution spatiotemporal fields.
which poses a very big challenge for learning due to the high correlations exhibited by spatiotemporal data and high dimensionality.
A workaround for this is to learn a latent representation which retains as much relevant information as possible for the given task.
In the ocean sciences, this is known as \textit{Reduced Order Modeling} (ROM) or more generally dimensionality reduction which has been frequently used for adaptive meshes for physical models~\cite{NEMOEOF}.
This could be used for pretraining fields to latent embeddings which could be useful for downstream tasks like anomaly detection~\cite{SSTFLOWANOMALY}.
% In modern diffusion models, most of the operations are within the latent domain without any loss of quality in the generative results.


\textbf{Surrogate Modeling}. 
Physical model simulations are very expensive and ML has played a role in learning surrogate models to descrease the computational intensity~\cite{ML4OCN,MLCLOSURE}.
We have a decently long spatiotemporal field over a region of interest which could be used to learn a surrogate model to mimic the dynamics of that region.
This is also very useful for hybrid schemes whereby we have parameterizations to account for processes that are missing from low resolution simulations.~\cite{MLOCNPARAMETERIZATION,MLOCNPARAMETERIZATION2, MLOCNPARAMETERIZATION3, MLOCNPARAMETERIZATION4}.



% \subsection{Main Tasks}

% \subsubsection{Interpolation/Extrapolation}

% The two main tasks we can define from this problem setup are: 1) interpolation/extrapolation and 2) forecasting (which can be seen as extrapolation in Time). 
% Both interpolation and extrapolation are when the domain for the $\mathcal{T}$ always falls between the past and the present, i.e., $\mathcal{T}\in[0, T]$ where $T$ is the current Time. 
% We define interpolation as the case where the observations along the boundary of the domain are equal to the boundary of the desired reconstruction domain, i.e., $\partial\boldsymbol{\Omega}_{obs} = \partial\boldsymbol{\Omega}_g$, and extrapolation as the case where the boundary of the observation domain lies entirely inside of the boundary of the reconstructed domain, i.e., $\partial\boldsymbol{\Omega}_{obs} \subset \partial\boldsymbol{\Omega}_g$. 
% Forecasting refers to the problem when the temporal domain $\mathcal{T}$ lies outside of the present, i.e., $T+\tau$ where $\tau$ is some step within the future that is outside of the temporal observation domain, $\mathcal{T}\in[0, T]$. 
% Note that this is irrespective of the spatial domain or its boundaries. 
% In the rest of this paper, we will look exclusively at the interpolation problem, but we refer to the reader to section~\ref{sec:other_tasks} in the appendix for a more detailed look at the other tasks.

% \begin{equation}
%     \mathbf{y} = \mathcal{H}(\mathbf{x})
% \end{equation}


% where $\mathbf{y}\in\boldsymbol{\Omega_p}$ is the incomplete observation within some subdomain and $\mathbf{x}\in\boldsymbol{\Omega}$  is the true observation over the full domain.



% \subsubsection{Forecasting}

% \begin{equation}
%     \boldsymbol{u}_{t+\delta t} = \mathcal{M}(\boldsymbol{u}(\mathbf{x}, t+\delta t), \delta t; \boldsymbol{\theta})
% \end{equation}

% \subsection{SubTasks}

% \subsubsection{Surrogate/Hybrid/Parameterizations}

% \begin{equation}
%     \frac{\partial u}{\partial t} = \mathcal{M}(\boldsymbol{u}(\mathbf{x},t), \mathbf{x}, t; \theta)
% \end{equation}





% \newpage
% \section{Machine Learning Method Ontology} \label{sec:ml_ontology}

Although this paper does not focus on the explicit methods used for SSH interpolation, we would like to give a readers a brief overview of some of the most popular methods in the literature.

\subsection{Coordinate-Based methods}

These methods learn a direct mapping between the coordinate vectors to the scalar or vector values. 
%
\begin{align}
    \boldsymbol{y}_{obs} &= \boldsymbol{f}(\mathbf{x},t;\boldsymbol{\theta})+\boldsymbol{\epsilon}(\mathbf{x},t)
\end{align}
%
This is better known as \textit{functa}~\cite{FUNCTA} which parameterizes the field directly as a model.

\textbf{Functional}. Optimal Interpolation (OI) is the most common method used for many of the operational methods~\cite{DUACS}. It is a non-parametric, functional method which is built upon covariance and precision matrices. In the machine learning community, these methods are known as Gaussian Process~\cite{GPsBIGDATA} and in the geostatistics community, this is known as Kriging~\cite{KRIGINGREVIEW}.

\textbf{Basis Function}. This is an easy simplification to the functional by introducing parametric basis functions. In particular, the MIOST~\cite{MIOST} algorithm will be adopted in the new operational products for SSH interpolation. It is a custom basis function based on Wavelet analysis which is scale-aware and scalable.

\textbf{Neural Fields}. Neural fields (NerFs) are a very popular set of methods that use neural networks to effectively learn the basis function through a composition of weights, biases and activations~\cite{NERFSSSH}.
Furthermore, one can add physics-informed constraints to the loss function which mirror that of a PDE~\cite{PINNS}.
In many cases, especially with many auxillary inputs, we don't have access the PDE so one fit a NN directly to the observations with a fully connected neural network~\cite{SOCAT}.


\subsection{Grid-Based Methods}

In practice, we often consider the field at a specific discretized setting like a uniform grid or mesh. 
This is because we typically operate on and store these fields as multi-dimensional arrays which are only defined on a subspace of the entire continuous domain. 
We denote a discretized spatial representation as $\boldsymbol{\Omega}_g\subset\mathbb{R}^{N_s}$. 
We can simplify this notation by including the domain within the operator. So equation~\ref{eq:interp_problem} like so:
\begin{equation}\label{eq:interp_problem_discretized}
    \boldsymbol{\eta}(\boldsymbol{\Omega}_{obs},t ) = \mathcal{H}\left(\boldsymbol{\eta}(\boldsymbol{\Omega}_g,t), t, \boldsymbol{\mu},  \boldsymbol{\varepsilon} \right) 
    % + \boldsymbol{\varepsilon}(\boldsymbol{\Omega}_g, t)
\end{equation} 
%
In this equation, $\mathcal{H}$ is the observation operator that transforms the field from the full discretized domain, $\boldsymbol{\Omega}_g$, to the observation domain, $\boldsymbol{\Omega}_{obs}\subset\mathbb{R}^{N_{obs}}$.

\textbf{Direct Methods}. 
These methods take the noisy, incomplete observations and directly feed it to a model that returns the full reconstructed field.
They typically involve training a convolutional neural network or recurrent neural network on pairs of corrupted observations to learn the reconstruction~\cite{SuperResSurvey,IMAGE2IMAGETRANSLATION, IMAGE2IMAGETRANSLATION2}.
This has seem some sucess in applications related to SSH interpolation~\cite{SSHInterpUNet,SSHInterpConvLSTM, SSHInterpAttention}.

\textbf{Traditional Data Assimilation.}
There are many traditional methods that are rooted in data assimilation~\cite{DAGEOSCIENCE}.
For example, the GLORYS~\cite{GLORYS12} method propagates the physical model forwards in time and then \textit{updates} the state based on observations periodically.
A simpler approach is to use a nudging scheme coupled with a simpler physical model~\cite{BFNQG}.


\textbf{End-to-End Learning}. These methods try to solve the problem by learning and end-to-end scheme to solve the model inversion problem.
This is very similar to implicit methods that define a cost function to minimize instead of a minimizing the parameters of a prior model.
Plug-in-Play priors are a popular class of methods that pre-train priors on auxillary observations and then use the prior in the inversion scheme~\cite{DEEPUNFOLDING}.
This has seen a lot of success in SSH interpolation~\cite{4DVARNETSWOT,4DVARNETSST,4DVarNetSSC}.



% \newpage
% In practice, we only consider the field at a specific discretized setting like a uniform grid or mesh. 
% This is because we typically operate on and store these fields as multi-dimensional arrays which are only defined on a subspace of the entire continuous domain. We denote this discretized spatial representation as $\boldsymbol{\Omega}_g\subset\mathbb{R}^{N_s}$. We can simplify this notation by including the domain within the operator, like so:
% %
% \begin{equation} \label{eq:ssh_field_discretized}
%     \eta =\boldsymbol{\eta}(\boldsymbol{\Omega}_g,t),
% \end{equation}
% %
% This is more reflective of how we use operators in practice as we typically insert the field as a grid or multi-dimensional spatial array through a series of mathematical operations.
% We can further discretized this field through the time domain whereby we have a finite set of observations in time, $D_t$. Let's say that we have $N_t$ ordered samples in time between a defined interval of $[0,T]$, i.e. $\mathcal{T}=\left\{ t_t \in [0,T]\right\}_i^{N_{t}}$. In some settings, this could be a uniform observation that is hourly for the period of 1 day, i.e. $N_t=24$, or daily over the period of 10 years, i.e. $N_t=3650$. In more realistic settings, this could be an irregular pattern.
% Equation~\eqref{eq:ssh_field_discretized} assumes the full field is observed. In practice, we observe a corrupted, incomplete version of this SSH field s.t.
% %
% \begin{equation}\label{eq:obs_operator_discretized}
%     \boldsymbol{\eta}(\boldsymbol{\Omega}_{obs},t ) = \mathcal{H}\left(\boldsymbol{\eta}(\boldsymbol{\Omega}_g,t), t, \boldsymbol{\mu},  \boldsymbol{\varepsilon} \right) 
%     % + \boldsymbol{\varepsilon}(\boldsymbol{\Omega}_g, t)
% \end{equation} 

% In this equation, $\mathcal{H}$ is the observation operator that transforms the field from the full discretized domain, $\boldsymbol{\Omega}_g$, to the observation domain, $\boldsymbol{\Omega}_{obs}\subset\mathbb{R}^{N_{obs}}$. 
% The observation domain, $\boldsymbol{\Omega}_{obs}$, are the spatial coordinates, $\mathbf{x}$, where the SSH has been observed which is a proper subset of the full domain, i.e. $\boldsymbol{\Omega}_{obs}\subseteq\boldsymbol{\Omega}_g$.

% \begin{align} \label{eq:interp_problem}
%     \mathcal{M}_{\boldsymbol{\theta}} &: \boldsymbol{\eta}_{obs}(\mathbf{x}, t, \boldsymbol{\mu}) \rightarrow \boldsymbol{\eta}(\mathbf{x},t) \hspace{10mm}
%      \boldsymbol{\Omega}_{obs} \in \mathbb{R}^{N_{obs}} \hspace{5mm} 
%     \boldsymbol{\Omega}_g \in \mathbb{R}^{N_s} \hspace{5mm} 
%     t \in \mathcal{T}.
% \end{align}


% \subsubsection{Hybrid Schemes} \label{sec:bfn}

% This includes Backwards-Forwards Nudging with a simpler model like the Quasi-Geostrophic equations.

% \subsubsection{Direct Methods}

% These feature methods that directly take in the sparse SSH fields with \tocite{Recent Papers}

% \subsubsection{Data Assimilation}

% \begin{align}
%     \boldsymbol{\theta}^* &= \underset{\boldsymbol{\theta}}{\text{argmin}} \hspace{2mm} \mathcal{L}(\mathbf{x}_{gt}, \mathbf{x}(\boldsymbol{\theta})) \\
%     \mathbf{x}^*(\boldsymbol{\theta}) &= \underset{ \mathbf{x}^*}{\text{argmin}}  \hspace{2mm} \mathcal{U}\left(\mathbf{x}(\boldsymbol{\theta}) \right)
% \end{align}

% where $\mathcal{L}$ is some loss function to find the best parameters and $\mathcal{U}$ is some energy function to minimize the state.

% \subsubsection{End-to-End 4DVariational Methods} \label{sec:4dvarnet}

% These methods take a variational approach whereby we directly define the cost function we wish to solve \tocite{Ronan}. To speed up the convergence of these methods, a meta-learning based method was introduced to learn the gradient descent\tocite{Ronan}. In addition, this has been improved with the addition of sea surface temperature as an additional term in the cost function. 



% % \newpage
% % \section{Target Audience}
% \subsection{Project Vision}
While this tool is general in scope, we specifically target three audiences: 1) the domain expert who may want to use and understand and investigate SSH in relation to other important EO quantities, 2) machine learning researchers who may want to investigate how to make a better model for SSH interpolation, and 3) downstream users who are interested in adopting some techniques for their own domain-specific applications that may rely on SSH like tracking ocean currents~\tocite{}, investigating biogeochemical transport~\tocite{}, or global climate change~\tocite{}. In the following subsections, we give a more detailed description about the users and how might they benefit from \texttt{OceanBench}.

\textbf{Domain Experts}. We consider \textit{domain experts} who are experts in different domains of oceanography. ...


\textbf{ML Researchers}. We consider those who are expert ML researchers but may lack...



\textbf{Down Stream Users}. We also envision a broader adoption of the framework across research labs interested in having standardized data challenges for their own research purposes. There are independent groups choosing datasets for their specific use cases however, this framework can serve as an easy way to integrate their existing methodologies into the set of common tools to be used and improved by multiple communities. By establishing a relatively consistent problem set, we hope that any innovation can be easily understood and transferred across domains.
% \newpage
% \section{Limitations} \label{sec:appendix_limitations}

\subsection{Framework Limitations}

While we have advertised \texttt{OceanBench} as a unifying framework that provides standardized processing steps that comply with domain-expert standards, we also highlight some potential limitations that could hinder its adoption for the wider community.

\textbf{Data Serving}. We provide a few datasets but we omit some of the original simulations. We found that the original simulations are terabytes/petabytes of data which becomes infeasible for most modest users (even with adequate CPU resources).  
This is very big problem and if we want to have a bigger impact, we may need to do more close collaborations with specified platforms like the Marine Data Store~\citep{MDSOCEANPHYSICS,MDSBIOGEOCHEMICAL,MDSOCEANPHYSICSENS,MDSINSITU,MDSWAVES,MDSALONGTRACK,MDSSSH} or the Climate Data Store~\citep{CDSREANALYSISSST,CDSOBSSST,CDSOBSOC,CDSOBSSSTENS}. Furthermore, there are many people that will not be able to do a lot of heavy duty research which indirectly favours institutions with adequate resources and marginalizing others. 
This is also problematic as those communities tend to be the ones who need the most support from the products of such frameworks.
We hope that leaving this open-source at least ensure that the knowledge is public.

\textbf{Framework Dependence}. The user has to "buy-into" the \texttt{hydra} framework to really take advantage of \texttt{OceanBench}. This adds a layer of abstraction and a new tool to learn. 
However, we designed the project so that high level usage does not require in-depth knowledge of the framework. 
In addition, we hope that, despite the complexity of project, users will appreciate the flexibility and extensibility of this framework.


\textbf{Lack of Metrics}. We do not provide the most exhaustive list of metrics available with the ocean community. In fact, we also believe that many of these metrics are often poor and do not effectively assess the goodness of our reconstructions. 
However, we do provide a platform that will hopefully be useful and easy to implement new and improved metrics.
Furthermore, having a wide range of metrics that are trusted across communities may help to improve the overall assessment of the different model performances~\cite{METRICSAVERAGE}.

\textbf{Limited ML Scope}. 
The framework does not support nor promote any machine learning methods and we lack any indication of comparing ML training and inference performance. 
However, we argue that a benchmark framework will allow us to effectively compare whichever ML methods are demonstratively the best which is a necessary preliminary step which offers users more flexibility in the long-run.

\textbf{Broad Oceans Application Scope}. 
We have targeted a broad ocean-application scope of state estimation.
However, there may be more urgent applications such as maritime monitoring, object tracking, and general ocean health.
However, we feel that many downstream applications require high-quality maps.
In addition, those downstream applications tend to be very complicated and are not always straightforward to apply ML under those instances.

\textbf{Full Pipeline Transparency}. We use a lot of different \texttt{xarray}-specific packages which have different design principles, assumptions and implementations. This may give the users an illusion of simplicity and transparency to real-world use. However, there are many underlying assumptions within each of the packages that may occlude a lot of design decisions.
Despite this limitation, we believe that being transparent about the processing steps and being consistent with the evaluation procedure will be beneficial for the ML research community.

\textbf{Scalability}. Scaling this to many terabytes or petabytes of data is easily the biggest limitation of the framework. In addition, we have only showcased demonstrations for 2D+T fields which are much less expensive than 3D+T fields.

\textbf{Deployability}. MLOPs has many wheels and it is not easy to integrate into existing systems. We offer no solutions to this. 
However, we believe that our framework is fully transparent in the assumptions and use cases which will facilitate some adoption into operational systems where they can further modify it for their use cases (see the evolution of \texttt{WeatherBench} and \texttt{ClimateBench}).

\textbf{Visualization Tools}.
We do not incorporate a high quality visualization tool that allows users to do pre- and post-analysis at a large scale. 
We do provide some simple visualization steps that are ML-relevant (see the GitHub repo) but it is very limited to ML standards.
One solution is to interface our pipeline with the source of many ocean datasets, e.g. Climate Data Store~\citep{CDSREANALYSISSST} or Marine Data Store~\citep{MDSOCEANPHYSICS}, then we can offset this task to them where they can offer better quality visualization tools.

\newpage
\subsection{Data Challenge Limitations}

We have showcased the SSH interpolation edition as a data challenge which could be helpful for real applications. 
However, in section~\ref{sec:problem_scope} we alluded to the greater task of general ocean state estimation which is more pertinent to the ocean sciences yet we don't address this directly with our data challenges.
Furthermore, we claim that the data challenges presented will help the ocean community with using ML for SSH interpolation.
Below, we outline some limitations which address these criticisms.

\textbf{Not the overall objective}. 
We recognized that we are far away from the actual reanalysis and forecasting goals of full state estimation. 
However, we argue that that is a rather ambitious challenge which will require a lot of interdisciplinary work across communities. 
In the meantime while we work towards that goal, operational centers could possibly improve their current products from ML-based techniques would would benefit downstream applications that deal directly with SSH.
Furthermore, SSH is an important variable in describing the full ocean state.
So a robust set of techniques that are able to solve the interpolation tasks could (in principal) be used to solve extra tasks.

\textbf{Small Region \& Period}.
We only feature a small region and period over the Gulfstream which is not representative of the different global regimes. 
This also does not take into account real things like \textit{data drift} which will inevitable occur in operational settings.
However, this is a dynamical regime and a well-studied area which does have some importance for specific communities and the results obtained offer some transferability to other dynamical regimes.
In addition, this area will have good coverage due to the new SWOT mission~\cite{SWOT} which will allow for further validation in the future.
Lastly, the area is small enough where the beginning stages for ML researchers is not overwhelmed with problems involving scale (even though we eventually want to arrive at global schemes).
We hope to extend our challenges to more relevant scenarios~\cite{MDSALONGTRACK}.

\textbf{Simulations versus Reanalysis}. We use simulations for the OSSE experiments instead of reanalysis. This is an open research question as it is unclear whether it's better to pretrain models on simulated ocean data or reanalysis ocean data. In future updates, we plan to add the reanalysis data to extend the challenge.

\textbf{Efficacy of OSSE Experiments}. We alluded to the idea that the OSSE experiments may not reflect the overarching goal of the user yet we provide more OSSE experiments than OSE experiments.
We acknowledged that it often does not coincide exactly with the OSE experiments which may give users a false sense of accomplishment and immediate transferability. 
However, we try to provide a framework where one could thoroughly experiment with the learning problem on OSSE configurations which can facilitate transfer learning to other domain-specific tasks.
We also anticipate that new \textit{real} SWOT data~\cite{SWOT} will start to become more available which will allow us to design better, realistic OSE experiments.

\textbf{Noise Characterization}.
Real data has noise to content with and we do not account for that within the SSH interpolation experiments.
The true noise we see in operational settings is structured and this would require more knowledge outside the scope of our teams expertise.
A more improved challenge would take these considerations into account.
We leave this as a future challenge for the community and we hope our platform can help facilitate this.

\textbf{Uncertainty Quantitification}.
We prefaced the problem statement with the idea of data assimilation which is the notion of \textit{state/parameter estimation under uncertain conditions and incomplete information}~\citep{DAGEOSCIENCE}.
However, we have not addressed any notion of uncertainty at all throughout the paper.
Uncertainty is difficult to quantify and we don't want to impose too many restrictions until we more sure about the efficacy of ML for easier problems.
However, to move the problem setting towards a more realistic setting, we can start to introduce metrics and additional requirements from future challenges, e.g. mean and standard deviation estimates or ensemble predictions.


\textbf{Operational Constraints}.
The real use case of SSH interpolation will involve global data and/or high-resolution data. 
This involves dealing with very high-dimensional spatiotemporal global state-space.
In practice, the necessity for the scalability of the method is of paramount importance.
However, there are also areas within the ML research community who are looking into many ways we can scale up physical models~\citep{VEROS,OCEANANIGANS} and machine learn models for geoscience tasks~\citep{SFNO}.
We anticipate that once a set of solutions are excepted by a community, the scalability will come later.




% \newpage
% \subsection{Dataset Limitations}



% \textbf{Limitations}. The scope of this is very specific to researchers who are interested in ML. It only serves as a baseline design. If one would like to scale, it would require much more engineering work to get everything to connect. This is already seen if we are to load datasets of PBs. In addition, this requires researchers to have access to considerably large machines to be able to run their own preprocessing schemes. We do our best to provide toy datasets of a modest size however, inevitably, one will probably need to work with larger and large datasets.

% \subsection*{Broader Impact}
% \label{sec:impact}

% The theme of interpolation is present in many applied communities with different names, e.g. kriging in ecology/hydrology, Optimal interpolation in oceanography, and Gaussian processes in statistics. We hope that this work bridges this gap between the communities and we invite other works to try to highlight concrete ways that machine learning and classic physics have commonalities.

% In the oceanography community in particular, we especially hope to see more adoption of machine learning methods for interpolation. DUACS is ultimately a closed-system so the wider scientific community does not have access to the algorithm. Our OI baseline hopefully unveils some of the finer details of the method. However, in general, the standard OI methods used in the applied community cannot keep up with the massive influxes of observations we receive. So this work is a first step in demonstrating that neural networks (in particular NerFs) are a viable, simpler, and scalable alternative.

\putbib[./00_Oceanbench/content/bibliographies/full.bib]
\end{bibunit}

% \bibliography{
% content/biblio, 
% content/bibliographies/software, 
% content/bibliographies/machine_learning, 
% content/bibliographies/sea_surface_height,
% content/bibliographies/ocean,
% content/bibliographies/data,
% content/bibliographies/applications
% }

% \section*{References}

% References follow the acknowledgments. Use unnumbered first-level heading for
% the references. Any choice of citation style is acceptable as long as you are
% consistent. It is permissible to reduce the font size to \verb+small+ (9 point)
% when listing the references.
% Note that the Reference section does not count towards the page limit.
% \medskip

% {
% \small

% [1] Alexander, J.A.\ \& Mozer, M.C.\ (1995) Template-based algorithms for
% connectionist rule extraction. In G.\ Tesauro, D.S.\ Touretzky and T.K.\ Leen
% (eds.), {\it Advances in Neural Information Processing Systems 7},
% pp.\ 609--616. Cambridge, MA: MIT Press.

% [2] Bower, J.M.\ \& Beeman, D.\ (1995) {\it The Book of GENESIS: Exploring
%   Realistic Neural Models with the GEneral NEural SImulation System.}  New York:
% TELOS/Springer--Verlag.

% [3] Hasselmo, M.E., Schnell, E.\ \& Barkai, E.\ (1995) Dynamics of learning and
% recall at excitatory recurrent synapses and cholinergic modulation in rat
% hippocampal region CA3. {\it Journal of Neuroscience} {\bf 15}(7):5249-5262.
% }

%%%%%%%%%%%%%%%%%%%%%%%%%%%%%%%%%%%%%%%%%%%%%%%%%%%%%%%%%%%%
\newpage
% \appendix
% \section*{\textsc{OceanBench}: The Sea Surface Height Edition - Supplementary Material}
% % \newpage
% \section{Data Challenges} \label{sec:data_challenges_extended}

In this section, we highlight some details that were omitted in section~\ref{sec:data_challenges}.
This includes details about the simulation type, the data structures, and the training/evaluation periods.

\subsection{OSSE NADIR}

The reference simulation is the \textit{NATL60} simulation based on the NEMO model~\cite{NEMOAJAYI2020}. 
This particular simulation was run over an entire year without any tidal forcing.
The simulation provides the outputs of SSH, SST, sea surface salinity (SSS) and the u,v velocities every 1 hour.
For the purposes of this data challenge, the spatial domain is over the Gulfstream with a spatial domain of $[-65^\circ, -55^\circ]$ longitude and $[33^\circ, 43^\circ]$ latitude.
The resolution of the original simulation is 1/60$^\circ$ resolution with hourly snapshots, and we consider a daily downsampled trajectory at 1/20$^\circ$ for the data challenge which results in a 365x200x200 spatio-temporal grid.
This simulation resolves finescale dynamical processes ($\sim$15km) which makes it a good test bed for creating an OSSE environment for mapping.
The SSH observations include simulations of ocean satellite NADIR tracks.
In particular, they are simulations of Topex-Poseidon, Jason 1, Geosat Follow-On, and Envisat.
There is no observation error considered within the challenge.
We use a the entire period from 2012-10-10 until 2013-09-30.
A training period is only from 2013-01-02 to 2013-09-30 where the users can use the reference simulation as well as all available simulated observations.
The evaluation period is from 2012-10-22 to 2012-12-02 (i.e. 41 days) which is considered decorrelated from the training period. 
During the evaluation period, the user cannot use the reference NATL60 simulation but they can use all available simulated observations. There is also a spin-up period allowance from 2012-10-01 where the user can also use all available simulated observations.

\subsection{OSSE SWOT \& OSSE SST}

For the OSSE SWOT and OSSE SST experiments, the reference simulation, domain, and evaluation period is the same as the OSSE NADIR experiment.
However, the OSSE SWOT includes simulated observations of the novel KaRIN sensor recently deployed during the SWOT mission, the pseudo-observations were generated using the SWOT simulator~\cite{SWOT}. 
This OSSE SST experiment allows the users to utilize the full fields of SST as inputs to help reconstruct the SSH field in conjunction with the NADIR and SWOT SSH observation.
Because the SST comes from the same NATL60 simulation, the geometry characteristics SST and SSH are exactly the same.

\subsection{OSE NADIR}

The OSE NADIR experiment only uses real observations aggregated from different altimeters. These SSH observations include observations from the SARAL/Altika, Jason 2, Jason 3, Sentinel 3A, Haiyang-2A and Cryosat-2 altimeters. The Cryosat-2 altimeter is used as the independent evaluation track used to assess the performance of the reconstructed SSH field.

\subsection{Results}

We use \texttt{OceanBench} to generate maps of relevant quantities from the 4DVarNet method~\cite{4DVARNETSWOT,4DVARNETSST}.
Figure~\ref{fig:oceanbench_maps_4dvarnet} showcases some demo maps for some key physical variables outlined in section~\ref{sec:physical_variables}.
We showcase the 4DVarNet method because it is the SOTA method that was applied to each of the data challenges.
We can see that the addition of more information, i.e. NADIR -> SWOT -> SST, results in maps look more similar to the NEMO simulation in the OSSE challenges.
It also produces sensible maps for the OSE challenge as well.

\texttt{OceanBench} also generated figure~\ref{fig:oceanbench_psd_4dvarnet} which shows plots of the PSD and PSD scores of SSH for the different challenges.
Again, as we increase the efficacy of the observations via SWOT and allow for more external factors like the SST, we get an improvement in the isotropic and spacetime PSD scores.
In addition, we see that the PSD plots for the OSE task look very similar to the OSE challenges. 

Lastly, we used \texttt{OceanBench} to generate a leaderboard of metrics for a diverse set of algorithms where the maps were available online.
Table~\ref{tb:exp-results-mega} displays all of the key metrics outlined in section~\ref{sec:metrics} including the normalized RMSE and various spectral scores which are appropriate for the challenge.
We see that as the complexity of the method increases, the metrics improve. 
In addition, the methods that involve end-to-end learning perform the best overall, i.e. 4DVarNet.

\begin{figure}[ht!]
\small
\begin{center}
\setlength{\tabcolsep}{1pt}
\begin{tabular}{cccc}
\hspace{3mm} Task OSSE & 
\hspace{3mm} Task OSSE & 
\hspace{2mm} Task OSSE & 
Task OSE \\
\hspace{3mm}  Nadir & 
\hspace{3mm} Nadir + SWOT & 
\hspace{2mm} Nadir + SST & 
Nadir \\
%\vspace{-2mm}
%%%%% SSH %%%%%%%%
\includegraphics[trim={0 13mm 22mm 0},clip, width=3.60cm,height=3.2cm]{content/figures/fourdvarnet_figs/osse_gf_nadir_ssh.png} &
\includegraphics[trim={13mm 13mm 22mm 0},clip, width=3.2cm,height=3.2cm]{content/figures/fourdvarnet_figs/osse_gf_nadirswot_ssh.png} &
\includegraphics[trim={13mm 13mm 22mm 0},clip, width=3.2cm,height=3.2cm]{content/figures/fourdvarnet_figs/osse_gf_nadir_sst_ssh.png} &
\includegraphics[trim={13mm 13mm 0 0},clip,width=4.0cm,height=3.2cm]{content/figures/fourdvarnet_figs/ose_gf_ssh.png} \\
%\vspace{3mm}
%%%%% KINETIC ENERGY %%%%%%%%
\includegraphics[trim={0 13mm 22mm 5mm}, clip, width=3.60cm,height=3cm]{content/figures/fourdvarnet_figs/osse_gf_nadir_ke.png} &
\includegraphics[trim={13mm 13mm 22mm 5mm},clip, width=3.2cm,height=3cm]{content/figures/fourdvarnet_figs/osse_gf_nadirswot_ke.png} &
\includegraphics[trim={13mm 13mm 22mm 5mm},clip, width=3.2cm,height=3cm]{content/figures/fourdvarnet_figs/osse_gf_nadir_sst_ke.png} &
\includegraphics[trim={13mm 13mm 0 5mm},clip,width=4cm,height=3cm]{content/figures/fourdvarnet_figs/ose_gf_ke.png} \\
%%%%% RELATIVE VORTICITY %%%%%%%%
\includegraphics[trim={0 13mm 21.2mm 5mm},clip, width=3.60cm,height=3cm]{content/figures/fourdvarnet_figs/osse_gf_nadir_vort_r.png} &
\includegraphics[trim={13mm 13mm 21.2mm 5mm},clip, width=3.2cm,height=3cm]{content/figures/fourdvarnet_figs/osse_gf_nadirswot_vort_r.png} &
\includegraphics[trim={13mm 13mm 21.2mm 5mm},clip, width=3.2cm,height=3cm]{content/figures/fourdvarnet_figs/osse_gf_nadir_sst_vort_r.png} &
\includegraphics[trim={13mm 13mm 0 5mm},clip,width=4.0cm,height=3cm]{content/figures/fourdvarnet_figs/ose_gf_vort_r.png} \\
%%%%% STRAIN %%%%%%%%
\includegraphics[trim={0 0 19mm 5mm},clip, width=3.60cm,height=3.4cm]{content/figures/fourdvarnet_figs/osse_gf_nadir_strain.png} &
\includegraphics[trim={13mm 0 19mm 5mm},clip, width=3.2cm,height=3.4cm]{content/figures/fourdvarnet_figs/osse_gf_nadirswot_strain.png} &
\includegraphics[trim={13mm 0 19mm 5mm},clip, width=3.2cm,height=3.4cm]{content/figures/fourdvarnet_figs/osse_gf_nadir_sst_strain.png} &
\includegraphics[trim={13mm 0 0 5mm},clip,width=4.0cm,height=3.4cm]{content/figures/fourdvarnet_figs/ose_gf_strain.png} \\
% \vspace{-2mm}
(a) & (b) & (c) & (d)
\end{tabular}
\vspace{-3mm}
% \caption{Row I - Isotrophic PSD. Row 2 - Isotrophic PSD Score}
\caption{
Reconstructed quantities by the 4dVarNet method for each of the four tasks.
Each row showcases the following physical variables found in appendix~\ref{sec:physical_variables}: (a) Sea Surface Height, (b) Kinetic Energy, (c) Relative Vorticity, and (d) Strain. 
Each column showcase the reconstructed from the tasks (a) OSSE using only Nadir tracks: (b) OSSE using Nadir tracks and SWOT swath, (c) Multimodal using Nadir tracks and sea surface temperature, and (d) Reconstruction using real nadir altimetry tracks.}
\vspace{-5mm}
\label{fig:oceanbench_maps_4dvarnet}
\end{center}
\end{figure}





% \begin{figure}[ht!]
\small
\begin{center}
\setlength{\tabcolsep}{1pt}
\begin{tabular}{cccc}
\hspace{3mm} Task OSSE & 
\hspace{3mm} Task OSSE & 
\hspace{2mm} Task OSSE & 
Task OSE \\
\hspace{3mm}  Nadir & 
\hspace{3mm} Nadir + SWOT & 
\hspace{2mm} Nadir + SST & 
Nadir \\
%\vspace{-2mm}
%%%%% SSH %%%%%%%%
\includegraphics[trim={0 13mm 22mm 0},clip, width=3.60cm,height=3.2cm]{00_Oceanbench/content/figures/fourdvarnet_figs/osse_gf_nadir_ssh.png} &
\includegraphics[trim={13mm 13mm 22mm 0},clip, width=3.2cm,height=3.2cm]{00_Oceanbench/content/figures/fourdvarnet_figs/osse_gf_nadirswot_ssh.png} &
\includegraphics[trim={13mm 13mm 22mm 0},clip, width=3.2cm,height=3.2cm]{00_Oceanbench/content/figures/fourdvarnet_figs/osse_gf_nadir_sst_ssh.png} &
\includegraphics[trim={13mm 13mm 0 0},clip,width=4.0cm,height=3.2cm]{00_Oceanbench/content/figures/fourdvarnet_figs/ose_gf_ssh.png} \\
%\vspace{3mm}
%%%%% KINETIC ENERGY %%%%%%%%
\includegraphics[trim={0 13mm 22mm 5mm}, clip, width=3.60cm,height=3cm]{00_Oceanbench/content/figures/fourdvarnet_figs/osse_gf_nadir_ke.png} &
\includegraphics[trim={13mm 13mm 22mm 5mm},clip, width=3.2cm,height=3cm]{00_Oceanbench/content/figures/fourdvarnet_figs/osse_gf_nadirswot_ke.png} &
\includegraphics[trim={13mm 13mm 22mm 5mm},clip, width=3.2cm,height=3cm]{00_Oceanbench/content/figures/fourdvarnet_figs/osse_gf_nadir_sst_ke.png} &
\includegraphics[trim={13mm 13mm 0 5mm},clip,width=4cm,height=3cm]{00_Oceanbench/content/figures/fourdvarnet_figs/ose_gf_ke.png} \\
%%%%% RELATIVE VORTICITY %%%%%%%%
\includegraphics[trim={0 13mm 21.2mm 5mm},clip, width=3.60cm,height=3cm]{00_Oceanbench/content/figures/fourdvarnet_figs/osse_gf_nadir_vort_r.png} &
\includegraphics[trim={13mm 13mm 21.2mm 5mm},clip, width=3.2cm,height=3cm]{00_Oceanbench/content/figures/fourdvarnet_figs/osse_gf_nadirswot_vort_r.png} &
\includegraphics[trim={13mm 13mm 21.2mm 5mm},clip, width=3.2cm,height=3cm]{00_Oceanbench/content/figures/fourdvarnet_figs/osse_gf_nadir_sst_vort_r.png} &
\includegraphics[trim={13mm 13mm 0 5mm},clip,width=4.0cm,height=3cm]{00_Oceanbench/content/figures/fourdvarnet_figs/ose_gf_vort_r.png} \\
%%%%% STRAIN %%%%%%%%
\includegraphics[trim={0 0 19mm 5mm},clip, width=3.60cm,height=3.4cm]{00_Oceanbench/content/figures/fourdvarnet_figs/osse_gf_nadir_strain.png} &
\includegraphics[trim={13mm 0 19mm 5mm},clip, width=3.2cm,height=3.4cm]{00_Oceanbench/content/figures/fourdvarnet_figs/osse_gf_nadirswot_strain.png} &
\includegraphics[trim={13mm 0 19mm 5mm},clip, width=3.2cm,height=3.4cm]{00_Oceanbench/content/figures/fourdvarnet_figs/osse_gf_nadir_sst_strain.png} &
\includegraphics[trim={13mm 0 0 5mm},clip,width=4.0cm,height=3.4cm]{00_Oceanbench/content/figures/fourdvarnet_figs/ose_gf_strain.png} \\
% \vspace{-2mm}
(a) & (b) & (c) & (d)
\end{tabular}
\vspace{-3mm}
% \caption{Row I - Isotrophic PSD. Row 2 - Isotrophic PSD Score}
\caption{
Reconstructed quantities by the 4dVarNet method for each of the four tasks.
Each row showcases the following physical variables found in appendix~\ref{sec:physical_variables}: (a) Sea Surface Height, (b) Kinetic Energy, (c) Relative Vorticity, and (d) Strain. 
Each column showcase the reconstructed from the tasks (a) OSSE using only Nadir tracks: (b) OSSE using Nadir tracks and SWOT swath, (c) Multimodal using Nadir tracks and sea surface temperature, and (d) Reconstruction using real nadir altimetry tracks.}
\vspace{-5mm}
\label{fig:oceanbench_maps_4dvarnet}
\end{center}
\end{figure}



\begin{table}[ht]
\caption{This table showcases all of the summary statistics for some methods for each of the data challenges listed in section~\ref{sec:data_challenges} and~\ref{sec:data_challenges_extended}. The summary statistics shown are the normalized RMSE and the effective resolution in the spectral domain. The spectral metrics for the effective resolution that were outlined in section~\ref{sec:metrics} are: i) $\lambda_a$ is the spatial score for the alongtrack PSD score, ii) $\lambda_r$ is the spatial score for the isotropic PSD, iii) $\lambda_x$ is the spatial score for space-time PSD score, and iv) $\lambda_t$ is the temporal score for the space-time PSD score.}
% \caption{This table highlights some of the results for the OSSE experiments outlined in section~\ref{sec:osse} and~\ref{sec:other_tasks}.

% This table highlights the performance statistically in the real and spectral space; the normalized RMSE for the real space and the minimum spatial and temporal scales resolved in the spectral domain. 
% For more information about the class of models displayed and class of metrics, see section~\ref{sec:ml_ontology} and section~\ref{sec:metrics} respectively.}
\label{tb:exp-results-mega}
\centering
\begin{tabular}{llcccccc}
 \toprule
% Experiment & Configuration & Method & nRMSE & Resolved Scale [km]    \\ \midrule
% \multirow{2}{*}{Experiment} & \multirow{2}{*}{Algorithm} & \multirow{2}{*}{Algorithm Class} & \multirow{2}{*}{nRMSE} & \multicolumn{2}{c}{Effective Resolution} \\ 
% &  &   &  & Wavelength [km]  & Period [days]      \\ \midrule
% \multirow{2}{*}{Experiment} & \multirow{2}{*}{Algorithm} & \multirow{2}{*}{Algorithm Class} & \multirow{2}{*}{nRMSE} & \multicolumn{2}{c}{Effective Resolution} \\ 
\multirow{2}{*}{Experiment} &  \multirow{2}{*}{Algorithm} &   \multirow{2}{*}{nRMSE} &
\multicolumn{4}{c}{Effective Resolution} \\
& & & $\lambda_{a}$ [km] & $\lambda_{r}$ [km]   &  $\lambda_{\mathbf{x}}$ [km]  &   $\lambda_{t}$ [days]      \\ \midrule
OSSE NADIR     &  OI & 0.92 & - & 123 & 174 & 10.8 \\
OSSE NADIR     &  MIOST &  0.93 & - & 100 & 157 & 10.1 \\
OSSE NADIR     &  BFNQG & 0.93 & - & 88 & 139 & 10.4 \\
OSSE NADIR &  4DVarNet &  \textbf{0.94} & - & \textbf{65} & \textbf{117} & \textbf{7.7} \\
\midrule
OSSE SWOT     &  OI & 0.92 & - & 106 & 139 & 11.7 \\
OSSE SWOT     &  MIOST &  0.94 & - & 88 & 131 & 10.1 \\
OSSE SWOT     &  BFNQG & 0.94 & - & 64 & 118 & 36.5 \\
OSSE SWOT &  4DVarNet &  \textbf{0.96} & - & \textbf{47} & \textbf{77} & \textbf{5.6} \\
\midrule
OSSE SST     &  Musti & 0.95 & - & 46 & 138 & 4.1 \\
OSSE SST &  4DVarNet &  \textbf{0.96} & - & \textbf{46} & \textbf{87} & \textbf{3.7} \\
\midrule
OSE NADIR     &  OI & 0.88 & 151 & - &  - &  -\\
OSE NADIR     &  MIOST &  0.90 & 135 & - &  - &  -\\
OSE NADIR     &  BFNQG & 0.88 & 122 & - & - &  -\\
OSE NADIR &  ConvLSTM &  0.89 & 113 &- &  - &  -\\
OSE NADIR &  4DVarNet & \textbf{0.91} & \textbf{98} & - &  -  &  -\\
\bottomrule
\end{tabular}
\end{table}

% \subsection{Simulated Altimetry Tracks} \label{sec:dc_osse_nadir}

% \textcolor{red}{
% The most commonly used SSH maps, the Developing Use of Altimetry for Climate Studies (DUACS) products, are derived from a statistical space–time interpolation of nadir altimeter observations. This intrinsically limits the effective resolution [as defined in Skamarock (2004), i.e., the fully resolved scales] of DUACS SSH maps to 150–200 km at middle latitudes (Ballarotta et al. 2019). The SSH mapping algorithm was developed by CNES and CLS in 1997, as part of the DUACS project, and has been continuously improved since then (Taburet et al. 2019). The DUACS products are now distributed by the Copernicus Marine Environment Monitoring Service (CMEMS). DUACS algorithm implements a statistical interpolation of SSH satellite data in space and time to produce global daily maps (Le Traon et al. 1998). The data are collected by a constellation of 2 to 4 nadir-looking altimeters (sometimes referred to as conventional altimeters), and characterized by large data gaps reaching 200 km in the zonal direction at the equator.
% }

% \subsection{Simulated SWOT Tracks} \label{sec:dc_osse_swot}


% \textcolor{red}{
% The Surface Water and Ocean Topography (SWOT; Fu et al. 2012; Morrow et al. 2019) altimetry mission, to be launched in early 2022, will open the way to SSH maps with resolution significantly higher than 150 km at midlatitudes, but this perspective entails a thorough revisit of the mapping algorithm. SWOT will considerably increase the measurement density at the surface of the oceans thanks to SSH measurements at a kilometric pixel resolution over a swath 120 km wide. On the swath, SWOT is expected to resolve scales down to 15 km at low latitude and 30–45 km at mid- and high latitudes (Wang et al. 2019). In its science phase, SWOT will have a 21 days repeat orbit, allowing an average revisit time of 11 days in most of the globe. Some of the dynamical processes observable by SWOT evolve over time scales on the order of 1 day, much shorter than the satellite revisit time. Consequently, the mapping method implemented in the current DUACS system will certainly not be sufficient to draw the maximum benefit from SWOT. A linear interpolation will filter most of the observed small-scale signals between two passes of the satellite, as anticipated by Gaultier et al. (2016).These authors advocate for using more advanced methods to build SSH maps.
% }

% \subsection{Multimodal with Sea Surface Temperature}  \label{sec:dc_osse_sst}


% \subsection{Real Altimetry Tracks}  \label{sec:dc_ose_nadir}
% \newpage
% \section{Physical Variables} \label{sec:physical_variables}

As alluded to in the main body of the paper, we have access to many physical quantities which can be derived from  sea surface height. 
This gives us a way to analyze how effective and trustworthy are our reconstructions. 
Many machine learning methods are unconstrained so they may provide solutions that are physically inconsistent and visualizing the field is a very easy eye test to assess the validity. 
In addition to post analysis, one could include some of these derived quantities maybe useful as additional inputs to the system and/or constraints to the loss function. 
Recall the spatiotemporal coordinates from equation~\ref{eq:spatiotemporal_coords}, 
we use the same coordinates for the subsequent physical quantities. \textbf{Sea Surface Height} is the deviation of the height of the ocean surface from the geoid of the Earth. We can define it as:
\begin{align}
	\text{Sea Surface Height }[m]:&& \quad
 \eta &= \boldsymbol{\eta}(\mathbf{x},t)&& \quad \Omega\times \mathcal{T}\rightarrow\mathbb{R} \label{eq:ssh}
\end{align}
This quantity is the actual value that is given from the satellite altimeters and is presented in the products for SSH maps~\cite{DUACS}. An example can be seen in the first row of figure~\ref{fig:oceanbench_maps_4dvarnet}.

\textbf{Sea Surface Anomaly} is the anomaly wrt to the spatial mean which is defined by
\begin{align}
	\text{Sea Level Anomaly }[m]:&& \quad
 \bar{\eta} &= \boldsymbol{\eta}(\mathbf{x},t) - \bar{\eta}(t) &&
 \quad \Omega\times \mathcal{T}\rightarrow\mathbb{R} \label{eq:sla}
\end{align}
where $\bar{\eta}(t)$ is the spatial average of the field at each time step.  
An example can be seen in the first row of figure~\ref{fig:oceanbench_maps}.

Another important quantity is the \textbf{geostrophic velocities} in the zonal and meridional directions. This is given by
\begin{align}
	\text{Zonal Velocity}[ms^{-2}]:&& \quad
 u &= -\frac{g}{f_0}\frac{\partial \eta}{\partial y} &&
 \quad \Omega\times \mathcal{T}\rightarrow\mathbb{R} \label{eq:u_vel} \\
	\text{Meridional Velocity}[ms^{-2}]:&& \quad
 v &= \frac{g}{f_0}\frac{\partial \eta}{\partial x} &&
 \quad \Omega\times \mathcal{T}\rightarrow\mathbb{R} \label{eq:v_vel}
\end{align}
where $g$ is the gravitational constant and $f_0$ is the mean Coriolis parameter. These quantities are important as they can be an related to the sea surface current. The geostrophic assumption is a very strong assumption however it can still be an important indicator variable. The \textbf{kinetic energy} is a way to summarize the (geostrophic) velocities as the total energy of the system. This is given by
\begin{equation} \label{eq:kineticenergy}
    KE = \frac{1}{2}\left(u^2 + v^2\right)
\end{equation}
An example can be seen in the second row of figure~\ref{fig:oceanbench_maps_4dvarnet}.

Another very important quantity is the \textit{vorticity} which measures the spin and rotation of a fluid. In geophysical fluid dynamics, we use the \textbf{relative vorticity} which is the vorticity observed within at rotating frame.
This is given by
\begin{equation} \label{eq:relvorticity}
    \zeta = \frac{\partial v}{\partial x} - \frac{\partial u}{\partial y}
\end{equation}
An example can be seen in the third row of figure~\ref{fig:oceanbench_maps_4dvarnet}.

% \subsection{Absolute Vorticity}

% \begin{equation} \label{eq:absvorticity}
%     |\zeta| = \frac{\partial v}{\partial x} + \frac{\partial u}{\partial y}
% \end{equation}

We can also use the \textbf{Enstrophy} to summarize the relative voriticty to measure the total contribution which is given by
\begin{equation} \label{eq:enstrophy}
    E = \frac{1}{2}\zeta^2
\end{equation}

The \textbf{Strain} is a measure of deformation of a fluid flow.

\begin{equation} \label{eq:strain}
    \sigma = \sqrt{\sigma_n^2 + \sigma_s^2}
\end{equation}

where $\sigma_n$ is the shear strain (aka the shearing deformation) and $\sigma_s$ is the normal strain (aka stretching deformation). An example can be seen in the fourth row of figure~\ref{fig:oceanbench_maps_4dvarnet}.

The \textbf{Okubo-Weiss Parameter} is high-order quantity which is a linear combination of the strain and the relative vorticity.

\begin{equation} \label{eq:okuboweiss}
    \sigma_{ow} = \sigma_n^2 + \sigma_s^2 - \zeta^2
\end{equation}

This quantity is often used as a threshold for determining the location of Eddies in sea surface height and sea surface current fields~\cite{OKUBO, WEISS, OKUBOWEISS}.

% \begin{table}[h!]
%   \caption{Table of nomanclature}
%   \label{sample-table}
%   \centering
%   \begin{tabular}{ccl}
%     \toprule
%     Symbol & Size & Description  \\
%     \midrule
%     $\mathbf{x}_s$ & $\mathbb{R}^{D_s}$ & Spatial Coordinates  \\
%     $t$ & $\mathbb{R}^{D_t}$ & Temporal Coordinates  \\
%    $\boldsymbol{f}(\mathbf{x}_s, t)$ & $\mathbb{R}^{D}$ & Spatiotemporal Field  \\
%    $\boldsymbol{y}_{obs}(\mathbf{x}_s, t)$ & $\mathbb{R}^{D_{obs}}$ & Spatiotemporal Observations  \\
%    $\eta$ & $\mathbb{R}$ & Sea Surface Height $[m]$ \\
%    $\bar{\eta}$ & $\mathbb{R}$ & Sea Surface Anomaly $[m]$ \\
%    $u$ & $\mathbb{R}$ & Zonal Velocity $[ms^{-2}]$ \\
%    $v$ & $\mathbb{R}$ & Meridional Velocity $[ms^{-2}]$ \\
%     \bottomrule
%   \end{tabular}
%   \label{tb:notation}
%  \end{table}


% \subsection{Coordinates}
% \textbf{SpatioTemporal Coordiantes}. We define some generic spatiotemporal coordinates.
% 
% We are dealing with satellite observations, so we are interested in the domain across the Earth's surface. 
% Let us define the Earth's domain by some spatial coordinates, $\mathbf{x} = [\text{Longitude},\text{Latitude}]^\top \in\mathbb{R}^{D_s}$, and temporal coordinates, $t=[\text{Time}]\in\mathbb{R}^+$, where $D_s$ is the dimensionality of the coordinate vector.  
% We can define some spatial (sub-)domain, $\Omega\subseteq\mathbb{R}^{D_s}$, and a temporal (sub-)domain, $\mathcal{T}\subseteq\mathbb{R}^+$. 
% This domain could be the entire globe for 10 years or a small region within the North Atlantic for 1 year.
%
%
% \begin{align} \label{eq:spatiotemporal_coords}
%     \text{Spatial Coordinates}: && \mathbf{x} &\in \Omega \subseteq \mathbb{R}^{D_s}\\ 
%     \text{Temporal Coordinates}: && t &\in \mathcal{T} \subseteq \mathbb{R}^{D_t}.
% \end{align}
% %
%
% In this case $D_s=2$ because we only have a two coordinates, however we can do some coordinate transformations like spherical to Cartesian. Likewise, we can do some coordinate transformation for the temporal coordinates like cyclic transformations or sinusoidal embeddings~\cite{ATTENTION}. We have two fields of interest from these spatiotemporal coordinates: the state and the observations.
% %
% %
% \begin{align} \label{eq:state_obs}
%     \text{State}: && \boldsymbol{u}(\mathbf{x},t) &: \Omega\times\mathcal{T}\rightarrow\mathbb{R}^{D_u} \\
%     \text{Observations}: && \boldsymbol{y}_{obs}(\mathbf{x},t) &: \Omega\times\mathcal{T}\rightarrow\mathbb{R}^{D_{obs}}
% \end{align}
% %
% %
% The state domain, $u\in\mathcal{U}$, is a scalar or vector-valued field of size $D_u$ which is typically the quantity of interest and the observation domain, $y_{obs}\in\mathcal{Y}_{obs}$, is the observable quantity which is also a scalar or vector-valued field of size $D_{obs}$. Now, we make the assumption that we have an operator $\mathcal{H}$ that transforms the field from the state space, $\boldsymbol{u}$, to the observation space, $\boldsymbol{y}_{obs}$.
% %
% %
% \begin{align} \label{eq:prob_definition}
%     \boldsymbol{y}_{obs}(\mathbf{x},t) = \mathcal{H}\left(\boldsymbol{u}(\mathbf{x},t), t, \boldsymbol{\varepsilon}, \boldsymbol{\mu}\right) 
% \end{align}
% %
% %


% % \subsection{Field}

% \textbf{Field}. We have the generic definition of a scalar or vector-valued field.

% \begin{align} \label{eq:field}
% \text{Field}:&& f&=\boldsymbol{f}(\mathbf{x},t), && \quad \Omega\times \mathcal{T}\rightarrow\mathbb{R}^{D}
% \end{align}
% \newpage
% \section{Metrics} \label{sec:metrics}

There are many metrics that are standard within the ML community but unconvincing for many parts the geoscience community. 
Specifically, many of these standard scores do not capture the important optimization criteria in the scientific machine learning tasks.
However, there is not consensus within domain-specific communities about the perfect metric which captures every aspect we are interested.
Therefore, we should have a variety of scores from different perspectives to really assess the pros and cons of each method we wish to evaluate thoroughly. 
Below, we outline two sets of scores we use within this framework: skill scores and spectral scores.

\subsection{Skill Scores}

We classify one set of metrics as \textit{skill scores}. 
These are globally averaged metrics which tend to operate within the real space.
Some examples include the root mean squared error (RMSE) and the normalized root mean squared (nRMSE) error.
The RMSE metric can also be calculated w.r.t. the spatial domain, temporal domain or both. 
For example, figure~\ref{fig:oceanbench_psd} showcases the nRMSE calculated only on the spatial domain and visualized for each time step.
%
\begin{align}
    \text{RMSE}: &&\text{RMSE}(\eta,\hat{\eta}) &= ||\eta - \hat{\eta}||_2 \label{eq:RMSE}\\
    % \text{RMSE}_t: &&\text{RMSE}_t(\eta,\hat{\eta}; t) &= ||\eta(t) - \hat{\eta}(t)||_2 \label{eq:RMSE_t}\\
    \text{nRMSE}: &&\text{nRMSE}(\eta,\hat{\eta}) &= 1 - \frac{\text{RMSE}(\eta - \hat{\eta})}{\text{RMSE}(\eta)} \label{eq:nRMSE}
\end{align}
%
However, we are not limited to just the standard MSE metrics.
We can easily incorporate more higher-order statistics like the Centered Kernel Alignment (CKA)~\cite{METRICSCKA} or information theory metrics like mutual information (MI)~\cite{METRICSITRBIG,METRICSITRBIG2}.
In addition, we could also utilize the same metrics in the frequency domain as is done in~\citep{PDEBench}.

\subsection{Spectral Scores}

Another class of scores that we use in \texttt{OceanBench} are the \textit{spectral scores}. These scores are calculated within the spectral space via the wavenumber power spectral density (PSD). 
This provides a spatial-scale-dependent metric which is useful for identifying the largest and smallest scales that were resolved by the reconstruction map. 
In general, we use these to measure the expected energy at different spatiotemporal scales and we can also construct custom score functions which gives us a summary statistic for how well we reconstructed certain scales.
%
\begin{align}
    \text{PSD}: &&\text{PSD}(\eta) &= \sum_{k_{min}}^{k_{max}}\|\mathcal{\mathcal{F}(\eta)}\|^2\label{psd}\\
    \text{PSD}_{score}: &&\text{PSD}_{score}(\eta,\hat{\eta}) &= 1 - \frac{\text{PSD}(\eta - \hat{\eta})}{\text{PSD}(\eta)} \label{eq:psd_score}
\end{align}
%
where $\mathcal{F}$ is the Fast Fourier Transformation (FFT). 
In our application, there are various ways to construct the PSD which depend on the FFT transformation.
We denote the \textit{space-time PSD} as $\lambda_\mathbf{x}$ which does the 2D FFT in the longitude and time direction, then takes the average over the latitude.
We denote the \textit{space-time PSD} as $\lambda_\mathbf{t}$ which does the 2D FFT in the longitude and latitude direction, then takes the average over the time.
We denote the \textit{isotropic PSD} as $\lambda_r$ which assumes a radial relationship in the spatial domain and then averages over the temporal domain.
Lastly, we denote the standard PSD score as $\lambda_a$ which is the 1D FFT over a prescribed distance along the satellite track; this is what is done for the OSE NADIR experiment.
We recognize that the FFT configurations are limited due to their global treatment of the spectral domain and we need more specialized metrics to handle the local scales.
This opens the door to new metrics that handle such cases such as the Wavelet transformation~\cite{METRICSWAVELET}.

\begin{figure}[t!]
\small
\begin{center}
\setlength{\tabcolsep}{1pt}
\begin{tabular}{cccc}
\hspace{3mm} Task OSSE & 
 Task OSSE & 
\hspace{-10mm} Task OSSE & 
\hspace{-10mm}Task OSE \\
\hspace{3mm}  Nadir & 
 Nadir + SWOT & 
\hspace{-10mm} Nadir + SST & 
\hspace{-10mm}Nadir \\
%\vspace{-2mm}
%%%%% SSH %%%%%%%%
\includegraphics[trim={0 0 0 0},clip, width=3.70cm,height=3.5cm]{00_Oceanbench/content/figures/fourdvarnet_figs/osse_gf_nadir_isotrop.png} &
\includegraphics[trim={18mm 0 0 0},clip, width=3.3cm,height=3.5cm]{00_Oceanbench/content/figures/fourdvarnet_figs/osse_gf_nadirswot_isotrop.png} &
\hspace{-5mm}\includegraphics[trim={18mm 0 0 0},clip, width=3.3cm,height=3.5cm]{00_Oceanbench/content/figures/fourdvarnet_figs/osse_gf_nadir_sst_isotrop.png} &
\hspace{-10mm}\includegraphics[trim={18mm 0 0 0},clip,width=3.3cm,height=3.5cm]{00_Oceanbench/content/figures/fourdvarnet_figs/ose_gf_isotrop.png} \\
%\vspace{3mm}
%%%%% KINETIC ENERGY %%%%%%%%
\includegraphics[trim={0 0 0 0}, clip, width=3.70cm,height=3.5cm]{00_Oceanbench/content/figures/fourdvarnet_figs/osse_gf_nadir_1d_psd_score.png} &
\hspace{1mm}\includegraphics[trim={18mm 0 0 0},clip, width=3.3cm,height=3.5cm]{00_Oceanbench/content/figures/fourdvarnet_figs/osse_gf_nadirswot_1d_psd_score.png} &
\hspace{-4mm}\includegraphics[trim={18mm 0 0 0},clip, width=3.3cm,height=3.5cm]{00_Oceanbench/content/figures/fourdvarnet_figs/osse_gf_nadir_sst_1d_psd_score.png} &
\hspace{-10mm}\includegraphics[trim={18mm 0 0 0},clip,width=3.3cm,height=3.5cm]{00_Oceanbench/content/figures/fourdvarnet_figs/ose_gf_1d_psd_score.png} \\
%%%%% RELATIVE VORTICITY %%%%%%%%
\hspace{-4mm}\includegraphics[trim={0 0 23mm 0},clip, width=3.65cm,height=3.5cm]{00_Oceanbench/content/figures/fourdvarnet_figs/osse_gf_nadir_psd_spacetime.png} &
\includegraphics[trim={14mm 0 23mm 0},clip, width=3cm,height=3.5cm]{00_Oceanbench/content/figures/fourdvarnet_figs/osse_gf_nadirswot_psd_spacetime.png} &
\hspace{-5mm}\includegraphics[trim={14mm 0 23mm 0},clip, width=3cm,height=3.5cm]{00_Oceanbench/content/figures/fourdvarnet_figs/osse_gf_nadir_sst_psd_spacetime.png} &
\hspace{-5mm}\includegraphics[trim={14mm 0 0 0},clip,width=3.8cm,height=3.5cm]{00_Oceanbench/content/figures/fourdvarnet_figs/ose_gf_psd_spacetime.png} \\
%%%%% STRAIN %%%%%%%%
\hspace{-4mm}\includegraphics[trim={0 0 23mm 0},clip, width=3.70cm,height=3.5cm]{00_Oceanbench/content/figures/fourdvarnet_figs/osse_gf_nadir_psd_spacetime_score.png} &
\hspace{-2mm}\includegraphics[trim={13mm 0 23mm 0},clip, width=3.1cm,height=3.5cm]{00_Oceanbench/content/figures/fourdvarnet_figs/osse_gf_nadirswot_psd_spacetime_score.png} &
\hspace{1mm}\includegraphics[trim={13mm 0 0 0},clip, width=3.8cm,height=3.5cm]{00_Oceanbench/content/figures/fourdvarnet_figs/osse_gf_nadir_sst_psd_spacetime_score.png} &
 \\
% \vspace{-2mm}
 \hspace{1mm} (a) & \hspace{-5mm} (b) & \hspace{-8mm}(c) & \hspace{-10mm}(d)
\end{tabular}
\vspace{-3mm}
% \caption{Row I - Isotrophic PSD. Row 2 - Isotrophic PSD Score}
\caption{
Power spectrum and associated scores of the 4dVarNet method for each of the four tasks.
The row display in order: (1) the isotropic PSD, (2) the spatial PSD score (using the isotropic PSD for the first three rows and along track PSD for the last row), (3) the space-time PSD, (4) The spacetime PSD score available only in OSSE task.  }

\vspace{-5mm}
\label{fig:oceanbench_psd_4dvarnet}
\end{center}
\end{figure}


% \begin{figure}[h]
% \small
% \begin{center}
% \setlength{\tabcolsep}{1pt}
% \begin{tabular}{cccc}
% \hspace{3mm} NEMO Simulation & 
% \hspace{3mm} MIOST & 
% \hspace{3mm} BFNQG & 
% 4DVarNet \\
% \vspace{-2mm}
% %%%%% SSH %%%%%%%%
% \includegraphics[trim={0 0 38mm 0},clip, width=3.20cm,height=3cm]{content/figures/psd_spacetime/dc20a/nadir4/dc20a_psd_spacetime_nemo_nadir4_ssh.png} &
% \includegraphics[trim={0 0 40mm 0},clip, width=3.2cm,height=3cm]{content/figures/psd_spacetime/dc20a/nadir4/dc20a_psd_spacetime_miost_nadir4_ssh.png} &
% \includegraphics[trim={0 0 38mm 0},clip, width=3.2cm,height=3cm]{content/figures/psd_spacetime/dc20a/nadir4/dc20a_psd_spacetime_bfnqg_nadir4_ssh.png} &
% \includegraphics[width=4.0cm,height=3cm]{content/figures/psd_spacetime/dc20a/nadir4/dc20a_psd_spacetime_4dvarnet_nadir4_ssh.png} \\
% \end{tabular}
% % \vspace{-3mm}
% % \caption{Row I - Isotrophic PSD. Row 2 - Isotrophic PSD Score}
% \caption{The space-time power spectrum decomposition.}
% % \vspace{-5mm}
% \label{fig:appendix_psd_spacetime_NADIR}
% \end{center}
% \end{figure}


% \begin{figure}[h]
% \small
% \begin{center}
% \setlength{\tabcolsep}{1pt}
% \begin{tabular}{cccc}
% \hspace{3mm} DUACS & 
% \hspace{3mm} MIOST & 
% \hspace{3mm} BFNQG & 
% 4DVarNet \\
% % \vspace{-2mm}
% %%%%% SSH %%%%%%%%
% \includegraphics[trim={0 0 38mm 0},clip, width=3.20cm,height=3cm]{content/figures/psd_spacetime/dc20a/nadir4/dc20a_psd_spacetime_score_duacs_nadir4_ssh.png} &
% \includegraphics[trim={0 0 40mm 0},clip, width=3.2cm,height=3cm]{content/figures/psd_spacetime/dc20a/nadir4/dc20a_psd_spacetime_score_miost_nadir4_ssh.png} &
% \includegraphics[trim={0 0 38mm 0},clip, width=3.2cm,height=3cm]{content/figures/psd_spacetime/dc20a/nadir4/dc20a_psd_spacetime_score_bfnqg_nadir4_ssh.png} &
% \includegraphics[width=4.0cm,height=3cm]{content/figures/psd_spacetime/dc20a/nadir4/dc20a_psd_spacetime_score_4dvarnet_nadir4_ssh.png} \\
% \end{tabular}
% % \caption{Row I - Isotrophic PSD. Row 2 - Isotrophic PSD Score}
% \caption{The space-time power spectrum score decomposition.}
% % \vspace{-5mm}
% \label{fig:appendix_psd_score_spacetime_NADIR}
% \end{center}
% \end{figure}
% \newpage
% \section{Use Case II: Hydra Recipes} \label{sec:hydra_recipes}

This framework has drastically reduced the overhead for the ML researcher while also enhancing the reprducibility and replicability of the preprocessing steps. In this section we showcase a few examples for how one can use oceanbench in conjunction with hydra to provide recipes for some standard processes.

\subsection{GeoProcessing Recipe}

In this example, we showcase how one can pipe a sequential transformation through the hydra framework. In this example, we open the dataset, validate the coordinates to comply to our standards, select the region of interest, subset the data, regrid the alongtrack data to a uniform grid, and save the data to a netcdf file. See the listing~\ref{hydraconfig:geoprocess} for more information.


\begin{listing}[ht!]
\begin{minted}[frame=lines]{yaml}
# Target Function to initialize
_target_: "oceanbench._src.dataset.pipe"
# netcdf file to be loaded
inp: "${data_directory}/nadir_tracks.nc"
# sequential transformations to be applied
fns:
    # Load Dataset
    - {_target_: "xarray.open_dataset", _partial_: True}
    # Validate LatLonTime Coordinates
    - {_target_: "oceanbench.validate_latlon", _partial_: True}
    - {_target_: "oceanbench.validate_time", _partial_: True}
    # Select Specific Region (Spatial | Temporal)
    - {_target_: "xarray.Dataset.sel", args: ${domain}, _partial_: True}
    # Take Subset of Data
    - {_target_: "oceanbench.subset", num_samples: 1500, _partial_: True}
    # Regridding (AlongTrack -> Uniform Grid)
    - {
        _target_: "oceanbench.regrid", 
        target_grid: ${grid.high_res}, 
        _partial_: True
      }
    # Save Dataset
    - {
        _target_: "xarray.Dataset.to_netcdf", 
        save_name: "demo.nc", 
        _partial_: True
      }
\end{minted}
\label{hydraconfig:geoprocess}
\caption{This is a \texttt{.yaml} which showcases how we can communicate with \texttt{Hydra} framework to list a predefined set of transformations to be \textit{piped} through sequentiall. In this example, we showcase some standard pre-processing strategies to be saved to another netcdf file.}
\end{listing}




\newpage
\subsection{Evaluation Recipe - OSSE}

In this example, we showcase how one can use hydra to do the evaluation procedure. This is the same evaluation procedure that is used to evaluate the effectiveness of the OSSE NADIR experiment. From code snippet~\ref{hydraconfig:geoprocess}, we see that we choose which target function to initialize and we choose the data directory where the \texttt{.netcdf} file is located. Then, we pipe some transformations for the \texttt{.netcdf} file: 1) validate the spatiotemporal coordinates, 2) we select the evaluation region, 3) we regrid it to the target get, 4) we fill in the nans with a Gauss-Seidel procedure, 5) we rescale the coordinates to be in meters and days, and 6) we perform the isotropic power spectrum transformation to get the effective resolution outlined in section~\ref{sec:metrics}.

\begin{listing}[ht!]
\begin{minted}[frame=lines]{yaml}
# Target Function to initialize
_target_: "oceanbench._src.dataset.pipe"
# netcdf file to be loaded
inp: "${data_directory}/ml_result.nc"
# sequential transformations to be applied
fns:
    # Load Dataset
    - {_target_: "xarray.open_dataset", _partial_: True}
    # Validate LatLonTime Coordinates
    - {_target_: "oceanbench.validate_latlon", _partial_: True}
    - {_target_: "oceanbench.validate_time", _partial_: True}
    # Select Specific Region (Spatial | Temporal)
    - {_target_: "xarray.Dataset.sel", args: ${domain}, _partial_: True}
    # Regridding (Uniform Grid -> Uniform Grid)
    - {_target_: "oceanbench.regrid", 
       target_grid: ${grid.reference}, _partial_: True}
    # Fill NANS (around the corners)
    - {_target_: "oceanbench.fill_nans", 
       method: "gauss_seidel", _partial_: True}
    # Coordinate Change (degree -> meters, ns -> days)
    - {_target_: "oceanbench.latlon_deg2m", _partial_: True}
    - {_target_: "oceanbench.time_rescale", 
       freq: 1, unit: "days", _partial_: True}
    # Calculate Isotropic Power Spectrum
    - {_target_: "oceanbench.power_spectrum_isotropic", 
       reference: ${grid.reference}, _partial_: True}
    # Calculate Resolved Spatial Scale
    - {_target_: "oceanbench.resolved_scale", _partial_: True}
    # Save Dataset
    - {_target_: "xarray.Dataset.to_netcdf", 
       save_name: "ml_result_psd.nc", _partial_: True}
\end{minted}
\label{hydraconfig:evaluation}
\caption{This is a \texttt{.yaml} which showcases how we can communicate with \texttt{Hydra} framework to list a predefined set of transformations to be \textit{piped} through sequentiall. In this example, we showcase some standard pre-processing strategies to be saved to another netcdf file.}
\end{listing}





% \newpage
% \section{Use Case III: XRPatcher}
\label{sec:xrpatcher}

There are many usecases for the \texttt{XRPatcher}. For example, we can do 1D Time chunking, 2D Spatial-Temporal Patches, or 3D Spatial-Temporal Cubes.


\begin{listing}[!ht]
\small
\begin{minted}[frame=lines]{python}
import xarray as xr
import torch
import itertools
from oceanbench import XRPatcher
# Easy Integration with PyTorch Datasets (and DataLoaders)
class XRTorchDataset(torch.utils.data.Dataset):
    def __init__(self, batcher: XRPatcher, item_postpro=None):
        self.batcher = batcher
        self.postpro = item_postpro
    def __getitem__(self, idx: int) -> torch.Tensor:
        item = self.batcher[idx].load().values
        if self.postpro:
            item = self.postpro(item)
        return item
    def reconstruct_from_batches(
            self, batches: list(torch.Tensor), **rec_kws
        ) -> xr.Dataset:
        return self.batcher.reconstruct(
            [*itertools.chain(*batches)], **rec_kws
        )
    def __len__(self) -> int:
        return len(self.batcher)
# load demo dataset
data = xr.tutorial.load_dataset("eraint_uvz")
# Instantiate the patching logic for training
patches = dict(longitude=30, latitude=30)
train_patcher = XRPatcher(
    da=data,
    patches=patches,
    strides=patches,        # No Overlap
    check_full_scan=True    # check no extra dimensions
)
# Instantiate the patching logic for testing
patches = dict(longitude=30, latitude=30)
strides = dict(longitude=5, latitude=5)
test_patcher = XRPatcher(
    da=data,
    patches=patches,
    strides=strides,        # Overlap
    check_full_scan=True    # check no extra dimensions
)
# instantiate PyTorch DataSet
train_ds = XRTorchDataset(train_patcher, item_postpro=TrainingItem._make)
test_ds = XRTorchDataset(test_patcher, item_postpro=TrainingItem._make)
# instantiate PyTorch DataLoader
train_dl = torch.utils.data.DataLoader(train_ds, batch_size=4, shuffle=False)
test_dl = torch.utils.data.DataLoader(test_ds, batch_size=4, shuffle=False)
\end{minted}
\label{listing:xrpatcher}
\caption{\texttt{XRPatcher} integration in Pytorch. We define a PyTorch dataset that handles the \texttt{XRPatcher}. We load an arbitrary dataset with \texttt{xarray}, then we instantiate the \texttt{XRPatcher} with the patching logic, then we instantiate the PyTorch dataset and dataloaders.}
\end{listing}





% \newpage
% \section{Additional Tasks}\label{sec:other_tasks}

In the main paper, we thoroughly outlined the interpolation task to showcase how \texttt{OceanBench} can be used to create automated pipelines for processing and evaluation procedures.
However, there are many other additional tasks that can make use of the \texttt{OceanBench} features. 

\textbf{Denoising}. A simpler problem for interpolation tasks is the denoising problem~\cite{DENOISESURVEY,DENOISESURVEY2}.
The SSH and SST measurements we obtain have inherent noise from the sensors.
A key problem is to calibrate the observations by separating the known noise patterns and the true signal.
There has already been a lot of work from the ML side ranging from amortized predictions~\cite{DENOISESWOT} to end-to-end learning schemes~\cite{DENOISESWOT2}.
Much of this work has been facilitated by the \textit{Ocean-Data-Challenge} group which have a few data challenges related to the denoising problem.
Just like \texttt{OceanBench} was able to create reproducible pipelines from the SSH interpolation challenge listed in section~\ref{sec:data_challenges}, we also believe that one could extend the denoising challenge in the same manner.

\textbf{Forecasting}. This is a special form of extrapolation whereby the temporal domain of the state variable is sufficiently outside of the domain of the observation domain. 
Many previous benchmarking suites already look at forecasting for weather~\cite{weatherbench} and climate~\cite{ClimateBench}.
However, in oceanography, it is also advantageous to do forecasting for problems involving currents~\cite{MLSSC,4DVarNetSSC} and eddies~\cite{OCEANEddyTracking,OKUBO,OKUBOWEISS}.
The \texttt{xrpatcher} will work out of the box for forecasting problems and contributions can be made to \texttt{OceanBench} to include some specific metrics for forecasting as were outlined in~\cite{weatherbench,ClimateBench,ENS10Bench}.

\textbf{Proxy Variables}. There are many other control variables that one could use to improve the interpolation or extrapolation task.
We mentioned SST in section~\ref{sec:data_challenges} because it is the most abundant observations available.
However, there are other important observed variables which could be useful, e.g. Ocean colour, Bigeochemical parameters, and atmospheric variables.
In many other downstream applications, the oceanography community often uses SSH and SST as proxy variables to predict important quantities related to the carbon uptake, e.g. SOCAT~\cite{SOCAT}.
It would be straightforward to include a specific variable (and the associated preprocessing operations) into \texttt{OceanBench}.

\textbf{Dimension Reduction}. We often have very resolution spatiotemporal fields.
which poses a very big challenge for learning due to the high correlations exhibited by spatiotemporal data and high dimensionality.
A workaround for this is to learn a latent representation which retains as much relevant information as possible for the given task.
In the ocean sciences, this is known as \textit{Reduced Order Modeling} (ROM) or more generally dimensionality reduction which has been frequently used for adaptive meshes for physical models~\cite{NEMOEOF}.
This could be used for pretraining fields to latent embeddings which could be useful for downstream tasks like anomaly detection~\cite{SSTFLOWANOMALY}.
% In modern diffusion models, most of the operations are within the latent domain without any loss of quality in the generative results.


\textbf{Surrogate Modeling}. 
Physical model simulations are very expensive and ML has played a role in learning surrogate models to descrease the computational intensity~\cite{ML4OCN,MLCLOSURE}.
We have a decently long spatiotemporal field over a region of interest which could be used to learn a surrogate model to mimic the dynamics of that region.
This is also very useful for hybrid schemes whereby we have parameterizations to account for processes that are missing from low resolution simulations.~\cite{MLOCNPARAMETERIZATION,MLOCNPARAMETERIZATION2, MLOCNPARAMETERIZATION3, MLOCNPARAMETERIZATION4}.



% \subsection{Main Tasks}

% \subsubsection{Interpolation/Extrapolation}

% The two main tasks we can define from this problem setup are: 1) interpolation/extrapolation and 2) forecasting (which can be seen as extrapolation in Time). 
% Both interpolation and extrapolation are when the domain for the $\mathcal{T}$ always falls between the past and the present, i.e., $\mathcal{T}\in[0, T]$ where $T$ is the current Time. 
% We define interpolation as the case where the observations along the boundary of the domain are equal to the boundary of the desired reconstruction domain, i.e., $\partial\boldsymbol{\Omega}_{obs} = \partial\boldsymbol{\Omega}_g$, and extrapolation as the case where the boundary of the observation domain lies entirely inside of the boundary of the reconstructed domain, i.e., $\partial\boldsymbol{\Omega}_{obs} \subset \partial\boldsymbol{\Omega}_g$. 
% Forecasting refers to the problem when the temporal domain $\mathcal{T}$ lies outside of the present, i.e., $T+\tau$ where $\tau$ is some step within the future that is outside of the temporal observation domain, $\mathcal{T}\in[0, T]$. 
% Note that this is irrespective of the spatial domain or its boundaries. 
% In the rest of this paper, we will look exclusively at the interpolation problem, but we refer to the reader to section~\ref{sec:other_tasks} in the appendix for a more detailed look at the other tasks.

% \begin{equation}
%     \mathbf{y} = \mathcal{H}(\mathbf{x})
% \end{equation}


% where $\mathbf{y}\in\boldsymbol{\Omega_p}$ is the incomplete observation within some subdomain and $\mathbf{x}\in\boldsymbol{\Omega}$  is the true observation over the full domain.



% \subsubsection{Forecasting}

% \begin{equation}
%     \boldsymbol{u}_{t+\delta t} = \mathcal{M}(\boldsymbol{u}(\mathbf{x}, t+\delta t), \delta t; \boldsymbol{\theta})
% \end{equation}

% \subsection{SubTasks}

% \subsubsection{Surrogate/Hybrid/Parameterizations}

% \begin{equation}
%     \frac{\partial u}{\partial t} = \mathcal{M}(\boldsymbol{u}(\mathbf{x},t), \mathbf{x}, t; \theta)
% \end{equation}





% \newpage
% \section{Machine Learning Method Ontology} \label{sec:ml_ontology}

Although this paper does not focus on the explicit methods used for SSH interpolation, we would like to give a readers a brief overview of some of the most popular methods in the literature.

\subsection{Coordinate-Based methods}

These methods learn a direct mapping between the coordinate vectors to the scalar or vector values. 
%
\begin{align}
    \boldsymbol{y}_{obs} &= \boldsymbol{f}(\mathbf{x},t;\boldsymbol{\theta})+\boldsymbol{\epsilon}(\mathbf{x},t)
\end{align}
%
This is better known as \textit{functa}~\cite{FUNCTA} which parameterizes the field directly as a model.

\textbf{Functional}. Optimal Interpolation (OI) is the most common method used for many of the operational methods~\cite{DUACS}. It is a non-parametric, functional method which is built upon covariance and precision matrices. In the machine learning community, these methods are known as Gaussian Process~\cite{GPsBIGDATA} and in the geostatistics community, this is known as Kriging~\cite{KRIGINGREVIEW}.

\textbf{Basis Function}. This is an easy simplification to the functional by introducing parametric basis functions. In particular, the MIOST~\cite{MIOST} algorithm will be adopted in the new operational products for SSH interpolation. It is a custom basis function based on Wavelet analysis which is scale-aware and scalable.

\textbf{Neural Fields}. Neural fields (NerFs) are a very popular set of methods that use neural networks to effectively learn the basis function through a composition of weights, biases and activations~\cite{NERFSSSH}.
Furthermore, one can add physics-informed constraints to the loss function which mirror that of a PDE~\cite{PINNS}.
In many cases, especially with many auxillary inputs, we don't have access the PDE so one fit a NN directly to the observations with a fully connected neural network~\cite{SOCAT}.


\subsection{Grid-Based Methods}

In practice, we often consider the field at a specific discretized setting like a uniform grid or mesh. 
This is because we typically operate on and store these fields as multi-dimensional arrays which are only defined on a subspace of the entire continuous domain. 
We denote a discretized spatial representation as $\boldsymbol{\Omega}_g\subset\mathbb{R}^{N_s}$. 
We can simplify this notation by including the domain within the operator. So equation~\ref{eq:interp_problem} like so:
\begin{equation}\label{eq:interp_problem_discretized}
    \boldsymbol{\eta}(\boldsymbol{\Omega}_{obs},t ) = \mathcal{H}\left(\boldsymbol{\eta}(\boldsymbol{\Omega}_g,t), t, \boldsymbol{\mu},  \boldsymbol{\varepsilon} \right) 
    % + \boldsymbol{\varepsilon}(\boldsymbol{\Omega}_g, t)
\end{equation} 
%
In this equation, $\mathcal{H}$ is the observation operator that transforms the field from the full discretized domain, $\boldsymbol{\Omega}_g$, to the observation domain, $\boldsymbol{\Omega}_{obs}\subset\mathbb{R}^{N_{obs}}$.

\textbf{Direct Methods}. 
These methods take the noisy, incomplete observations and directly feed it to a model that returns the full reconstructed field.
They typically involve training a convolutional neural network or recurrent neural network on pairs of corrupted observations to learn the reconstruction~\cite{SuperResSurvey,IMAGE2IMAGETRANSLATION, IMAGE2IMAGETRANSLATION2}.
This has seem some sucess in applications related to SSH interpolation~\cite{SSHInterpUNet,SSHInterpConvLSTM, SSHInterpAttention}.

\textbf{Traditional Data Assimilation.}
There are many traditional methods that are rooted in data assimilation~\cite{DAGEOSCIENCE}.
For example, the GLORYS~\cite{GLORYS12} method propagates the physical model forwards in time and then \textit{updates} the state based on observations periodically.
A simpler approach is to use a nudging scheme coupled with a simpler physical model~\cite{BFNQG}.


\textbf{End-to-End Learning}. These methods try to solve the problem by learning and end-to-end scheme to solve the model inversion problem.
This is very similar to implicit methods that define a cost function to minimize instead of a minimizing the parameters of a prior model.
Plug-in-Play priors are a popular class of methods that pre-train priors on auxillary observations and then use the prior in the inversion scheme~\cite{DEEPUNFOLDING}.
This has seen a lot of success in SSH interpolation~\cite{4DVARNETSWOT,4DVARNETSST,4DVarNetSSC}.



% \newpage
% In practice, we only consider the field at a specific discretized setting like a uniform grid or mesh. 
% This is because we typically operate on and store these fields as multi-dimensional arrays which are only defined on a subspace of the entire continuous domain. We denote this discretized spatial representation as $\boldsymbol{\Omega}_g\subset\mathbb{R}^{N_s}$. We can simplify this notation by including the domain within the operator, like so:
% %
% \begin{equation} \label{eq:ssh_field_discretized}
%     \eta =\boldsymbol{\eta}(\boldsymbol{\Omega}_g,t),
% \end{equation}
% %
% This is more reflective of how we use operators in practice as we typically insert the field as a grid or multi-dimensional spatial array through a series of mathematical operations.
% We can further discretized this field through the time domain whereby we have a finite set of observations in time, $D_t$. Let's say that we have $N_t$ ordered samples in time between a defined interval of $[0,T]$, i.e. $\mathcal{T}=\left\{ t_t \in [0,T]\right\}_i^{N_{t}}$. In some settings, this could be a uniform observation that is hourly for the period of 1 day, i.e. $N_t=24$, or daily over the period of 10 years, i.e. $N_t=3650$. In more realistic settings, this could be an irregular pattern.
% Equation~\eqref{eq:ssh_field_discretized} assumes the full field is observed. In practice, we observe a corrupted, incomplete version of this SSH field s.t.
% %
% \begin{equation}\label{eq:obs_operator_discretized}
%     \boldsymbol{\eta}(\boldsymbol{\Omega}_{obs},t ) = \mathcal{H}\left(\boldsymbol{\eta}(\boldsymbol{\Omega}_g,t), t, \boldsymbol{\mu},  \boldsymbol{\varepsilon} \right) 
%     % + \boldsymbol{\varepsilon}(\boldsymbol{\Omega}_g, t)
% \end{equation} 

% In this equation, $\mathcal{H}$ is the observation operator that transforms the field from the full discretized domain, $\boldsymbol{\Omega}_g$, to the observation domain, $\boldsymbol{\Omega}_{obs}\subset\mathbb{R}^{N_{obs}}$. 
% The observation domain, $\boldsymbol{\Omega}_{obs}$, are the spatial coordinates, $\mathbf{x}$, where the SSH has been observed which is a proper subset of the full domain, i.e. $\boldsymbol{\Omega}_{obs}\subseteq\boldsymbol{\Omega}_g$.

% \begin{align} \label{eq:interp_problem}
%     \mathcal{M}_{\boldsymbol{\theta}} &: \boldsymbol{\eta}_{obs}(\mathbf{x}, t, \boldsymbol{\mu}) \rightarrow \boldsymbol{\eta}(\mathbf{x},t) \hspace{10mm}
%      \boldsymbol{\Omega}_{obs} \in \mathbb{R}^{N_{obs}} \hspace{5mm} 
%     \boldsymbol{\Omega}_g \in \mathbb{R}^{N_s} \hspace{5mm} 
%     t \in \mathcal{T}.
% \end{align}


% \subsubsection{Hybrid Schemes} \label{sec:bfn}

% This includes Backwards-Forwards Nudging with a simpler model like the Quasi-Geostrophic equations.

% \subsubsection{Direct Methods}

% These feature methods that directly take in the sparse SSH fields with \tocite{Recent Papers}

% \subsubsection{Data Assimilation}

% \begin{align}
%     \boldsymbol{\theta}^* &= \underset{\boldsymbol{\theta}}{\text{argmin}} \hspace{2mm} \mathcal{L}(\mathbf{x}_{gt}, \mathbf{x}(\boldsymbol{\theta})) \\
%     \mathbf{x}^*(\boldsymbol{\theta}) &= \underset{ \mathbf{x}^*}{\text{argmin}}  \hspace{2mm} \mathcal{U}\left(\mathbf{x}(\boldsymbol{\theta}) \right)
% \end{align}

% where $\mathcal{L}$ is some loss function to find the best parameters and $\mathcal{U}$ is some energy function to minimize the state.

% \subsubsection{End-to-End 4DVariational Methods} \label{sec:4dvarnet}

% These methods take a variational approach whereby we directly define the cost function we wish to solve \tocite{Ronan}. To speed up the convergence of these methods, a meta-learning based method was introduced to learn the gradient descent\tocite{Ronan}. In addition, this has been improved with the addition of sea surface temperature as an additional term in the cost function. 



% % \newpage
% % \section{Target Audience}
% \subsection{Project Vision}
While this tool is general in scope, we specifically target three audiences: 1) the domain expert who may want to use and understand and investigate SSH in relation to other important EO quantities, 2) machine learning researchers who may want to investigate how to make a better model for SSH interpolation, and 3) downstream users who are interested in adopting some techniques for their own domain-specific applications that may rely on SSH like tracking ocean currents~\tocite{}, investigating biogeochemical transport~\tocite{}, or global climate change~\tocite{}. In the following subsections, we give a more detailed description about the users and how might they benefit from \texttt{OceanBench}.

\textbf{Domain Experts}. We consider \textit{domain experts} who are experts in different domains of oceanography. ...


\textbf{ML Researchers}. We consider those who are expert ML researchers but may lack...



\textbf{Down Stream Users}. We also envision a broader adoption of the framework across research labs interested in having standardized data challenges for their own research purposes. There are independent groups choosing datasets for their specific use cases however, this framework can serve as an easy way to integrate their existing methodologies into the set of common tools to be used and improved by multiple communities. By establishing a relatively consistent problem set, we hope that any innovation can be easily understood and transferred across domains.
% \newpage
% \section{Limitations} \label{sec:appendix_limitations}

\subsection{Framework Limitations}

While we have advertised \texttt{OceanBench} as a unifying framework that provides standardized processing steps that comply with domain-expert standards, we also highlight some potential limitations that could hinder its adoption for the wider community.

\textbf{Data Serving}. We provide a few datasets but we omit some of the original simulations. We found that the original simulations are terabytes/petabytes of data which becomes infeasible for most modest users (even with adequate CPU resources).  
This is very big problem and if we want to have a bigger impact, we may need to do more close collaborations with specified platforms like the Marine Data Store~\citep{MDSOCEANPHYSICS,MDSBIOGEOCHEMICAL,MDSOCEANPHYSICSENS,MDSINSITU,MDSWAVES,MDSALONGTRACK,MDSSSH} or the Climate Data Store~\citep{CDSREANALYSISSST,CDSOBSSST,CDSOBSOC,CDSOBSSSTENS}. Furthermore, there are many people that will not be able to do a lot of heavy duty research which indirectly favours institutions with adequate resources and marginalizing others. 
This is also problematic as those communities tend to be the ones who need the most support from the products of such frameworks.
We hope that leaving this open-source at least ensure that the knowledge is public.

\textbf{Framework Dependence}. The user has to "buy-into" the \texttt{hydra} framework to really take advantage of \texttt{OceanBench}. This adds a layer of abstraction and a new tool to learn. 
However, we designed the project so that high level usage does not require in-depth knowledge of the framework. 
In addition, we hope that, despite the complexity of project, users will appreciate the flexibility and extensibility of this framework.


\textbf{Lack of Metrics}. We do not provide the most exhaustive list of metrics available with the ocean community. In fact, we also believe that many of these metrics are often poor and do not effectively assess the goodness of our reconstructions. 
However, we do provide a platform that will hopefully be useful and easy to implement new and improved metrics.
Furthermore, having a wide range of metrics that are trusted across communities may help to improve the overall assessment of the different model performances~\cite{METRICSAVERAGE}.

\textbf{Limited ML Scope}. 
The framework does not support nor promote any machine learning methods and we lack any indication of comparing ML training and inference performance. 
However, we argue that a benchmark framework will allow us to effectively compare whichever ML methods are demonstratively the best which is a necessary preliminary step which offers users more flexibility in the long-run.

\textbf{Broad Oceans Application Scope}. 
We have targeted a broad ocean-application scope of state estimation.
However, there may be more urgent applications such as maritime monitoring, object tracking, and general ocean health.
However, we feel that many downstream applications require high-quality maps.
In addition, those downstream applications tend to be very complicated and are not always straightforward to apply ML under those instances.

\textbf{Full Pipeline Transparency}. We use a lot of different \texttt{xarray}-specific packages which have different design principles, assumptions and implementations. This may give the users an illusion of simplicity and transparency to real-world use. However, there are many underlying assumptions within each of the packages that may occlude a lot of design decisions.
Despite this limitation, we believe that being transparent about the processing steps and being consistent with the evaluation procedure will be beneficial for the ML research community.

\textbf{Scalability}. Scaling this to many terabytes or petabytes of data is easily the biggest limitation of the framework. In addition, we have only showcased demonstrations for 2D+T fields which are much less expensive than 3D+T fields.

\textbf{Deployability}. MLOPs has many wheels and it is not easy to integrate into existing systems. We offer no solutions to this. 
However, we believe that our framework is fully transparent in the assumptions and use cases which will facilitate some adoption into operational systems where they can further modify it for their use cases (see the evolution of \texttt{WeatherBench} and \texttt{ClimateBench}).

\textbf{Visualization Tools}.
We do not incorporate a high quality visualization tool that allows users to do pre- and post-analysis at a large scale. 
We do provide some simple visualization steps that are ML-relevant (see the GitHub repo) but it is very limited to ML standards.
One solution is to interface our pipeline with the source of many ocean datasets, e.g. Climate Data Store~\citep{CDSREANALYSISSST} or Marine Data Store~\citep{MDSOCEANPHYSICS}, then we can offset this task to them where they can offer better quality visualization tools.

\newpage
\subsection{Data Challenge Limitations}

We have showcased the SSH interpolation edition as a data challenge which could be helpful for real applications. 
However, in section~\ref{sec:problem_scope} we alluded to the greater task of general ocean state estimation which is more pertinent to the ocean sciences yet we don't address this directly with our data challenges.
Furthermore, we claim that the data challenges presented will help the ocean community with using ML for SSH interpolation.
Below, we outline some limitations which address these criticisms.

\textbf{Not the overall objective}. 
We recognized that we are far away from the actual reanalysis and forecasting goals of full state estimation. 
However, we argue that that is a rather ambitious challenge which will require a lot of interdisciplinary work across communities. 
In the meantime while we work towards that goal, operational centers could possibly improve their current products from ML-based techniques would would benefit downstream applications that deal directly with SSH.
Furthermore, SSH is an important variable in describing the full ocean state.
So a robust set of techniques that are able to solve the interpolation tasks could (in principal) be used to solve extra tasks.

\textbf{Small Region \& Period}.
We only feature a small region and period over the Gulfstream which is not representative of the different global regimes. 
This also does not take into account real things like \textit{data drift} which will inevitable occur in operational settings.
However, this is a dynamical regime and a well-studied area which does have some importance for specific communities and the results obtained offer some transferability to other dynamical regimes.
In addition, this area will have good coverage due to the new SWOT mission~\cite{SWOT} which will allow for further validation in the future.
Lastly, the area is small enough where the beginning stages for ML researchers is not overwhelmed with problems involving scale (even though we eventually want to arrive at global schemes).
We hope to extend our challenges to more relevant scenarios~\cite{MDSALONGTRACK}.

\textbf{Simulations versus Reanalysis}. We use simulations for the OSSE experiments instead of reanalysis. This is an open research question as it is unclear whether it's better to pretrain models on simulated ocean data or reanalysis ocean data. In future updates, we plan to add the reanalysis data to extend the challenge.

\textbf{Efficacy of OSSE Experiments}. We alluded to the idea that the OSSE experiments may not reflect the overarching goal of the user yet we provide more OSSE experiments than OSE experiments.
We acknowledged that it often does not coincide exactly with the OSE experiments which may give users a false sense of accomplishment and immediate transferability. 
However, we try to provide a framework where one could thoroughly experiment with the learning problem on OSSE configurations which can facilitate transfer learning to other domain-specific tasks.
We also anticipate that new \textit{real} SWOT data~\cite{SWOT} will start to become more available which will allow us to design better, realistic OSE experiments.

\textbf{Noise Characterization}.
Real data has noise to content with and we do not account for that within the SSH interpolation experiments.
The true noise we see in operational settings is structured and this would require more knowledge outside the scope of our teams expertise.
A more improved challenge would take these considerations into account.
We leave this as a future challenge for the community and we hope our platform can help facilitate this.

\textbf{Uncertainty Quantitification}.
We prefaced the problem statement with the idea of data assimilation which is the notion of \textit{state/parameter estimation under uncertain conditions and incomplete information}~\citep{DAGEOSCIENCE}.
However, we have not addressed any notion of uncertainty at all throughout the paper.
Uncertainty is difficult to quantify and we don't want to impose too many restrictions until we more sure about the efficacy of ML for easier problems.
However, to move the problem setting towards a more realistic setting, we can start to introduce metrics and additional requirements from future challenges, e.g. mean and standard deviation estimates or ensemble predictions.


\textbf{Operational Constraints}.
The real use case of SSH interpolation will involve global data and/or high-resolution data. 
This involves dealing with very high-dimensional spatiotemporal global state-space.
In practice, the necessity for the scalability of the method is of paramount importance.
However, there are also areas within the ML research community who are looking into many ways we can scale up physical models~\citep{VEROS,OCEANANIGANS} and machine learn models for geoscience tasks~\citep{SFNO}.
We anticipate that once a set of solutions are excepted by a community, the scalability will come later.




% \newpage
% \subsection{Dataset Limitations}



% \textbf{Limitations}. The scope of this is very specific to researchers who are interested in ML. It only serves as a baseline design. If one would like to scale, it would require much more engineering work to get everything to connect. This is already seen if we are to load datasets of PBs. In addition, this requires researchers to have access to considerably large machines to be able to run their own preprocessing schemes. We do our best to provide toy datasets of a modest size however, inevitably, one will probably need to work with larger and large datasets.

% \subsection*{Broader Impact}
% \label{sec:impact}

% The theme of interpolation is present in many applied communities with different names, e.g. kriging in ecology/hydrology, Optimal interpolation in oceanography, and Gaussian processes in statistics. We hope that this work bridges this gap between the communities and we invite other works to try to highlight concrete ways that machine learning and classic physics have commonalities.

% In the oceanography community in particular, we especially hope to see more adoption of machine learning methods for interpolation. DUACS is ultimately a closed-system so the wider scientific community does not have access to the algorithm. Our OI baseline hopefully unveils some of the finer details of the method. However, in general, the standard OI methods used in the applied community cannot keep up with the massive influxes of observations we receive. So this work is a first step in demonstrating that neural networks (in particular NerFs) are a viable, simpler, and scalable alternative.

\putbib[./00_Oceanbench/content/bibliographies/full.bib]
\end{bibunit}

% \clearemptydoublepage
% \begin{bibunit}[IEEEtran.bst]

\clearemptydoublepage
\chapter{Problem Formulation and Ontology of Approaches}
\label{chap:1}


\section{The Thermometer Calibration Example}

Journey from an observation problem to a learning problem.
In order to contextualize the work in this thesis, I aim to introduce and explicit the different sources of decision and assumption making when solving an observation problem. This will naturally introduce the concepts used when working with learning-based approaches. Furthermore it will provide a useful perspective for interpreting the research by pointing where learning approaches make different assumptions compared to more traditional approaches.
In order to elucidate the proposed ontology, let's delve into the process of calibrating a thermometer. This example was selected due to its relevance to our study and because it is easier to reason about the underlying physical processes and quantities than in the case of satellite altimetry. As a starting point we can state the problem of calibrating a thermometer as: given an ungraduated thermometer, how can we interpret the level of the liquid as a temperature?

\subsection{Calibrating a thermometer}

The overarching goal of this calibration is to be able to know the temperature at some place by putting the thermometer in a specific location, observe the height of the liquid, and infer the temperature at the point of measurement.

The first step involves accumulating theories and assumptions to construct a model linking the observed level and the actual temperature. For instance, based on our knowledge of fluid dilation in response to temperature, assuming the diameter of the tube is constant with height, we can posit that the level is linearly correlated with the temperature. This model introduces two parameters: the slope and offset of our linear model that need to be ascertained.

The second step involves determining these parameters. For this purpose, we traditionally immerse the thermometer in icing and boiling water to acquire the levels corresponding to 0°C and 100°C. A simple linear system can then be used to solve for the parameters.
This second step rely on data consisting of input-output pairs of the function we're looking for. 

A few notes on this example:
\begin{itemize}
\item The calibration procedure consisted in two steps that respectively rely on conceptual knowledge and data;
\item The two steps are coupled
\item If we loosen the assumption about the tube's constant diameter, we need to incorporate a parameterization of the tube diameter into the model, adding more parameters and necessitating additional data for calibration;
\item If we have a well calibrated thermometer that provides us as many data as we want, we could make very little assumptions and just mark each graduation using data from the calibrated thermometer.
\item By adding the knowledge that the thermometer is in boiling water, our mapping is reduced to a constant function returning 100°C by convention.
  \end{itemize}
% use a report the data points in the graduationswe could limit our assumptions to stationarity (a level that corresponded to a temperature in the past implies the same level corresponds to the same temperature) and smoothness (levels that are close to each other correspond to temperatures that are close to each other), and our temperature estimate could simply be the nearest observation from the data points (which would be a model without parameters);

We now have graduations on our thermometer and can use the level as a proxy for the temperature without further thought!... Although how do we know if our calibrated thermometer is good ? 


\subsection{Evaluation}

Let's define "evaluation" as quantifying quality through metrics.

In our case the most intuitive metric for characterizing our thermometer's quality would be the accuracy of the temperature it gives. However this is by no means obvious, some situation may put greater importance on the speed of the thermometer or the range at which it's functional. 
  Furthermore, in order to properly evaluate our calibrated instrument, we need to test it in conditions corresponding to its intended use, (indeed for a domestic thermometer, testing it it on Mars or 5000 meter underwater would not provide helpful information).

 In order to clarify its intended use, we need to explicit some silent assumptions made on what we would consider a good thermometer.
  For example that it needs to "be accurate to the half of degree", "have response time under 10 minutes", "work between -30°C and 200°C" "work at a reasonnable athmospheric pressure" etc...

Then, using a trustworthy reference like a third-party well-calibrated thermometer, we could compare the measurements of the reference with the one given by our solution.
  An example evaluation procedure could be to confront the measurements of the two instruments at different temperatures such as: in a freezer, in a fridge, at ambiant room temperature and in an oven.

Some remarks about the evaluation:
\begin{itemize}
\item Data is needed for computing metrics
\item The evaluation depend on the metric chosen and the way we compute it.
\item Defining relevant metrics requires intimate knowledge of the intended goal of the instrument.
\item Different metric can produce different rankings, therefore the evaluation is relative to the metric's choice
\item Both the single accuracy in the oven and the mean or standard deviations of the different measured accuracies can be considered as metrics
\item An evaluation can use multiple metrics, therefore no ranking between two methods is guaranteed
\item If the metrics' choice and computation are not suited to the intended use of the instrument, the evaluation will be flawed.
\item If the reference thermometer is biased (not well calibrated) a good metric will not define a thermometer of quality
\item The criterium on the range of the thermometer implies that the thermometer needs to give a temperature even for levels it were not calibrated on (since only finite number of observations were used to calibrate it).
\item If the evaluation only measured the accuracy at the observed temperature 0°C and 100°C, any procedures that fit the observations would get the highest metric even if all other graduations were non-sense.
\end{itemize}

 \subsection{Sources of errors}

Given an evaluation procedure, the errors are the gap to the reference and can be attributed to three sources.
 The model is a source of error if the assumptions made were inacurate. For example if the diameter of the tube is not constant with height the linear correlation between level and temperature is not fallacious and will induce errors when interpreting the level.
 Even with perfect assumptions, the noisy data can introduce errors in the calibration. If we interpreted our 0°C and 100°C in icing and boiling water at the top of a montain with lower athmospheric pressure, we will have calibrated our parameters with erroneous measurements and the subsequent graduation of our thermometer will be inaccurate.
 Finally even with perfect assumption and perfect data, the procedure used to find the solution's parameters can be a source of errors if it fails to find the optimal parameters. For example if we solve for the parameters using a gradient descent method, using a step size too big will prevent finding the exact parameters which will also generate errors in the subsequent measurements.


Once we have an evaluation of our instrument, the next natural step is to aim at reducing the errors. Doing this we're not just thinking about the level-temperature relationship but about the whole method used to determine this relationship. We need to incroporate in our reasoning the how the calibration method (assumptions, data, optimization) relates to the errors we measure. We can state this as a higher level problem: "How to find the best thermometer calibration procedure?"

\subsection{Finding the best thermometer calibration procedure}
In this section, we propose a parallel between the two problems of "knowing the temperature when looking at the thermometer" and  "knowing the calibration procedure". we respectively call first order and second order what refers to the former and latter.

  The second order objective is to find a function that takes in a set of level-temperature observations, and returns the level to temperature mapping (the first order solution)

In order to find such a procedure, we can apply the same two steps as before.

  First, how to model the link from the input observations and the output mapping. This would include (first order) assumptions about the relationship between level and temperature as well as assumptions about the type of errors in the observations and the optimization procedure. These different assumptions are compiled into a set of candidate calibration procedures characterized by some configuration (the second order parameters).

  The second step consists then in determining the optimal configuration among the different candidates.
  When considering the first order problem, the data used to determine the parameter were levels with corresponding temperature. These data are  the inputs and outputs of the function we are trying to determine.
  If we reason about the inputs and outputs of our calibration procedure, the inputs consists in a set of level-temperature pairs  and the output are calibrated thermometer.
  Indeed 
  
  In order to do so, we need data points as well as an evaluation procedure.


A few notes on this example:
\begin{itemize}
\item The same two steps applied for the first and second order problem: defining candidates using theory and finding the optimal solution using data
\item A second order solution is a function that takes as inputs first order data and outputs first order solution
\item Second order parameters can be discrete choices like different first order assumptions 'considering the diameter is constant or not'
\item Second order parameters can be discrete choices between two different optimization procedure
\item Second order parameters can be constants in the level-temperature mappings
\item Second order parameters can be parameters of an optimization procedure like step size
\item The first order assumptions are about the physics of the system 
\item The second order assumptions includes considerations about the methods through data (errors) and optimization
\end{itemize}

We now have a great calibration procedure!... Or do we?  Wait what is a "great" calibration procedure?

\subsection{Evaluation of a calibration procedure}

A first question we need to ask to evaluate the procedure in itsefl is: what would be the desired domain (i.e. range) of a calibration procedure ? Or put differentely, on what calibration problems do we intend to use it.
We therefore need to explicit what kind of thermometer we want the calibration procedure to be applied to, and what observations we expect to have.
In our example, I implicitely wanted the calibration procedure to work on any standard thermometer, having at least two level-temperature observations without noise.

Now we need to determine a metric and an evaluation procedure.

Having determined a satisfying metric for evaluating a thermometer, a natural metric for the procedure is to compute the calibration metric on results of the procedure
The evaluation procedure can then be to use a few different thermometers a verify that the caliration works well on them.


A few remarks here:
\begin{itemize}
\item other metrics can relate to second order aspects of the problem, for example the robustness to noise or the computational complexity.
\item Different calibration tasks need not be on multiple thermometers, but could the same thermometer with different observation data points
\item if we tuned our procedure and computed our metric on a single thermometer, the metric would not informed us if the calibration worked on different thermometer.
\end{itemize}


Using this evaluation procedure, we could quantify the second order errors that could be classified in the same categories.
The model errors now includes the repercussions of wrong assumptions and parameterizations of errors in the data and optimization dynamics.
Data errors would for example be due an inaccurate reference thermometer used tune the second order parameters.
And optimization errors denote the gap between the solution found compared to the best possible candidate when evaluated on second step's data.


\subsection{Introducing space}
  As mentioned in the notes ultimate objective (zero order) of the calibration is to know the temperature at a single place and time given by the location of the thermometer. 
  This is a particular case  -- related to the calibration task  -- of the classes of problem we're interested in.
  The more generic class of problems would be knowing the temperature field over a spatio-temporal domain.

We could then update the associated hierarchy of problems as:
  - Zero order: Know the temperature given a location and a time within a spatio-temporal frame, 
  - First order: Find the field of temperature given observations of thermometers (potentially at different places and times)
  - Second order: Determine a procedure that can map a set of observation to the temperature field

And the conceptual blocks introduce above apply in a similar manner.


  As mentioned in the notes ultimate objective (zero order) of the calibration is to know the temperature at a single place and time given by the location of the thermometer. 
  This is a particular case  -- related to the calibration task  -- of the generic classes of problem we're interested in.
  The more associated generic class of problems would be knowing the temperature field over a spatio-temporal domain.

We could then update the associated hierarchy of problems as:
  - Zero order: Know the temperature given a location and a time within a spatio-temporal frame, 
  - First order: Find the field of temperature given observations of thermometer levels (potentially at different places and times)
  - Second order: Determine a procedure that can map a set of observation to the temperature field

And the conceptual blocks introduce above apply in a similar manner.


\subsection{In a nutshell}

A necessary step before developing a solution to a problem is to be able to evaluate it, this requires informed knowledge and assumptions on the intended use of the solution as well as a choice of metric computation and the associated data.

Then to actually develop the solution, three components are needed: a model of the relationship between input and output of the solution as well algorithm to chose the model configuration given some data.


To solve the calibration problem we need to come up with a calibration procedure that will take level-temperature observations to produce the calibrated thermometer.

To evaluate the calibration procedure we evaluate the calibrated thermometers over a range of representative tasks.
To evaluate a calibrated thermometer, we compare the temperatures given by the thermometer over a range of representative situations.

To develop the calibration procedure we need three components:
The data: an ensemble of calibration task that we can evaluate (of calibrated thermometers)
The algorithm that choses the best calibration procedures 
The model that is the set of candidates calibration procedures that differ by one or more component:
  The model which is the set of candidate of level-temperature mapping
  The algorithm that choses the best mapping given some level-temperature observations







Solving an observation problem can be structured in two levels:
- Finding the best estimate of the temperature when looking at a thermometer
- Finding the best way to convert observations 
- Linking the observation to the quantity of interest

\subsection{Domain experts versus deep learning approaches}
  We group what we describe above in the domain experts category of methods, with a tendency 




\section{Generic formulation}

  The zero order objective of the class of problem we're interested in is to find a function $f_0=\hat{u}$  that approximates a quantity of interest $u(t, x)$ on a domain $\Gamma_u$ with values in $\mathbb{R}^d_u$. 
  In geoscience we can generally consider $\Gamma$ to be spatio-temporal, and define the spatial and temporal domains  as {\Omega = {x, (t,x) \in \Gamma} and {\Tau = {t, (t,x) \in \Gamma}

   $u$ is defined 
  conditioned on a set of observations $\cal{D}_0$.

  The first order objective is therefore to formulates $f_1$ that outputs $f_0$ from  a set of observations $\cal{D}_0$
  with $\cal{D}_0 = \{(p_1, y(p_1)), ..., (p_{N_0}, y(p_{N_0})) \}$ with  $p_i \in \Gamma_y$ and $y: \Gamma_y \to \mathbb{R}^d_y$.

  The second order objective is therefore to formulates $f_2$ that outputs $f_1$ from a dataset $\cal{D}_1$ .
  with $\cal{D}_1 = \{(cal{D}_0^1, Y^1),... (cal{D}_0^{N_1}, Y^2) \}$ with  $cal{D}_0^{i}$ different sets of observations and $Y^i$ additional data that can be used to evaluate first order solutions.

  We note $\cal{F}_1$  the set of candidates $f_1$ and $\theta_1 \in \mathbb{R}^p_1$ the parameters of $f_1$
  We note $\cal{F}_2$  the set of candidates $f_2$ and $\theta_2 \in \mathbb{R}^p_2$ the parameters of $f_2$



% We could then state that a second order metric is can be composed of first order metrics.
%
% Some 
%
% The domain 
%
% We can then 
%
% We can use analog steps as for the evaluation of the calibrated thermometer to reason about the assessment of the quality of the calibration procedure.
%
% 1) What assumptions did we make about what is a good calibration procedure ?
% 2) What metric would be a good way a quantifying them ?
% 3) How can we compute this metric in a representative way ?
%
%
% A first idea to assess if our calibration procedure is good, would be to evaluate our thermometer once calibrated with it.
% However we would then be faced with a dilemna, 
% The issue encountered is that if we use the same evaluation procedure we 
% The most straightforward idea for a good k
%
%  Indeed the best possible calibration procedure will certainly produce a weakly calibrated thermometer if the input observations associate random levels with random temperatures.
%
%
%
% Those are the three sources of error from our calibration procedure, however note that in the absence of perfect reference we cannot directly quantify them.
%  Then as is often the case, some proxy is used as another source of error is even trickier and comes from the gap betkno In the absence of such reference we could 
% A reference 
% We would like to clarify that the evaluation procedure needs to be done on data that were not used 
%  Evaluating a second order solution on a data point uis
% Evaluating a second order solution only makes sense if the evalua
% We first want to point out the importance of the scope of evaluation, the second order problem only makes sense
%
% \subsection{Introducing time}
% The response time of the thermometer is the duration before the temperature indicated by the level reflects the temperature of the location it's in. This is related to the fact that the level is actually related to the temperature of the liquid which will take some time to adjusts to the location temperature.
% One could aim at using the thermometer to estimate the instantaneous temperature.
% However, in order to do so multiple recent observations at the same location  would be needed to take into account the dynamics.
% The mapping between recent observations and instantaneous temperature would include a parameterization of the diffusion process.
% The evaluation should also be consciensiously chosen to measure the dynamical aspects.
%
% Note that this could be treated as an end to calibration problem if we consider the levels as the thermometer as inputs, or as a separate problemif we consider the thermometer already graduated. This would impact the different hypothesis made.
%
%
%
%
% %###################
%
%
% \begin{itemize}
%   \item The optimization procedure also relies on some assumptions and parameters
%   \item We can then think of the tuning of hyper-parameters as an a higher order problem
%   \item The solution of this problem takes in a set of data points and produces a mapping between levels and temperatures.
%   \item Such solution could be applied to a range of different thermometer given that the set of available data points meets the requirements
%   \item The accuracy of a calibrated thermometer would depend on the validity of the assumptions made
% \end{itemize}
%
% \section{Problem Formulation Bis}
% \label{sec:chap1_problem_formbis}
%
%   The end objective (zero order) of the class of problem we're interested in is to find a function $f_0=\hat{u}$  that approximates a quantity of interest $u(t, x)$ conditioned on a set of observations $\cal{D}_0$.
%    $u$ is defined on a domain $\Gamma_u$ with values in $\mathbb{R}^d_u$.
%   In geoscience we can generally consider $\Gamma$ to be spatio-temporal, and define the spatial and temporal domains  as {\Omega = {x, (t,x) \in \Gamma} and {\Tau = {t, (t,x) \in \Gamma}
%   A set of observations is in the form $\cal{D}_0 = \{(p_0, y(p_0)), ..., (p_{N_y}, y(p_{N_y})) \}$ with  $p_i \in \Gamma_y$ and $y(p_i) \in \mathbb{R}^d_y$.
%
%
%   Therefore, $f_0 \in \cal{F}_0 = { h \to \alpha h + \beta , (\alpha, \beta) \in |R^2 }$.
%
%   More generally, $f_0 \in \cal{F}_0 = \{ f: y; z \to f(y; z), z \in |R^{N_0} \}$ with f the modelisation of the first order problem parameterizes by $z$.
%
%
%   Solving for $f_0$ consists in two steps: 
%   Defining $\cal{F}_0$  the set of candidate $f_0$ and $\theta_0 \in \mathbb{R}^p_0$ the parameters of $f_0$
%   Defining the procedure $f_1$ that outputs $f_0$ from $\cal{D}_0$
%
%
%   The first order objective is therefore to formulates $f_1$ that outputs $f_0$ from $\cal{D}_0$
%   We note $\cal{F}_1$  the set of candidate $f_1$ and $\theta_1 \in \mathbb{R}^p_1$ the parameters of $f_1$
%
%   We can similarly introduce a second order objective which is to find $f_2$ that outputs $f_1$ from  $\cal{D}_1$ which groups multiple observation sets 
%   We note $\cal{F}_1$  the set of candidate $f_1$ and $\theta_1 \in \mathbb{R}^p_1$ the parameters of $f_1$
%
% All along this section we will illustrate our ontology of problems and methods using an example of thermometer calibration. We chose this use case due to its relevance to our work given that it's about sensor calibration and because the underlying physical processes are much simpler that the ones involved in the earth system wich makes for a comprehensive and useful usecase.
%
% In the case of our thermometer the quantity of interest is the temperature  $T$ which is a scalar field and $\Omega_u$ is the thermometer location  $x_t$ at time $t_t$. 
%   We want to estimate it given a single observation of the level of the thermometer $h_t$ at the same point: $\cal{D}_0 = \{((x_t, t_t), h_t$.
%
%   In the case of our thermometer, $f_1: \{(t_t, x_t, h_t)\} -> \{f_0(t_t, x_t)\}$ can be reduced  to finding the relationship between the level $h_t$ and the temperature $T_t$ of the thermometer.
%
%
%   In order to solve the first order problem, the first step involves compiling our theoretical knowledge and making assumption to come up with a model. For example, given our understanding of fluid dilation in response to temperature, under the assumption that the diameter of the tube is constant with height, can assume that level is linearly correlated with the temperature. Therefore, $f_1 \in \cal{F}_1 = { h \to \alpha h + \beta , (\alpha, \beta)=\theta_1 \in |R^2 }$.
%     A second necessary step is to determine the values of $\alpha$ and $\beta$. To this end we require a set of data points $\cal{D}_1 = \{\cal{D}_0^1=(h^1, T^1), ..., \cal{D}_0^{N_1}=(h^{N_1}, T^{N_1})\}$ to calibrate our model, traditionally obtained by placing the thermometer in icy and boiling water to get the levels corresponding to 0°C and 100°C. We can then solve for the parameters with a simple linear system.
%
%
%
%
%   The second order objective is then to formulates $f_2$ that outputs $f_1$ from  $\cal{D}_1$
%
%
%   We can now formulate the second order problem associated with 
%   The 
%    To solve this 
%   The end objective of the class of problem we're interested in is to find a function $f_0$  that approximates a quantity of interest $u(t, x)$ on a domain $\Gamma_u$ with values in $\mathbb{R}^d_u$.
%   In order to estimate find $f_0$, we have observation data $\cal{D}_0 = \{(p_0, y(p_0)), ..., (p_{N_y}, y(p_{N_y})) \}$ with  $p_i \in \Gamma_y$ and $y(p_i) \in \mathbb{R}^d_y$.
%   In geoscience we can generally consider $\Gamma$ to be spatio-temporal, and define the spatial and temporal domains  as {\Omega = {x, (t,x) \in \Gamma} and {\Tau = {t, (t,x) \in \Gamma}
%
%
% \section{Problem Formulation}
% \label{sec:chap1_problem_form}
%
%
%
% The broadest formulation the problems we're interested in is finding a function $f$ which maps the available inputs $y$ to the desired outputs $u$.
%
%
%
%
%
% All along this section we will illustrate our ontology of problems and methods using an example of thermometer calibration. We chose this use case due to its relevance to our work given that it's about sensor calibration and because the underlying physical processes are much simpler that the ones involved in the earth system.
%
% This formulation encompasses different order of problem that we detail below.
%
% \subsection{First order: Calibrating the thermometer}
%   \label{ssec:firstorder}
% The first order of problem would be to find the mapping $f_0$ between a given level $h$ of the thermometer and the associated temperature $T$ of the liquid within.
%
% A first step would involve compiling our theoretical knowledge and making assumption to come up with a model. For example, given our understanding of fluid dilation in response to temperature, under the assumption that the diameter of the tube is constant with height, if we don't know the volume of liquid and the dilation rate we can model an affine  that the level is linearly correlated with the temperature. Therefore, $f_0 \in \cal{F}_0 = { h \to \alpha h + \beta , (\alpha, \beta) \in |R^2 }$.
%
%   More generally, $f_0 \in \cal{F}_0 = \{ f: y; z \to f(y; z), z \in |R^{N_0} \}$ with f the modelisation of the first order problem parameterizes by $z$.
%
%   A second necessary step is to determine the values of $\alpha$ and $\beta$. To this end we require a set of data points $\cal{D}_0 = \{(h^1, T^1), ..., (h^{d_0}, T^{d_0})\}$ to calibrate our model, traditionally obtained by placing the thermometer in icy and boiling water to get the levels corresponding to 0°C and 100°C. We can then solve for the parameters with a simple linear system.
%
%
%   The first and second step are coupled. Indeed, the more parameters $f_0$ depends on, the more data points will be needed in $\cal{D}_0$
%   For example if some assumptions are considered too restrictive like "the diameter of the thermometer is constant with height", they can be relaxed.
%   However this expands the class of functions $F_0$, by requiring the incorporation of a model of the diameter in function of height, which will introduce new parameters. Let's assume that the diameter is linear per part for every $K$mm section, and the corresponding parameters to find would be the $N_k$ value of the diameter every $K$mm: $(d_1, ... d_{N_k}) \in |R^{N_k}$ 
%
%   In this new formulation, note that chosing $d_i$ values determi
%   In order to find adequate $d_i$ values, more data points would be needed in $\cal{D}_0$
%
% \subsection{Second order: Finding a thermometer calibration procedure}
%   The resolution of a first order problem can be stated as the following: How to find the mapping $f_1$ between a set of data points $\cal{D}_0$ and $f_0$.
%   We refer to this formulation as a second order problem.
%
%   The second order solution $f_1$, is a procedure that outputs the best first order solution given a set of data points $\cal{D}_0$
%   This procedure is an algorithm chosen in regard of some assumptions on the level of noise in the input $\cal{D}_0$ and the landscape of the search space $F_0$. These assumptions define the classes of functions $\cal{F}_1 = \{f: \cal{D}_0; \theta \to f_0, \theta \in |R^{N_1}\}$ conditionned on $\theta$ which are the constant parameters across first order instances.
%   In the above example $f_1$ could be the resolution of a simple linear system without any parameters, but could also be a gradient descent procedure relying on a definition of an objective function and with parameters such as the step size. 
%   More generally, when focusing on iterative optimization procedure such as gradient descent based algorithms, the parameters \theta can be characterize initialization schemes, the step computation or regularization parameters that reduce the search space $F_0$.
%
%
%   % Note that regularization can be loosely defined as assumptions used to reduce the search space $\cal{F}_0$ across all tasks such that $f_0 \in \cal{F}_0 = \{ f: y; z \to f(y; z), z \in \cal{Z}(\theta) \subset |R^{N_0} \}$.
%   % An example of regularization assumption could be a smoothness a priori on the diameter of the thermometer conditionned on a parameter $\lambda$ such that the parameters $(d_1, ... d_{N_k}) \in |R^{N_k}$ are constrained by  $\forall i \in [|1, N_k-1|], d_i - d_{i+1} < \lambda$ effectively reducing the search space.
%
%   The parameters $\theta$ are then tuned using a dataset $\cal{D}_1 = \{\cal{D}_0^1, ..., \cal{D}_0^{d_1}\}$ comprised of multiple instances of first order problems.
%
%  For example one can use the different examples in  $\cal{D}_1$ 
%
% %   We when modeling The function $f_1$ also depends on parameters that are 
% %   Additionally, for second order problems, the solution $f_1$ performs the search for the parameters of $f_0$ i.e. an optimization procedure. This procedure also depends on assumptions such as the level of noise in the input $\cal{D}_0$, and the landscape of the search space $F_0$. This will define parameters to find for $f_1$. Such parameters are tuned using a dataset $\cal{D}_1 = \{\cal{D}_0^1, ..., \cal{D}_0^{d_1}\}$ comprised of multiple first order problems.
% %
% %
% % Finally some parameters of the modelisation of first order problem can be unknown but common across problems and therefore are parameters of 
%
%
% \subsection{Second order ++: Finding a generic procedure to solve a class of problem}
%
% In the above second order problem, note that the class of function $F_0$ considered needs now to apply to every thermometers we want to calibrate using $f_1$.
%
%   This can motivate the elaboration of more generic modelisation $f_0$ with higher dimensional $cal{\F}_0$.
%   However this requires more data.
%   The two sources of data $\cal{D}_0$ and $\cal{D}_1$ are complimentary to this end.
%   More data in $\cal{D}_1$ allow for more tuning more parameters generic across and therefore reducing the search space of $cal{\F}_0$ during the second order resolution.
%   whereas a larger $\cal{D}_0$ can be used to search higher dimensional $cal{\F}_0$ during the first order resolution.
%
%   In the thermometer example above assuming the determination of the diameter every $K$mm requires too much datapoints compare to what we expect to have in $\cal{D}_0$, but that we have a many examples of thermometers in $\cal{D}_1$.
%   We can learn the distribution 
%   We could then introduce regularization parameters to be fitted across different first order instances and that would couple the $d_i$ together such as reducing the search space of $F_0$.
%   For example coupling the 
%   degrees of freedom that can be solved for each $\cal{D}_0$
%
%
% In the previous example, we saw an example of broadening the space of possible first order solution by relaxing some assumptions and then searching this larger space relying on more data.
% Going even further we could aim at finding a generic approach for finding calibration operator of differnt kind of sensors or even generic 
%
%
%
%
%   The model of the thermometer made for the first order problem needs to be questioned and 
%   When solving a second order problem, the assumptions made to 
%   $f_1$ then incorporate the search for $f_0$'s parameters.
%
%
%
% Given a non graduated thermometer we want to find 
%
%
% In geoscience, $u$ and $y$ can generally be defined as $D_u$ and $D_y$ dimensional vector fields defined on spatio-temporal domains $\Omega_u$ and $\Omega_y$. Here, we consider scalar fields and discrete fields as particular cases of this generic formulation. The different problems can be characterized by the types of quantities $u$ and $y$ and the spatio-temporal domains $\Omega_u$ and $\Omega_y$ on which they are defined. Figure \ref{fig:planet_drawings} illustrates how our two use cases of sensor calibration and SSH mapping easily fit into such formulation. Additionally, Figure \ref{fig:task_ontology} shows how tasks such as calibration, mapping, and forecasting can differ through the domain of definition of $u$ and $y$.
%
% \begin{figure}[htbp]
% \begin{center}
% \includegraphics[width=0.8\linewidth]{Chapitre1/Ch1-Figures/Cal_drawing.png} 
% \includegraphics[width=0.8\linewidth]{Chapitre1/Ch1-Figures/Mapping_drawing.png} 
% \end{center}
% \caption[Swot calibration and altimetry mapping problem illustration]
% {\footnotesize The calibration problem (top row) consists in finding the mapping $f$ that estimates the observed SSH $u$ from the SWOT satellite given the actual noisy measurement and ancillary calibrated measures ($y$).
% The mapping task (bottom row) consist in finding an operator $f$ that maps partial measurements of the SSH $y$ to a map of SSH $u$}
% \label{fig:planet_drawings}
% \end{figure}
%
%
% % \begin{figure}[htbp]
% % \begin{center}
% % \begin{tabular}[c]
% % \includegraphics[width=0.8\linewidth]{Chapitre1/Ch1-Figures/Cal_drawing.png} \\
% % \includegraphics[width=0.8\linewidth]{Chapitre1/Ch1-Figures/Mapping_drawing.png} \\
% % \end{tabular}
% % \end{center}
% % \caption[Swot calibration and altimetry mapping problem illustration]
% % {\footnotesize The calibration problem (top row) consists in finding the mapping $f$ that estimates the observed SSH $u$ from the SWOT satellite given the actual noisy measurement and ancillary calibrated measures ($y$).
% % The mapping task (bottom row) consist in finding an operator $f$ that maps partial measurements of the SSH $y$ to a map of SSH $u$}
% % \label{fig:planet_drawings}
% % \end{figure}
%
% \begin{figure}[htbp]
% \begin{center}
% \includegraphics[width=0.8\linewidth]{Chapitre1/Ch1-Figures/Task_ontology.png}
% \end{center}
% \caption[Task characterization through the domains $\Omega_u$ and $\Omega_y$ of $u$ and $y$]
% {\footnotesize Using the perspective provided by the problem definition, we can easily categorize earth observation problems.
% The calibration consist of estimating the field $u$ on a subset of the observation domain, the mapping consist in estimating $u$ on the same temporal domain but extending the spatial domain.
% And finally forecast can considered as wanting to estimate a quantity on an unobserved future domain.}
% \label{fig:task_ontology}
% \end{figure}
%
% \section{Method Ontology}
% We aim to characterize and organize the various methods used to tackle the class of problem introduced in \ref{sec:chap1_problem_form}. We propose that all methods can be decomposed into the following two steps:
% \begin{itemize}
% \item Step 1: Define the set $\cal{F}$ of possible $f$ using theoretical knowledge and making assumptions about the problem (conceptual models)
% \item Step 2: Search $\cal{F}$ for an optimal $f$ using factual knowledge (data) and making assumptions about the data
% % \item The first and second steps respectively rely on conceptual knowledge and available data;
% % \item If the data had included the volume of the liquid, the tube's diameter, and the dilation rate of the liquid, we might have needed theoretical knowledge about the volume of a tube to construct a relationship between the level and the temperature;
% % \item If only the diameter were unknown in a similar situation, a single data point would have been sufficient to calibrate the model;
% \end{itemize}
%
%   To illustrate our point let's consider the simple problem of thermometer calibration.
%
%
%
%
%
% To illustrate our point, consider a simple example of building a thermometer by placing a liquid in a tube and wanting to interpret the level of the liquid as a temperature. According to our previous notations, $y$ is the level of the liquid and $u$ is the temperature of the liquid inside, and we seek to find the mapping $f$ between the two.
%
% Step 1 involves compiling our theoretical knowledge on the problem to define the class of function. Given our understanding of fluid dilation in response to temperature, under the assumption that the diameter of the tube is constant with height, we can state that the level is linearly correlated with the temperature. Therefore, $f$ will be part of $\cal{F} = { y: \alpha y + \beta , (\alpha, \beta) \in |R^2 }$.
%
% In Step 2, to find $\alpha$ and $\beta$, we require two data points to calibrate our model, traditionally obtained by placing the thermometer in icy and boiling water at 1 bar of pressure to get the levels corresponding to 0°C and 100°C. This method relies on strong theoretical foundations and assumptions to reduce the dimensionality of the search space $\cal{F}$, thus facilitating the parameter search with relatively few data points.
%
% However, if we clearly see that the diameter of our thermometer is not constant, the model needs to incorporate that the evolution of the temperature depends on the diameter at each height. This expands the class of functions, necessitating the incorporation of a model of the evolution of the diameter in function of the height, which will introduce new parameters. We could assume that the diameter is linear for every 5mm section, and the corresponding parameters to search would be the value of the diameter every 5mm.
%
% To estimate these new parameters, we need more data which could be direct measures of the diameter or measures of temperature every 5mm. We could directly model $f$ as linear per part, thereby reducing the number of parameters to estimate (no more $\alpha$ and $\beta$). If we have measurements of the temperature, this also alleviates the need to explicitly model the relationship between diameter and temperature.
%
%
%
%
%
%
% % Tasks
% % Simple exemple
% % Calibration and mapping example
%
% % Tasks of interests can be summed up as finding f
% % Finding f takes two steps: defining the set of possible fs, searching the set for the best f
% % Formulating the sets of F requires theoritical knowledge
% % Searching the sets of F requires data
%
% % From theory to sets of 
% %   inverse problems state x
% %   data assimilation: dynamical model
% %   spatio temporal correlation: Covariance model
% %   deep learning
% %   locality: convolution
% %   temporal dependence RNN LSTM
%
%
% Lorem ipsum dolor sit amet, consectetuer adipiscing elit. Maecenas fermentum, elit non lobortis cursus, orci velit suscipit est, id mollis turpis mi eget orci.
%
% \section{Première section du chapitre}
%
% Lorem ipsum dolor sit amet, consectetuer adipiscing elit. Maecenas fermentum, elit non lobortis cursus, orci velit suscipit est, id mollis turpis mi eget orci.
%
% \subsection{Première sous-section}
%
% Lorem ipsum dolor sit amet, consectetuer adipiscing elit. Maecenas fermentum, elit non lobortis cursus, orci velit suscipit est, id mollis turpis mi eget orci.
%
% Voir figure \ref{fig:mafigure2}.
%
%
% \begin{figure}[htbp]
%    \begin{center}
%       \includegraphics[width=0.8\linewidth]{Chapitre1/Ch1-Figures/comparison.png}
%    \end{center}
%    \caption[titre court pour la liste des figures]
%    {\footnotesize Titre plus long avec des explications.}
%    \label{fig:mafigure2}
% \end{figure}
%
% \subsection{Deuxième sous-section}
%
% The calibration procedure described above followed an intuitive flow of framing the problem using assumptions on the physics and solving it using data.
% The calibration procedure described above followed an intuitive flow of framing the problem using assumptions on the physics and solving it using data.
%
% The above section detailed how to find a function that maps an observation of the level of the thermometer to the temperature.
% We saw how this function is the result of an procedure that uses some assumptions about the problem and takes as input a set of data points.
%
% We propose here a parallel between the calibration problem as stated above and the optimization problem of infering optimal parameters from data-points.
%
% The first step 
%
% Lorem ipsum dolor sit amet, consectetuer adipiscing elit. Maecenas fermentum, elit non lobortis cursus, orci velit suscipit est, id mollis turpis mi eget orci.
%
% \section{Conclusion du premier chapitre}
%
% Lorem ipsum dolor sit amet, consectetuer adipiscing elit. Maecenas fermentum, elit non lobortis cursus, orci velit suscipit est, id mollis turpis mi eget orci.
%
% In this manuscript I'd like to cite \cite{remo3,remo4}.

\addcontentsline{toc}{section}{Bibliography}
\putbib[./Chapitre1/Ch1-Biblio.bib]
\end{bibunit}


% %Input your chapter 2
% \clearemptydoublepage
% \input{./Chapitre2/Chapitre2}

\clearemptydoublepage
\backmatter
% \begin{bibunit}[IEEEtran.bst]

\chapter*{Conclusions and perspectives}
\label{chap:conclusions}
\addcontentsline{toc}{chapter}{Conclusion and perspectives}
\chaptermark{Conclusion}
We review in this chapter the primary contributions outlined in this manuscript and the future avenues of research they open.

\section*{Contributions Summary}
\addcontentsline{toc}{section}{Contributions Summary}
This thesis is part of a broader movement towards developing deep learning methods to address observation challenges in ocean science. It emphasizes altimetry applications, especially in the context of the recent launch of the SWOT mission.

The first contribution highlights the successful application of deep learning for bias correction of simulated SWOT observation data. While standard deep learning architectures struggled to differentiate fine SSH signatures from high amplitude bias, we demonstrated that deep learning methods could be tailored to suit the unique characteristics of altimetry data. We employed SWOT mission's error specifications to craft a custom architecture focused on calibrating SWOT's correlated errors. This study is promising, yet the method developed was calibrated and assessed using simulated data, bringing up questions about its applicability to actual SWOT observations.

The second study delves into how learning-based altimetry methods, once calibrated on simulated data, can be applied to real data. We evaluated the 4dVarNet mapping schemes on real altimetry after calibration on simulated data. The findings indicate strong generalization capabilities even with coarse simulations, while more accurate simulations enhance the mapping performance.

The initial two studies shed light on the potential of applying learning-based approaches to ocean science's observational challenges. Yet, they also spotlight the complexities in melding expertise in observation, simulation data, deep learning techniques, and domain-specific evaluation methodologies. This spurred the creation of the specialized toolset, Oceanbench, aiming to narrow the gap between deep learning and ocean science experts. Oceanbench enables ocean scientists to flexibly design evaluation setups using data and metrics. These setups come with the essential tools for deep learning practitioners to access and prepare the data in view of training their models.

\newpage
\subsection{Limitations}

We have showcased the SSH interpolation edition as a data challenge which could be helpful for real applications. 
However, in section~\ref{sec:problem_scope} we alluded to the greater task of general ocean state estimation which is more pertinent to the ocean sciences yet we don't address this directly with our data challenges.
Furthermore, we claim that the data challenges presented will help the ocean community with using ML for SSH interpolation.
Below, we outline some limitations which address these criticisms.

\textbf{Not the overall objective}. 
We recognized that we are far away from the actual reanalysis and forecasting goals of full state estimation. 
However, we argue that that is a rather ambitious challenge which will require a lot of interdisciplinary work across communities. 
In the meantime while we work towards that goal, operational centers could possibly improve their current products from ML-based techniques would would benefit downstream applications that deal directly with SSH.
Furthermore, SSH is an important variable in describing the full ocean state.
So a robust set of techniques that are able to solve the interpolation tasks could (in principal) be used to solve extra tasks.

\textbf{Small Region \& Period}.
We only feature a small region and period over the Gulfstream which is not representative of the different global regimes. 
This also does not take into account real things like \textit{data drift} which will inevitable occur in operational settings.
However, this is a dynamical regime and a well-studied area which does have some importance for specific communities and the results obtained offer some transferability to other dynamical regimes.
In addition, this area will have good coverage due to the new SWOT mission~\cite{SWOT} which will allow for further validation in the future.
Lastly, the area is small enough where the beginning stages for ML researchers is not overwhelmed with problems involving scale (even though we eventually want to arrive at global schemes).
We hope to extend our challenges to more relevant scenarios~\cite{MDSALONGTRACK}.


\textbf{Noise Characterization}.
Real data has noise to content with and we do not account for that within the SSH interpolation experiments.
The true noise we see in operational settings is structured and this would require more knowledge outside the scope of our teams expertise.
A more improved challenge would take these considerations into account.
We leave this as a future challenge for the community and we hope our platform can help facilitate this.

\textbf{Uncertainty Quantitification}.
We prefaced the problem statement with the idea of data assimilation which is the notion of \textit{state/parameter estimation under uncertain conditions and incomplete information}~\citep{DAGEOSCIENCE}.
However, we have not addressed any notion of uncertainty at all throughout the paper.
Uncertainty is difficult to quantify and we don't want to impose too many restrictions until we more sure about the efficacy of ML for easier problems.
However, to move the problem setting towards a more realistic setting, we can start to introduce metrics and additional requirements from future challenges, e.g. mean and standard deviation estimates or ensemble predictions.


\textbf{Operational Constraints}.
The real use case of SSH interpolation will involve global data and/or high-resolution data. 
This involves dealing with very high-dimensional spatiotemporal global state-space.
In practice, the necessity for the scalability of the method is of paramount importance.
However, there are also areas within the ML research community who are looking into many ways we can scale up physical models~\citep{VEROS,OCEANANIGANS} and machine learn models for geoscience tasks~\citep{SFNO}.
We anticipate that once a set of solutions are excepted by a community, the scalability will come later.


\section*{Future Perspectives}
\addcontentsline{toc}{section}{Future Perspectives}

Several avenues can be explored to further extend the work presented in this thesis.

While deep learning methods offer promising results, it's understandable that some ocean scientists may remain cautious, even if these methods demonstrate superior performance to current operational approaches. Their concerns regarding the interpretability of deep learning models and their relative robustness compared to physically descriptive systems are valid points of discussion.
To address these concerns, I propose two potential paths forward. First, emphasizing the importance of quantifying the uncertainty associated with model estimations. Such uncertainty quantification (UQ) is crucial when addressing ill-posed inverse problems and can play a significant role in bolstering confidence in the results. Second, exploring the realm of physics-informed deep learning, which marries our physical understanding of the ocean with the adaptive nature of deep learning models. There are existing studies that have dabbled in approaches involving dynamical systems, which, while simpler than the ocean, can provide valuable insights.

This thesis predominantly centers on SSH, a surface field that is relatively well-observed in the realm of ocean quantities. Exploring other quantities, observed through different instrument, with different sampling or even not observed directly at all would introduce many more challenges requiring domain-informed problem specifications that deep learning could adress.

Furthermore, the research presented herein pertains to a particular region. Transitioning these findings to functional products would entail considerable scaling challenges. These encompass both scientific aspects, such as dealing with coastlines and varying ocean regimes, and engineering concerns like handling expansive datasets for the training and assessment of the models.



% We review in this chapter the main contributions that have been described in this manuscript as well as the future perspectives introduced through them.

% \section*{Contributions summary}
% \addcontentsline{toc}{section}{Contributions summary}
% Overall this thesis is part of a momentum in developing deep learning methods applied to observation challenges in geosciences.
% More specifically focusing on altimetry applications in the context of the recent lauch of the SWOT mission.

% The first contribution is the successfull of deep learning to bias corrections of simulated SWOT observation data.
% Although off-the-shelf deep learning architectures fail to separate fine SSH signatures from high amplitude bias, we've shown that deep learning methodology can be adapted to the specifities of altimetry data.
% We leveraged SWOT mission's error specifications to design a custom architecture tailored to the calibration of SWOT correlated errrors.
% This study is encouraging but the method developped has been calibrated and evaluated on simulated data, which raises the concerns of applicability to real swot observations.

% The second study shows how learning-based altimetry methods calibrated on simulated data can be applied to real data.
% This is done through the evaluation of 4dVarNet mapping schemes on OSE setups after calibration on simulated data. 
% The results show robust generalization with even coarse simulations while more realistic simulations improve the mapping performances.

% The two first studies opened appealing perspectives for applying learning-based methodology to observation challenges in geosciences.
% However they also highlighted the challenges in combining expertise  observation and simulation data, deep learning methods and domain-informed evaluation procedures.
% This motivated the implementation of an opinionated suite of tools Oceanbench to help bridge the distance between the deep learning and ocean science communities.
% Oceanbench provides a way for ocean scientist to flexibly design evaluation setups with data and metrics. These setups are defined along the necessary tools for deep learning practicioners to train and test their models.


% \section{Perspectives}
% There are a few ways in which the work presented here can extended in interesting ways.

% The element absent from this thesis is the uncertainty quantification which is critical in addressing ill-posed inverse problem. Coming back to the methodological framework introduced in the first chapters, incorporating uncertainty comes down to choosing a probabilistic representation of the estimated SSH.


% This thesis focused on SSH which is a surface field relatively well observed. Other quantities in particular with depth would add significant challenges.

% The evaluations were performed on a specific region, converting the results to operational products would introduce significant scaling challenges, Among which handling coasts and different dynamics.


% Finally, a recurring criticism point to the lack of physical interpretability of deep learning methods. 





% \addcontentsline{toc}{section}{Bibliography}
% \putbib[./Conclusion/End-Biblio.bib]
\end{bibunit}


% \clearemptydoublepage
\addcontentsline{toc}{chapter}{List of publications}

Scale-aware neural calibration for wide swath altimetry observations https://doi.org/10.48550/arXiv.2302.04497
Learning operational altimetry mapping from ocean models https://doi.org/10.5194/egusphere-egu23-8288
Joint Calibration and Mapping of Satellite Altimetry Data Using Trainable Variational Models https://doi.org/10.1109/ICASSP43922.2022.9746889
4DVarNet-SSH: end-to-end learning of variational interpolation schemes for nadir and wide-swath satellite altimetry https://doi.org/10.5194/gmd-16-2119-2023
% \chapter*{List of publications}
%
% \section*{International Journals}
%
% \section*{International Conferences}


% \begin{bibunit}[IEEEtran.bst]
% \renewcommand\bibname{List of publications}
% \nocite{*}
% \putbib[./PublicationList/PubList-Biblio.bib]
\end{bibunit}





\clearemptydoublepage
\begin{bibunit}[IEEEtran.bst]

\chapter*{Appendix}
\addcontentsline{toc}{chapter}{Appendix}

\section*{\textsc{OceanBench}: The Sea Surface Height Edition - Supplementary Material}
% \newpage
\section{Data Challenges} \label{sec:data_challenges_extended}

In this section, we highlight some details that were omitted in section~\ref{sec:data_challenges}.
This includes details about the simulation type, the data structures, and the training/evaluation periods.

\subsection{OSSE NADIR}

The reference simulation is the \textit{NATL60} simulation based on the NEMO model~\cite{NEMOAJAYI2020}. 
This particular simulation was run over an entire year without any tidal forcing.
The simulation provides the outputs of SSH, SST, sea surface salinity (SSS) and the u,v velocities every 1 hour.
For the purposes of this data challenge, the spatial domain is over the Gulfstream with a spatial domain of $[-65^\circ, -55^\circ]$ longitude and $[33^\circ, 43^\circ]$ latitude.
The resolution of the original simulation is 1/60$^\circ$ resolution with hourly snapshots, and we consider a daily downsampled trajectory at 1/20$^\circ$ for the data challenge which results in a 365x200x200 spatio-temporal grid.
This simulation resolves finescale dynamical processes ($\sim$15km) which makes it a good test bed for creating an OSSE environment for mapping.
The SSH observations include simulations of ocean satellite NADIR tracks.
In particular, they are simulations of Topex-Poseidon, Jason 1, Geosat Follow-On, and Envisat.
There is no observation error considered within the challenge.
We use a the entire period from 2012-10-10 until 2013-09-30.
A training period is only from 2013-01-02 to 2013-09-30 where the users can use the reference simulation as well as all available simulated observations.
The evaluation period is from 2012-10-22 to 2012-12-02 (i.e. 41 days) which is considered decorrelated from the training period. 
During the evaluation period, the user cannot use the reference NATL60 simulation but they can use all available simulated observations. There is also a spin-up period allowance from 2012-10-01 where the user can also use all available simulated observations.

\subsection{OSSE SWOT \& OSSE SST}

For the OSSE SWOT and OSSE SST experiments, the reference simulation, domain, and evaluation period is the same as the OSSE NADIR experiment.
However, the OSSE SWOT includes simulated observations of the novel KaRIN sensor recently deployed during the SWOT mission, the pseudo-observations were generated using the SWOT simulator~\cite{SWOT}. 
This OSSE SST experiment allows the users to utilize the full fields of SST as inputs to help reconstruct the SSH field in conjunction with the NADIR and SWOT SSH observation.
Because the SST comes from the same NATL60 simulation, the geometry characteristics SST and SSH are exactly the same.

\subsection{OSE NADIR}

The OSE NADIR experiment only uses real observations aggregated from different altimeters. These SSH observations include observations from the SARAL/Altika, Jason 2, Jason 3, Sentinel 3A, Haiyang-2A and Cryosat-2 altimeters. The Cryosat-2 altimeter is used as the independent evaluation track used to assess the performance of the reconstructed SSH field.

\subsection{Results}

We use \texttt{OceanBench} to generate maps of relevant quantities from the 4DVarNet method~\cite{4DVARNETSWOT,4DVARNETSST}.
Figure~\ref{fig:oceanbench_maps_4dvarnet} showcases some demo maps for some key physical variables outlined in section~\ref{sec:physical_variables}.
We showcase the 4DVarNet method because it is the SOTA method that was applied to each of the data challenges.
We can see that the addition of more information, i.e. NADIR -> SWOT -> SST, results in maps look more similar to the NEMO simulation in the OSSE challenges.
It also produces sensible maps for the OSE challenge as well.

\texttt{OceanBench} also generated figure~\ref{fig:oceanbench_psd_4dvarnet} which shows plots of the PSD and PSD scores of SSH for the different challenges.
Again, as we increase the efficacy of the observations via SWOT and allow for more external factors like the SST, we get an improvement in the isotropic and spacetime PSD scores.
In addition, we see that the PSD plots for the OSE task look very similar to the OSE challenges. 

Lastly, we used \texttt{OceanBench} to generate a leaderboard of metrics for a diverse set of algorithms where the maps were available online.
Table~\ref{tb:exp-results-mega} displays all of the key metrics outlined in section~\ref{sec:metrics} including the normalized RMSE and various spectral scores which are appropriate for the challenge.
We see that as the complexity of the method increases, the metrics improve. 
In addition, the methods that involve end-to-end learning perform the best overall, i.e. 4DVarNet.

\begin{figure}[ht!]
\small
\begin{center}
\setlength{\tabcolsep}{1pt}
\begin{tabular}{cccc}
\hspace{3mm} Task OSSE & 
\hspace{3mm} Task OSSE & 
\hspace{2mm} Task OSSE & 
Task OSE \\
\hspace{3mm}  Nadir & 
\hspace{3mm} Nadir + SWOT & 
\hspace{2mm} Nadir + SST & 
Nadir \\
%\vspace{-2mm}
%%%%% SSH %%%%%%%%
\includegraphics[trim={0 13mm 22mm 0},clip, width=3.60cm,height=3.2cm]{content/figures/fourdvarnet_figs/osse_gf_nadir_ssh.png} &
\includegraphics[trim={13mm 13mm 22mm 0},clip, width=3.2cm,height=3.2cm]{content/figures/fourdvarnet_figs/osse_gf_nadirswot_ssh.png} &
\includegraphics[trim={13mm 13mm 22mm 0},clip, width=3.2cm,height=3.2cm]{content/figures/fourdvarnet_figs/osse_gf_nadir_sst_ssh.png} &
\includegraphics[trim={13mm 13mm 0 0},clip,width=4.0cm,height=3.2cm]{content/figures/fourdvarnet_figs/ose_gf_ssh.png} \\
%\vspace{3mm}
%%%%% KINETIC ENERGY %%%%%%%%
\includegraphics[trim={0 13mm 22mm 5mm}, clip, width=3.60cm,height=3cm]{content/figures/fourdvarnet_figs/osse_gf_nadir_ke.png} &
\includegraphics[trim={13mm 13mm 22mm 5mm},clip, width=3.2cm,height=3cm]{content/figures/fourdvarnet_figs/osse_gf_nadirswot_ke.png} &
\includegraphics[trim={13mm 13mm 22mm 5mm},clip, width=3.2cm,height=3cm]{content/figures/fourdvarnet_figs/osse_gf_nadir_sst_ke.png} &
\includegraphics[trim={13mm 13mm 0 5mm},clip,width=4cm,height=3cm]{content/figures/fourdvarnet_figs/ose_gf_ke.png} \\
%%%%% RELATIVE VORTICITY %%%%%%%%
\includegraphics[trim={0 13mm 21.2mm 5mm},clip, width=3.60cm,height=3cm]{content/figures/fourdvarnet_figs/osse_gf_nadir_vort_r.png} &
\includegraphics[trim={13mm 13mm 21.2mm 5mm},clip, width=3.2cm,height=3cm]{content/figures/fourdvarnet_figs/osse_gf_nadirswot_vort_r.png} &
\includegraphics[trim={13mm 13mm 21.2mm 5mm},clip, width=3.2cm,height=3cm]{content/figures/fourdvarnet_figs/osse_gf_nadir_sst_vort_r.png} &
\includegraphics[trim={13mm 13mm 0 5mm},clip,width=4.0cm,height=3cm]{content/figures/fourdvarnet_figs/ose_gf_vort_r.png} \\
%%%%% STRAIN %%%%%%%%
\includegraphics[trim={0 0 19mm 5mm},clip, width=3.60cm,height=3.4cm]{content/figures/fourdvarnet_figs/osse_gf_nadir_strain.png} &
\includegraphics[trim={13mm 0 19mm 5mm},clip, width=3.2cm,height=3.4cm]{content/figures/fourdvarnet_figs/osse_gf_nadirswot_strain.png} &
\includegraphics[trim={13mm 0 19mm 5mm},clip, width=3.2cm,height=3.4cm]{content/figures/fourdvarnet_figs/osse_gf_nadir_sst_strain.png} &
\includegraphics[trim={13mm 0 0 5mm},clip,width=4.0cm,height=3.4cm]{content/figures/fourdvarnet_figs/ose_gf_strain.png} \\
% \vspace{-2mm}
(a) & (b) & (c) & (d)
\end{tabular}
\vspace{-3mm}
% \caption{Row I - Isotrophic PSD. Row 2 - Isotrophic PSD Score}
\caption{
Reconstructed quantities by the 4dVarNet method for each of the four tasks.
Each row showcases the following physical variables found in appendix~\ref{sec:physical_variables}: (a) Sea Surface Height, (b) Kinetic Energy, (c) Relative Vorticity, and (d) Strain. 
Each column showcase the reconstructed from the tasks (a) OSSE using only Nadir tracks: (b) OSSE using Nadir tracks and SWOT swath, (c) Multimodal using Nadir tracks and sea surface temperature, and (d) Reconstruction using real nadir altimetry tracks.}
\vspace{-5mm}
\label{fig:oceanbench_maps_4dvarnet}
\end{center}
\end{figure}





% \begin{figure}[ht!]
\small
\begin{center}
\setlength{\tabcolsep}{1pt}
\begin{tabular}{cccc}
\hspace{3mm} Task OSSE & 
\hspace{3mm} Task OSSE & 
\hspace{2mm} Task OSSE & 
Task OSE \\
\hspace{3mm}  Nadir & 
\hspace{3mm} Nadir + SWOT & 
\hspace{2mm} Nadir + SST & 
Nadir \\
%\vspace{-2mm}
%%%%% SSH %%%%%%%%
\includegraphics[trim={0 13mm 22mm 0},clip, width=3.60cm,height=3.2cm]{00_Oceanbench/content/figures/fourdvarnet_figs/osse_gf_nadir_ssh.png} &
\includegraphics[trim={13mm 13mm 22mm 0},clip, width=3.2cm,height=3.2cm]{00_Oceanbench/content/figures/fourdvarnet_figs/osse_gf_nadirswot_ssh.png} &
\includegraphics[trim={13mm 13mm 22mm 0},clip, width=3.2cm,height=3.2cm]{00_Oceanbench/content/figures/fourdvarnet_figs/osse_gf_nadir_sst_ssh.png} &
\includegraphics[trim={13mm 13mm 0 0},clip,width=4.0cm,height=3.2cm]{00_Oceanbench/content/figures/fourdvarnet_figs/ose_gf_ssh.png} \\
%\vspace{3mm}
%%%%% KINETIC ENERGY %%%%%%%%
\includegraphics[trim={0 13mm 22mm 5mm}, clip, width=3.60cm,height=3cm]{00_Oceanbench/content/figures/fourdvarnet_figs/osse_gf_nadir_ke.png} &
\includegraphics[trim={13mm 13mm 22mm 5mm},clip, width=3.2cm,height=3cm]{00_Oceanbench/content/figures/fourdvarnet_figs/osse_gf_nadirswot_ke.png} &
\includegraphics[trim={13mm 13mm 22mm 5mm},clip, width=3.2cm,height=3cm]{00_Oceanbench/content/figures/fourdvarnet_figs/osse_gf_nadir_sst_ke.png} &
\includegraphics[trim={13mm 13mm 0 5mm},clip,width=4cm,height=3cm]{00_Oceanbench/content/figures/fourdvarnet_figs/ose_gf_ke.png} \\
%%%%% RELATIVE VORTICITY %%%%%%%%
\includegraphics[trim={0 13mm 21.2mm 5mm},clip, width=3.60cm,height=3cm]{00_Oceanbench/content/figures/fourdvarnet_figs/osse_gf_nadir_vort_r.png} &
\includegraphics[trim={13mm 13mm 21.2mm 5mm},clip, width=3.2cm,height=3cm]{00_Oceanbench/content/figures/fourdvarnet_figs/osse_gf_nadirswot_vort_r.png} &
\includegraphics[trim={13mm 13mm 21.2mm 5mm},clip, width=3.2cm,height=3cm]{00_Oceanbench/content/figures/fourdvarnet_figs/osse_gf_nadir_sst_vort_r.png} &
\includegraphics[trim={13mm 13mm 0 5mm},clip,width=4.0cm,height=3cm]{00_Oceanbench/content/figures/fourdvarnet_figs/ose_gf_vort_r.png} \\
%%%%% STRAIN %%%%%%%%
\includegraphics[trim={0 0 19mm 5mm},clip, width=3.60cm,height=3.4cm]{00_Oceanbench/content/figures/fourdvarnet_figs/osse_gf_nadir_strain.png} &
\includegraphics[trim={13mm 0 19mm 5mm},clip, width=3.2cm,height=3.4cm]{00_Oceanbench/content/figures/fourdvarnet_figs/osse_gf_nadirswot_strain.png} &
\includegraphics[trim={13mm 0 19mm 5mm},clip, width=3.2cm,height=3.4cm]{00_Oceanbench/content/figures/fourdvarnet_figs/osse_gf_nadir_sst_strain.png} &
\includegraphics[trim={13mm 0 0 5mm},clip,width=4.0cm,height=3.4cm]{00_Oceanbench/content/figures/fourdvarnet_figs/ose_gf_strain.png} \\
% \vspace{-2mm}
(a) & (b) & (c) & (d)
\end{tabular}
\vspace{-3mm}
% \caption{Row I - Isotrophic PSD. Row 2 - Isotrophic PSD Score}
\caption{
Reconstructed quantities by the 4dVarNet method for each of the four tasks.
Each row showcases the following physical variables found in appendix~\ref{sec:physical_variables}: (a) Sea Surface Height, (b) Kinetic Energy, (c) Relative Vorticity, and (d) Strain. 
Each column showcase the reconstructed from the tasks (a) OSSE using only Nadir tracks: (b) OSSE using Nadir tracks and SWOT swath, (c) Multimodal using Nadir tracks and sea surface temperature, and (d) Reconstruction using real nadir altimetry tracks.}
\vspace{-5mm}
\label{fig:oceanbench_maps_4dvarnet}
\end{center}
\end{figure}



\begin{table}[ht]
\caption{This table showcases all of the summary statistics for some methods for each of the data challenges listed in section~\ref{sec:data_challenges} and~\ref{sec:data_challenges_extended}. The summary statistics shown are the normalized RMSE and the effective resolution in the spectral domain. The spectral metrics for the effective resolution that were outlined in section~\ref{sec:metrics} are: i) $\lambda_a$ is the spatial score for the alongtrack PSD score, ii) $\lambda_r$ is the spatial score for the isotropic PSD, iii) $\lambda_x$ is the spatial score for space-time PSD score, and iv) $\lambda_t$ is the temporal score for the space-time PSD score.}
% \caption{This table highlights some of the results for the OSSE experiments outlined in section~\ref{sec:osse} and~\ref{sec:other_tasks}.

% This table highlights the performance statistically in the real and spectral space; the normalized RMSE for the real space and the minimum spatial and temporal scales resolved in the spectral domain. 
% For more information about the class of models displayed and class of metrics, see section~\ref{sec:ml_ontology} and section~\ref{sec:metrics} respectively.}
\label{tb:exp-results-mega}
\centering
\begin{tabular}{llcccccc}
 \toprule
% Experiment & Configuration & Method & nRMSE & Resolved Scale [km]    \\ \midrule
% \multirow{2}{*}{Experiment} & \multirow{2}{*}{Algorithm} & \multirow{2}{*}{Algorithm Class} & \multirow{2}{*}{nRMSE} & \multicolumn{2}{c}{Effective Resolution} \\ 
% &  &   &  & Wavelength [km]  & Period [days]      \\ \midrule
% \multirow{2}{*}{Experiment} & \multirow{2}{*}{Algorithm} & \multirow{2}{*}{Algorithm Class} & \multirow{2}{*}{nRMSE} & \multicolumn{2}{c}{Effective Resolution} \\ 
\multirow{2}{*}{Experiment} &  \multirow{2}{*}{Algorithm} &   \multirow{2}{*}{nRMSE} &
\multicolumn{4}{c}{Effective Resolution} \\
& & & $\lambda_{a}$ [km] & $\lambda_{r}$ [km]   &  $\lambda_{\mathbf{x}}$ [km]  &   $\lambda_{t}$ [days]      \\ \midrule
OSSE NADIR     &  OI & 0.92 & - & 123 & 174 & 10.8 \\
OSSE NADIR     &  MIOST &  0.93 & - & 100 & 157 & 10.1 \\
OSSE NADIR     &  BFNQG & 0.93 & - & 88 & 139 & 10.4 \\
OSSE NADIR &  4DVarNet &  \textbf{0.94} & - & \textbf{65} & \textbf{117} & \textbf{7.7} \\
\midrule
OSSE SWOT     &  OI & 0.92 & - & 106 & 139 & 11.7 \\
OSSE SWOT     &  MIOST &  0.94 & - & 88 & 131 & 10.1 \\
OSSE SWOT     &  BFNQG & 0.94 & - & 64 & 118 & 36.5 \\
OSSE SWOT &  4DVarNet &  \textbf{0.96} & - & \textbf{47} & \textbf{77} & \textbf{5.6} \\
\midrule
OSSE SST     &  Musti & 0.95 & - & 46 & 138 & 4.1 \\
OSSE SST &  4DVarNet &  \textbf{0.96} & - & \textbf{46} & \textbf{87} & \textbf{3.7} \\
\midrule
OSE NADIR     &  OI & 0.88 & 151 & - &  - &  -\\
OSE NADIR     &  MIOST &  0.90 & 135 & - &  - &  -\\
OSE NADIR     &  BFNQG & 0.88 & 122 & - & - &  -\\
OSE NADIR &  ConvLSTM &  0.89 & 113 &- &  - &  -\\
OSE NADIR &  4DVarNet & \textbf{0.91} & \textbf{98} & - &  -  &  -\\
\bottomrule
\end{tabular}
\end{table}

% \subsection{Simulated Altimetry Tracks} \label{sec:dc_osse_nadir}

% \textcolor{red}{
% The most commonly used SSH maps, the Developing Use of Altimetry for Climate Studies (DUACS) products, are derived from a statistical space–time interpolation of nadir altimeter observations. This intrinsically limits the effective resolution [as defined in Skamarock (2004), i.e., the fully resolved scales] of DUACS SSH maps to 150–200 km at middle latitudes (Ballarotta et al. 2019). The SSH mapping algorithm was developed by CNES and CLS in 1997, as part of the DUACS project, and has been continuously improved since then (Taburet et al. 2019). The DUACS products are now distributed by the Copernicus Marine Environment Monitoring Service (CMEMS). DUACS algorithm implements a statistical interpolation of SSH satellite data in space and time to produce global daily maps (Le Traon et al. 1998). The data are collected by a constellation of 2 to 4 nadir-looking altimeters (sometimes referred to as conventional altimeters), and characterized by large data gaps reaching 200 km in the zonal direction at the equator.
% }

% \subsection{Simulated SWOT Tracks} \label{sec:dc_osse_swot}


% \textcolor{red}{
% The Surface Water and Ocean Topography (SWOT; Fu et al. 2012; Morrow et al. 2019) altimetry mission, to be launched in early 2022, will open the way to SSH maps with resolution significantly higher than 150 km at midlatitudes, but this perspective entails a thorough revisit of the mapping algorithm. SWOT will considerably increase the measurement density at the surface of the oceans thanks to SSH measurements at a kilometric pixel resolution over a swath 120 km wide. On the swath, SWOT is expected to resolve scales down to 15 km at low latitude and 30–45 km at mid- and high latitudes (Wang et al. 2019). In its science phase, SWOT will have a 21 days repeat orbit, allowing an average revisit time of 11 days in most of the globe. Some of the dynamical processes observable by SWOT evolve over time scales on the order of 1 day, much shorter than the satellite revisit time. Consequently, the mapping method implemented in the current DUACS system will certainly not be sufficient to draw the maximum benefit from SWOT. A linear interpolation will filter most of the observed small-scale signals between two passes of the satellite, as anticipated by Gaultier et al. (2016).These authors advocate for using more advanced methods to build SSH maps.
% }

% \subsection{Multimodal with Sea Surface Temperature}  \label{sec:dc_osse_sst}


% \subsection{Real Altimetry Tracks}  \label{sec:dc_ose_nadir}
\newpage
\section{Physical Variables} \label{sec:physical_variables}

As alluded to in the main body of the paper, we have access to many physical quantities which can be derived from  sea surface height. 
This gives us a way to analyze how effective and trustworthy are our reconstructions. 
Many machine learning methods are unconstrained so they may provide solutions that are physically inconsistent and visualizing the field is a very easy eye test to assess the validity. 
In addition to post analysis, one could include some of these derived quantities maybe useful as additional inputs to the system and/or constraints to the loss function. 
Recall the spatiotemporal coordinates from equation~\ref{eq:spatiotemporal_coords}, 
we use the same coordinates for the subsequent physical quantities. \textbf{Sea Surface Height} is the deviation of the height of the ocean surface from the geoid of the Earth. We can define it as:
\begin{align}
	\text{Sea Surface Height }[m]:&& \quad
 \eta &= \boldsymbol{\eta}(\mathbf{x},t)&& \quad \Omega\times \mathcal{T}\rightarrow\mathbb{R} \label{eq:ssh}
\end{align}
This quantity is the actual value that is given from the satellite altimeters and is presented in the products for SSH maps~\cite{DUACS}. An example can be seen in the first row of figure~\ref{fig:oceanbench_maps_4dvarnet}.

\textbf{Sea Surface Anomaly} is the anomaly wrt to the spatial mean which is defined by
\begin{align}
	\text{Sea Level Anomaly }[m]:&& \quad
 \bar{\eta} &= \boldsymbol{\eta}(\mathbf{x},t) - \bar{\eta}(t) &&
 \quad \Omega\times \mathcal{T}\rightarrow\mathbb{R} \label{eq:sla}
\end{align}
where $\bar{\eta}(t)$ is the spatial average of the field at each time step.  
An example can be seen in the first row of figure~\ref{fig:oceanbench_maps}.

Another important quantity is the \textbf{geostrophic velocities} in the zonal and meridional directions. This is given by
\begin{align}
	\text{Zonal Velocity}[ms^{-2}]:&& \quad
 u &= -\frac{g}{f_0}\frac{\partial \eta}{\partial y} &&
 \quad \Omega\times \mathcal{T}\rightarrow\mathbb{R} \label{eq:u_vel} \\
	\text{Meridional Velocity}[ms^{-2}]:&& \quad
 v &= \frac{g}{f_0}\frac{\partial \eta}{\partial x} &&
 \quad \Omega\times \mathcal{T}\rightarrow\mathbb{R} \label{eq:v_vel}
\end{align}
where $g$ is the gravitational constant and $f_0$ is the mean Coriolis parameter. These quantities are important as they can be an related to the sea surface current. The geostrophic assumption is a very strong assumption however it can still be an important indicator variable. The \textbf{kinetic energy} is a way to summarize the (geostrophic) velocities as the total energy of the system. This is given by
\begin{equation} \label{eq:kineticenergy}
    KE = \frac{1}{2}\left(u^2 + v^2\right)
\end{equation}
An example can be seen in the second row of figure~\ref{fig:oceanbench_maps_4dvarnet}.

Another very important quantity is the \textit{vorticity} which measures the spin and rotation of a fluid. In geophysical fluid dynamics, we use the \textbf{relative vorticity} which is the vorticity observed within at rotating frame.
This is given by
\begin{equation} \label{eq:relvorticity}
    \zeta = \frac{\partial v}{\partial x} - \frac{\partial u}{\partial y}
\end{equation}
An example can be seen in the third row of figure~\ref{fig:oceanbench_maps_4dvarnet}.

% \subsection{Absolute Vorticity}

% \begin{equation} \label{eq:absvorticity}
%     |\zeta| = \frac{\partial v}{\partial x} + \frac{\partial u}{\partial y}
% \end{equation}

We can also use the \textbf{Enstrophy} to summarize the relative voriticty to measure the total contribution which is given by
\begin{equation} \label{eq:enstrophy}
    E = \frac{1}{2}\zeta^2
\end{equation}

The \textbf{Strain} is a measure of deformation of a fluid flow.

\begin{equation} \label{eq:strain}
    \sigma = \sqrt{\sigma_n^2 + \sigma_s^2}
\end{equation}

where $\sigma_n$ is the shear strain (aka the shearing deformation) and $\sigma_s$ is the normal strain (aka stretching deformation). An example can be seen in the fourth row of figure~\ref{fig:oceanbench_maps_4dvarnet}.

The \textbf{Okubo-Weiss Parameter} is high-order quantity which is a linear combination of the strain and the relative vorticity.

\begin{equation} \label{eq:okuboweiss}
    \sigma_{ow} = \sigma_n^2 + \sigma_s^2 - \zeta^2
\end{equation}

This quantity is often used as a threshold for determining the location of Eddies in sea surface height and sea surface current fields~\cite{OKUBO, WEISS, OKUBOWEISS}.

% \begin{table}[h!]
%   \caption{Table of nomanclature}
%   \label{sample-table}
%   \centering
%   \begin{tabular}{ccl}
%     \toprule
%     Symbol & Size & Description  \\
%     \midrule
%     $\mathbf{x}_s$ & $\mathbb{R}^{D_s}$ & Spatial Coordinates  \\
%     $t$ & $\mathbb{R}^{D_t}$ & Temporal Coordinates  \\
%    $\boldsymbol{f}(\mathbf{x}_s, t)$ & $\mathbb{R}^{D}$ & Spatiotemporal Field  \\
%    $\boldsymbol{y}_{obs}(\mathbf{x}_s, t)$ & $\mathbb{R}^{D_{obs}}$ & Spatiotemporal Observations  \\
%    $\eta$ & $\mathbb{R}$ & Sea Surface Height $[m]$ \\
%    $\bar{\eta}$ & $\mathbb{R}$ & Sea Surface Anomaly $[m]$ \\
%    $u$ & $\mathbb{R}$ & Zonal Velocity $[ms^{-2}]$ \\
%    $v$ & $\mathbb{R}$ & Meridional Velocity $[ms^{-2}]$ \\
%     \bottomrule
%   \end{tabular}
%   \label{tb:notation}
%  \end{table}


% \subsection{Coordinates}
% \textbf{SpatioTemporal Coordiantes}. We define some generic spatiotemporal coordinates.
% 
% We are dealing with satellite observations, so we are interested in the domain across the Earth's surface. 
% Let us define the Earth's domain by some spatial coordinates, $\mathbf{x} = [\text{Longitude},\text{Latitude}]^\top \in\mathbb{R}^{D_s}$, and temporal coordinates, $t=[\text{Time}]\in\mathbb{R}^+$, where $D_s$ is the dimensionality of the coordinate vector.  
% We can define some spatial (sub-)domain, $\Omega\subseteq\mathbb{R}^{D_s}$, and a temporal (sub-)domain, $\mathcal{T}\subseteq\mathbb{R}^+$. 
% This domain could be the entire globe for 10 years or a small region within the North Atlantic for 1 year.
%
%
% \begin{align} \label{eq:spatiotemporal_coords}
%     \text{Spatial Coordinates}: && \mathbf{x} &\in \Omega \subseteq \mathbb{R}^{D_s}\\ 
%     \text{Temporal Coordinates}: && t &\in \mathcal{T} \subseteq \mathbb{R}^{D_t}.
% \end{align}
% %
%
% In this case $D_s=2$ because we only have a two coordinates, however we can do some coordinate transformations like spherical to Cartesian. Likewise, we can do some coordinate transformation for the temporal coordinates like cyclic transformations or sinusoidal embeddings~\cite{ATTENTION}. We have two fields of interest from these spatiotemporal coordinates: the state and the observations.
% %
% %
% \begin{align} \label{eq:state_obs}
%     \text{State}: && \boldsymbol{u}(\mathbf{x},t) &: \Omega\times\mathcal{T}\rightarrow\mathbb{R}^{D_u} \\
%     \text{Observations}: && \boldsymbol{y}_{obs}(\mathbf{x},t) &: \Omega\times\mathcal{T}\rightarrow\mathbb{R}^{D_{obs}}
% \end{align}
% %
% %
% The state domain, $u\in\mathcal{U}$, is a scalar or vector-valued field of size $D_u$ which is typically the quantity of interest and the observation domain, $y_{obs}\in\mathcal{Y}_{obs}$, is the observable quantity which is also a scalar or vector-valued field of size $D_{obs}$. Now, we make the assumption that we have an operator $\mathcal{H}$ that transforms the field from the state space, $\boldsymbol{u}$, to the observation space, $\boldsymbol{y}_{obs}$.
% %
% %
% \begin{align} \label{eq:prob_definition}
%     \boldsymbol{y}_{obs}(\mathbf{x},t) = \mathcal{H}\left(\boldsymbol{u}(\mathbf{x},t), t, \boldsymbol{\varepsilon}, \boldsymbol{\mu}\right) 
% \end{align}
% %
% %


% % \subsection{Field}

% \textbf{Field}. We have the generic definition of a scalar or vector-valued field.

% \begin{align} \label{eq:field}
% \text{Field}:&& f&=\boldsymbol{f}(\mathbf{x},t), && \quad \Omega\times \mathcal{T}\rightarrow\mathbb{R}^{D}
% \end{align}
\newpage
\section{Metrics} \label{sec:metrics}

There are many metrics that are standard within the ML community but unconvincing for many parts the geoscience community. 
Specifically, many of these standard scores do not capture the important optimization criteria in the scientific machine learning tasks.
However, there is not consensus within domain-specific communities about the perfect metric which captures every aspect we are interested.
Therefore, we should have a variety of scores from different perspectives to really assess the pros and cons of each method we wish to evaluate thoroughly. 
Below, we outline two sets of scores we use within this framework: skill scores and spectral scores.

\subsection{Skill Scores}

We classify one set of metrics as \textit{skill scores}. 
These are globally averaged metrics which tend to operate within the real space.
Some examples include the root mean squared error (RMSE) and the normalized root mean squared (nRMSE) error.
The RMSE metric can also be calculated w.r.t. the spatial domain, temporal domain or both. 
For example, figure~\ref{fig:oceanbench_psd} showcases the nRMSE calculated only on the spatial domain and visualized for each time step.
%
\begin{align}
    \text{RMSE}: &&\text{RMSE}(\eta,\hat{\eta}) &= ||\eta - \hat{\eta}||_2 \label{eq:RMSE}\\
    % \text{RMSE}_t: &&\text{RMSE}_t(\eta,\hat{\eta}; t) &= ||\eta(t) - \hat{\eta}(t)||_2 \label{eq:RMSE_t}\\
    \text{nRMSE}: &&\text{nRMSE}(\eta,\hat{\eta}) &= 1 - \frac{\text{RMSE}(\eta - \hat{\eta})}{\text{RMSE}(\eta)} \label{eq:nRMSE}
\end{align}
%
However, we are not limited to just the standard MSE metrics.
We can easily incorporate more higher-order statistics like the Centered Kernel Alignment (CKA)~\cite{METRICSCKA} or information theory metrics like mutual information (MI)~\cite{METRICSITRBIG,METRICSITRBIG2}.
In addition, we could also utilize the same metrics in the frequency domain as is done in~\citep{PDEBench}.

\subsection{Spectral Scores}

Another class of scores that we use in \texttt{OceanBench} are the \textit{spectral scores}. These scores are calculated within the spectral space via the wavenumber power spectral density (PSD). 
This provides a spatial-scale-dependent metric which is useful for identifying the largest and smallest scales that were resolved by the reconstruction map. 
In general, we use these to measure the expected energy at different spatiotemporal scales and we can also construct custom score functions which gives us a summary statistic for how well we reconstructed certain scales.
%
\begin{align}
    \text{PSD}: &&\text{PSD}(\eta) &= \sum_{k_{min}}^{k_{max}}\|\mathcal{\mathcal{F}(\eta)}\|^2\label{psd}\\
    \text{PSD}_{score}: &&\text{PSD}_{score}(\eta,\hat{\eta}) &= 1 - \frac{\text{PSD}(\eta - \hat{\eta})}{\text{PSD}(\eta)} \label{eq:psd_score}
\end{align}
%
where $\mathcal{F}$ is the Fast Fourier Transformation (FFT). 
In our application, there are various ways to construct the PSD which depend on the FFT transformation.
We denote the \textit{space-time PSD} as $\lambda_\mathbf{x}$ which does the 2D FFT in the longitude and time direction, then takes the average over the latitude.
We denote the \textit{space-time PSD} as $\lambda_\mathbf{t}$ which does the 2D FFT in the longitude and latitude direction, then takes the average over the time.
We denote the \textit{isotropic PSD} as $\lambda_r$ which assumes a radial relationship in the spatial domain and then averages over the temporal domain.
Lastly, we denote the standard PSD score as $\lambda_a$ which is the 1D FFT over a prescribed distance along the satellite track; this is what is done for the OSE NADIR experiment.
We recognize that the FFT configurations are limited due to their global treatment of the spectral domain and we need more specialized metrics to handle the local scales.
This opens the door to new metrics that handle such cases such as the Wavelet transformation~\cite{METRICSWAVELET}.

\begin{figure}[t!]
\small
\begin{center}
\setlength{\tabcolsep}{1pt}
\begin{tabular}{cccc}
\hspace{3mm} Task OSSE & 
 Task OSSE & 
\hspace{-10mm} Task OSSE & 
\hspace{-10mm}Task OSE \\
\hspace{3mm}  Nadir & 
 Nadir + SWOT & 
\hspace{-10mm} Nadir + SST & 
\hspace{-10mm}Nadir \\
%\vspace{-2mm}
%%%%% SSH %%%%%%%%
\includegraphics[trim={0 0 0 0},clip, width=3.70cm,height=3.5cm]{00_Oceanbench/content/figures/fourdvarnet_figs/osse_gf_nadir_isotrop.png} &
\includegraphics[trim={18mm 0 0 0},clip, width=3.3cm,height=3.5cm]{00_Oceanbench/content/figures/fourdvarnet_figs/osse_gf_nadirswot_isotrop.png} &
\hspace{-5mm}\includegraphics[trim={18mm 0 0 0},clip, width=3.3cm,height=3.5cm]{00_Oceanbench/content/figures/fourdvarnet_figs/osse_gf_nadir_sst_isotrop.png} &
\hspace{-10mm}\includegraphics[trim={18mm 0 0 0},clip,width=3.3cm,height=3.5cm]{00_Oceanbench/content/figures/fourdvarnet_figs/ose_gf_isotrop.png} \\
%\vspace{3mm}
%%%%% KINETIC ENERGY %%%%%%%%
\includegraphics[trim={0 0 0 0}, clip, width=3.70cm,height=3.5cm]{00_Oceanbench/content/figures/fourdvarnet_figs/osse_gf_nadir_1d_psd_score.png} &
\hspace{1mm}\includegraphics[trim={18mm 0 0 0},clip, width=3.3cm,height=3.5cm]{00_Oceanbench/content/figures/fourdvarnet_figs/osse_gf_nadirswot_1d_psd_score.png} &
\hspace{-4mm}\includegraphics[trim={18mm 0 0 0},clip, width=3.3cm,height=3.5cm]{00_Oceanbench/content/figures/fourdvarnet_figs/osse_gf_nadir_sst_1d_psd_score.png} &
\hspace{-10mm}\includegraphics[trim={18mm 0 0 0},clip,width=3.3cm,height=3.5cm]{00_Oceanbench/content/figures/fourdvarnet_figs/ose_gf_1d_psd_score.png} \\
%%%%% RELATIVE VORTICITY %%%%%%%%
\hspace{-4mm}\includegraphics[trim={0 0 23mm 0},clip, width=3.65cm,height=3.5cm]{00_Oceanbench/content/figures/fourdvarnet_figs/osse_gf_nadir_psd_spacetime.png} &
\includegraphics[trim={14mm 0 23mm 0},clip, width=3cm,height=3.5cm]{00_Oceanbench/content/figures/fourdvarnet_figs/osse_gf_nadirswot_psd_spacetime.png} &
\hspace{-5mm}\includegraphics[trim={14mm 0 23mm 0},clip, width=3cm,height=3.5cm]{00_Oceanbench/content/figures/fourdvarnet_figs/osse_gf_nadir_sst_psd_spacetime.png} &
\hspace{-5mm}\includegraphics[trim={14mm 0 0 0},clip,width=3.8cm,height=3.5cm]{00_Oceanbench/content/figures/fourdvarnet_figs/ose_gf_psd_spacetime.png} \\
%%%%% STRAIN %%%%%%%%
\hspace{-4mm}\includegraphics[trim={0 0 23mm 0},clip, width=3.70cm,height=3.5cm]{00_Oceanbench/content/figures/fourdvarnet_figs/osse_gf_nadir_psd_spacetime_score.png} &
\hspace{-2mm}\includegraphics[trim={13mm 0 23mm 0},clip, width=3.1cm,height=3.5cm]{00_Oceanbench/content/figures/fourdvarnet_figs/osse_gf_nadirswot_psd_spacetime_score.png} &
\hspace{1mm}\includegraphics[trim={13mm 0 0 0},clip, width=3.8cm,height=3.5cm]{00_Oceanbench/content/figures/fourdvarnet_figs/osse_gf_nadir_sst_psd_spacetime_score.png} &
 \\
% \vspace{-2mm}
 \hspace{1mm} (a) & \hspace{-5mm} (b) & \hspace{-8mm}(c) & \hspace{-10mm}(d)
\end{tabular}
\vspace{-3mm}
% \caption{Row I - Isotrophic PSD. Row 2 - Isotrophic PSD Score}
\caption{
Power spectrum and associated scores of the 4dVarNet method for each of the four tasks.
The row display in order: (1) the isotropic PSD, (2) the spatial PSD score (using the isotropic PSD for the first three rows and along track PSD for the last row), (3) the space-time PSD, (4) The spacetime PSD score available only in OSSE task.  }

\vspace{-5mm}
\label{fig:oceanbench_psd_4dvarnet}
\end{center}
\end{figure}


% \begin{figure}[h]
% \small
% \begin{center}
% \setlength{\tabcolsep}{1pt}
% \begin{tabular}{cccc}
% \hspace{3mm} NEMO Simulation & 
% \hspace{3mm} MIOST & 
% \hspace{3mm} BFNQG & 
% 4DVarNet \\
% \vspace{-2mm}
% %%%%% SSH %%%%%%%%
% \includegraphics[trim={0 0 38mm 0},clip, width=3.20cm,height=3cm]{content/figures/psd_spacetime/dc20a/nadir4/dc20a_psd_spacetime_nemo_nadir4_ssh.png} &
% \includegraphics[trim={0 0 40mm 0},clip, width=3.2cm,height=3cm]{content/figures/psd_spacetime/dc20a/nadir4/dc20a_psd_spacetime_miost_nadir4_ssh.png} &
% \includegraphics[trim={0 0 38mm 0},clip, width=3.2cm,height=3cm]{content/figures/psd_spacetime/dc20a/nadir4/dc20a_psd_spacetime_bfnqg_nadir4_ssh.png} &
% \includegraphics[width=4.0cm,height=3cm]{content/figures/psd_spacetime/dc20a/nadir4/dc20a_psd_spacetime_4dvarnet_nadir4_ssh.png} \\
% \end{tabular}
% % \vspace{-3mm}
% % \caption{Row I - Isotrophic PSD. Row 2 - Isotrophic PSD Score}
% \caption{The space-time power spectrum decomposition.}
% % \vspace{-5mm}
% \label{fig:appendix_psd_spacetime_NADIR}
% \end{center}
% \end{figure}


% \begin{figure}[h]
% \small
% \begin{center}
% \setlength{\tabcolsep}{1pt}
% \begin{tabular}{cccc}
% \hspace{3mm} DUACS & 
% \hspace{3mm} MIOST & 
% \hspace{3mm} BFNQG & 
% 4DVarNet \\
% % \vspace{-2mm}
% %%%%% SSH %%%%%%%%
% \includegraphics[trim={0 0 38mm 0},clip, width=3.20cm,height=3cm]{content/figures/psd_spacetime/dc20a/nadir4/dc20a_psd_spacetime_score_duacs_nadir4_ssh.png} &
% \includegraphics[trim={0 0 40mm 0},clip, width=3.2cm,height=3cm]{content/figures/psd_spacetime/dc20a/nadir4/dc20a_psd_spacetime_score_miost_nadir4_ssh.png} &
% \includegraphics[trim={0 0 38mm 0},clip, width=3.2cm,height=3cm]{content/figures/psd_spacetime/dc20a/nadir4/dc20a_psd_spacetime_score_bfnqg_nadir4_ssh.png} &
% \includegraphics[width=4.0cm,height=3cm]{content/figures/psd_spacetime/dc20a/nadir4/dc20a_psd_spacetime_score_4dvarnet_nadir4_ssh.png} \\
% \end{tabular}
% % \caption{Row I - Isotrophic PSD. Row 2 - Isotrophic PSD Score}
% \caption{The space-time power spectrum score decomposition.}
% % \vspace{-5mm}
% \label{fig:appendix_psd_score_spacetime_NADIR}
% \end{center}
% \end{figure}
\newpage
\section{Use Case II: Hydra Recipes} \label{sec:hydra_recipes}

This framework has drastically reduced the overhead for the ML researcher while also enhancing the reprducibility and replicability of the preprocessing steps. In this section we showcase a few examples for how one can use oceanbench in conjunction with hydra to provide recipes for some standard processes.

\subsection{GeoProcessing Recipe}

In this example, we showcase how one can pipe a sequential transformation through the hydra framework. In this example, we open the dataset, validate the coordinates to comply to our standards, select the region of interest, subset the data, regrid the alongtrack data to a uniform grid, and save the data to a netcdf file. See the listing~\ref{hydraconfig:geoprocess} for more information.


\begin{listing}[ht!]
\begin{minted}[frame=lines]{yaml}
# Target Function to initialize
_target_: "oceanbench._src.dataset.pipe"
# netcdf file to be loaded
inp: "${data_directory}/nadir_tracks.nc"
# sequential transformations to be applied
fns:
    # Load Dataset
    - {_target_: "xarray.open_dataset", _partial_: True}
    # Validate LatLonTime Coordinates
    - {_target_: "oceanbench.validate_latlon", _partial_: True}
    - {_target_: "oceanbench.validate_time", _partial_: True}
    # Select Specific Region (Spatial | Temporal)
    - {_target_: "xarray.Dataset.sel", args: ${domain}, _partial_: True}
    # Take Subset of Data
    - {_target_: "oceanbench.subset", num_samples: 1500, _partial_: True}
    # Regridding (AlongTrack -> Uniform Grid)
    - {
        _target_: "oceanbench.regrid", 
        target_grid: ${grid.high_res}, 
        _partial_: True
      }
    # Save Dataset
    - {
        _target_: "xarray.Dataset.to_netcdf", 
        save_name: "demo.nc", 
        _partial_: True
      }
\end{minted}
\label{hydraconfig:geoprocess}
\caption{This is a \texttt{.yaml} which showcases how we can communicate with \texttt{Hydra} framework to list a predefined set of transformations to be \textit{piped} through sequentiall. In this example, we showcase some standard pre-processing strategies to be saved to another netcdf file.}
\end{listing}




\newpage
\subsection{Evaluation Recipe - OSSE}

In this example, we showcase how one can use hydra to do the evaluation procedure. This is the same evaluation procedure that is used to evaluate the effectiveness of the OSSE NADIR experiment. From code snippet~\ref{hydraconfig:geoprocess}, we see that we choose which target function to initialize and we choose the data directory where the \texttt{.netcdf} file is located. Then, we pipe some transformations for the \texttt{.netcdf} file: 1) validate the spatiotemporal coordinates, 2) we select the evaluation region, 3) we regrid it to the target get, 4) we fill in the nans with a Gauss-Seidel procedure, 5) we rescale the coordinates to be in meters and days, and 6) we perform the isotropic power spectrum transformation to get the effective resolution outlined in section~\ref{sec:metrics}.

\begin{listing}[ht!]
\begin{minted}[frame=lines]{yaml}
# Target Function to initialize
_target_: "oceanbench._src.dataset.pipe"
# netcdf file to be loaded
inp: "${data_directory}/ml_result.nc"
# sequential transformations to be applied
fns:
    # Load Dataset
    - {_target_: "xarray.open_dataset", _partial_: True}
    # Validate LatLonTime Coordinates
    - {_target_: "oceanbench.validate_latlon", _partial_: True}
    - {_target_: "oceanbench.validate_time", _partial_: True}
    # Select Specific Region (Spatial | Temporal)
    - {_target_: "xarray.Dataset.sel", args: ${domain}, _partial_: True}
    # Regridding (Uniform Grid -> Uniform Grid)
    - {_target_: "oceanbench.regrid", 
       target_grid: ${grid.reference}, _partial_: True}
    # Fill NANS (around the corners)
    - {_target_: "oceanbench.fill_nans", 
       method: "gauss_seidel", _partial_: True}
    # Coordinate Change (degree -> meters, ns -> days)
    - {_target_: "oceanbench.latlon_deg2m", _partial_: True}
    - {_target_: "oceanbench.time_rescale", 
       freq: 1, unit: "days", _partial_: True}
    # Calculate Isotropic Power Spectrum
    - {_target_: "oceanbench.power_spectrum_isotropic", 
       reference: ${grid.reference}, _partial_: True}
    # Calculate Resolved Spatial Scale
    - {_target_: "oceanbench.resolved_scale", _partial_: True}
    # Save Dataset
    - {_target_: "xarray.Dataset.to_netcdf", 
       save_name: "ml_result_psd.nc", _partial_: True}
\end{minted}
\label{hydraconfig:evaluation}
\caption{This is a \texttt{.yaml} which showcases how we can communicate with \texttt{Hydra} framework to list a predefined set of transformations to be \textit{piped} through sequentiall. In this example, we showcase some standard pre-processing strategies to be saved to another netcdf file.}
\end{listing}





\newpage
\section{Use Case III: XRPatcher}
\label{sec:xrpatcher}

There are many usecases for the \texttt{XRPatcher}. For example, we can do 1D Time chunking, 2D Spatial-Temporal Patches, or 3D Spatial-Temporal Cubes.


\begin{listing}[!ht]
\small
\begin{minted}[frame=lines]{python}
import xarray as xr
import torch
import itertools
from oceanbench import XRPatcher
# Easy Integration with PyTorch Datasets (and DataLoaders)
class XRTorchDataset(torch.utils.data.Dataset):
    def __init__(self, batcher: XRPatcher, item_postpro=None):
        self.batcher = batcher
        self.postpro = item_postpro
    def __getitem__(self, idx: int) -> torch.Tensor:
        item = self.batcher[idx].load().values
        if self.postpro:
            item = self.postpro(item)
        return item
    def reconstruct_from_batches(
            self, batches: list(torch.Tensor), **rec_kws
        ) -> xr.Dataset:
        return self.batcher.reconstruct(
            [*itertools.chain(*batches)], **rec_kws
        )
    def __len__(self) -> int:
        return len(self.batcher)
# load demo dataset
data = xr.tutorial.load_dataset("eraint_uvz")
# Instantiate the patching logic for training
patches = dict(longitude=30, latitude=30)
train_patcher = XRPatcher(
    da=data,
    patches=patches,
    strides=patches,        # No Overlap
    check_full_scan=True    # check no extra dimensions
)
# Instantiate the patching logic for testing
patches = dict(longitude=30, latitude=30)
strides = dict(longitude=5, latitude=5)
test_patcher = XRPatcher(
    da=data,
    patches=patches,
    strides=strides,        # Overlap
    check_full_scan=True    # check no extra dimensions
)
# instantiate PyTorch DataSet
train_ds = XRTorchDataset(train_patcher, item_postpro=TrainingItem._make)
test_ds = XRTorchDataset(test_patcher, item_postpro=TrainingItem._make)
# instantiate PyTorch DataLoader
train_dl = torch.utils.data.DataLoader(train_ds, batch_size=4, shuffle=False)
test_dl = torch.utils.data.DataLoader(test_ds, batch_size=4, shuffle=False)
\end{minted}
\label{listing:xrpatcher}
\caption{\texttt{XRPatcher} integration in Pytorch. We define a PyTorch dataset that handles the \texttt{XRPatcher}. We load an arbitrary dataset with \texttt{xarray}, then we instantiate the \texttt{XRPatcher} with the patching logic, then we instantiate the PyTorch dataset and dataloaders.}
\end{listing}





\newpage
\section{Additional Tasks}\label{sec:other_tasks}

In the main paper, we thoroughly outlined the interpolation task to showcase how \texttt{OceanBench} can be used to create automated pipelines for processing and evaluation procedures.
However, there are many other additional tasks that can make use of the \texttt{OceanBench} features. 

\textbf{Denoising}. A simpler problem for interpolation tasks is the denoising problem~\cite{DENOISESURVEY,DENOISESURVEY2}.
The SSH and SST measurements we obtain have inherent noise from the sensors.
A key problem is to calibrate the observations by separating the known noise patterns and the true signal.
There has already been a lot of work from the ML side ranging from amortized predictions~\cite{DENOISESWOT} to end-to-end learning schemes~\cite{DENOISESWOT2}.
Much of this work has been facilitated by the \textit{Ocean-Data-Challenge} group which have a few data challenges related to the denoising problem.
Just like \texttt{OceanBench} was able to create reproducible pipelines from the SSH interpolation challenge listed in section~\ref{sec:data_challenges}, we also believe that one could extend the denoising challenge in the same manner.

\textbf{Forecasting}. This is a special form of extrapolation whereby the temporal domain of the state variable is sufficiently outside of the domain of the observation domain. 
Many previous benchmarking suites already look at forecasting for weather~\cite{weatherbench} and climate~\cite{ClimateBench}.
However, in oceanography, it is also advantageous to do forecasting for problems involving currents~\cite{MLSSC,4DVarNetSSC} and eddies~\cite{OCEANEddyTracking,OKUBO,OKUBOWEISS}.
The \texttt{xrpatcher} will work out of the box for forecasting problems and contributions can be made to \texttt{OceanBench} to include some specific metrics for forecasting as were outlined in~\cite{weatherbench,ClimateBench,ENS10Bench}.

\textbf{Proxy Variables}. There are many other control variables that one could use to improve the interpolation or extrapolation task.
We mentioned SST in section~\ref{sec:data_challenges} because it is the most abundant observations available.
However, there are other important observed variables which could be useful, e.g. Ocean colour, Bigeochemical parameters, and atmospheric variables.
In many other downstream applications, the oceanography community often uses SSH and SST as proxy variables to predict important quantities related to the carbon uptake, e.g. SOCAT~\cite{SOCAT}.
It would be straightforward to include a specific variable (and the associated preprocessing operations) into \texttt{OceanBench}.

\textbf{Dimension Reduction}. We often have very resolution spatiotemporal fields.
which poses a very big challenge for learning due to the high correlations exhibited by spatiotemporal data and high dimensionality.
A workaround for this is to learn a latent representation which retains as much relevant information as possible for the given task.
In the ocean sciences, this is known as \textit{Reduced Order Modeling} (ROM) or more generally dimensionality reduction which has been frequently used for adaptive meshes for physical models~\cite{NEMOEOF}.
This could be used for pretraining fields to latent embeddings which could be useful for downstream tasks like anomaly detection~\cite{SSTFLOWANOMALY}.
% In modern diffusion models, most of the operations are within the latent domain without any loss of quality in the generative results.


\textbf{Surrogate Modeling}. 
Physical model simulations are very expensive and ML has played a role in learning surrogate models to descrease the computational intensity~\cite{ML4OCN,MLCLOSURE}.
We have a decently long spatiotemporal field over a region of interest which could be used to learn a surrogate model to mimic the dynamics of that region.
This is also very useful for hybrid schemes whereby we have parameterizations to account for processes that are missing from low resolution simulations.~\cite{MLOCNPARAMETERIZATION,MLOCNPARAMETERIZATION2, MLOCNPARAMETERIZATION3, MLOCNPARAMETERIZATION4}.



% \subsection{Main Tasks}

% \subsubsection{Interpolation/Extrapolation}

% The two main tasks we can define from this problem setup are: 1) interpolation/extrapolation and 2) forecasting (which can be seen as extrapolation in Time). 
% Both interpolation and extrapolation are when the domain for the $\mathcal{T}$ always falls between the past and the present, i.e., $\mathcal{T}\in[0, T]$ where $T$ is the current Time. 
% We define interpolation as the case where the observations along the boundary of the domain are equal to the boundary of the desired reconstruction domain, i.e., $\partial\boldsymbol{\Omega}_{obs} = \partial\boldsymbol{\Omega}_g$, and extrapolation as the case where the boundary of the observation domain lies entirely inside of the boundary of the reconstructed domain, i.e., $\partial\boldsymbol{\Omega}_{obs} \subset \partial\boldsymbol{\Omega}_g$. 
% Forecasting refers to the problem when the temporal domain $\mathcal{T}$ lies outside of the present, i.e., $T+\tau$ where $\tau$ is some step within the future that is outside of the temporal observation domain, $\mathcal{T}\in[0, T]$. 
% Note that this is irrespective of the spatial domain or its boundaries. 
% In the rest of this paper, we will look exclusively at the interpolation problem, but we refer to the reader to section~\ref{sec:other_tasks} in the appendix for a more detailed look at the other tasks.

% \begin{equation}
%     \mathbf{y} = \mathcal{H}(\mathbf{x})
% \end{equation}


% where $\mathbf{y}\in\boldsymbol{\Omega_p}$ is the incomplete observation within some subdomain and $\mathbf{x}\in\boldsymbol{\Omega}$  is the true observation over the full domain.



% \subsubsection{Forecasting}

% \begin{equation}
%     \boldsymbol{u}_{t+\delta t} = \mathcal{M}(\boldsymbol{u}(\mathbf{x}, t+\delta t), \delta t; \boldsymbol{\theta})
% \end{equation}

% \subsection{SubTasks}

% \subsubsection{Surrogate/Hybrid/Parameterizations}

% \begin{equation}
%     \frac{\partial u}{\partial t} = \mathcal{M}(\boldsymbol{u}(\mathbf{x},t), \mathbf{x}, t; \theta)
% \end{equation}





\newpage
\section{Machine Learning Method Ontology} \label{sec:ml_ontology}

Although this paper does not focus on the explicit methods used for SSH interpolation, we would like to give a readers a brief overview of some of the most popular methods in the literature.

\subsection{Coordinate-Based methods}

These methods learn a direct mapping between the coordinate vectors to the scalar or vector values. 
%
\begin{align}
    \boldsymbol{y}_{obs} &= \boldsymbol{f}(\mathbf{x},t;\boldsymbol{\theta})+\boldsymbol{\epsilon}(\mathbf{x},t)
\end{align}
%
This is better known as \textit{functa}~\cite{FUNCTA} which parameterizes the field directly as a model.

\textbf{Functional}. Optimal Interpolation (OI) is the most common method used for many of the operational methods~\cite{DUACS}. It is a non-parametric, functional method which is built upon covariance and precision matrices. In the machine learning community, these methods are known as Gaussian Process~\cite{GPsBIGDATA} and in the geostatistics community, this is known as Kriging~\cite{KRIGINGREVIEW}.

\textbf{Basis Function}. This is an easy simplification to the functional by introducing parametric basis functions. In particular, the MIOST~\cite{MIOST} algorithm will be adopted in the new operational products for SSH interpolation. It is a custom basis function based on Wavelet analysis which is scale-aware and scalable.

\textbf{Neural Fields}. Neural fields (NerFs) are a very popular set of methods that use neural networks to effectively learn the basis function through a composition of weights, biases and activations~\cite{NERFSSSH}.
Furthermore, one can add physics-informed constraints to the loss function which mirror that of a PDE~\cite{PINNS}.
In many cases, especially with many auxillary inputs, we don't have access the PDE so one fit a NN directly to the observations with a fully connected neural network~\cite{SOCAT}.


\subsection{Grid-Based Methods}

In practice, we often consider the field at a specific discretized setting like a uniform grid or mesh. 
This is because we typically operate on and store these fields as multi-dimensional arrays which are only defined on a subspace of the entire continuous domain. 
We denote a discretized spatial representation as $\boldsymbol{\Omega}_g\subset\mathbb{R}^{N_s}$. 
We can simplify this notation by including the domain within the operator. So equation~\ref{eq:interp_problem} like so:
\begin{equation}\label{eq:interp_problem_discretized}
    \boldsymbol{\eta}(\boldsymbol{\Omega}_{obs},t ) = \mathcal{H}\left(\boldsymbol{\eta}(\boldsymbol{\Omega}_g,t), t, \boldsymbol{\mu},  \boldsymbol{\varepsilon} \right) 
    % + \boldsymbol{\varepsilon}(\boldsymbol{\Omega}_g, t)
\end{equation} 
%
In this equation, $\mathcal{H}$ is the observation operator that transforms the field from the full discretized domain, $\boldsymbol{\Omega}_g$, to the observation domain, $\boldsymbol{\Omega}_{obs}\subset\mathbb{R}^{N_{obs}}$.

\textbf{Direct Methods}. 
These methods take the noisy, incomplete observations and directly feed it to a model that returns the full reconstructed field.
They typically involve training a convolutional neural network or recurrent neural network on pairs of corrupted observations to learn the reconstruction~\cite{SuperResSurvey,IMAGE2IMAGETRANSLATION, IMAGE2IMAGETRANSLATION2}.
This has seem some sucess in applications related to SSH interpolation~\cite{SSHInterpUNet,SSHInterpConvLSTM, SSHInterpAttention}.

\textbf{Traditional Data Assimilation.}
There are many traditional methods that are rooted in data assimilation~\cite{DAGEOSCIENCE}.
For example, the GLORYS~\cite{GLORYS12} method propagates the physical model forwards in time and then \textit{updates} the state based on observations periodically.
A simpler approach is to use a nudging scheme coupled with a simpler physical model~\cite{BFNQG}.


\textbf{End-to-End Learning}. These methods try to solve the problem by learning and end-to-end scheme to solve the model inversion problem.
This is very similar to implicit methods that define a cost function to minimize instead of a minimizing the parameters of a prior model.
Plug-in-Play priors are a popular class of methods that pre-train priors on auxillary observations and then use the prior in the inversion scheme~\cite{DEEPUNFOLDING}.
This has seen a lot of success in SSH interpolation~\cite{4DVARNETSWOT,4DVARNETSST,4DVarNetSSC}.



% \newpage
% In practice, we only consider the field at a specific discretized setting like a uniform grid or mesh. 
% This is because we typically operate on and store these fields as multi-dimensional arrays which are only defined on a subspace of the entire continuous domain. We denote this discretized spatial representation as $\boldsymbol{\Omega}_g\subset\mathbb{R}^{N_s}$. We can simplify this notation by including the domain within the operator, like so:
% %
% \begin{equation} \label{eq:ssh_field_discretized}
%     \eta =\boldsymbol{\eta}(\boldsymbol{\Omega}_g,t),
% \end{equation}
% %
% This is more reflective of how we use operators in practice as we typically insert the field as a grid or multi-dimensional spatial array through a series of mathematical operations.
% We can further discretized this field through the time domain whereby we have a finite set of observations in time, $D_t$. Let's say that we have $N_t$ ordered samples in time between a defined interval of $[0,T]$, i.e. $\mathcal{T}=\left\{ t_t \in [0,T]\right\}_i^{N_{t}}$. In some settings, this could be a uniform observation that is hourly for the period of 1 day, i.e. $N_t=24$, or daily over the period of 10 years, i.e. $N_t=3650$. In more realistic settings, this could be an irregular pattern.
% Equation~\eqref{eq:ssh_field_discretized} assumes the full field is observed. In practice, we observe a corrupted, incomplete version of this SSH field s.t.
% %
% \begin{equation}\label{eq:obs_operator_discretized}
%     \boldsymbol{\eta}(\boldsymbol{\Omega}_{obs},t ) = \mathcal{H}\left(\boldsymbol{\eta}(\boldsymbol{\Omega}_g,t), t, \boldsymbol{\mu},  \boldsymbol{\varepsilon} \right) 
%     % + \boldsymbol{\varepsilon}(\boldsymbol{\Omega}_g, t)
% \end{equation} 

% In this equation, $\mathcal{H}$ is the observation operator that transforms the field from the full discretized domain, $\boldsymbol{\Omega}_g$, to the observation domain, $\boldsymbol{\Omega}_{obs}\subset\mathbb{R}^{N_{obs}}$. 
% The observation domain, $\boldsymbol{\Omega}_{obs}$, are the spatial coordinates, $\mathbf{x}$, where the SSH has been observed which is a proper subset of the full domain, i.e. $\boldsymbol{\Omega}_{obs}\subseteq\boldsymbol{\Omega}_g$.

% \begin{align} \label{eq:interp_problem}
%     \mathcal{M}_{\boldsymbol{\theta}} &: \boldsymbol{\eta}_{obs}(\mathbf{x}, t, \boldsymbol{\mu}) \rightarrow \boldsymbol{\eta}(\mathbf{x},t) \hspace{10mm}
%      \boldsymbol{\Omega}_{obs} \in \mathbb{R}^{N_{obs}} \hspace{5mm} 
%     \boldsymbol{\Omega}_g \in \mathbb{R}^{N_s} \hspace{5mm} 
%     t \in \mathcal{T}.
% \end{align}


% \subsubsection{Hybrid Schemes} \label{sec:bfn}

% This includes Backwards-Forwards Nudging with a simpler model like the Quasi-Geostrophic equations.

% \subsubsection{Direct Methods}

% These feature methods that directly take in the sparse SSH fields with \tocite{Recent Papers}

% \subsubsection{Data Assimilation}

% \begin{align}
%     \boldsymbol{\theta}^* &= \underset{\boldsymbol{\theta}}{\text{argmin}} \hspace{2mm} \mathcal{L}(\mathbf{x}_{gt}, \mathbf{x}(\boldsymbol{\theta})) \\
%     \mathbf{x}^*(\boldsymbol{\theta}) &= \underset{ \mathbf{x}^*}{\text{argmin}}  \hspace{2mm} \mathcal{U}\left(\mathbf{x}(\boldsymbol{\theta}) \right)
% \end{align}

% where $\mathcal{L}$ is some loss function to find the best parameters and $\mathcal{U}$ is some energy function to minimize the state.

% \subsubsection{End-to-End 4DVariational Methods} \label{sec:4dvarnet}

% These methods take a variational approach whereby we directly define the cost function we wish to solve \tocite{Ronan}. To speed up the convergence of these methods, a meta-learning based method was introduced to learn the gradient descent\tocite{Ronan}. In addition, this has been improved with the addition of sea surface temperature as an additional term in the cost function. 



% \newpage
% \section{Target Audience}
% \subsection{Project Vision}
While this tool is general in scope, we specifically target three audiences: 1) the domain expert who may want to use and understand and investigate SSH in relation to other important EO quantities, 2) machine learning researchers who may want to investigate how to make a better model for SSH interpolation, and 3) downstream users who are interested in adopting some techniques for their own domain-specific applications that may rely on SSH like tracking ocean currents~\tocite{}, investigating biogeochemical transport~\tocite{}, or global climate change~\tocite{}. In the following subsections, we give a more detailed description about the users and how might they benefit from \texttt{OceanBench}.

\textbf{Domain Experts}. We consider \textit{domain experts} who are experts in different domains of oceanography. ...


\textbf{ML Researchers}. We consider those who are expert ML researchers but may lack...



\textbf{Down Stream Users}. We also envision a broader adoption of the framework across research labs interested in having standardized data challenges for their own research purposes. There are independent groups choosing datasets for their specific use cases however, this framework can serve as an easy way to integrate their existing methodologies into the set of common tools to be used and improved by multiple communities. By establishing a relatively consistent problem set, we hope that any innovation can be easily understood and transferred across domains.
\newpage
\section{Limitations} \label{sec:appendix_limitations}

\subsection{Framework Limitations}

While we have advertised \texttt{OceanBench} as a unifying framework that provides standardized processing steps that comply with domain-expert standards, we also highlight some potential limitations that could hinder its adoption for the wider community.

\textbf{Data Serving}. We provide a few datasets but we omit some of the original simulations. We found that the original simulations are terabytes/petabytes of data which becomes infeasible for most modest users (even with adequate CPU resources).  
This is very big problem and if we want to have a bigger impact, we may need to do more close collaborations with specified platforms like the Marine Data Store~\citep{MDSOCEANPHYSICS,MDSBIOGEOCHEMICAL,MDSOCEANPHYSICSENS,MDSINSITU,MDSWAVES,MDSALONGTRACK,MDSSSH} or the Climate Data Store~\citep{CDSREANALYSISSST,CDSOBSSST,CDSOBSOC,CDSOBSSSTENS}. Furthermore, there are many people that will not be able to do a lot of heavy duty research which indirectly favours institutions with adequate resources and marginalizing others. 
This is also problematic as those communities tend to be the ones who need the most support from the products of such frameworks.
We hope that leaving this open-source at least ensure that the knowledge is public.

\textbf{Framework Dependence}. The user has to "buy-into" the \texttt{hydra} framework to really take advantage of \texttt{OceanBench}. This adds a layer of abstraction and a new tool to learn. 
However, we designed the project so that high level usage does not require in-depth knowledge of the framework. 
In addition, we hope that, despite the complexity of project, users will appreciate the flexibility and extensibility of this framework.


\textbf{Lack of Metrics}. We do not provide the most exhaustive list of metrics available with the ocean community. In fact, we also believe that many of these metrics are often poor and do not effectively assess the goodness of our reconstructions. 
However, we do provide a platform that will hopefully be useful and easy to implement new and improved metrics.
Furthermore, having a wide range of metrics that are trusted across communities may help to improve the overall assessment of the different model performances~\cite{METRICSAVERAGE}.

\textbf{Limited ML Scope}. 
The framework does not support nor promote any machine learning methods and we lack any indication of comparing ML training and inference performance. 
However, we argue that a benchmark framework will allow us to effectively compare whichever ML methods are demonstratively the best which is a necessary preliminary step which offers users more flexibility in the long-run.

\textbf{Broad Oceans Application Scope}. 
We have targeted a broad ocean-application scope of state estimation.
However, there may be more urgent applications such as maritime monitoring, object tracking, and general ocean health.
However, we feel that many downstream applications require high-quality maps.
In addition, those downstream applications tend to be very complicated and are not always straightforward to apply ML under those instances.

\textbf{Full Pipeline Transparency}. We use a lot of different \texttt{xarray}-specific packages which have different design principles, assumptions and implementations. This may give the users an illusion of simplicity and transparency to real-world use. However, there are many underlying assumptions within each of the packages that may occlude a lot of design decisions.
Despite this limitation, we believe that being transparent about the processing steps and being consistent with the evaluation procedure will be beneficial for the ML research community.

\textbf{Scalability}. Scaling this to many terabytes or petabytes of data is easily the biggest limitation of the framework. In addition, we have only showcased demonstrations for 2D+T fields which are much less expensive than 3D+T fields.

\textbf{Deployability}. MLOPs has many wheels and it is not easy to integrate into existing systems. We offer no solutions to this. 
However, we believe that our framework is fully transparent in the assumptions and use cases which will facilitate some adoption into operational systems where they can further modify it for their use cases (see the evolution of \texttt{WeatherBench} and \texttt{ClimateBench}).

\textbf{Visualization Tools}.
We do not incorporate a high quality visualization tool that allows users to do pre- and post-analysis at a large scale. 
We do provide some simple visualization steps that are ML-relevant (see the GitHub repo) but it is very limited to ML standards.
One solution is to interface our pipeline with the source of many ocean datasets, e.g. Climate Data Store~\citep{CDSREANALYSISSST} or Marine Data Store~\citep{MDSOCEANPHYSICS}, then we can offset this task to them where they can offer better quality visualization tools.

\newpage
\subsection{Data Challenge Limitations}

We have showcased the SSH interpolation edition as a data challenge which could be helpful for real applications. 
However, in section~\ref{sec:problem_scope} we alluded to the greater task of general ocean state estimation which is more pertinent to the ocean sciences yet we don't address this directly with our data challenges.
Furthermore, we claim that the data challenges presented will help the ocean community with using ML for SSH interpolation.
Below, we outline some limitations which address these criticisms.

\textbf{Not the overall objective}. 
We recognized that we are far away from the actual reanalysis and forecasting goals of full state estimation. 
However, we argue that that is a rather ambitious challenge which will require a lot of interdisciplinary work across communities. 
In the meantime while we work towards that goal, operational centers could possibly improve their current products from ML-based techniques would would benefit downstream applications that deal directly with SSH.
Furthermore, SSH is an important variable in describing the full ocean state.
So a robust set of techniques that are able to solve the interpolation tasks could (in principal) be used to solve extra tasks.

\textbf{Small Region \& Period}.
We only feature a small region and period over the Gulfstream which is not representative of the different global regimes. 
This also does not take into account real things like \textit{data drift} which will inevitable occur in operational settings.
However, this is a dynamical regime and a well-studied area which does have some importance for specific communities and the results obtained offer some transferability to other dynamical regimes.
In addition, this area will have good coverage due to the new SWOT mission~\cite{SWOT} which will allow for further validation in the future.
Lastly, the area is small enough where the beginning stages for ML researchers is not overwhelmed with problems involving scale (even though we eventually want to arrive at global schemes).
We hope to extend our challenges to more relevant scenarios~\cite{MDSALONGTRACK}.

\textbf{Simulations versus Reanalysis}. We use simulations for the OSSE experiments instead of reanalysis. This is an open research question as it is unclear whether it's better to pretrain models on simulated ocean data or reanalysis ocean data. In future updates, we plan to add the reanalysis data to extend the challenge.

\textbf{Efficacy of OSSE Experiments}. We alluded to the idea that the OSSE experiments may not reflect the overarching goal of the user yet we provide more OSSE experiments than OSE experiments.
We acknowledged that it often does not coincide exactly with the OSE experiments which may give users a false sense of accomplishment and immediate transferability. 
However, we try to provide a framework where one could thoroughly experiment with the learning problem on OSSE configurations which can facilitate transfer learning to other domain-specific tasks.
We also anticipate that new \textit{real} SWOT data~\cite{SWOT} will start to become more available which will allow us to design better, realistic OSE experiments.

\textbf{Noise Characterization}.
Real data has noise to content with and we do not account for that within the SSH interpolation experiments.
The true noise we see in operational settings is structured and this would require more knowledge outside the scope of our teams expertise.
A more improved challenge would take these considerations into account.
We leave this as a future challenge for the community and we hope our platform can help facilitate this.

\textbf{Uncertainty Quantitification}.
We prefaced the problem statement with the idea of data assimilation which is the notion of \textit{state/parameter estimation under uncertain conditions and incomplete information}~\citep{DAGEOSCIENCE}.
However, we have not addressed any notion of uncertainty at all throughout the paper.
Uncertainty is difficult to quantify and we don't want to impose too many restrictions until we more sure about the efficacy of ML for easier problems.
However, to move the problem setting towards a more realistic setting, we can start to introduce metrics and additional requirements from future challenges, e.g. mean and standard deviation estimates or ensemble predictions.


\textbf{Operational Constraints}.
The real use case of SSH interpolation will involve global data and/or high-resolution data. 
This involves dealing with very high-dimensional spatiotemporal global state-space.
In practice, the necessity for the scalability of the method is of paramount importance.
However, there are also areas within the ML research community who are looking into many ways we can scale up physical models~\citep{VEROS,OCEANANIGANS} and machine learn models for geoscience tasks~\citep{SFNO}.
We anticipate that once a set of solutions are excepted by a community, the scalability will come later.




% \newpage
% \subsection{Dataset Limitations}



% \textbf{Limitations}. The scope of this is very specific to researchers who are interested in ML. It only serves as a baseline design. If one would like to scale, it would require much more engineering work to get everything to connect. This is already seen if we are to load datasets of PBs. In addition, this requires researchers to have access to considerably large machines to be able to run their own preprocessing schemes. We do our best to provide toy datasets of a modest size however, inevitably, one will probably need to work with larger and large datasets.

% \subsection*{Broader Impact}
% \label{sec:impact}

% The theme of interpolation is present in many applied communities with different names, e.g. kriging in ecology/hydrology, Optimal interpolation in oceanography, and Gaussian processes in statistics. We hope that this work bridges this gap between the communities and we invite other works to try to highlight concrete ways that machine learning and classic physics have commonalities.

% In the oceanography community in particular, we especially hope to see more adoption of machine learning methods for interpolation. DUACS is ultimately a closed-system so the wider scientific community does not have access to the algorithm. Our OI baseline hopefully unveils some of the finer details of the method. However, in general, the standard OI methods used in the applied community cannot keep up with the massive influxes of observations we receive. So this work is a first step in demonstrating that neural networks (in particular NerFs) are a viable, simpler, and scalable alternative.
% \section*{User Guide}
% \addcontentsline{toc}{section}{User Guide}

% \addcontentsline{toc}{section}{Bibliography}
  \putbib[00_Oceanbench/content/bibliographies/full.bib]
\end{bibunit}


\clearemptydoublepage
% Pour avoir la quatrième de couverture sur une page paire
% To have the back cover on an even page
\cleartoevenpage[\thispagestyle{empty}]
\markboth{}{}
% Plus petite marge du bas pour la quatrième de couverture
% Shorter bottom margin for the back cover
\newgeometry{inner=30mm,outer=20mm,top=40mm,bottom=20mm}

%insertion de l'image de fond du dos (resume)
%background image for resume (back)
\backcoverheader

% Switch font style to back cover style
\selectfontbackcover{ % Font style change is limited to this page using braces, just in case

\titleFR{Apprentissage profond pour l'altimétrie satellitaire océanique : sp\'{e}cificit\'{e}s et implications pratiques.}

\keywordsFR{Apprentissage profond, Altimétrie, SWOT}

\abstractFR{Cette thèse explore comment les avancées en apprentissage profond peuvent aider à l'analyse des mesures satellitaires de la hauteur de surface de la mer (SSH). Les altimètres actuels fournissent des données échantillonnées de manière irrégulière limitant l'observation des processus les plus fins. Repousser cette limite améliorerait nos capacités de surveillance du climat. D'excitantes opportunités ont émergées avec la mission SWOT. Les approches d'apprentissage ont démontré des capacités remarquables dans de nombreux domaines. Cette thèse aborde les considérations spécifiques de l'application de l'apprentissage profond aux données altimétriques en trois parties.

Premièrement, à travers l'étalonnage du capteur KaRIn, nous démontrons comment des connaissances spécifiques du domaine peuvent être intégrées dans les cadres d'apprentissage profond. Deuxièmement, nous abordons la rareté des données de vérité terrain lors de l'apprentissage de méthodes  d'interpolation de données altimétriques. Nous illustrons comment les simulations de modèles océaniques et de systèmes d'observation peuvent surmonter ce défi en fournissant des environnements d'entraînement supervisés qui se généralisent aux données réelles.Enfin, notre troisième contribution traite des défis rencontrés pour combler le fossé entre les communautés "océan" et "apprentissage profond". Nous décrivons comment nous avons abordé ces aspects lors du développement du projet OceanBench.}



\titleEN{Deep Learning for ocean satellite altimetry : specificities and
practical implications}

\keywordsEN{Deep Learning, Altimetry, SWOT}

\abstractEN{This thesis explores how advancements in deep learning can aid in the analysis of satellite measurements of sea surface height (SSH). Current altimeters provide data sampled in an irregular manner, limiting the observation of finer processes. Pushing this limit would enhance our climate monitoring capabilities. Exciting opportunities have emerged with the SWOT mission. Learning approaches have shown remarkable capabilities in many areas. This thesis addresses the specific considerations of applying deep learning to altimetry data in three parts.

First, through the calibration of the KaRIn sensor, we demonstrate how specific domain knowledge can be integrated into deep learning frameworks. Second, we address the scarcity of ground truth data when learning altimetry data interpolation methods. We illustrate how ocean model simulations and observation systems can overcome this challenge by providing supervised training environments that generalize to real data. Lastly, our third contribution discusses the challenges faced in bridging the gap between the "ocean" and "deep learning" communities. We describe how we approached these aspects during the development of the OceanBench project.}

}

% Rétablit les marges d'origines
% Restore original margin settings
\restoregeometry

\end{document}
