\begin{bibunit}[IEEEtran.bst]

\chapter*{Résumé en français}
Cette thèse explore comment les avancées en apprentissage profond peuvent aider à l'analyse des mesures satellitaires de la hauteur de surface de la mer (SSH). Les altimètres actuels fournissent des données échantillonnées de manière irrégulière avec un faible couverture spatiale, limitant ainsi l'observation de processus prenant places aux petites échelles. Repousser cette limite améliorerait notre connaissance des dynamique de la surface de l'océan ainsi que nos capacités de surveillance du climat. De nouvelles opportunités pour renforcer nos capacités d'observation ont émergé avec le déploiement du capteur KaRIn lors de la mission SWOT.

Ces dernières années, les approches d'apprentissage ont démontré des capacités remarquables en vision par ordinateur et en traitement du langage naturel. Contrairement à de telles tâches, les problèmes d'observations océaniques peuvent impliquer de forts a priori physiques ainsi que peu de données annotées. Ce travail aborde les considérations spécifiques de l'application de l'apprentissage profond aux données altimétriques en trois parties.

Premièrement, à travers l'étalonnage du capteur KaRIn, nous démontrons comment des connaissances spécifiques du domaine peuvent être intégrées dans les cadres d'apprentissage profond. Nous montrons spécifiquement comment le budget d'erreur spectral de la mission SWOT peut informer la conception d'architectures neuronales.

Deuxièmement, nous abordons la rareté des données de vérité terrain lors de la conception de méthodes neuronales d'interpolation de données altimétriques. Nous illustrons comment les simulations de modèles océaniques et de systèmes d'observation peuvent surmonter ce défi en fournissant des environnements d'entraînement supervisés qui se généralisent aux données réelles.

Enfin, notre troisième contribution traite des défis rencontrés pour combler le fossé entre les communautés "océan" et "apprentissage profond". Une collaboration efficace nécessite que des experts de l'océan définissent les défis d'intérêt ainsi que la méthode adéquate d'évaluer les solutions. Le praticien en apprentissage automatique nécessite de son côté les outils nécessaires pour accéder et manipuler les différentes données pertinentes.  Nous décrivons comment nous avons abordé ces aspects lors du développement du projet OceanBench.

% \section*{Motivations}

% \section*{Objectifs}

% \section*{Contributions}

% \section*{Contenu du manuscrit}

% In this manuscript I'd like to cite \cite{remo1,remo2}.

% \putbib[./Resume-Francais/Res-Biblio.bib]
\end{bibunit}